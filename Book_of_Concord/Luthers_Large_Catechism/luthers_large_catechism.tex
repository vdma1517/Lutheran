The Large Catechism by Dr. Martin Luther

Translated by F. Bente and W. H. T. Dau


Published in:
Triglot Concordia: The Symbolical Books of the Ev. Lutheran Church.
St. Louis: Concordia Publishing House, 1921), pp. 565-773

Preface

A Christian, Profitable, and Necessary Preface and Faithful, Earnest
Exhortation of Dr. Martin Luther to All Christians, but Especially to
All Pastors and Preachers, that They Should Daily Exercise Themselves
in the Catechism, which is a Short Summary and Epitome of the Entire
Holy Scriptures, and that they May Always Teach the Same.

We have no slight reasons for treating the Catechism so constantly [in
Sermons] and for both desiring and beseeching others to teach it, since
we see to our sorrow that many pastors and preachers are very negligent
in this, and slight both their office and this teaching; some from
great and high art [giving their mind, as they imagine, to much higher
matters], but others from sheer laziness and care for their paunches,
assuming no other relation to this business than if they were pastors
and preachers for their bellies' sake, and had nothing to do but to
[spend and] consume their emoluments as long as they live, as they have
been accustomed to do under the Papacy.


 And although they have now everything that they are to preach and
teach placed before them so abundantly, clearly, and easily, in so many
[excellent and] helpful books, and the true Sermones per se loquentes,
Dormi secure, Paratos et Thesauros, as they were called in former
times; yet they are not so godly and honest as to buy these books, or
even when they have them, to look at them or read them. Alas! they are
altogether shameful gluttons and servants of their own bellies who
ought to be more properly swineherds and dog-tenders than care-takers
of souls and pastors.

And now that they are delivered from the unprofitable and burdensome
babbling of the Seven Canonical Hours, oh, that, instead thereof, they
would only, morning, noon, and evening, read a page or two in the
Catechism, the Prayer-book, the New Testament, or elsewhere in the
Bible, and pray the Lord's Prayer for themselves and their
parishioners, so that they might render, in return, honor and thanks to
the Gospel, by which they have been delivered from burdens and troubles
so manifold, and might feel a little shame because like pigs and dogs
they retain no more of the Gospel than such a lazy, pernicious,
shameful, carnal liberty! For, alas! as it is, the common people regard
the Gospel altogether too lightly, and we accomplish nothing
extraordinary even though we use all diligence. What, then, will be
achieved if we shall be negligent and lazy as we were under the Papacy?


To this there is added the shameful vice and secret infection of
security and satiety, that is, that many regard the Catechism as a
poor, mean teaching, which they can read through at one time, and then
immediately know it, throw the book into a corner, and be ashamed, as
it were, to read in it again.

Yea, even among the nobility there may be found some louts and
scrimps, who declare that there is no longer any need either of
pastors or preachers; that we have everything in books, and every one
can easily learn it by himself; and so they are content to let the
parishes decay and become desolate, and pastors and preachers to suffer
distress and hunger a plenty, just as it becomes crazy Germans to do.
For we Germans have such disgraceful people, and must endure them.

But for myself I say this: I am also a doctor and preacher, yea, as
learned and experienced as all those may be who have such presumption
and security; yet I do as a child who is being taught the Catechism,
and every morning, and whenever I have time, I read and say, word for
word, the Ten Commandments, the Creed, the Lord's Prayer, the Psalms,
etc. And I must still read and study daily, and yet I cannot master it
as I wish, but must remain a child and pupil of the Catechism, and am
glad so to remain. And yet these delicate, fastidious fellows would
with one reading promptly be doctors above all doctors, know everything
and be in need of nothing. Well, this, too, is indeed a sure sign that
they despise both their office and the souls of the people, yea, even
God and His Word. They do not have to fall, they are already fallen all
too horribly, they would need to become children, and begin to learn
their alphabet, which they imagine that they have long since outgrown.

Therefore I beg such lazy paunches or presumptuous saints to be
persuaded and believe for God's sake that they are verily, verily! not
so learned or such great doctors as they imagine; and never to presume
that they have finished learning this [the parts of the Catechism], or
know it well enough in all points, even though they think that they
know it ever so well. For though they should know and understand it
perfectly (which, however, is impossible in this life), yet there are
manifold benefits and fruits still to be obtained, if it be daily read
and practiced in thought and speech; namely, that the Holy Ghost is
present in such reading and repetition and meditation, and bestows ever
new and more light and devoutness, so that it is daily relished and
appreciated better, as Christ promises, Matt. 18, 20: Where two or
three are gathered together in My name, there am I in the midst of
them.

Besides, it is an exceedingly effectual help against the devil, the
world, and the flesh and all evil thoughts to be occupied with the Word
of God, and to speak of it, and meditate upon it, so that the First
Psalm declares those blessed who meditate upon the law of God day and
night. Undoubtedly, you will not start a stronger incense or other
fumigation against the devil than by being engaged upon God's
commandments and words, and speaking, singing, or thinking of them. For
this is indeed the true holy water and holy sign from which he flees,
and by which he may be driven away.

Now, for this reason alone you ought gladly to read, speak, think and
treat of these things if you had no other profit and fruit from them
than that by doing so you can drive away the devil and evil thoughts.
For he cannot hear or endure God's Word; and God's Word is not like
some other silly prattle, as that about Dietrich of Berne, etc., but as
St. Paul says, Rom. 1, 16, the power of God. Yea, indeed, the power of
God which gives the devil burning pain, and strengthens, comforts, and
helps us beyond measure.

And what need is there of many words ? If I were to recount all the
profit and fruit which God's Word produces, whence would I get enough
paper and time? The devil is called the master of a thousand arts. But
what shall we call God's Word, which drives away and brings to naught
this master of a thousand arts with all his arts and power? It must
indeed be the master of more than a hundred thousand arts. And shall we
frivolously despise such power, profit, strength, and fruit -- we,
especially, who claim to be pastors and preachers? If so, we should not
only have nothing given us to eat, but be driven out, being baited with
dogs, and pelted with dung, because we not only need all this every day
as we need our daily bread but must also daily use it against the daily
and unabated attacks and lurking of the devil, the master of a thousand
arts.

And if this were not sufficient to admonish us to read the Catechism
daily, yet we should feel sufficiently constrained by the command of
God alone, who solemnly enjoins in Deut. 6, 6 ff. that we should always
meditate upon His precepts, sitting, walking, standing, Lying down, and
rising, and have them before our eyes and in our hands as a constant
mark and sign. Doubtless He did not so solemnly require and enjoin this
without a purpose; but because He knows our danger and need, as well as
the constant and furious assaults and temptations of devils, He wishes
to warn, equip, and preserve us against them, as with a good armor
against their fiery darts and with good medicine against their evil
infection and suggestion.

Oh, what mad, senseless fools are we that, while we must ever live and
dwell among such mighty enemies as the devils are, we nevertheless
despise our weapons and defense, and are too lazy to look at or think
of them! And what else are such supercilious, presumptuous saints, who
are unwilling to read and study the Catechism daily, doing than
esteeming themselves much more learned than God Himself with all His
saints, angels [patriarchs], prophets, apostles, and all Christians For
inasmuch as God Himself is not ashamed to teach these things daily, as
knowing nothing better to teach, and always keeps teaching the same
thing, and does not take up anything new or different, and all the
saints know nothing better or different to learn, and cannot finish
learning this, are we not the finest of all fellows to imagine, if we
have once read or heard it, that we know it all, and have no further
need to read and learn, but can finish learning in one hour what God
Himself cannot finish teaching, although He is engaged in teaching it
from the beginning to the end of the world, and all prophets, together
with all saints, have been occupied with learning it and have ever
remained pupils, and must continue to be such ?

For it needs must be that whoever knows the Ten Commandments perfectly
must know all the Scriptures, so that, in all affairs and cases, he can
advise, help, comfort, judge, and decide both spiritual and temporal
matters and is qualified to sit in judgment upon all doctrines,
estates, spirits, laws, and whatever else is in the world. And what,
indeed, is the entire Psalter but thoughts and exercises upon the First
Commandment? Now I know of a truth that such lazy paunches and
presumptuous spirits do not understand a single psalm, much less the
entire Holy Scriptures; and yet they pretend to know and despise the
Catechism, which is a compend and brief summary of all the Holy
Scriptures.

Therefore I again implore all Christians, especially pastors and
preachers, not to be doctors too soon, and imagine that they know
everything (for imagination and cloth unshrunk [and false weights] fall
far short of the measure), but that they daily exercise themselves well
in these studies and constantly treat them; moreover, that they guard
with all care and diligence against the poisonous infection of such
security and vain imagination, but steadily keep on reading, teaching,
learning, pondering, and meditating, and do not cease until they have
made a test and are sure that they have taught the devil to death and
have become more learned than God Himself and all His saints.

If they manifest such diligence, then I will promise them, and they
shall also perceive, what fruit they will obtain, and what excellent
men God will make of them, so that in due time they themselves will
acknowledge that the longer and the more they study the Catechism, the
less they know of it, and the more they find yet to learn; and then
only, as hungry and thirsty ones, will they truly relish that which now
they cannot endure because of great abundance and satiety. To this end
may God grant His grace! Amen.


 SHORT PREFACE OF DR. MARTIN LUTHER.


This sermon is designed and undertaken that it might be an instruction
for children and the simple-minded. Hence of old it was called in Greek
catechism, i.e., instruction for children, what every Christian must
needs know, so that he who does not know this could not be numbered
with the Christians nor be admitted to any Sacrament, just as a
mechanic who does not understand the rules and customs of his trade is
expelled and considered incapable. Therefore we must have the young
learn the parts which belong to the Catechism or instruction for
children well and fluently and diligently exercise themselves in them
and keep them occupied with them.

Therefore it is the duty of every father of a family to question and
examine his children and servants at least once a week and to
ascertain what they know of it, or are learning and, if they do not
know it, to keep them faithfully at it. For I well remember the time,
indeed, even now it is a daily occurrence that one finds rude, old
persons who knew nothing and still know nothing of these things, and
who, nevertheless, go to Baptism and the Lord's Supper, and use
everything belonging to Christians, notwithstanding that those who come
to the Lord's Supper ought to know more and have a fuller understanding
of all Christian doctrine than children and new scholars. However, for
the common people we are satisfied with the three parts, which have
remained in Christendom from of old, though little of it has been
taught and treated correctly until both young and old who are called
and wish to be Christians, are well trained in them and familiar with
them. These are the following:


First.

THE TEN COMMANDMENTS OF GOD.


1. Thou shalt have no other gods before Me.

2. Thou shalt not take the name of the Lord, thy God, in vain [for the
Lord will not hold him guiltless that taketh His name in vain].

3. Thou shalt sanctify the holy-day. [Remember the Sabbath-day to keep
it holy.]

4. Thou shalt honor thy father and mother [that thou mayest live long
upon the earth].

5. Thou shalt not kill.

6. Thou shalt not commit adultery.

7. Thou shalt not steal.

8. Thou shalt not bear false witness against thy neighbor.

9. Thou shalt not covet thy neighbor's house.

10. Thou shalt not covet thy neighbor's wife, nor his man-servant, nor
his maidservant, nor his cattle [ox, nor his ass], nor anything that is
his.

Secondly.

THE CHIEF ARTICLES OF OUR FAITH.


1. I believe in God the Father Almighty, Maker of heaven and earth.

2. And in Jesus Christ, His only Son, our Lord; who was conceived by
the Holy Ghost, born of the Virgin Mary; suffered under Pontius Pilate,
was crucified, dead and buried; He descended into hell; the third day
He rose again from the dead; He ascended into heaven, and sitteth on
the right hand of God the Father Almighty; from thence He shall come to
judge the quick and the dead.

3. I believe in the Holy Ghost, the holy Christian Church, the
communion of saints, the forgiveness of sins, the resurrection of the
body, and the life everlasting. Amen.


Thirdly.

THE PRAYER, OR "OUR FATHER," WHICH CHRIST TAUGHT

Our Father who art in heaven.

1. Hallowed be Thy name.

2. Thy kingdom come.

3. Thy will be done on earth as it is in heaven.

4. Give us this day our daily bread.

5. And forgive us our trespasses as we forgive those who trespass
against us.

6. And lead us not into temptation.

7. But deliver us from evil. [For Thine is the kingdom and the power
and the glory, forever and ever.] Amen.


These are the most necessary parts which one should first learn to
repeat word for word and which our children should be accustomed to
recite daily when they arise in the morning when they sit down to their
meals, and when they retire at night; and until they repeat them, they
should be given neither food nor drink. Likewise every head of a
household is obliged to do the same with respect to his domestics,
ma-servants and maid-servants and not to keep them in his house if they
do not know these things and are unwilling to learn them. For a person
who is so rude and unruly as to be unwilling to learn these things is
not to be tolerated, for in these three parts everything that we have
in the Scriptures is comprehended in short, pain, and simple terms. For
the holy Fathers or apostles (whoever they were) have thus embraced in
a summary the doctrine, life, wisdom, and art of Christians, of which
they speak and treat, and with which they are occupied. Now, when these
three arts are apprehended, it behooves a person also to know what to
say concerning our Sacraments, which Christ Himself instituted, Baptism
and the holy body and blood of Christ, namely, the text which Matthew
[28, 19 ff.] and Mark [16, 15 f.] record at the close of their Gospels
when Christ said farewell to His disciples and sent them forth.

OF BAPTISM.

Go ye and teach all nations, baptizing them in the name of the Father,
and of the Son, and of the Holy Ghost. He that believeth and is
baptized shall be saved; but he that believeth not shall be damned. So
much is sufficient for a simple person to know from the Scriptures
concerning Baptism. In like manner, also, concerning the other
Sacrament in short, simple words, namely the text of St. Paul [1 Cor.
11, 23 f.].

OF THE SACRAMENT

Our Lord Jesus Christ, the same night in which He was betrayed, took
bread; and when He had given thanks, He brake it, and gave it to His
disciples and said, Take, eat; this is My body, which is given for you:
this do in remembrance of Me.

After the same manner also He took the cup, when He had supped, gave
thanks, and gave it to them, saying, Drink ye all of it; this cup is
the new testament in My blood, which is shed for you for the remission
of sins: this do ye, as oft as ye drink it, in remembrance of Me.

Thus, ye would have, in all, five parts of the entire Christian
doctrine which should be constantly treated and required [of children]
and heard recited word for word. For you must not rely upon it that the
young people will learn and retain these things from the sermon alone.
When these parts have been well learned, you may, as a supplement and
to fortify them. lay before them also some psalms or hymns, which have
been composed on these parts, and thus lead the young into the
Scriptures, and make daily progress therein.

However, it is not enough for them to comprehend and recite these
parts according to the words only, but the young people should also be
made to attend the preaching, especially during the time which is
devoted to the Catechism, that they may hear it explained and may learn
to understand what every part contains, so as to be able to recite it
as they have heard it, and, when asked, may give a correct answer, so
that the preaching may not be without profit and fruit. For the reason
why we exercise such diligence in preaching the Catechism so often is
that it may be inculcated on our youth, not in a high and subtle
manner, but briefly and with the greatest simplicity, so as to enter
the mind readily and be fixed in the memory. Therefore we shall now
take up the above mentioned articles one by one and in the plainest
manner possible say about them as much as is necessary.



 Part First. The Ten Commandments.


The First Commandment.


Thou shalt have no other gods before Me.


That is: Thou shalt have [and worship] Me alone as thy God. What is the
force of this, and how is it to be understood? What does it mean to
have a god? or, what is God? Answer: A god means that from which we are
to expect all good and to which we are to take refuge in all distress,
so that to have a God is nothing else than to trust and believe Him
from the [whole] heart; as I have often said that the confidence and
faith of the heart alone make both God and an idol. If your faith and
trust be right, then is your god also true; and, on the other hand, if
your trust be false and wrong, then you have not the true God; for
these two belong together faith and God. That now, I say, upon which
you set your heart and put your trust is properly your god.

Therefore it is the intent of this commandment to require true faith
and trust of the heart which settles upon the only true God and clings
to Him alone. That is as much as to say: "See to it that you let Me
alone be your God, and never seek another," i.e.: Whatever you lack of
good things, expect it of Me, and look to Me for it, and whenever you
suffer misfortune and distress, creep and cling to Me. I, yes, I, will
give you enough and help you out of every need; only let not your heart
cleave to or rest in any other.

This I must unfold somewhat more plainly, that it may be understood and
perceived by ordinary examples of the contrary. Many a one thinks that
he has God and everything in abundance when he has money and
possessions; he trusts in them and boasts of them with such firmness
and assurance as to care for no one. Lo, such a man also has a god,
Mammon by name, i.e., money and possessions, on which he sets all his
heart, and which is also the most common idol on earth. He who has
money and possessions feels secure, and is joyful and undismayed as
though he were sitting in the midst of Paradise. On the other hand, he
who has none doubts and is despondent, as though he knew of no God. For
very few are to be found who are of good cheer, and who neither mourn
nor complain if they have not Mammon. This [care and desire for money]
sticks and clings to our nature, even to the grave.

So, too, whoever trusts and boasts that he possesses great skill,
prudence, power, favor friendship, and honor has also a god, but not
this true and only God. This appears again when you notice how
presumptuous, secure, and proud people are because of such
possessions, and how despondent when they no longer exist or are
withdrawn. Therefore I repeat that the chief explanation of this point
is that to have a god is to have something in which the heart entirely
trusts.

Besides, consider what in our blindness, we have hitherto been
practicing and doing under the Papacy. If any one had toothache, he
fasted and honored St. Apollonia [lacerated his flesh by voluntary
fasting to the honor of St. Apollonia]; if he was afraid of fire, he
chose St. Lawrence as his helper in need; if he dreaded pestilence, he
made a vow to St. Sebastian or Rochio, and a countless number of such
abominations, where every one selected his own saint, worshiped him,
and called for help to him in distress. Here belong those also, as,
e.g., sorcerers and magicians, whose idolatry is most gross, and who
make a covenant with the devil, in order that he may give them plenty
of money or help them in love-affairs, preserve their cattle, restore
to them lost possessions, etc. For all these place their heart and
trust elsewhere than in the true God, look for nothing good to Him nor
seek it from Him.

Thus you can easily understand what and how much this commandment
requires, namely, that man's entire heart and all his confidence be
placed in God alone, and in no one else. For to have God, you can
easily perceive, is not to lay hold of Him with our hands or to put Him
in a bag [as money], or to lock Him in a chest [as silver vessels]. But
to apprehend Him means when the heart lays hold of Him and clings to
Him. But to cling to Him with the heart is nothing else than to trust
in Him entirely. For this reason He wishes to turn us away from
everything else that exists outside of Him, and to draw us to Himself,
namely, because He is the only eternal good. As though He would say:
Whatever you have heretofore sought of the saints, or for whatever
[things] you have trusted in Mammon or anything else, expect it all of
Me, and regard Me as the one who will help you and pour out upon you
richly all good things.

Lo, here you have the meaning of the true honor and worship of God,
which pleases God, and which He commands under penalty of eternal
wrath, namely, that the heart know no other comfort or confidence than
in Him, and do not suffer itself to be torn from Him, but, for Him,
risk and disregard everything upon earth. On the other hand, you can
easily see and judge how the world practices only false worship and
idolatry. For no people has ever been so reprobate as not to institute
and observe some divine worship; every one has set up as his special
god whatever he looked to for blessings, help, and comfort.

Thus, for example, the heathen who put their trust in power and
dominion elevated Jupiter as the supreme god; the others, who were bent
upon riches, happiness, or pleasure, and a life of ease, Hercules,
Mercury, Venus or others; women with child, Diana or Lucina, and so on;
thus every one made that his god to which his heart was inclined, so
that even in the mind of the heathen to have a god means to trust and
believe. But their error is this that their trust is false and wrong
for it is not placed in the only God, besides whom there is truly no
God in heaven or upon earth. Therefore the heathen really make their
self-invented notions and dreams of God an idol, and put their trust in
that which is altogether nothing. Thus it is with all idolatry; for it
consists not merely in erecting an image and worshiping it, but rather
in the heart, which stands gaping at something else, and seeks help and
consolation from creatures saints, or devils, and neither cares for
God, nor looks to Him for so much good as to believe that He is willing
to help, neither believes that whatever good it experiences comes from
God.

Besides, there is also a false worship and extreme idolatry, which we
have hitherto practiced, and is still prevalent in the world, upon
which also all ecclesiastical orders are founded, and which concerns
the conscience alone that seeks in its own works help, consolation, and
salvation, presumes to wrest heaven from God, and reckons how many
bequests it has made, how often it has fasted, celebrated Mass, etc.
Upon such things it depends, and of them boasts, as though unwilling to
receive anything from God as a gift, but desires itself to earn or
merit it superabundantly, just as though He must serve us and were our
debtor, and we His liege lords. What is this but reducing God to an
idol, yea, [a fig image or] an apple-god, and elevating and regarding
ourselves as God ? But this is slightly too subtle, and is not for
young pupils.

But let this be said to the simple, that they may well note and
remember the meaning of this commandment, namely, that we are to trust
in God alone, and look to Him and expect from Him naught but good, as
from one who gives us body, life, food, drink, nourishment, health,
protection, peace, and all necessaries of both temporal and eternal
things. He also preserves us from misfortune, and if any evil befall
us, delivers and rescues us, so that it is God alone (as has been
sufficiently said) from whom we receive all good, and by whom we are
delivered from all evil. Hence also, I think, we Germans from ancient
times call God (more elegantly and appropriately than any other
language) by that name from the word good as being an eternal fountain
which gushes forth abundantly nothing but what is good, and from which
flows forth all that is and is called good.

For even though otherwise we experience much good from men, still
whatever we receive by His command or arrangement is all received from
God. For our parents, and all rulers, and every one besides with
respect to his neighbor, have received from God the command that they
should do us all manner of good, so that we receive these blessings not
from them, but, through them, from God. For creatures are only the
hands, channels, and means whereby God gives all things, as He gives to
the mother breasts and milk to offer to her child, and corn and all
manner of produce from the earth for nourishment, none of which
blessings could be produced by any creature of itself.

Therefore no man should presume to take or give anything except as God
has commanded, in order that it may be acknowledged as God's gift, and
thanks may be rendered Him for it, as this commandment requires. On
this account also these means of receiving good gifts through creatures
are not to be rejected, neither should we in presumption seek other
ways and means than God has commanded. For that would not be receiving
from God, hut seeking of ourselves.

Let every one, then, see to it that he esteem this commandment great
and high above all things, and do not regard it as a joke. Ask and
examine your heart diligently, and you will find whether it cleaves to
God alone or not. If you have a heart that can expect of Him nothing
but what is good, especially in want and distress, and that, moreover
renounces and forsakes everything that is not God, then you have the
only true God. If on the contrary, it cleaves to anything else, of
which it expects more good and help than of God, and does not take
refuge in Him, but in adversity flees from Him, then you have an idol,
another god.

In order that it may be seen that God will not have this commandment
thrown to the winds, but will most strictly enforce it, He has attached
to it first a terrible threat, and then a beautiful, comforting promise
which is also to be urged and impressed upon young people, that they
may take it to heart and retain it:

[Exposition of the Appendix to the First Commandment.]


For I am the Lord, thy God, strong and jealous, visiting the iniquity
of the fathers upon the children unto the third and fourth generation
of them that hate Me; and showing mercy unto thousands of them that
love Me and keep My commandments.

Although these words relate to all the commandments (as we shall
hereafter learn), yet they are joined to this chief commandment
because it is of first importance that men have a right head; for
where the head is right, the whole life must be right, and vice versa.
Learn, therefore, from these words how angry God is with those who
trust in anything but Him, and again, how good and gracious He is to
those who trust and believe in Him alone with the whole heart; so that
His anger does not cease until the fourth generation, while, on the
other hand, His blessing and goodness extend to many thousands lest you
live in such security and commit yourself to chance, as men of brutal
heart, who think that it makes no great difference [how they live]. He
is a God who will not leave it unavenged if men turn from Him, and will
not cease to be angry until the fourth generation, even until they are
utterly exterminated. Therefore He is to be feared, and not to be
desisted.

He has also demonstrated this in all history, as the Scriptures
abundantly show and daily experience still teaches. For from the
beginning He has utterly extirpated all idolatry, and, on account of
it, both heathen and Jews; even as at the present day He overthrows all
false worship, so that all who remain therein must finally perish.
Therefore, although proud, powerful, and rich worldlings
[Sardanapaluses and Phalarides, who surpass even the Persians in
wealth] are now to be found, who boast defiantly of their Mammon, with
utter disregard whether God is angry at or smiles on them, and dare to
withstand His wrath, yet they shall not succeed, but before they are
aware, they shall be wrecked, with all in which they trusted; as all
others have perished who have thought themselves more secure or
powerful. And just because of such hardened heads who imagine because
God connives and allows them to rest in security, that He either is
entirely ignorant or cares nothing about such matters, He must deal a
smashing blow and punish them, so that He cannot forget it unto
children's children; so that every one may take note and see that this
is no joke to Him. For they are those whom He means when He says: Who
hate Me, i.e., those who persist in their defiance and pride; whatever
is preached or said to them, they will not listen; when they are
reproved, in order that they may learn to know themselves and amend
before the punishment begins, they become mad and foolish so as to
fairly merit wrath, as now we see daily in bishops and princes.

But terrible as are these threatenings, so much the more powerful is
the consolation in the promise, that those who cling to God alone
should be sure that He will show them mercy that is, show them pure
goodness and blessing not only for themselves, but also to their
children and children's children, even to the thousandth generation and
beyond that. This ought certainly to move and impel us to risk our
hearts in all confidence with God, if we wish all temporal and eternal
good, since the Supreme Majesty makes such sublime offers and presents
such cordial inducements and such rich promises.

Therefore let everyone seriously take this to heart, lest it be
regarded as though a man had spoken it. For to you it is a question
either of eternal blessing, happiness, and salvation, or of eternal
wrath, misery, and woe. What more would you have or desire than that He
so kindly promises to be yours with every blessing, and to protect and
help you in all need?

But, alas! here is the failure, that the world believes nothing of
this, nor regards it as God's Word, because it sees that those who
trust in God and not in Mammon suffer care and want, and the devil
opposes and resists them, that they have neither money, favor, nor
honor, and, besides, can scarcely support life; while, on the other
hand, those who serve Mammon have power, favor, honor, possessions, and
every comfort in the eyes of the world. For this reason, these words
must be grasped as being directed against such appearances; and we must
consider that they do not lie or deceive, but must come true.

Reflect for yourself or make inquiry and tell me: Those who have
employed all their care and diligence to accumulate great possessions
and wealth, what have they finally attained? You will find that they
have wasted their toil and labor, or even though they have amassed
great treasures, they have been dispersed and scattered, so that the
themselves have never found happiness in their wealth, and afterwards
never reached the third generation. Instances of this you will find a
plenty in all histories, also in the memory of aged and experienced
people. Only observe and ponder them.

Saul was a great king, chosen of God and a godly man; but when he was
established on his throne, and let his heart decline from God, and put
his trust in his crown and power, he had to perish with all that he
had, so that none even of his children remained. David, on the other
hand, was a poor, despised man, hunted down and chased, so that he
nowhere felt secure of his life; yet he had to remain in spite of Saul,
and become king. For these words had to abide and come true, since God
cannot lie or deceive. Only let not the devil and the world deceive you
with their show, which indeed remains for a time, but finally is
nothing.

Let us, then, learn well the First Commandment, that we may see how God
will tolerate no presumption nor any trust in any other object, and how
He requires nothing higher of us than confidence from the heart for
everything good, so that we may proceed right and straightforward and
use all the blessings which God gives no farther than as a shoemaker
uses his needle, awl, and thread for work, and then lays them aside, or
as a traveler uses an inn, and food, and his bed only for temporal
necessity, each one in his station, according to God's order, and
without allowing any of these things to be our food or idol. Let this
suffice with respect to the First Commandment, which we have had to
explain at length, since it is of chief importance, because, as before
said, where the heart is rightly disposed toward God and this
commandment is observed, all the others follow.



The Second Commandment.

Thou shalt not take the name of the Lord, thy God, in vain.


As the First Commandment has instructed the heart and taught [the
basis of] faith, so this commandment leads us forth and directs the
mouth and tongue to God. For the first objects that spring from the
heart and manifest themselves are words. Now, as I have taught above
how to answer the question, what it is to have a god, so you must learn
to comprehend simply the meaning of this and all the commandments, and
to apply it to yourself. If, then, it be asked: How do you understand
the Second Commandment, or what is meant by taking in vain, or misusing
God's name? answer briefly thus: It is misusing God's name when we call
upon the Lord God no matter in what way, for purposes of falsehood or
wrong of any kind. Therefore this commandment enjoins this much, that
God's name must not be appealed to falsely, or taken upon the lips
while the heart knows well enough, or should know, differently; as
among those who take oaths in court, where one side lies against the
other. For God's name cannot be misused worse than for the support of
falsehood and deceit. Let4this remain the exact German and simplest
meaning of this commandment.

From this every one can readily infer when and in how many ways God's
name is misused, although it is impossible to enumerate all its
misuses. Yet, to tell it in a few words, all misuse of the divine name
occurs, first, in worldly business and in matters which concern money,
possessions, honor, whether it be publicly in court, in the market, or
wherever else men make false oaths in God's name, or pledge their souls
in any matter. And this is especially prevalent in marriage affairs
where two go and secretly betroth themselves to one another, and
afterward abjure [their plighted troth].

But. the greatest abuse occurs in spiritual matters, which pertain to
the conscience, when false preachers rise up and offer their Lying
vanities as God's Word. Behold, all this is decking one's self out with
God's name, or making a pretty show, or claiming to be right, whether
it occur in gross, worldly business or in sublime, subtle matters of
faith and doctrine. And among liars belong also blasphemers, not alone
the very gross, well known to every one, who disgrace God's name
without fear (these are not for us, but for the hangman to discipline);
but also those who publicly traduce the truth and God's Word and
consign it to the devil. Of this there is no need now to speak further.


Here, then, let us learn and take to heart the great importance of this
commandment, that with all diligence we may guard against and dread
every misuse of the holy name, as the greatest sin that can be
outwardly committed. For to lie and deceive is in itself a great sin,
but is greatly aggravated when we attempt to justify it, and seek to
confirm it by invoking the name of God and using it as a cloak for
shame, so that from a single lie a double lie, nay, manifold lies,
result.

For this reason, too, God has added a solemn threat to this
commandment, to wit: For the Lord will not hold him guiltless that
taketh His name in van. That is: It shall not be condoned to any one
nor pass unpunished. For as little as He will leave it unavenged if any
one turn his heart from Him, as little will He suffer His name to be
employed for dressing up a lie. Now alas! it is a common calamity in
all the word that there are as few who are not using the name of God
for purposes of Lying and all wickedness as there are those who with
their heart trust alone in God. For by nature we all have within us
this beautiful virtue, to wit, that whoever has committed a wrong would
like to cover up and adorn his disgrace, so that no one may see it or
know it; and no one is so bold as to boast to all the world of the
wickedness he has perpetrated, all wish to act by stealth and without
any one being aware of what thy do. Then, if any one be arraigned, the
name of God is dragged into the affair and must make the villainy look
like godliness, and the shame like honor. This is the common course of
the world, which, like a great deluge, has flooded all lands. Hence we
have also as our reward what we seek and deserve: pestilences wars,
famines, conflagrations, floods, wayward wives, children, servants, and
all sorts of defilement. Whence else should so much misery come? It is
still a great mercy that the earth bears and supports us.

Therefore, above all things, our young people should have this
commandment earnestly enforced upon them, and they should be trained to
hold this and the First Commandment in high regard; and whenever they
transgress, we must at once be after them with the rod and hold the
commandment before them, and constantly inculcate it, so as to bring
them up not only with punishment, but also in the reverence and fear of
God.

Thus you now understand what. it is to take God's name in vain, that is
(to recapitulate briefly), either simply for purposes of falsehood, and
to allege God's name for something that is not so, or to curse, swear,
conjure, and, in short, to practice whatever wickedness one may.
Besides this you must also know how to use the name [of God] aright.
For when saying: Thou shalt not take the name of the Lord thy God, in
vain, He gives us to understand at the same time that it is to be used
properly. For it has been revealed and given to us for the very purpose
that it may be of constant use and profit. Hence it is a natural
inference, since using the holy name for falsehood or wickedness is
here forbidden, that we are, on the other hand, commanded to employ it
for truth and for all good, as when one swears truly where there is
need and it is demanded. So also when there is right teaching, and when
the name is invoked in trouble or praised and thanked in prosperity
etc.; all of which is comprehended summarily and commanded in the
passage Ps. 50, 15: Call upon Me in the days of trouble; I will deliver
thee, and thou shalt glorify Me. For all this is bringing 't into the
service of truth, and using it in a blessed way, and thus His name is
hallowed, as we pray in the Lord's Prayer.

Thus you have the sum of the entire commandment explained. And with
this understanding the question with which many teachers have troubled
themselves has been easily solved, to wit, why swearing is prohibited
in the Gospel, and yet Christ, St. Paul, and other saints often swore.
The explanation is briefly this: We are not to swear in support of
evil, that is, of falsehood, and where there is no need or use; but for
the support of good and the advantage of our neighbor we should swear.
For it is a truly good work, by which God is praised, truth and right
are established, falsehood is refuted, peace is made among men,
obedience is rendered, and quarrels are settled. For in this way God
Himself interposes and separates between right and wrong, good and
evil. If one part swears falsely, he has his sentence that he shall not
escape punishment, ad though it be deferred a long time, he shall not
succeed; that all that he may gain thereby will slip out of his hands,
and he will never enjoy it; as I have seen in the case of many who
perjured themselves in their marriage-vows, that they have never had a
happy hour or a healthful day, and thus perished miserably in body,
soul, and possessions.

Therefore I advise and exhort as before that by means of warning and
threatening, restraint and punishment, the children be trained betimes
to shun falsehood, and especially to avoid the use of God's name in its
support. For where they are allowed to do as they please, no good will
result, as is even now evident that the world is worse than it has ever
been and that there is no government, no obedience, no fidelity, no
faith, but only daring, unbridled men, whom no teaching or reproof
helps; all of which is God's wrath and punishment for such wanton
contempt of this commandment.

On the other hand, they should be constantly urged and incited to
honor God's name, and to have it always upon their lips in everything
that may happen to them or come to their notice: For that is the true
honor of His Name, to look to it and implore it for all consolation, so
that (as we have heard above) first the heart by faith gives God the
honor due Him, and afterwards the lips by confession.

This is also a blessed and useful habit and very effectual against the
devil, who is ever about us, and lies in wait to bring us into sin and
shame, calamity and trouble, but who is very loath to hear God's name,
and cannot remain long where it is uttered and called upon from the
heart. And, indeed, many a terrible and shocking calamity would befall
us if, by our calling upon His name, God did not preserve us. I have
myself tried it, and learned by experience that often sudden great
calamity was immediately averted and removed during such invocation. To
vex the devil, I say, we should always have this holy name in our
mouth, so that he may not be able to injure us as he wishes.

For this end it is also of service that we form the habit of daily
commending ourselves to God, with soul and body, wife, children,
servants, and all that we have, against every need that may occur;
whence also the blessing and thanksgiving at meals, and other prayers,
morning and evening, have originated and remain in use. Likewise the
practices of children to cross themselves when anything monstrous or
terrible is seen or heard, and to exclaim: "Lord God, protect us!"
"Help, dear Lord Jesus!" etc. Thus, too, if any one meets with
unexpected good fortune, however trivial, that he say: "God be praised
and thanked; this God has bestowed on me!" etc., as formerly the
children were accustomed to fast and pray to St. Nicholas and other
saints. This would be more pleasing and acceptable to God than all
monasticism and Carthusian sanctity.

Behold, thus we might train our youth in a childlike way and playfully
in the fear and honor of God, so that the First and Second Commandments
might be well observed and in constant practice. Then some good might
take root, spring up and bear fruit, and men grow up whom an entire
land might relish and enjoy. Moreover, this would be the true way to
bring Up children well as long as they can become trained with kindness
and delight. For what must be enforced with rods and blows only will
not develop into a good breed and at best they will remain godly under
such treatment no longer than while the rod is upon their back.

But this [manner of training] so spreads its roots in the heart that
they fear God more than rods and clubs. This I say with such
simplicity for the sake of the young, that it may penetrate their
minds. For since we are preaching to children, we must also prattle
with them. Thus we have prevented the abuse and have taught the right
use of the divine name, which should consist not only in words, but
also in practices and life, so that we may know that God is well
pleased with this and will as richly reward it as He will terribly
punish the abuse.

The Third Commandment.

Thou shalt sanctify the holy day.
[Remember the Sabbath day to keep it holy.]


The word holy day (Feiertag) is rendered from the Hebrew word Sabbath
which properly signifies to rest, that is, to abstain from labor. Hence
we are accustomed to say, Feierbend machen [that is, to cease working],
or heiligen Abend geben [sanctify the Sabbath]. Now, in the Old
Testament, God separated the seventh day, and appointed it for rest,
and commanded that it should be regarded as holy above all others. As
regards this external observance, this commandment was given to the
Jews alone, that they should abstain from toilsome work, and rest, so
that both man and beast might recuperate, and not be weakened by
unremitting labor. Although they afterwards restricted this too
closely, and grossly abused it, so that they traduced and could not
endure in Christ those works which they themselves were accustomed to
do on that day, as we read in the Gospel just as though the commandment
were fulfilled by doing no external [manual] work whatever, which,
however, was not the meaning, but, as we shall hear, that they sanctify
the holy day or day of rest.

This commandment, therefore, according to its gross sense, does not
concern us Christians; for it is altogether an external matter, like
other ordinances of the Old Testament, which were attached to
particular customs, persons, times, and places, and now have been made
free through Christ. But to grasp a Christian meaning for the simple as
to what God requires in this commandment, note that we keep holy days
not for the sake of intelligent and learned Christians (for they have
no need of it [holy days]), but first of all for bodily causes and
necessities, which nature teaches and requires; for the common people,
man-servants and maid-servants, who have been attending to their work
and trade the whole week, that for a day they may retire in order to
rest and be refreshed.

Secondly, and most especially, that on such day of rest (since we can
get no other opportunity) freedom and time be taken to attend divine
service, so that we come together to hear and treat of God's and then
to praise God, to sing and pray.

However, this, I say, is not so restricted to any time, as with the
Jews, that it must be just on this or that day; for in itself no one
day is better than another; but this should indeed be done daily;
however, since the masses cannot give such attendance, there must be at
least one day in the week set apart. But since from of old Sunday [the
Lord's Day] has been appointed for this purpose, we also should
continue the same, in order that everything be done in harmonious
order, and no one create disorder by unnecessary innovation.

Therefore this is the simple meaning of the commandment: since
holidays are observed anyhow, such observance should be devoted to
hearing God's Word, so that the special function of this day should be
the ministry of the Word for the young and the mass of poor people, yet
that the resting be not so strictly interpreted as to forbid any other
incidental work that cannot be avoided.

Accordingly, when asked, What is meant by the commandment: Thou shalt
sanctify the holy day? answer: To sanctify the holy day is the same as
to keep it holy. But what is meant by keeping it holy? Nothing else
than to be occupied in holy words, works, and life. For the day needs
no sanctification for itself; for in itself it has been created holy
[from the beginning of the creation it was sanctified by its Creator].
But God desires it to be holy to you. Therefore it becomes holy or
unholy on your account, according as you are occupied on the same with
things that are holy or unholy.

How, then, does such sanctification take place? Not in this manner,
that [with folded hands] we sit behind the stove and do no rough
[external] work, or deck ourselves with a wreath and put on our best
clothes, but (as has been said) that we occupy ourselves with God's
Word, and exercise ourselves therein.

And, indeed, we Christians ought always to keep such a holy day, and be
occupied with nothing but holy things, i.e., daily be engaged upon
God's Word, and carry it in our hearts and upon our lips. But (as has
been said) since we do not at all times have leisure, we must devote
several hours a week for the sake of the young, or at least a day for
the sake of the entire multitude, to being concerned about this alone,
and especially urge the Ten Commandments, the Creed, and the Lord's
Prayer, and thus direct our whole life and being according to God's
Word. At whatever time, then, this is being observed and practiced,
there a true holy day is being kept; otherwise it shall not be called a
Christians' holy day. For, indeed, non-Christians can also cease from
work and be idle, just as the entire swarm of our ecclesiastics, who
stand daily in the churches, singing, and ringing bells but keeping no
holy day holy, because they neither preach nor practices God's Word,
but teach and live contrary to it.

For the Word of God is the sanctuary above all sanctuaries, yea, the
only one which we Christians know and have. For though we had the bones
of all the saints or all holy and consecrated garments upon a heap,
still that would help us nothing; for all that is a dead thing which
can sanctify nobody. But God's Word is the treasure which sanctifies
everything, and by which even all the saints themselves were
sanctified. At whatever hour then, God's Word is taught, preached,
heard, read or meditated upon, there the person, day, and work are
sanctified thereby, not because of the external work, but because of
the Word which makes saints of us all. Therefore I constantly say that
all our life and work must be ordered according to God's Word, if it is
to be God-pleasing or holy. Where this is done, this commandment is in
force and being fulfilled.

On the contrary, any observance or work that is practiced without
God's Word is unholy before God, no matter how brilliantly it may
shine! even though it be covered with relics, such as the fictitious
spiritual orders which know nothing of God's Word and seek holiness in
their own works. Note, therefore, that the force and power of this
commandment lies not in the resting but in the sanctifying so that to
this day belongs a special holy exercise. For other works and
occupations are not properly called holy exercises, unless the man
himself be first holy. But here a work is to be done by which man is
himself made holy, which is done (as we have heard ) alone through
God's Word. For this, then, fixed places, times, persons, and the
entire external order of worship have been created and appointed, so
that it may be publicly in operation.

Since, therefore, so much depends upon God's Word that without it no
holy day can be sanctified, we must know that God insists upon a strict
observance of this commandment, and will punish all who despise His
Word and are not willing to hear and learn it, especially at the time
appointed for the purpose.

Therefore not only those sin against this commandment who grossly
misuse and desecrate the holy day, as those who on account of their
greed or frivolity neglect to hear God's Word or lie in taverns and are
dead drunk like swine; but also that other crowd, who listen to God's
Word as to any other trifle, and only from custom come to preaching,
and go away again, and at the end of the year know as little of it as
at the beginning. For hitherto the opinion prevailed that you had
properly hallowed Sunday when you had heard a mass or the Gospel read;
but no one cared for God's Word, as also no one taught it. Now, while
we have God's Word we nevertheless do not correct the abuse; we suffer
ourselves to be preached to and admonished, but we listen without
seriousness and care.

Know, therefore, that you must be concerned not only about hearing, but
also about learning and retaining it in memory, and do not think that
it is optional with you or of no great importance, but that it is God's
commandment, who will require of you how you have heard, learned, and
honored His Word.

Likewise those fastidious spirits are to be reproved who, when they
have heard a sermon or two, find it tedious and dull, thinking that
they know all that well enough, and need no more instruction. For just
that is the sin which has been hitherto reckoned among mortal sins, and
is called _achedia_, i.e., torpor or satiety, a malignant, dangerous
plague with which the devil bewitches and deceives the hearts of many,
that he may surprise us and secretly withdraw God's Word from us.

For let me tell you this, even though you know it perfectly and be
already master in all things, still you are daily in the dominion of
the devil, who ceases neither day nor night to steal unawares upon you,
to kindle in your heart unbelief and wicked thoughts against the
foregoing and all the commandments. Therefore you must always have
God's Word in your heart, upon your lips, and in your ears. But where
the heart is idle, and the Word does not sound, he breaks in and has
done the damage before we are aware. On the other hand, such is the
efficacy of the Word, whenever it is seriously contemplated heard, and
used, that it is bound never to be without fruit, but always awakens
new understanding, pleasure, and devoutness, and produces a pure heart
and pure thoughts. For these words are not inoperative or dead, but
creative, living words. And even though no other interest or necessity
impel us, yet this ought to urge every one thereunto, because thereby
the devil is put to flight and driven away, and, besides, this
commandment is fulfilled, and [this exercise in the Word] is more
pleasing to God than any work of hypocrisy, however brilliant.

The Fourth Commandment.

Thus far we have learned the first three commandments, which relate to
God. First that with our whole heart we trust in Him, and fear and love
Him throughout all our life. Secondly, that we do not misuse His holy
name in the support of falsehood or any bad work, but employ it to the
praise of God and the profit and salvation of our neighbor and
ourselves. Thirdly, that on holidays and when at rest we diligently
treat and urge God's Word, so that all our actions and our entire life
be ordered according to it. Now follow the other seven, which relate to
our neighbor among which the first and greatest is:

Thou shalt honor thy father and thy mother.

To this estate of fatherhood and motherhood God has given the special
distinction above all estates that are beneath it that He not simply
commands us to love our parents, but to honor them. For with respect to
brothers, sisters, and our neighbors in general He commands nothing
higher than that we love them, so that He separates and distinguishes
father and mother above all other persons upon earth, and places them
at His side. For it is a far higher thing to honor than to love one,
inasmuch as it comprehends not only love, but also modesty, humility,
and deference as to a majesty there hidden, and requires not only that
they be addressed kindly and with reverence, but, most of all that both
in heart and with the body we so act as to show that we esteem them
very highly, and that, next to God, we regard them as the very highest.
For one whom we are to honor from the heart we must truly regard as
high and great.

We must, therefore impress it upon the young that they should regard
their parents as in God's stead, and remember that however lowly, poor,
frail, and queer they may be, nevertheless they are father and mother
given them by God. They are not to be deprived of their honor because
of their conduct or their failings. Therefore we are not to regard
their persons, how they may be, but the will of God who has thus
created and ordained. In other respects we are, indeed, all alike in
the eyes of God; but among us there must necessarily be such inequality
and ordered difference, and therefore God commands it to be observed,
that you obey me as your father, and that I have the supremacy.

Learn, therefore, first, what is the honor towards parents required by
this commandment to wit, that they be held in distinction and esteem
above all things, as the most precious treasure on earth. Furthermore,
that also in our words we observe modesty toward them, do not accost
them roughly, haughtily, and defiantly, but yield to them and be silent
even though they go too far. Thirdly, that we show them such honor also
by works, that is, with our body and possessions, that we serve them,
help them, and provide for them when they are old, sick, infirm, or
poor, and all that not only gladly, but with humility and reverence, as
doing it before God. For he who knows how to regard them in his heart
will not allow them to suffer want or hunger, but will place them above
him and at his side, and will share with them whatever he has and
possesses.

Secondly, notice how great, good, and holy a work is here assigned
children, which is alas! utterly neglected and disregarded, and no one
perceives that God has commanded it or that it is a holy, divine Word
and doctrine. For if it had been regarded as such, every one could have
inferred that they must be holy men who live according to these words.
Thus there would have been no need of inventing monasticism nor
spiritual orders, but every child would have abided by this
commandment, and could have directed his conscience to God and said:
"If I am to do good and holy works, I know of none better than to
render all honor and obedience to my parents, because God has Himself
commanded it. For what God commands must be much and far nobler than
everything that we may devise ourselves, and since there is no higher
or better teacher to be found than God, there can be no better
doctrine, indeed, than He gives forth. Now, He teaches fully what we
should do if we wish to perform truly good works, and by commanding
them, He shows that they please Him. If, then, it is God who commands
this, and who knows not how to appoint anything better, I will never
improve upon it."

Behold, in this manner we would have had a godly child properly
taught, reared in true blessedness, and kept at home in obedience to
his parents and in their service, so that men should have had blessing
and joy from the spectacle. However, God's commandment was not
permitted to be thus [with such care and diligence] commended, but had
to be neglected and trampled under foot, so that a child could not lay
it to heart, and meanwhile gaped [like a panting wolf] at the devices
which we set up, without once [consulting or] giving reverence to God.

Let us, therefore, learn at last, for God's sake, that, placing all
other things out of sight, our youths look first to this commandment,
if they wish to serve God with truly good works, that they do what is
pleasing to their fathers and mothers, or to those to whom they may be
subject in their stead. For every child that knows and does this has,
in the first place, this great consolation in his heart that he can
joyfully say and boast (in spite of and against all who are occupied
with works of their own choice): "Behold, this work is well pleasing to
my God in heaven that I know for certain." Let them all come together
with their many great, distressing, and difficult works and make their
boast, we will see whether they can show one that is greater and
nobler than obedience to father and mother, to whom God has appointed
and commanded obedience next to His own majesty; so that if God's Word
and will are in force and being accomplished nothing shall be esteemed
higher than the will and word of parents; yet so that it, too, is
subordinated to obedience toward God and is not opposed to the
preceding commandments.

Therefore you should be heartily glad and thank God that He has chosen
you and made you worthy to do a work so precious and pleasing to Him.
Only see that, although it be regarded as the most humble and despised
you esteem it great and precious, not on account of our worthiness, but
because it is comprehended in, and controlled by, the jewel and
sanctuary, namely, the Word and commandment of God. Oh, what a high
price would all; Carthusians, monks, and nuns pay, if in all their
religious doings they could bring into God's presence a single work
done by virtue of His commandment, and be able before His face to say
with joyful heart: "Now I know that this work is well pleasing to
Thee." Where will these poor wretched persons hide when in the sight of
God and all the world they shall blush with shame before a young child
who has lived according to this commandment, and shall have to confess
that with their whole life they are not worthy to give it a drink of
water? And it serves them right for their devilish perversion in
treading God's commandment under foot that they must vainly torment
themselves with works of their own device, and, in addition, have scorn
and loss for their reward.

Should not the heart, then, leap and melt for joy when going to work
and doing what is commanded, saying: Lo, this is better than all
holiness of the Carthusians, even though they kill themselves fasting
and praying upon their knees without ceasing? For here you have a sure
text and a divine testimony that He has enjoined this, but concerning
the other He did not command a word. But this is the plight and
miserable blindness of the world that no one believes these things; to
such an extent the devil has deceived us with false holiness and the
glamour of our own works.

Therefore I would be very glad (I say it again) if men would open
their eyes and ears and take this to heart, lest some time we may
again be led astray from the pure Word of God to the lying vanities of
the devil. Then, too, all would be well; for parents would have more
joy, love, friendship, and concord in their houses; thus the children
could captivate their parents' hearts. On the other hand, when they are
obstinate, and will not do what they ought until a rod is laid upon
their back, they anger both God and their parents, whereby they deprive
themselves of this treasure and joy of conscience and lay up for
themselves only misfortune. Therefore, as every one complains, the
course of the world now is such that both young and old are altogether
dissolute and beyond control, have no reverence nor sense of honor, do
nothing except as they are driven to it by blows, and perpetrate what
wrong and detraction they can behind each other's back; therefore God
also punishes them, that they sink into all kinds of filth and misery.
As a rule, the parents, too, are themselves stupid and ignorant; one
fool trains [teaches] another, and as they have lived, so live their
children after them.

 This, now, I say should be the first and most important consideration
to urge us to the observance of this commandment; on which account,
even if we had no father and mother we ought to wish that God would set
up wood and stone before Us, whom we might call father and mother. How
much more, since He has given us living parents, should we rejoice to
show them honor and obedience, because we know it is so highly pleasing
to the Divine Majesty and to all angels, and vexes all devils, and is,
besides, the highest work which we can do, after the sublime divine
worship comprehended in the previous commandments, so that giving of
alms and every other good work toward our neighbor are not equal to
this. For God has assigned this estate the highest place, yea, has set
it up in His own stead, upon earth. This will and pleasure of God ought
to be a sufficient reason and incentive to us to do what we can with
good will and pleasure.

Besides this, it is our duty before the world to be grateful for
benefits and every good which we have of our parents. But here again
the devil rules in the world, so that the children forget their
parents, as we all forget God, and no one considers how God nourishes,
protects, and defends us, and bestows so much good on body and soul;
especially when an evil hour comes we are angry and grumble with
impatience and all the good which we have received throughout our life
is wiped out [from our memory]. Just so we do also with our parents,
and there is no child that understands and considers this [what the
parents have endured while nourishing and fostering him], except the
Holy Ghost grant him this grace.

God knows very well this perverseness of the world; therefore He
admonishes and urges by commandments that every one consider what his
parents have done for him and he will find that he has from them body
and life, moreover, that he has been fed and reared when otherwise he
would have perished a hundred times in his own filth. Therefore it is a
true and good saying of old and wise men: Deo, parentibus et magistris
non potest satis gratiae rependi, that is, To God, to parents, and to
teachers we can never render sufficient gratitude and compensation. He
that regards and considers this will indeed without compulsion do all
honor to his parents, and bear them up on his hands as those through
whom God has done him all good.

Over and above all this, another great reason that should incite us the
more [to obedience to this commandment] is that God attaches to this
commandment a temporal promise and says: That thou mayest live long
upon the land which the Lord, thy God, giveth thee.

Here you can see yourself how much God is in earnest in respect to this
commandment, inasmuch as He not only declares that it is well pleasing
to Him, and that He has joy and delight therein; but also that it shall
be for our prosperity and promote our highest good; so that we may have
a pleasant and agreeable life, furnished with every good thing.
Therefore also St. Paul greatly emphasizes the same and rejoices in it
when he says, Eph. 6, 2. 3: This is the first commandment with promise:
That it may be well with thee, and thou mayest live long on the earth.
For although the rest also have their promises contained in them, yet
in none is it so plainly and explicitly stated.

Here, then, you have the fruit and the reward, that whoever observes
this commandment shall have happy days, fortune, and prosperity; and on
the other hand, the punishment, that whoever is disobedient shall the
sooner perish, and never enjoy life. For to have long life in the sense
of the Scriptures is not only to become old, but to have everything
which belongs to long life, such as health, wife, and children,
livelihood, peace, good government, etc., without which this life can
neither be enjoyed in cheerfulness nor long endure. If, therefore, you
will not obey father and mother and submit to their discipline, then
obey the hangman; if you will not obey him, then submit to the
skeleton-man, i.e., death [death the all-subduer, the teacher of
wicked children]. For on this God insists peremptorily: Either if you
obey Him rendering love and service, He will reward you abundantly with
all good, or if you offend Him, He will send upon you both death and
the hangman.

Whence come so many knaves that must daily be hanged, beheaded, broken
upon the wheel, but from disobedience [to parents], because they will
not submit to discipline in kindness, so that, by the punishment of
God, they bring it about that we behold their misfortune and grief? For
it seldom happens that such perverse people die a natural or timely
death.

But the godly and obedient have this blessing, that they live long in
pleasant quietness and see their children's children (as said above) to
the third and fourth generation. Thus experience also teaches, that
where there are honorable, old families who fare well and have many
children, they owe their origin to the fact, to be sure, that some of
them were brought up well and were regardful of their parents. On the
other hand, it is written of the wicked, Ps. 109,13: Let his posterity
be cut off; and in the generation following let their name be blotted
out. Therefore heed well how great a thing in God's sight obedience is
since He so highly esteems it, is so highly pleased with it, and
rewards it so richly, and besides enforces punishment so rigorously on
those who act contrariwise.

All this I say that it may be well impressed upon the young. For no one
believes how necessary this commandment is, although it has not been
esteemed and taught hitherto under the papacy. These are simple and
easy words, and everybody thinks he knew them a fore; therefore men
pass them lightly by, are gaping after other matters, and do not see
and believe that God is so greatly offended if they be disregarded, nor
that one does a work so well pleasing and precious if he follows them.

In this commandment belongs a further statement regarding all kinds of
obedience to persons in authority who have to command and to govern.
For all authority flows and is propagated from the authority of
parents. For where a father is unable alone to educate his [rebellious
and irritable] child, he employs a schoolmaster to instruct him; if he
be too weak, he enlists the aid of his friends and neighbors; if he
departs this life, he delegates and confers his authority and
government upon others who are appointed for the purpose. Likewise, he
must have domestics, man-servants and maid-servants, under himself for
the management of the household, so that all whom we call masters are
in the place of parents and must derive their power and authority to
govern from them. Hence also they are all called fathers in the
Scriptures, as those who in their government perform the functions of a
father, and should have a paternal heart toward their subordinates. As
also from antiquity the Romans and other nations called the masters and
mistresses of the household patres- et matresfamiliae that is,
housefathers and housemothers. So also they called their national
rulers and overlords patres patriae, that is fathers of the entire
country, for a great shame to us who would be Christians that we do not
likewise call them so, or, at least do not esteem and honor them as
such.

Now, what a child owes to father and mother, the same owe all who are
embraced in the household. Therefore man-servants and maid-servants
should be careful not only to be obedient to their masters and
mistresses but also to honor them as their own fathers and mothers, and
to do everything which they know is expected of them, not from
compulsion and with reluctance, but with pleasure and joy for the cause
just mentioned, namely that it is God's command and is pleasing to Him
above all other works. Therefore they ought rather to pay wages in
addition and be glad that they may obtain masters and mistresses to
have such joyful consciences and to know how they may do truly golden
works; a matter which has hitherto been neglected and despised, when,
instead, everybody ran in the devil's name, into convents or to
pilgrimages and indulgences, with loss [of time and money] and with an
evil conscience.

If this truth, then, could be impressed upon the poor people, a
servant-girl would leap and praise and thank God; and with her tidy
work for which she receives support and wages she would acquire such a
treasure as all that are esteemed the greatest saints have not
obtained. Is it not an excellent boast to know and say that, if you
perform your daily domestic task, this is better than all the sanctity
and ascetic life of monks? And you have the promise, in addition, that
you shall prosper in all good and fare well. How can you lead a more
blessed or holier life as far as your works are concerned? For in the
sight of God faith is what really renders a person holy, and alone
serves Him, but the works are for the service of man. There you have
everything good, protection and defense in the Lord, a joyful
conscience and a gracious God besides, who will reward you a
hundredfold, so that you are even a nobleman if you be only pious and
obedient. But if not, you have, in the first place, nothing but the
wrath and displeasure of God, no peace of heart, and afterwards all
manner of plagues and misfortunes.

Whoever will not be influenced by this and inclined to godliness we
hand over to the hangman and to the skeleton-man. Therefore let every
one who allows himself to be advised remember that God is not making
sport, and know that it is God who speaks with you and demands
obedience. If you obey Him, you are His dear child; but if you despise
to do it, then take shame, misery, and grief for your reward.

The same also is to be said of obedience to civil government, which (as
we have said) is all embraced in the estate of fatherhood and extends
farthest of all relations. For here the father is not one of a single
family, but of as many people as he has tenants, citizens, or subjects.
For through them, as through our parents, God gives to us food, house
and home, protection and security. Therefore since they bear such name
and title with all honor as their highest dignity, it is our duty to
honor them and to esteem them great as the dearest treasure and the
most precious jewel upon earth.

He, now, who is obedient here, is willing and ready to serve, and
cheerfully does all that pertains to honor, knows that he is pleasing
God and that he will receive joy and happiness for his reward. If he
will not do it in love, but despises and resists [authority] or rebels,
let him also know, on the other hand, that he shall have no favor nor
blessing, and where he thinks to gain a florin thereby, he will
elsewhere lose ten times as much, or become a victim to the hangman,
perish by war, pestilence, and famine, or experience no good in his
children, and be obliged to suffer injury, injustice, and violence at
the hands of his servants, neighbors, or strangers and tyrants; so that
what we seek and deserve is paid back and comes home to us.

If we would ever suffer ourselves to be persuaded that such works are
pleasing to God and have so rich a reward, we would be established in
altogether abundant possessions and have what our heart desires. But
because the word and command of God are so lightly esteemed, as though
some babbler had spoken it, let us see whether you are the man to
oppose Him. How difficult, do you think, it will be for Him to
recompense you! Therefore you would certainly live much better with the
divine favor, peace, and happiness than with His displeasure and
misfortune. Why, think you, is the world now so full of unfaithfulness,
disgrace, calamity, and murder, but because every one desires to be his
own master and free from the emperor, to care nothing for any one, and
do what pleases him? Therefore God punishes one knave by another, so
that, when you defraud and despise your master, another comes and deals
in like manner with you, yea, in your household you must suffer ten
times more from wife, children, or servants.

Indeed, we feel our misfortune, we murmur and complain of
unfaithfulness, violence, and injustice, but will not see that we
ourselves are knaves who have fully deserved this punishment, and yet
are not thereby reformed. We will have no favor and happiness,
therefore it is but fair that we have nothing but misfortune without
mercy. There must still be somewhere upon earth some godly people
because God continues to grant us so much good! On our own account we
should not have a farthing in the house nor a straw in the field. All
this I have been obliged to urge with so many words, in hope that some
one may take it to heart, that we may be relieved of the blindness and
misery in which we are steeped so deeply, and may truly understand the
Word and will of God, and earnestly accept it. For thence we would
learn how we could have joy, happiness, and salvation enough, both
temporal and eternal.

Thus we have two kinds of fathers presented in this commandment,
fathers in blood and fathers in office, or those to whom belongs the
care of the family, and those to whom belongs the care of the country.
Besides these there are yet spiritual fathers; not like those in the
Papacy, who have indeed had themselves called thus, but have performed
no function of the paternal office. For those only are called spiritual
fathers who govern and guide us by the Word of God; as St. Paul boasts
his fatherhood 1 Cor. 4, 15, where he says: In Christ Jesus I hove
begotten you through the Gospel. Now, since they are fathers they are
entitled to their honor, even above all others. But here it is bestowed
least; for the way which the world knows for honoring them is to drive
them out of the country and to grudge them a piece of bread and, in
short, they must be (as says St. Paul 1 Cor. 4, 13) as the filth of the
world and everybody's refuse and footrag.

Yet there is need that this also be urged upon the populace, that
those who would be Christians are under obligation in the sight of God
to esteem them worthy of double honor who minister to their souls, that
they deal well with them and provide for them. For that, God is willing
to add to you sufficient blessing and will not let you come to want.
But in this matter every one refuses and resists, and all are afraid
that they will perish from bodily want, and cannot now support one
respectable preacher, where formerly they filled ten fat paunches. In
this we also deserve that God deprive us of His Word and blessing, and
again allow preachers of lies to arise to lead us to the devil, and, in
addition, to drain our sweat and blood.

But those who keep in sight God's will and commandment have the
promise that everything which they bestow upon temporal and spiritual
fathers, and whatever they do to honor them, shall be richly
recompensed to them, so that they shall have, not bread, clothing, and
money for a year or two, but long life, support, and peace, and shall
be eternally rich and blessed. Therefore only do what is your duty, and
let God take care how He is to support you and provide for you
sufficiently. Since He has promised it, and has never yet lied, He will
not be found lying to you.

This ought indeed to encourage us, and give us hearts that would melt
in pleasure and love toward those to whom we owe honor, so that we
would raise our hands and joyfully thank God who has given us such
promises, for which we ought to run to the ends of the world [to the
remotest parts of India]. For although the whole world should combine,
it could not add an hour to our life or give us a single grain from the
earth. But God wishes to give you all exceeding abundantly according to
your heart's desire. He who despises and casts this to the winds is not
worthy ever to hear a word of God. This has now been stated more than
enough for all who belong under this commandment.

In addition, it would be well to preach to the parents also, and such
as bear their office, as to how they should deport themselves toward
those who are committed to them for their government. For although this
is not expressed in the Ten Commandments, it is nevertheless abundantly
enjoined in many places in the Scriptures. And God desires to have it
embraced in this commandment when He speaks of father and mother. For
He does not wish to have in this office and government knaves and
tyrants; nor does He assign to them this honor, that is, power and
authority to govern, that they should have themselves worshiped; but
they should consider that they are under obligations of obedience to
God; and that, first of all, they should earnestly and faithfully
discharge their office, not only to support and provide for the bodily
necessities of their children, servants, subjects, etc., but, most of
all, to train them to the honor and praise of God. Therefore do not
think that this is left to your pleasure and arbitrary will, but that
it is a strict command and injunction of God, to whom also you must
give account for it.

But here again the sad plight arises that no one perceives or heeds
this, and all live on as though God gave us children for our pleasure
or amusement, and servants that we should employ them like a cow or
ass, only for work, or as though we were only to gratify our wantonness
with our subjects, ignoring them, as though it were no concern of ours
what they learn or how they live; and no one is willing to see that
this is the command of the Supreme Majesty, who will most strictly call
us to account and punish us for it; nor that there is so great need to
be so seriously concerned about the young. For if we wish to have
excellent and apt persons both for civil and ecclesiastical government
we must spare no diligence, time, or cost in teaching and educating our
children, that they may serve God and the world, and we must not think
only how we may amass money and possessions for them. For God can
indeed without us support and make them rich, as He daily does. But for
this purpose He has given us children, and issued this command that we
should train and govern them according to His will, else He would have
no need of father and mother. Let every one know therefore, that it is
his duty, on peril of losing the divine favor, to bring up his
children above all things in the fear and knowledge of God, and if they
are talented, have them learn and study something, that they may be
employed for whatever need there is [to have them instructed and
trained in a liberal education, that men may be able to have their aid
in government and in whatever is necessary].

If that were done, God would also richly bless us and give us grace to
train men by whom land and people might be improved and likewise well
educated citizens, chaste and domestic wives, who afterwards would rear
godly children and servants. Here consider now what deadly injury you
are doing if you be negligent and fail on your part to bring up your
child to usefulness and piety, and how you bring upon yourself all sin
and wrath, thus earning hell by your own children, even though you be
otherwise pious and holy. And because this is disregarded, God so
fearfully punishes the world that there is no discipline, government,
or peace, of which we all complain, but do not see that it is our
fault; for as we train them, we have spoiled and disobedient children
and subjects. Let this be sufficient exhortation; for to draw this out
at length belongs to another time.

The Fifth Commandment.

Thou shalt not kill.

We have now completed both the spiritual and the temporal government,
that is, the divine and the paternal authority and obedience. But here
now we go forth from our house among our neighbors to learn how we
should live with one another, every one himself toward his neighbor.
Therefore God and government are not included in this commandment nor
is the power to kill, which they have taken away. For God has delegated
His authority to punish evil-doers to the government instead of
parents, who aforetime (as we read in Moses) were required to bring
their own children to judgment and sentence them to death. Therefore,
what is here forbidden is forbidden to the individual in his relation
to any one else, and not to the government.

Now this commandment is easy enough and has been often treated,
because we hear it annually in the Gospel of St. Matthew, 5, 21 ff.,
where Christ Himself explains and sums it up, namely, that we must not
kill neither with hand, heart, mouth, signs, gestures, help, nor
counsel. Therefore it is here forbidden to every one to be angry,
except those (as we said) who are in the place of God, that is, parents
and the government. For it is proper for God and for every one who is
in a divine estate to be angry, to reprove and punish, namely, on
account of those very persons who transgress this and the other
commandments.

But the cause and need of this commandment is that God well knows that
the world is evil, and that this life has much unhappiness; therefore
He has placed this and the other commandments between the good and the
evil. Now, as there are many assaults upon all commandments, so it
happens also in this commandment that we must live among many people
who do us harm, so that we have cause to be hostile to them.

As when your neighbor sees that you have a better house and home [a
larger family and more fertile fields], greater possessions and fortune
from God than he, he is sulky, envies you, and speaks no good of you.

Thus by the devil's incitement you will get many enemies who cannot
bear to see you have any good, either bodily or spiritual. When we see
such people, our hearts, in turn, would rage and bleed and take
vengeance. Then there arise cursing and blows, from which follow
finally misery and murder. Here, now, God like a kind father steps in
ahead of Us, interposes and wishes to have the quarrel settled, that no
misfortune come of it, nor one destroy another. And briefly He would
hereby protect, set free, and keep in peace every one against the crime
and violence of every one else; and would have this commandment placed
as a wall, fortress, and refuge about our neighbor, that we do him no
hurt nor harm in his body.

Thus this commandment aims at this, that no one offend his neighbor on
account of any evil deed, even though he have fully deserved it. For
where murder is forbidden, all cause also is forbidden whence murder
may originate. For many a one, although he does not kill, yet curses
and utters a wish, which would stop a person from running far if it
were to strike him in the neck [makes imprecations, which if fulfilled
with respect to any one, he would not live long]. Now since this
inheres in every one by nature and it is a common practice that no one
is willing to suffer at the hands of another, God wishes to remove the
root and source by which the heart is embittered against our neighbor,
and to accustom us ever to keep in view this commandment, always to
contemplate ourselves in it as in a mirror, to regard the will of God,
and with hearty confidence and invocation of His name to commit to Him
the wrong which we suffer. Thus we shall suffer our enemies to rage and
be angry, doing what they can, and we learn to calm our wrath, and to
have a patient, gentle heart, especially toward those who give us cause
to be angry, that is, our enemies.

Therefore the entire sum of what it means not to kill is to be
impressed most explicitly upon the simple-minded. In the first place
that we harm no one, first, with our hand or by deed. Then, that we do
not employ our tongue to instigate or counsel thereto. Further, that we
neither use nor assent to any kind of means or methods whereby any one
may be injured. And finally, that the heart be not ill disposed toward
any one, nor from anger and hatred wish him ill, so that body and soul
may be innocent in regard to every one, but especially those who wish
you evil or inflict such upon you. For to do evil to one who wishes and
does you good is not human, but diabolical.

Secondly, under this commandment not only he is guilty who does evil to
his neighbor, but he also who can do him good, prevent, resist evil,
defend and save him, so that no bodily harm or hurt happen to him and
yet does not do it. If, therefore, you send away one that is naked when
you could clothe him, you have caused him to freeze to death; you see
one suffer hunger and do not give him food, you have caused him to
starve. So also, if you see any one innocently sentenced to death or in
like distress, and do not save him, although you know ways and means to
do so, you have killed him. And it will not avail you to make the
pretext that you did not afford any help, counsel, or aid thereto for
you have withheld your love from him and deprived him of the benefit
whereby his life would have been saved.

Therefore God also rightly calls all those murderers who do not afford
counsel and help in distress and danger of body and life, and will pass
a most terrible sentence upon them in the last day, as Christ Himself
has announced when He shall say, Matt.25, 42f.: I was an hungered, and
ye gave Me no meat; I was thirsty, and ye gave Me no drink; I was a
stranger, and ye took Me not in; naked, and ye clothed Me not; sick and
in prison and ye visited Me not. That is: You would have suffered Me
and Mine to die of hunger thirst, and cold, would have suffered the
wild beasts to tear us to pieces, or left us to rot in prison or perish
in distress. What else is that but to reproach them as murderers and
bloodhounds? For although you have not actually done all this, you have
nevertheless, so far as you were concerned, suffered him to pine and
perish in misfortune.

It is just as if I saw some one navigating and laboring in deep water
[and struggling against adverse winds] or one fallen into fire, and
could extend to him the hand to pull him out and save him, and yet
refused to do it. What else would I appear, even in the eyes of the
world, than as a murderer and a criminal?

Therefore it is God's ultimate purpose that we suffer harm to befall no
man, but show him all good and love; and, as we have said it is
specially directed toward those who are our enemies. For to do good to
our friends is but an ordinary heathen virtue as Christ says Matt. 5,
46.

Here we have again the Word of God whereby He would encourage and urge
us to true noble and sublime works, as gentleness patience, and, in
short, love and kindness to our enemies, and would ever remind us to
reflect upon the First Commandment, that He is our God, that is, that
He will help, assist, and protect us, in order that He may thus quench
the desire of revenge in us.

This we ought to practice and inculcate and we would have our hands
full doing good works. But this would not be preaching for monks; it
would greatly detract from the religious estate, and infringe upon the
sanctity of Carthusians, and would even be regarded as forbidding good
works and clearing the convents. For in this wise the ordinary state of
Christians would be considered just as worthy, and even worthier, and
everybody would see how they mock and delude the world with a false,
hypocritical show of holiness, because they have given this and other
commandments to the winds, and have esteemed them unnecessary, as
though they were not commandments but mere counsels, and have at the
same time shamelessly proclaimed and boasted their hypocritical estate
and works as the most perfect life, in order that they might lead a
pleasant, easy life, without the cross and without patience, for which
reason, too, they have resorted to the cloisters, so that they might
not be obliged to suffer any wrong from any one or to do him any good.
But know now that these are the true, holy, and godly works, in which,
with all the angels He rejoices, in comparison with which all human
holiness is but stench and filth, and besides, deserves nothing but
wrath and damnation.

The Sixth Commandment.

Thou shalt not commit adultery.

These commandments now [that follow] are easily understood from [the
explanation of] the preceding; for they are all to the effect that we
[be careful to] avoid doing any kind of injury to our neighbor. But
they are arranged in fine [elegant] order. In the first place, they
treat of his own person. Then they proceed to the person nearest him,
or the closest possession next after his body namely, his wife, who is
one flesh and blood with him, so that we cannot inflict a higher injury
upon him in any good that is his. Therefore it is explicitly forbidden
here to bring any disgrace upon him in respect to his wife. And it
really aims at adultery, because among the Jews it was ordained and
commanded that every one must be married. Therefore also the young were
early provided for [married], so that the virgin state was held in
small esteem, neither were public prostitution and lewdness tolerated
(as now). Therefore adultery was the most common form of unchastity
among them.

But because among us there is such a shameful mess and the very dregs
of all vice and lewdness, this commandment is directed also against all
manner of unchastity, whatever it may be called; and not only is the
external act forbidden, but also every kind of cause, incitement, and
means, so that the heart, the lips, and the whole body may be chaste
and afford no opportunity, help, or persuasion to unchastity. And not
only this, but that we also make resistance, afford protection and
rescue wherever there is danger and need; and again, that we give help
and counsel, so as to maintain our neighbor's honor. For whenever you
omit this when you could make resistance, or connive at it as if it did
not concern you, you are as truly guilty as the one perpetrating the
deed. Thus, to state it in the briefest manner, there is required this
much, that every one both live chastely himself and help his neighbor
do the same, so that God by this commandment wishes to hedge round
about and protect [as with a rampart] every spouse that no one trespass
against them.

But since this commandment is aimed directly at the state of matrimony
and gives occasion to speak of the same, you must well understand and
mark, first, how gloriously God honors and extols this estate, inasmuch
as by His commandment He both sanctions and guards it. He has
sanctioned it above in the Fourth Commandment: Honor thy father and thy
mother; but here He has (as we said ) hedged it about and protected it.
Therefore He also wishes us to honor it, and to maintain and conduct it
as a divine and blessed estate; because, in the first place, He has
instituted it before all others, and therefore created man and woman
separately (as is evident), not for lewdness, but that they should
[legitimately] live together, be fruitful, beget children, and nourish
and train them to the honor of God.

Therefore God has also most richly blessed this estate above all
others, and, in addition, has bestowed on it and wrapped up in it
everything in the world, to the end that this estate might be well and
richly provided for. Married life is therefore no jest or presumption;
but it is an excellent thing and a matter of divine seriousness. For it
is of the highest importance to Him that persons be raised who may
serve the world and promote the knowledge of God, godly living, and all
virtues, to fight against wickedness and the devil.

Therefore I have always taught that this estate should not be despised
nor held in disrepute, as is done by the blind world and our false
ecclesiastics, but that it be regarded according to God's Word, by
which it is adorned and sanctified, so that it is not only placed on an
equality with other estates, but that it precedes and surpasses them
all, whether they be that of emperor, princes, bishops, or whoever they
please. For both ecclesiastical and civil estates must humble
themselves and all be found in this estate as we shall hear. Therefore
it is not a peculiar estate, but the most common and noblest estate,
which pervades all Christendom, yea which extends through all the
world.

In the second place, you must know also that it is not only an
honorable, but also a necessary state, and it is solemnly commanded by
God that, in general, in all conditions, men and women, who were
created for it, shall be found in this estate; yet with some exceptions
(although few) whom God has especially excepted, so that they are not
fit for the married estate, or whom He has released by a high,
supernatural gift that they can maintain chastity without this estate.
For where nature has its course, as it is implanted by God, it is not
possible to remain chaste without marriage. For flesh and blood remain
flesh and blood, and the natural inclination and excitement have their
course without let or hindrance, as everybody sees and feels. In
order, therefore, that it may be the more easy in some degree to avoid
unchastity, God has commanded the estate of matrimony, that every one
may have his proper portion and be satisfied therewith; although God's
grace besides is required in order that the heart also may be pure.

From this you see how this popish rabble, priests, monks, and nuns,
resist God's order and commandment, inasmuch as they despise and forbid
matrimony, and presume and vow to maintain perpetual chastity, and,
besides, deceive the simple-minded with lying words and appearances
[impostures]. For no one has so little love and inclination to chastity
as just those who because of great sanctity avoid marriage, and either
indulge in open and shameless prostitution, or secretly do even worse,
so that one dare not speak of it, as has, alas! been learned too fully.
And, in short, even though they abstain from the act, their hearts are
so full of unchaste thoughts and evil lusts that there is a continual
burning and secret suffering, which can be avoided in the married life.
Therefore all vows of chastity out of the married state are condemned
by this commandment, and free permission is granted, yea, even the
command is given, to all poor ensnared consciences which have been
deceived by their monastic vows to abandon the unchaste state and enter
the married life, considering that even if the monastic life were
godly, it would nevertheless not be in their power to maintain
chastity, and if they remain in it, they must only sin more and more
against this commandment.

Now, I speak of this in order that the young may be so guided that they
conceive a liking for the married estate, and know that it is a blessed
estate and pleasing to God. For in this way we might in the course of
time bring it about that married life be restored to honor, and that
there might be less of the filthy, dissolute, disorderly doings which
now run riot the world over in open prostitution and other shameful
vices arising from disregard of married life. Therefore it is the duty
of parents and the government to see to it that our youth be brought up
to discipline and respectability, and when they have come to years of
maturity, to provide for them [to have them married] in the fear of God
and honorably; He would not fail to add His blessing and grace, so that
men would have joy and happiness from the same.

Let me now say in conclusion that this commandment demands not only
that every one live chastely in thought, word, and deed in his
condition, that is, especially in the estate of matrimony, but also
that every one love and esteem the spouse given him by God. For where
conjugal chastity is to be maintained, man and wife must by all means
live together in love and harmony, that one may cherish the other from
the heart and with entire fidelity. For that is one of the principal
points which enkindle love and desire of chastity, so that, where this
is found, chastity will follow as a matter of course without any
command. Therefore also St. Paul so diligently exhorts husband and wife
to love and honor one another. Here you have again a precious, yea,
many and great good works, of which you can joyfully boast, against all
ecclesiastical estates, chosen without God's Word and commandment.

 The Seventh Commandment.

Thou shalt not steal.

After your person and spouse temporal property comes next. That also
God wishes to have protected, and He has commanded that no one shall
subtract from, or curtail, his neighbor's possessions. For to steal is
nothing else than to get possession of another's property wrongfully,
which briefly comprehends all kinds of advantage in all sorts of trade
to the disadvantage of our neighbor. Now, this is indeed quite a
wide-spread and common vice, but so little regarded and observed that
it exceeds all measure, so that if all who are thieves, and yet do not
wish to be called such, were to be hanged on gallows the world would
soon be devastated and there would be a lack both of executioners and
gallows. For, as we have just said, to steal is to signify not only to
empty our neighbor's coffer and pockets, but to be grasping in the
market, in all stores, booths, wine- and beer-cellars, workshops, and,
in short, wherever there is trading or taking and giving of money for
merchandise or labor.

As, for instance, to explain this somewhat grossly for the common
people, that it may be seen how godly we are: When a manservant or
maid-servant does not serve faithfully in the house, and does damage,
or allows it to be done when it could be prevented, or otherwise ruins
and neglects the goods entrusted to him, from indolence idleness, or
malice, to the spite and vexation of master and mistress, and in
whatever way this can be done purposely (for I do not speak of what
happens from oversight and against one's will), you can in a year
abscond thirty, forty florins, which if another had taken secretly or
carried away, he would be hanged with the rope. But here you [while
conscious of such a great theft] may even bid defiance and become
insolent, and no one dare call you a thief.

The same I say also of mechanics, workmen, and day-laborers, who all
follow their wanton notions, and never know enough ways to overcharge
people, while they are lazy and unfaithful in their work. All these are
far worse than sneak-thieves, against whom we can guard with locks and
bolts, or who, if apprehended, are treated in such a manner that they
will not do the same again. But against these no one can guard, no one
dare even look awry at them or accuse them of theft, so that one would
ten times rather lose from his purse. For here are my neighbors, good
friends, my own servants, from whom I expect good [every faithful and
diligent service], who defraud me first of all.

Furthermore, in the market and in common trade likewise, this practice
is in full swing and force to the greatest extent, where one openly
defrauds another with bad merchandise, false measures, weights, coins,
and by nimbleness and queer finances or dexterous tricks takes
advantage of him; likewise, when one overcharges a person in a trade
and wantonly drives a hard bargain, skins and distresses him. And who
can recount or think of all these things? To sum up, this is the
commonest craft and the largest guild on earth, and if we regard the
world throughout all conditions of life, it is nothing else than a
vast, wide stall, full of great thieves.

Therefore they are also called swivel-chair robbers, land- and
highway-robbers, not pick-locks and sneak-thieves who snatch away the
ready cash, but who sit on the chair [at home] and are styled great
noblemen, and honorable, pious citizens, and yet rob and steal under a
good pretext.

Yes, here we might be silent about the trifling individual thieves if
we were to attack the great, powerful arch-thieves with whom lords and
princes keep company, who daily plunder not only a city or two, but all
Germany. Yea, where should we place the head and supreme protector of
all thieves, the Holy Chair at Rome with all its retinue, which has
grabbed by theft the wealth of all the world, and holds it to this day?


This is, in short, the course of the world: whoever can steal and rob
openly goes free and secure, unmolested by any one, and even demands
that he be honored. Meanwhile the little sneak-thieves, who have once
trespassed, must bear the shame and punishment to render the former
godly and honorable. But let them know that in the sight of God they
are the greatest thieves, and that He will punish them as they are
worthy and deserve.

Now, since this commandment is so far-reaching [and comprehensive], as
just indicated, it is necessary to urge it well and to explain it to
the common people, not to let them go on in their wantonness and
security, but always to place before their eyes the wrath of God, and
inculcate the same. For we have to preach this not to Christians, but
chiefly to knaves and scoundrels, to whom it would be more fitting for
judges, jailers, or Master Hannes [the executioner] to preach.
Therefore let every one know that it is his duty, at the risk of God's
displeasure, not only to do no injury to his neighbor, nor to deprive
him of gain, nor to perpetrate any act of unfaithfulness or malice in
any bargain or trade, but faithfully to preserve his property for him,
to secure and promote his advantage, especially when one accepts money,
wages, and one's livelihood for such service.

He now who wantonly despises this may indeed pass along and escape the
hangman, but he shall not escape the wrath and punishment of God; and
when he has long practiced his defiance and arrogance, he shall yet
remain a tramp and beggar, and, in addition, have all plagues and
misfortune. Now you are going your way [wherever your heart's pleasure
calls you] while you ought to preserve the property of your master and
mistress, for which service you fill your crop and maw, take your wages
like a thief, have people treat you as a nobleman; for there are many
that are even insolent towards their masters and mistresses, and are
unwilling to do them a favor or service by which to protect them from
loss.

But reflect what you will gain when, having come into your own
property and being set up in your home (to which God will help with all
misfortunes), it [your perfidy] will bob up again and come home to you,
and you will find that where you have cheated or done injury to the
value of one mite, you will have to pay thirty again.

Such shall be the lot also of mechanics and day-laborers of whom we are
now obliged to hear and suffer such intolerable maliciousness, as
though they were noblemen in another's possessions, and every one were
obliged to give them what they demand. Just let them continue
practicing their exactions as long as they can; but God will not forget
His commandment, and will reward them according as they have served,
and will hang them, not upon a green gallows, but upon a dry one so
that all their life they shall neither prosper nor accumulate anything.
And indeed, if there were a well-ordered government in the land, such
wantonness might soon be checked and prevented, as was the custom in
ancient times among the Romans, where such characters were promptly
seized by the pate in a way that others took warning.

No more shall all the rest prosper who change the open free market into
a carrion-pit of extortion and a den of robbery, where the poor are
daily overcharged, new burdens and high prices are imposed, and every
one uses the market according to his caprice, and is even defiant and
brags as though it were his fair privilege and right to sell his goods
for as high a price as he please, and no one had a right to say a word
against it. We will indeed look on and let these people skin, pinch,
and hoard, but we will trust in God -- who will, however, do this of
His own accord, -- that, after you have been skinning and scraping for
a long time, He will pronounce such a blessing on your gains that your
grain in the garner, your beer in the cellar, your cattle in the stalls
shall perish; yea, where you have cheated and overcharged any one to
the amount of a florin, your entire pile shall be consumed with rust,
so that you shall never enjoy it.

And indeed, we see and experience this being fulfilled daily before our
eyes, that no stolen or dishonestly acquired possession thrives. How
many there are who rake and scrape day and night, and yet grow not a
farthing richer! And though they gather much, they must suffer so many
plagues and misfortunes that they cannot relish it with cheerfulness
nor transmit it to their children. But as no one minds it, and we go on
as though it did not concern us, God must visit us in a different way
and teach us manners by imposing one taxation after another, or
billeting a troop of soldiers upon us, who in one hour empty our
coffers and purses, and do not quit as long as we have a farthing
left, and in addition, by way of thanks, burn and devastate house and
home, and outrage and kill wife and children.

And, in short, if you steal much, depend upon it that again as much
will be stolen from you; and he who robs and acquires with violence and
wrong will submit to one who shall deal after the same fashion with
him. For God is master of this art, that since every one robs and
steals from the other, He punishes one thief by means of another. Else
where should we find enough gallows and ropes?

Now, whoever is willing to be instructed let him know that this is the
commandment of God, and that it must not be treated as a jest. For
although you despise us, defraud, steal, and rob, we will indeed manage
to endure your haughtiness, suffer, and, according to the Lord's
Prayer, forgive and show pity; for we know that the godly shall
nevertheless have enough, and you injure yourself more than another.

But beware of this: When the poor man comes to you (of whom there are
so many now) who must buy with the penny of his daily wages and live
upon it, and you are harsh to him, as though every one lived by your
favor, and you skin and scrape to the bone, and, besides, with pride
and haughtiness turn him off to whom you ought to give for nothing, he
will go away wretched and sorrowful, and since he can complain to no
one he will cry and call to heaven, -- then beware (I say again) as of
the devil himself. For such groaning and calling will be no jest, but
will have a weight that will prove too heavy for you and all the
world. For it will reach Him who takes care of the poor sorrowful
hearts, and will not allow them to go unavenged. But if you despise
this and become defiant, see whom you have brought upon you: if you
succeed and prosper, you may before all the world call God and me a
liar.

We have exhorted, warned, and protested enough; he who will not heed or
believe it may go on until he learns this by experience Yet it must be
impressed upon the young that they may be careful not to follow the old
lawless crowd, but keep their eyes fixed upon God's commandment, lest
His wrath and punishment come upon them too. It behooves us to do no
more than to instruct and reprove with God's Word; but to check such
open wantonness there is need of the princes and government, who
themselves would have eyes and the courage to establish and maintain
order in all manner of trade and commerce, lest the poor be burdened
and oppressed nor they themselves be loaded with other men's sins.

Let this suffice as an explanation of what stealing is, that it be not
taken too narrowly but made to extend as far as we have to do with our
neighbors. And briefly, in a summary, as in the former commandments, it
is herewith forbidden, in the first place, to do our neighbor any
injury or wrong (in whatever manner supposable, by curtailing,
forestalling, and withholding his possessions and property), or even to
consent or allow such a thing, but to interpose and prevent it. And, on
the other hand, it is commanded that we advance and improve his
possessions, and in case he suffers want, that we help, communicate,
and lend both to friends and foes.

Whoever now seeks and desires good works will find here more than
enough such as are heartily acceptable and pleasing to God, and in
addition are favored and crowned with excellent blessings, that we are
to be richly compensated for all that we do for our neighbor's good and
from friendship; as King Solomon also teaches Prov. 19, 17: He that
hath pity upon the poor lendeth unto the Lord; and that which he hath
given will He pay him again. Here, then you have a rich Lord, who is
certainly sufficient for you, and who will not suffer you to come short
in anything or to want; thus you can with a joyful conscience enjoy a
hundred times more than you could scrape together with unfaithfulness
and wrong. Now, whoever does not desire the blessing will find wrath
and misfortune enough.

The Eighth Commandment.

Thou shalt not bear false witness against thy neighbor.

Over and above our own body, spouse, and temporal possessions, we have
yet another treasure, namely, honor and good report [the illustrious
testimony of an upright and unsullied name and reputation], with which
we cannot dispense. For it is intolerable to live among men in open
shame and general contempt. Therefore God wishes the reputation, good
name, and upright character of our neighbor to be taken away or
diminished as little as his money and possessions, that every one may
stand in his integrity before wife, children, servants, and neighbors.
And in the first place, we take the plainest meaning of this
commandment according to the words (Thou shalt not bear false witness),
as pertaining to the public courts of justice, where a poor innocent
man is accused and oppressed by false witnesses in order to be punished
in his body, property, or honor.

Now, this appears as if it were of little concern to us at present; but
with the Jews it was quite a common and ordinary matter. For the people
were organized under an excellent and regular government; and where
there is still such a government, instances of this sin will not be
wanting. The cause of it is that where judges, burgomasters, princes,
or others in authority sit in judgment, things never fail to go
according to the course of the world; namely, men do not like to offend
anybody, flatter, and speak to gain favor, money, prospects, or
friendship; and in consequence a poor man and his cause must be
oppressed, denounced as wrong, and suffer punishment. And it is a
common calamity in the world that in courts of justice there seldom
preside godly men.

For to be a judge requires above all things a godly man, and not only a
godly, but also a wise, modest, yea, a brave and bold man; likewise, to
be a witness requires a fearless and especially a godly man. For a
person who is to judge all matters rightly and carry them through with
his decision will often offend good friends, relatives, neighbors, and
the rich and powerful, who can greatly serve or injure him. Therefore
he must be quite blind, have his eyes and ears closed, neither see nor
hear, but go straight forward in everything that comes before him, and
decide accordingly.

Therefore this commandment is given first of all that every one shall
help his neighbor to secure his rights, and not allow them to be
hindered or twisted, but shall promote and strictly maintain them, no
matter whether he be judge or witness, and let it pertain to whatsoever
it will. And especially is a goal set up here for our jurists that they
be careful to deal truly and uprightly with every case, allowing right
to remain right, and, on the other hand, not perverting anything [by
their tricks and technical points turning black into white and making
wrong out to be right], nor glossing it over or keeping silent
concerning it, irrespective of a person's money, possession, honor, or
power. This is one part and the plainest sense of this commandment
concerning all that takes place in court.

Next, it extends very much further, if we are to apply it to spiritual
jurisdiction or administration; here it is a common occurrence that
every one bears false witness against his neighbor. For wherever there
are godly preachers and Christians, they must bear the sentence before
the world that they are called heretics, apostates, yea, seditious and
desperately wicked miscreants. Besides the Word of God must suffer in
the most shameful and malicious manner, being persecuted blasphemed,
contradicted, perverted and falsely cited and interpreted. But let this
pass; for it is the way of the blind world that she condemns and
persecutes the truth and the children of God, and yet esteems it no
sin.

In the third place, what concerns us all, this commandment forbids all
sins of the tongue whereby we may injure or approach too closely to our
neighbor. For to bear false witness is nothing else than a work of the
tongue. Now, whatever is done with the tongue against a fellow-man God
would have prohibited, whether it be false preachers with their
doctrine and blasphemy, false judges and witnesses with their verdict,
or outside of court by lying and evil-speaking. Here belongs
particularly the detestable, shameful vice of speaking behind a
person's back and slandering, to which the devil spurs us on and of
which there would be much to be said. For it is a common evil plague
that every one prefers hearing evil to hearing good of his neighbor;
and although we ourselves are so bad that we cannot suffer that any one
should say anything bad about us, but every one would much rather that
all the world should speak of him in terms of gold, yet we cannot bear
that the best is spoken about others.

Therefore, to avoid this vice we should note that no one is allowed
publicly to judge and reprove his neighbor, although he may see him
sin, unless he have a command to judge and to reprove. For there is a
great difference between these two things, judging sin and knowing sin.
You may indeed know it, but you are not to judge it. I can indeed see
and hear that my neighbor sins, but I have no command to report it to
others. Now, if I rush in, judging and passing sentence, I fall into a
sin which is greater than his. But if you know it, do nothing else
than turn your ears into a grave and cover it, until you are appointed
to be judge and to punish by virtue of your office.

Those, then, are called slanderers who are not content with knowing a
thing, but proceed to assume jurisdiction, and when they know a slight
offense of another, carry it into every corner, and are delighted and
tickled that they can stir up another's displeasure [baseness], as
swine roll themselves in the dirt and root in it with the snout. This
is nothing else than meddling with the judgment and office of God, and
pronouncing sentence and punishment with the most severe verdict. For
no judge can punish to a higher degree nor go farther than to say: "He
is a thief, a murderer, a traitor," etc. Therefore, whoever presumes to
say the same of his neighbor goes just as far as the emperor and all
governments. For although you do not wield the sword, you employ your
poisonous tongue to the shame and hurt of your neighbor.

God therefore would have it prohibited that any one speak evil of
another even though he be guilty, and the latter know it right well;
much less if he do not know it, and have it only from hearsay. But you
say: Shall I not say it if it be the truth? Answer: Why do you not make
accusation to regular judges? Ah, I cannot prove it publicly, and hence
I might be silenced and turned away in a harsh manner [incur the
penalty of a false accusation]. "Ah, indeed, do you smell the roast?"
If you do not trust yourself to stand before the proper authorities and
to make answer, then hold your tongue. But if you know it, know it for
yourself and not for another. For if you tell it to others, although it
be true, you will appear as a liar, because you cannot prove it, and
you are, besides acting like a knave. For we ought never to deprive any
one of his honor or good name unless it be first taken away from him
publicly.

False witness, then, is everything which cannot be properly proved.
Therefore, what is not manifest upon sufficient evidence no one shall
make public or declare for truth; and in short, whatever is secret
should be allowed to remain secret, or, at any rate, should be secretly
reproved, as we shall hear. Therefore, if you encounter an idle tongue
which betrays and slanders some one, contradict such a one promptly to
his face, that he may blush thus many a one will hold his tongue who
else would bring some poor man into bad repute from which he would not
easily extricate himself. For honor and a good name are easily taken
away, but not easily restored.

Thus you see that it is summarily forbidden to speak any evil of our
neighbor, however the civil government, preachers, father and mother
excepted, on the understanding that this commandment does not allow
evil to go unpunished. Now, as according to the Fifth Commandment no
one is to be injured in body, and yet Master Hannes [the executioner]
is excepted, who by virtue of his office does his neighbor no good, but
only evil and harm, and nevertheless does not sin against God's
commandment, because God has on His own account instituted that office;
for He has reserved punishment for His own good pleasure, as He
threatens in the First Commandment, -- just so also, although no one
has a right in his own person to judge and condemn anybody, yet if they
to whose office it belongs fail to do it, they sin as well as he who
would do so of his own accord, without such office. For here necessity
requires one to speak of the evil, to prefer charges, to investigate
and testify; and it is not different from the case of a physician who
is sometimes compelled to examine and handle the patient whom he is to
cure in secret parts. Just so governments, father and mother, brothers
and sisters, and other good friends, are under obligation to each other
to reprove evil wherever it is needful and profitable.

But the true way in this matter would be to observe the order
according to the Gospel, Matt. 18, 15, where Christ says: If thy
brother shall trespass against thee, go and tell him his fault between
thee and him alone. Here you have a precious and excellent teaching for
governing well the tongue, which is to be carefully observed against
this detestable misuse. Let this, then, be your rule, that you do not
too readily spread evil concerning your neighbor and slander him to
others, but admonish him privately that he may amend [his life].
Likewise, also, if some one report to you what this or that one has
done, teach him, too, to go and admonish him personally if he have seen
it himself; but if not, that he hold his tongue.

The same you can learn also from the daily government of the
household. For when the master of the house sees that the servant does
not do what he ought, he admonishes him personally. But if he were so
foolish as to let the servant sit at home, and went on the streets to
complain of him to his neighbors, he would no doubt be told: "You fool,
what does that concern us? Why do you not tell it to him ?" Behold,
that would be acting quite brotherly, so that the evil would be stayed,
and your neighbor would retain his honor. As Christ also says in the
same place: If he hear thee, thou host gained thy brother. Then you
have done a great and excellent work; for do you think it is a little
matter to gain a brother? Let all monks and holy orders step forth,
with all their works melted together into one mass, and see if they
can boast that they have gained a brother.

Further, Christ teaches: But if he will not hear thee, then take with
thee one or two more, that in the mouth of two or three witnesses every
word may be established. So he whom it concerns is always to be treated
with personally, and not to be spoken of without his knowledge. But if
that do not avail, then bring it publicly before the community, whether
before the civil or the ecclesiastical tribunal. For then you do not
stand alone, but you have those witnesses with you by whom you can
convict the guilty one, relying on whom the judge can pronounce
sentence and punish. This is the right and regular course for checking
and reforming a wicked person. But if we gossip about another in all
corners and stir the filth, no one will be reformed, and afterwards
when we are to stand up and bear witness, we deny having said so.
Therefore it would serve such tongues right if their itch for slander
were severely punished, as a warning to others. If you were acting for
your neighbor's reformation or from love of the truth, you would not
sneak about secretly nor shun the day and the light.

All this has been said regarding secret sins. But where the sin is
quite public so that the judge and everybody know it you can without
any sin avoid him and let him go, because he has brought himself into
disgrace, and you may also publicly testify concerning him. For when a
matter is public in the light of day, there can be no slandering or
false judging or testifying; as, when we now reprove the Pope with his
doctrine, which is publicly set forth in books and proclaimed in all
the world. For where the sin is public, the reproof also must be
public, that every one may learn to guard against it.

Thus we have now the sum and general understanding of this
commandment, to wit, that no one do any injury with the tongue to his
neighbor, whether friend or foe, nor speak evil of him, no matter
whether it be true or false, unless it be done by commandment or for
his reformation, but that every one employ his tongue and make it serve
for the best of every one else, to cover up his neighbor's sins and
infirmities, excuse them, palliate and garnish them with his own
reputation. The chief reason for this should be the one which Christ
alleges in the Gospel, in which He comprehends all commandments
respecting our neighbor, Matt. 7, 12: Whatsoever ye would that men
should do to you, do ye even so to them.

Even nature teaches the same thing in our own bodies, as St. Paul
says, 1 Cor. 12, 22: Much more, those members of the body which seem to
be more feeble are necessary; and those members of the body which we
think to be less honorable, upon these we bestow more abundant honor;
and our uncomely parts have more abundant comeliness. No one covers his
face, eyes, nose, and mouth, for they, being in themselves the most
honorable members which we have, do not require it. But the most infirm
members, of which we are ashamed, we cover with all diligence; hands,
eyes, and the whole body must help to cover and conceal them. Thus also
among ourselves should we adorn whatever blemishes and infirmities we
find in our neighbor, and serve and help him to promote his honor to
the best of our ability, and, on the other hand, prevent whatever may
be discreditable to him. And it is especially an excellent and noble
virtue for one always to explain advantageously and put the best
construction upon all he may hear of his neighbor (if it be not
notoriously evil), or at any rate to condone it over and against the
poisonous tongues that are busy wherever they can pry out and discover
something to blame in a neighbor, and that explain and pervert it in
the worst way; as is done now especially with the precious Word of God
and its preachers.

There are comprehended therefore in this commandment quite a multitude
of good works which please God most highly, and bring abundant good and
blessing, if only the blind world and the false saints would recognize
them. For there is nothing on or in entire man which can do both
greater and more extensive good or harm in spiritual and in temporal
matters than the tongue, though it is the least and feeblest member.

The Ninth and Tenth Commandments

Thou shalt not covet thy neighbor's house. Thou shalt not covet thy
neighbor's wife, nor his man-servant, nor his maid-servant, nor his
cattle, nor anything that is his.

These two commandments are given quite exclusively to the Jews;
nevertheless, in part they also concern us. For they do not interpret
them as referring to unchastity or theft, because these are
sufficiently forbidden above. They also thought that they had kept all
those when they had done or not done the external act. Therefore God
has added these two commandments in order that it be esteemed as sin
and forbidden to desire or in any way to aim at getting our neighbor's
wife or possessions; and especially because under the Jewish government
man-servants and maid-servants were not free as now to serve for wages
as long as they pleased, but were their master's property with their
body and all they had, as cattle and other possessions. Moreover,
every man had power over his wife to put her away publicly by giving
her a bill of divorce, and to take another. Therefore they were in
constant danger among each other that if one took a fancy to another's
wife, he might allege any reason both to dismiss his own wife and to
estrange the other's wife from him, that he might obtain her under
pretext of right. That was not considered a sin nor disgrace with them;
as little as now with hired help, when a proprietor dismisses his
man-servant or maid-servant, or takes another's servants from him in
any way.

Therefore (I say) they thus interpreted these commandments, and that
rightly (although their scope reaches somewhat farther and higher),
that no one think or purpose to obtain what belongs to another, such as
his wife, servants, house and estate, land meadows, cattle, even with a
show of right or by a subterfuge, yet with injury to his neighbor. For
above, in the Seventh Commandment, the vice is forbidden where one
wrests to himself the possessions of others, or withholds them from his
neighbor, which he cannot do by right. But here it is also forbidden to
alienate anything from your neighbor, even though you could do so with
honor in the eyes of the world, so that no one could accuse or blame
you as though you had obtained it wrongfully.

For we are so inclined by nature that no one desires to see another
have as much as himself, and each one acquires as much as he can; the
other may fare as best he can. And yet we pretend to be godly, know how
to adorn ourselves most finely and conceal our rascality, resort to and
invent adroit devices and deceitful artifices (such as now are daily
most ingeniously contrived) as though they were derived from the law
codes; yea, we even dare impertinently to refer to it, and boast of it,
and will not have it called rascality, but shrewdness and caution. In
this lawyers and jurists assist, who twist and stretch the law to suit
it to their cause, stress words and use them for a subterfuge,
irrespective of equity or their neighbor's necessity. And, in short,
whoever is the most expert and cunning in these affairs finds most help
in law, as they themselves say: Vigilantibus iura subveniunt [that is,
The laws favor the watchful].

This last commandment therefore is given not for rogues in the eyes of
the world, but just for the most pious, who wish to be praised and be
called honest and upright people, since they have not offended against
the former commandments, as especially the Jews claimed to be, and even
now many great noblemen, gentlemen, and princes. For the other common
masses belong yet farther down, under the Seventh Commandment, as those
who are not much concerned whether they acquire their possessions with
honor and right.

Now, this occurs most frequently in cases that are brought into court,
where it is the purpose to get something from our neighbor and to force
him out of his own. As (to give examples), when people quarrel and
wrangle about a large inheritance, real estate, etc., they avail
themselves of, and resort to, whatever has the appearance of right, so
dressing and adorning everything that the law must favor their side,
and they keep the property with such title that no one can make
complaint or lay claim thereto. In like manner, if any one desire to
have a castle, city, duchy, or any other great thing, he practices so
much financiering through relationships, and by any means he can, that
the other is judicially deprived of it, and it is adjudicated to him,
and confirmed with deed and seal and declared to have been acquired by
princely title and honestly.

Likewise also in common trade where one dexterously slips something out
of another's hand, so that he must look after it, or surprises and
defrauds him in a matter in which he sees advantage and benefit for
himself, so that the latter, perhaps on account of distress or debt,
cannot regain or redeem it without injury, and the former gains the
half or even more; and yet this must not be considered as acquired by
fraud or stolen, but honestly bought. Here they say: First come, first
served, and every one must look to his own interest, let another get
what he can. And who can be so smart as to think of all the ways in
which one can get many things into his possession by such specious
pretexts? This the world does not consider wrong [nor is it punished by
laws], and will not see that the neighbor is thereby placed at a
disadvantage, and must sacrifice what he cannot spare without injury.
Yet there is no one who wishes this to be done to him; from which we
can easily perceive that such devices and pretexts are false.

Thus it was done formerly also with respect to wives: they knew such
devices that if one were pleased with another woman, he personally or
through others (as there were many ways and means to be invented)
caused her husband to conceive a displeasure toward her, or had her
resist him and so conduct herself that he was obliged to dismiss her
and leave her to the other. That sort of thing undoubtedly prevailed
much under the Law, as also we read in the (Gospel of King Herod that
he took his brother's wife while he was yet living, and yet wished to
be thought an honorable, pious man, as St. Mark also testifies of him.
But such an example, I trust, will not occur among us, because in the
New Testament those who are married are forbidden to be divorced,
except in such a case where one [shrewdly] by some stratagem takes away
a rich bride from another. But it is not a rare thing with us that one
estranges or alienates another's man-servant or maid-servant, or
entices them away by flattering words.

In whatever way such things happen, we must know that God does not wish
that you deprive your neighbor of anything that belongs to him so that
he suffer the loss and you gratify your avarice with it, even if you
could keep it honorably before the world; for it is a secret and
insidious imposition practiced under the hat, as we say, that it may
not be observed. For although you go your way as if you had done no one
any wrong, you have nevertheless injured your neighbor; and if it is
not called stealing and cheating, yet it is called coveting your
neighbor's property, that is, aiming at possession of it, enticing it
away from him without his will, and being unwilling to see him enjoy
what God has granted him. And although the judge and every one must
leave you in possession of it, yet God will not leave you therein; for
He sees the deceitful heart and the malice of the world, which is sure
to take an ell in addition wherever you yield to her a finger's
breadth, and at length public wrong and violence follow.

Therefore we allow these commandments to remain in their ordinary
meaning, that it is commanded, first, that we do not desire our
neighbor's damage, nor even assist, nor give occasion for it, but
gladly wish and leave him what he has, and, besides, advance and
preserve for him what may be for his profit and service, as we should
wish to be treated. Thus these commandments are especially directed
against envy and miserable avarice, God wishing to remove all causes
and sources whence arises everything by which we do injury to our
neighbor, and therefore He expresses it in plain words: Thou shalt not
covet, etc. For He would especially have the heart pure, although we
shall never attain to that as long as we live here; so that this
commandment will remain, like all the rest, one that will constantly
accuse us and show how godly we are in the sight of God!

 Conclusion of the Ten Commandments.

Thus we have the Ten Commandments, a compend of divine doctrine, as to
what we are to do in order that our whole life may be pleasing to God,
and the true fountain and channel from and in which everything must
arise and flow that is to be a good work, so that outside of the Ten
Commandments no work or thing can be good or pleasing to God, however
great or precious it be in the eyes of the world. Let us see now what
our great saints can boast of their spiritual orders and their great
and grievous works which they have invented and set up, while they let
these pass, as though they were far too insignificant, or had long ago
been perfectly fulfilled.

I am of opinion indeed, that here one will find his hands full, [and
will have enough] to do to observe these, namely, meekness, patience,
and love towards enemies, chastity, kindness, etc., and what such
virtues imply. But such works are not of value and make no display in
the eyes of the world; for they are not peculiar and conceited works
and restricted to particular times, places, rites, and customs, but are
common, every-day domestic works which one neighbor can practice toward
another; therefore they are not of high esteem.

But the other works cause people to open their eyes and ears wide, and
men aid to this effect by the great display, expense, and magnificent
buildings with which they adorn them, so that everything shines and
glitters. There they waft incense, they sing and ring bells, they light
tapers and candles, so that nothing else can be seen or heard. For when
a priest stands there in a surplice embroidered with gilt, or a layman
continues all day upon his knees in church, that is regarded as a most
precious work which no one can sufficiently praise. But when a poor
girl tends a little child and faithfully does what she is told that is
considered nothing; for else what should monks and nuns seek in their
cloisters?

But see, is not that a cursed presumption of those desperate saints who
dare to invent a higher and better life and estate than the Ten
Commandments teach, pretending (as we have said) that this is an
ordinary life for the common man, but that theirs is for saints and
perfect ones? And the miserable blind people do not see that no man can
get so far as to keep one of the Ten Commandments as it should be kept,
but both the Apostles' Creed and the Lord's Prayer must come to our aid
(as we shall hear), by which that [power and strength to keep the
commandments] is sought and prayed for and received continually.
Therefore all their boasting amounts to as much as if I boasted and
said: To be sure, I have not a penny to make payment with, but I
confidently undertake to pay ten florins.

All this I say and urge in order that men might become rid of the sad
misuse which has taken such deep root and still cleaves to everybody,
and in all estates upon earth become used to looking hither only, and
to being concerned about these matters. For it will be a long time
before they will produce a doctrine or estates equal to the Ten
Commandments, because they are so high that no one can attain to them
by human power; and whoever does attain to them is a heavenly, angelic
man far above all holiness of the world. Only occupy yourself with
them, and try your best, apply all power and ability and you will find
so much to do that you will neither seek nor esteem any other work or
holiness.

Let this be sufficient concerning the first part of the common
Christian doctrine, both for teaching and urging what is necessary. In
conclusion, however, we must repeat the text which belongs here, of
which we have treated already in the First Commandment, in order that
we may learn what pains God requires to the end we may learn to
inculcate and practice the Ten Commandments:

For I the Lord, thy God, am a jealous God, visiting the iniquity of the
fathers upon the children unto the third and fourth generation of them
that hate Me, and showing mercy unto thousands of them that love Me and
keep My commandments.

Although (as we have heard above) this appendix was primarily attached
to the First Commandment, it was nevertheless [we cannot deny that it
was] laid down for the sake of all the commandments, as all of them are
to be referred and directed to it. Therefore I have said that this,
too, should be presented to and inculcated upon the young, that they
may learn and remember it, in order to see what is to urge and compel
us to keep these Ten Commandments. And it is to be regarded as though
this part were specially added to each, so that it inheres in, and
pervades, them all.

Now, there is comprehended in these words (as said before) both an
angry word of threatening and a friendly promise to terrify and warn
us, and, moreover to induce and encourage us to receive and highly
esteem His Word as a matter of divine earnestness, because He Himself
declares how much He is concerned about it, and how rigidly He will
enforce it, namely, that He will horribly and terribly punish all who
despise and transgress His commandments; and again, how richly He will
reward, bless, and do all good to those who hold them in high esteem,
and gladly do and live according to them. Thus He demands that all our
works proceed from a heart which fears and regards God alone, and from
such fear avoids everything that is contrary to His will, lest it
should move Him to wrath; and, on the other hand, also trusts in Him
alone, and from love to Him does all He wishes, because he speaks to us
as friendly as a father, and offers us all grace and every good.

Just this is also the meaning and true interpretation of the first and
chief commandment, from which all the others must flow and proceed, so
that this word: Thou shalt have no other gods before Me, in its
simplest meaning states nothing else than this demand: Thou shalt fear,
love, and trust in Me as thine only true God. For where there is a
heart thus disposed towards God, the same has fulfilled this and all
the other commandments. On the other hand, whoever fears and loves
anything else in heaven and upon earth will keep neither this nor any.
Thus the entire scriptures have everywhere preached and inculcated this
commandment, aiming always at these two things: fear of God and trust
in Him. And especially the prophet David throughout the Psalms, as when
he says [Ps. 147,11]: The Lord taketh pleasure in them that fear Him,
in those that hope in His mercy. As if the entire commandment were
explained by one verse, as much as to say: The Lord taketh pleasure in
those who have no other gods.

Thus the First Commandment is to shine and impart its splendor to all
the others. Therefore you must let this declaration run through all the
commandments, like a hoop in a wreath, joining the end to the beginning
and holding them all together, that it be continually repeated and not
forgotten; as, namely, in the Second Commandment, that we fear God and
do not take His name in vain for cursing, lying, deceiving, and other
modes of leading men astray, or rascality, but make proper and good use
of it by calling upon Him in prayer, praise, and thanksgiving, derived
from love and trust according to the First Commandment. In like manner
such fear, love, and trust is to urge and force us not to despise His
Word, but gladly to learn, hear, and esteem it holy, and honor it.

Thus continuing through all the following commandments towards our
neighbor likewise, everything is to proceed by virtue of the First
Commandment, to wit, that we honor father and mother, masters, and all
in authority and be subject and obedient to them, not on their own
account, but for God's sake. For you are not to regard or fear father
or mother, or from love of them do or omit anything. But see to that
which God would have you do, and what He will quite surely demand of
you; if you omit that, you have an angry Judge, but in the contrary
case a gracious Father.

Again, that you do your neighbor no harm, injury, or violence, nor in
any wise encroach upon him as touching his body, wife, property, honor,
or rights, as all these things are commanded in their order, even
though you have opportunity and cause to do so and no man would reprove
you; but that you do good to all men, help them, and promote their
interest, howsoever and wherever you can, purely from love of God and
in order to please Him, in the confidence that He will abundantly
reward you for everything. Thus you see how the First Commandment is
the chief source and fountainhead which flows into all the rest, and
again, all return to that and depend upon it, so that beginning and end
are fastened and bound to each other.

This (I say) it is profitable and necessary always to teach to the
young people, to admonish them and to remind them of it, that they may
be brought up not only with blows and compulsion, like cattle, but in
the fear and reverence of God. For where this is considered and laid to
heart that these things are not human trifles, but the commandments of
the Divine Majesty, who insists upon them with such earnestness, is
angry with, and punishes those who despise them, and, on the other
hand, abundantly rewards those who keep them, there will be a
spontaneous impulse and a desire gladly to do the will of God.
Therefore it is not in vain that it is commanded in the Old Testament
to write the Ten Commandments on all walls and corners, yes, even on
the garments, not for the sake of merely having them written in these
places and making a show of them, as did the Jews, but that we might
have our eyes constantly fixed upon them, and have them always in our
memory, and that we might practice them in all our actions and ways,
and every one make them his daily exercise in all cases, in every
business and transaction, as though they were written in every place
wherever he would look, yea, wherever he walks or stands. Thus there
would be occasion enough, both at home in our own house and abroad with
our neighbors, to practice the Ten Commandments, that no one need run
far for them.

From this it again appears how highly these Ten Commandments are to be
exalted and extolled above all estates, commandments, and works which
are taught and practiced aside from them. For here we can boast and
say: Let all the wise and saints step forth and produce, if they can, a
[single] work like these commandments, upon which God insists with such
earnestness, and which He enjoins with His greatest wrath and
punishment, and, besides, adds such glorious promises that He will pour
out upon us all good things and blessings. Therefore they should be
taught above all others, and be esteemed precious and dear, as the
highest treasure given by God.

Part Second. OF THE CREED.

Thus far we have heard the first part of Christian doctrine, in which
we have seen all that God wishes us to do or to leave undone. Now,
there properly follows the Creed, which sets forth to us everything
that we must expect and receive from God, and, to state it quite
briefly, teaches us to know Him fully. And this is intended to help us
do that which according to the Ten Commandments we ought to do. For (as
said above) they are set so high that all human ability is far too
feeble and weak to [attain to or] keep them. Therefore it is as
necessary to learn this part as the former in order that we may know
how to attain thereto, whence and whereby to obtain such power. For if
we could by our own powers keep the Ten Commandments as they are to be
kept, we would need nothing further, neither the Creed nor the Lord's
Prayer. But before we explain this advantage and necessity of the
Creed, it is sufficient at first for the simple-minded that they learn
to comprehend and understand the Creed itself.

In the first place, the Creed has hitherto been divided into twelve
articles, although, if all points which are written in the Scriptures
and which belong to the Creed were to be distinctly set forth, there
would be far more articles, nor could they all be clearly expressed in
so few words. But that it may be most easily and clearly understood as
it is to be taught to children, we shall briefly sum up the entire
Creed in three chief articles, according to the three persons in the
Godhead, to whom everything that we believe is related, So that the
First Article, of God the Father, explains Creation, the Second
Article, of the Son, Redemption, and the Third, of the Holy Ghost,
Sanctification. Just as though the Creed were briefly comprehended in
so many words: I believe in God the Father, who has created me; I
believe in God the Son, who has redeemed me; I believe in the Holy
Ghost, who sanctifies me. One God and one faith, but three persons,
therefore also three articles or confessions. Let us briefly run over
the words.

Article I.

I believe in God the Father Almighty, Maker of heaven and earth.

This portrays and sets forth most briefly what is the essence, will,
activity, and work of God the Father. For since the Ten Commandments
have taught that we are to have not more than one God, the question
might be asked, What kind of a person is God? What does He do? How can
we praise or portray and describe Him, that He may be known? Now, that
is taught in this and in the following article, so that the Creed is
nothing else than the answer and confession of Christians arranged with
respect to the First Commandment. As if you were to ask a little child:
My dear, what sort of a God have you? What do you know of Him? he could
say: This is my God: first, the Father, who has created heaven and
earth; besides this only One I regard nothing else as God; for there is
no one else who could create heaven and earth.

But for the learned, and those who are somewhat advanced [have
acquired some Scriptural knowledge], these three articles may all be
expanded and divided into as many parts as there are words. But now for
young scholars let it suffice to indicate the most necessary points,
namely, as we have said, that this article refers to the Creation: that
we emphasize the words: Creator of heaven and earth But what is the
force of this, or what do you mean by these words: I believe in God the
Father Almighty, Maker, etc.? Answer: This is what I mean and believe,
that I am a creature of God; that is, that He has given and constantly
preserves to me my body, soul, and life, members great and small, all
my senses, reason, and understanding, and so on, food and drink,
clothing and support, wife and children, domestics, house and home,
etc. Besides, He causes all creatures to serve for the uses and
necessities of life -- sun, moon and stars in the firmament, day and
night, air, fire, water, earth, and whatever it bears and produces,
birds and fishes, beasts, grain, and all kinds of produce, and whatever
else there is of bodily and temporal goods, good government, peace,
security. Thus we learn from this article that none of us has of
himself, nor can preserve, his life nor anything that is here
enumerated or can be enumerated, however small and unimportant a thing
it might be, for all is comprehended in the word Creator.

Moreover, we also confess that God the Father has not only given us all
that we have and see before our eyes, but daily preserves and defends
us against all evil and misfortune, averts all sorts of danger and
calamity; and that He does all this out of pure love and goodness,
without our merit, as a benevolent Father, who cares for us that no
evil befall us. But to speak more of this belongs in the other two
parts of this article, where we say: Father Almighty
Now, since: all that we possess, and, moreover, whatever, in addition,
is in heaven and upon the earth, is daily given, preserved, and kept
for us by God, it is readily inferred and concluded that it is our duty
to love, praise, and thank Him for it without ceasing, and, in short,
to serve Him with all these things as He demands and has enjoined in
the Ten Commandments.

Here we could say much if we were to expatiate, how few there are that
believe this article. For we all pass over it, hear it and say it, but
neither see nor consider what the words teach us. For if we believed it
with the heart, we would also act accordingly, and not stalk about
proudly, act defiantly, and boast as though we had life, riches, power,
and honor, etc., of ourselves, so that others must fear and serve us,
as is the practice of the wretched, perverse world, which is drowned in
blindness, and abuses all the good things and gifts of God only for its
own pride, avarice, lust, and luxury, and never once regards God, so
as to thank Him or acknowledge Him as Lord and Creator.

Therefore, this article ought to humble and terrify us all, if we
believed it. For we sin daily with eyes, ears, hands, body and soul,
money and possessions, and with everything we have, especially those
who even fight against the Word of God. Yet Christians have this
advantage, that they acknowledge themselves in duty bound to serve God
for all these things, and to be obedient to Him [which the world knows
not how to do].

We ought, therefore, daily to practice this article, impress it upon
our mind, and to remember it in all that meets our eyes, and in all
good that falls to our lot, and wherever we escape from calamity or
danger, that it is God who gives and does all these things, that
therein we sense and see His paternal heart and His transcendent love
toward us. Thereby the heart would be warmed and kindled to be
thankful, and to employ all such good things to the honor and praise of
God.

Thus we have most briefly presented the meaning of this article, as
much as is at first necessary for the most simple to learn, both as to
what we have and receive from God, and what we owe in return, which is
a most excellent knowledge, but a far greater treasure. For here we see
how the Father has given Himself to us, together with all creatures,
and has most richly provided for us in this life, besides that He has
overwhelmed us with unspeakable, eternal treasures by His Son and the
Holy Ghost, as we shall hear.

Article II.

And in Jesus Christ, His only Son, our Lord, who was conceived by the
Holy Ghost, born of the Virgin Mary; suffered under Pontius Pilate, was
crucified, dead, and buried; He descended into hell; the third day He
rose again from the dead; He ascended into heaven, and sitteth on the
right hand of God the Father Almighty; from thence He shall come to
judge the quick and the dead.

Here we learn to know the Second Person of the Godhead, so that we see
what we have from God over and above the temporal goods
aforementioned; namely, how He has completely poured forth Himself and
withheld nothing from us that He has not given us. Now, this article is
very rich and broad; but in order to expound it also briefly and in a
childlike way, we shall take up one word and sum up in that the entire
article, namely (as we have said), that we may here learn how we have
been redeemed; and we shall base this on these words: In Jesus Christ,
our Lord.

If now you are asked, What do you believe in the Second Article of
Jesus Christ? answer briefly: I believe that Jesus Christ, true Son of
God, has become my Lord. But what is it to become Lord? It is this,
that He has redeemed me from sin, from the devil, from death, and all
evil. For before I had no Lord nor King, but was captive under the
power of the devil, condemned to death, enmeshed in sin and blindness.

For when we had been created by God the Father, and had received from
Him all manner of good, the devil came and led us into disobedience,
sin, death, and all evil, so that we fell under His wrath and
displeasure and were doomed to eternal damnation, as we had merited and
deserved. There was no counsel, help, or comfort until this only and
eternal Son of God in His unfathomable goodness had compassion upon our
misery and wretchedness, and came from heaven to help us. Those tyrants
and jailers, then, are all expelled now, and in their place has come
Jesus Christ, Lord of life, righteousness, every blessing, and
salvation, and has delivered us poor lost men from the jaws of hell,
has won us, made us free, and brought us again into the favor and grace
of the Father, and has taken us as His own property under His shelter
and protection, that He may govern us by His righteousness, wisdom,
power, life, and blessedness.

Let this then, be the sum of this article that the little word Lord
signifies simply as much as Redeemer, i.e., He who has brought us from
Satan to God, from death to life, from sin to righteousness, and who
preserves us in the same. But all the points which follow in order in
this article serve no other end than to explain and express this
redemption, how and whereby it was accomplished, that is, how much it
cost Him, and what He spent and risked that He might win us and bring
us under His dominion, namely, that He became man, conceived and born
without [any stain of] sin, of the Holy Ghost and of the Virgin Mary,
that He might overcome sin; moreover, that He suffered, died and was
buried, that He might make satisfaction for me and pay what I owe, not
with silver nor gold, but with His own precious blood. And all this, in
order to become my Lord; for He did none of these for Himself, nor had
He any need of it. And after that He rose again from the dead,
swallowed up and devoured death, and finally ascended into heaven and
assumed the government at the Father's right hand, so that the devil
and all powers must be subject to Him and lie at His feet, until
finally, at the last day, He will completely part and separate us from
the wicked world, the devil, death, sin, etc.

But to explain all these single points separately belongs not to brief
sermons for children, but rather to the ampler sermons that extend
throughout the entire year, especially at those times which are
appointed for the purpose of treating at length of each article -- of
the birth, sufferings, resurrection, ascension of Christ, etc.

Ay, the entire Gospel which we preach is based on this, that we
properly understand this article as that upon which our salvation and
all our happiness rest, and which is so rich and comprehensive that we
never can learn it fully.

Article III.

I believe in the Holy Ghost; the holy Christian Church, the communion
of saints; the forgiveness of sins; the resurrection of the body; and
the life everlasting. Amen.

This article (as I have said) I cannot relate better than to
Sanctification, that through the same the Holy Ghost, with His office,
is declared and depicted, namely, that He makes holy. Therefore we must
take our stand upon the word Holy Ghost, because it is so precise and
comprehensive that we cannot find another. For there are, besides, many
kinds of spirits mentioned in the Holy Scriptures, as, the spirit of
man, heavenly spirits, and evil spirits. But the Spirit of God alone is
called Holy Ghost, that is, He who has sanctified and still sanctifies
us. For as the Father is called Creator, the Son Redeemer, so the Holy
Ghost, from His work, must be called Sanctifier, or One that makes
holy. But how is such sanctifying done? Answer: Just as the Son obtains
dominion, whereby He wins us, through His birth, death, resurrection,
etc., so also the Holy Ghost effects our sanctification by the
following parts, namely, by the communion of saints or the Christian
Church, the forgiveness of sins, the resurrection of the body, and the
life everlasting; that is, He first leads us into His holy
congregation, and places us in the bosom of the Church, whereby He
preaches to us and brings us to Christ.

For neither you nor I could ever know anything of Christ, or believe on
Him, and obtain Him for our Lord, unless it were offered to us and
granted to our hearts by the Holy Ghost through the preaching of the
Gospel. The work is done and accomplished; for Christ has acquired and
gained the treasure for us by His suffering, death, resurrection, etc.
But if the work remained concealed so that no one knew of it, then it
would be in vain and lost. That this treasure, therefore, might not lie
buried, but be appropriated and enjoyed, God has caused the Word to go
forth and be proclaimed, in which He gives the Holy Ghost to bring this
treasure home and appropriate it to us. Therefore sanctifying is
nothing else than bringing us to Christ to receive this good, to which
we could not attain of ourselves.

Learn, then, to understand this article most clearly. If you are
asked: What do you mean by the words: I believe in the Holy Ghost? you
can answer: I believe that the Holy Ghost makes me holy, as His name
implies. But whereby does He accomplish this, or what are His method
and means to this end? Answer: By the Christian Church, the forgiveness
of sins, the resurrection of the body, and the life everlasting. For,
in the first place, He has a peculiar congregation in the world, which
is the mother that begets and bears every Christian through the Word of
God, which He reveals and preaches, [and through which] He illumines
and enkindles hearts, that they understand, accept it, cling to it, and
persevere in it.

For where He does not cause it to be preached and made alive in the
heart, so that it is understood, it is lost, as was the case under the
Papacy, where faith was entirely put under the bench, and no one
recognized Christ as his Lord or the Holy Ghost as his Sanctifier, that
is, no one believed that Christ is our Lord in the sense that He has
acquired this treasure for us, without our works and merit, and made us
acceptable to the Father. What, then, was lacking? This, that the Holy
Ghost was not there to reveal it and cause it to be preached; but men
and evil spirits were there, who taught us to obtain grace and be saved
by our works. Therefore it is not a Christian Church either; for where
Christ is not preached, there is no Holy Ghost who creates, calls, and
gathers the Christian Church, without which no one can come to Christ
the Lord. Let this suffice concerning the sum of this article. But
because the parts which are here enumerated are not quite clear to the
simple, we shall run over them also.

The Creed denominates the holy Christian Church, communionem
sanctorum, a communion of saints; for both expressions, taken
together, are identical. But formerly the one [the second] expression
was not there, and it has been poorly and unintelligibly translated
into German eine Gemeinschaft der Heiligen, a communion of saints. If
it is to be rendered plainly, it must be expressed quite differently in
the German idiom; for the word ecclesia properly means in German eine
Versammlung, an assembly. But we are accustomed to the word church, by
which the simple do not understand an assembled multitude, but the
consecrated house or building, although the house ought not to be
called a church, except only for the reason that the multitude
assembles there. For we who assemble there make and choose for
ourselves a particular place, and give a name to the house according to
the assembly.

Thus the word Kirche (church) means really nothing else than a common
assembly and is not German by idiom, but Greek (as is also the word
ecclesia); for in their own language they call it kyria, as in Latin it
is called curia. Therefore in genuine German, in our mother-tongue, it
ought to be called a Christian congregation or assembly (eine
christliche Gemeinde oder Sammlung), or, best of all and most clearly,
holy Christendom (eine heilige Christenheit).

So also the word communio, which is added, ought not to be rendered
communion (Gemeinschaft), but congregation (Gemeinde). And it is
nothing else than an interpretation or explanation by which some one
meant to explain what the Christian Church is. This our people, who
understood neither Latin nor German, have rendered Gemeinschaft der
Heiligen (communion of saints), although no German language speaks
thus, nor understands it thus. But to speak correct German, it ought to
be eine Gemeinde der Heiligen (a congregation of saints), that is, a
congregation made up purely of saints, or, to speak yet more plainly,
eine heilige Gemeinde, a holy congregation. I say this in order that
the words Gemeinschaft der Heiligen (communion of saints) may be
understood, because the expression has become so established by custom
that it cannot well be eradicated, and it is treated almost as heresy
if one should attempt to change a word.

But this is the meaning and substance of this addition: I believe that
there is upon earth a little holy group and congregation of pure
saints, under one head, even Christ, called together by the Holy Ghost
in one faith, one mind, and understanding, with manifold gifts, yet
agreeing in love, without sects or schisms. I am also a part and member
of the same a sharer and joint owner of all the goods it possesses,
brought to it and incorporated into it by the Holy Ghost by having
heard and continuing to hear the Word of God, which is the beginning of
entering it. For formerly, before we had attained to this, we were
altogether of the devil, knowing nothing of God and of Christ. Thus,
until the last day, the Holy Ghost abides with the holy congregation or
Christendom, by means of which He fetches us to Christ and which He
employs to teach and preach to us the Word, whereby He works and
promotes sanctification, causing it [this community] daily to grow and
become strong in the faith and its fruits which He produces.

We further believe that in this Christian Church we have forgiveness of
sin, which is wrought through the holy Sacraments and Absolution,
moreover, through all manner of consolatory promises of the entire
Gospel. Therefore, whatever is to be preached concerning the Sacraments
belongs here, and, in short, the whole Gospel and all the offices of
Christianity, which also must be preached and taught without ceasing.
For although the grace of God is secured through Christ, and
sanctification is wrought by the Holy Ghost through the Word of God in
the unity of the Christian Church, yet on account of our flesh which we
bear about with us we are never without sin.

Everything, therefore, in the Christian Church is ordered to the end
that we shall daily obtain there nothing but the forgiveness of sin
through the Word and signs, to comfort and encourage our consciences as
long as we live here. Thus, although we have sins, the [grace of the]
Holy Ghost does not allow them to injure us, because we are in the
Christian Church, where there is nothing but [continuous,
uninterrupted] forgiveness of sin, both in that God forgives us, and in
that we forgive, bear with, and help each other.

But outside of this Christian Church, where the Gospel is not, there is
no forgiveness, as also there can be no holiness [sanctification].
Therefore all who seek and wish to merit holiness [sanctification], not
through the Gospel and forgiveness of sin, but by their works, have
expelled and severed themselves [from this Church].

Meanwhile, however, while sanctification has begun and is growing
daily, we expect that our flesh will be destroyed and buried with all
its uncleanness, and will come forth gloriously, and arise to entire
and perfect holiness in a new eternal life. For now we are only half
pure and holy, so that the Holy Ghost has ever [some reason why] to
continue His work in us through the Word, and daily to dispense
forgiveness, until we attain to that life where there will be no more
forgiveness, but only perfectly pure and holy people, full of godliness
and righteousness, removed and free from sin, death, and all evil, in a
new, immortal, and glorified body.

Behold, all this is to be the office and work of the Holy Ghost, that
He begin and daily increase holiness upon earth by means of these two
things, the Christian Church and the forgiveness of sin. But in our
dissolution He will accomplish it altogether in an instant, and will
forever preserve us therein by the last two parts.

But the term Auferstehung des Fleisches (resurrection of the flesh)
here employed is not according to good German idiom. For when we
Germans hear the word Fleisch (flesh), we think no farther than of the
shambles. But in good German idiom we would say Auferstehung des
Leibes, or Leichnams (resurrection of the body). However, it is not a
matter of much moment, if we only understand the words aright.

This, now, is the article which must ever be and remain in operation.
For creation we have received; redemption, too, is finished. But the
Holy Ghost carries on His work without ceasing to the last day. And for
that purpose He has appointed a congregation upon earth by which He
speaks and does everything. For He has not yet brought together all His
Christian Church nor dispensed forgiveness. Therefore we believe in Him
who through the Word daily brings us into the fellowship of this
Christian Church, and through the same Word and the forgiveness of sins
bestows, increases, and strengthens faith in order that when He has
accomplished it all, and we abide therein, and die to the world and to
all evil, He may finally make us perfectly and forever holy; which now
we expect in faith through the Word.

Behold, here you have the entire divine essence, will, and work
depicted most exquisitely in quite short and yet rich words wherein
consists all our wisdom, which surpasses and exceeds the wisdom, mind,
and reason of all men. For although the whole world with all diligence
has endeavored to ascertain what God is, what He has in mind and does,
yet has she never been able to attain to [the knowledge and
understanding of] any of these things. But here we have everything in
richest measure; for here in all three articles He has Himself revealed
and opened the deepest abyss of his paternal heart and of His pure
unutterable love. For He has created us for this very object, that He
might redeem and sanctify us; and in addition to giving and imparting
to us everything in heaven and upon earth, He has given to us even His
Son and the Holy Ghost, by whom to bring us to Himself. For (as
explained above) we could never attain to the knowledge of the grace
and favor of the Father except through the Lord Christ, who is a mirror
of the paternal heart, outside of whom we see nothing but an angry and
terrible Judge. But of Christ we could know nothing either, unless it
had been revealed by the Holy Ghost.

These articles of the Creed, therefore, divide and separate us
Christians from all other people upon earth. For all outside of
Christianity, whether heathen, Turks, Jews, or false Christians and
hypocrites, although they believe in, and worship, only one true God,
yet know not what His mind towards them is, and cannot expect any love
or blessing from Him; therefore they abide in eternal wrath and
damnation. For they have not the Lord Christ, and, besides, are not
illumined and favored by any gifts of the Holy Ghost.

From this you perceive that the Creed is a doctrine quite different
from the Ten Commandments; for the latter teaches indeed what we ought
to do, but the former tells what God does for us and gives to us.
Moreover, apart from this, the Ten Commandments are written in the
hearts of all men; the Creed, however, no human wisdom can comprehend,
but it must be taught by the Holy Ghost alone. The latter doctrine [of
the Law], therefore makes no Christian, for the wrath and displeasure
of God abide upon us still, because we cannot keep what God demands of
us; but this [namely, the doctrine of faith] brings pure grace, and
makes us godly and acceptable to God. For by this knowledge we obtain
love and delight in all the commandments of God, because here we see
that God gives Himself entire to us, with all that He has and is able
to do, to aid and direct us in keeping the Ten Commandments -- the
Father, all creatures; the Son, His entire work; and the Holy Ghost,
all His gifts.

Let this suffice concerning the Creed to lay a foundation for the
simple, that they may not be burdened, so that, if they understand the
substance of it, they themselves may afterwards strive to acquire more,
and to refer to these parts whatever they learn in the Scriptures, and
may ever grow and increase in richer understanding. For as long as we
live here, we shall daily have enough to do to preach and to learn
this.

Part Third. OF PRAYER.

 The Lord's Prayer.

We have now heard what we must do and believe, in which things the best
and happiest life consists. Now follows the third part, how we ought to
pray. For since we are so situated that no man can perfectly keep the
Ten Commandments, even though he have begun to believe, and since the
devil with all his power together with the world and our own flesh,
resists our endeavors, nothing is so necessary as that we should
continually resort to the ear of God, call upon Him, and pray to Him,
that He would give, preserve, and increase in us faith and the
fulfillment of the Ten Commandments, and that He would remove
everything that is in our way and opposes us therein. But that we might
know what and how to pray, our Lord Christ has Himself taught us both
the mode and the words, as we shall see.

But before we explain the Lord's Prayer part by part, it is most
necessary first to exhort and incite people to prayer, as Christ and
the apostles also have done. And the first matter is to know that it is
our duty to pray because of God's commandment. For thus we heard in the
Second Commandment: Thou shalt not take the name of the lord, thy God,
in vain, that we are there required to praise that holy name, and call
upon it in every need, or to pray. For to call upon the name of God is
nothing else than to pray. Prayer is therefore as strictly and
earnestly commanded as all other commandments: to have no other God,
not to kill, not to steal, etc. Let no one think that it is all the
same whether he pray or not, as vulgar people do, who grope in such
delusion and ask Why should I pray? Who knows whether God heeds or will
hear my prayer? If I do not pray, some one else will. And thus they
fall into the habit of never praying, and frame a pretext, as though we
taught that there is no duty or need of prayer, because we reject false
and hypocritical prayers.

But this is true indeed that such prayers as have been offered
hitherto when men were babbling and bawling in the churches were no
prayers. For such external matters, when they are properly observed,
may be a good exercise for young children, scholars, and simple
persons, and may be called singing or reading, but not really praying.
But praying, as the Second Commandment teaches, is to call upon God in
every need. This He requires of us, and has not left it to our choice.
But it is our duty and obligation to pray if we would be Christians, as
much as it is our duty and obligation to obey our parents and the
government; for by calling upon it and praying the name of God is
honored and profitably employed. This you must note above all things,
that thereby you may silence and repel such thoughts as would keep and
deter us from prayer. For just as it would be idle for a son to say to
his father, "Of what advantage is my obedience? I will go and do what
I can; it is all the same"; but there stands the commandment, Thou
shalt and must do it, so also here it is not left to my will to do it
or leave it undone, but prayer shall and must be offered at the risk of
God's wrath and displeasure.

This is therefore to be understood and noted before everything else, in
order that thereby we may silence and repel the thoughts which would
keep and deter us from praying, as though it were not of much
consequence if we do not pray, or as though it were commanded those who
are holier and in better favor with God than we; as, indeed, the human
heart is by nature so despondent that it always flees from God and
imagines that He does not wish or desire our prayer, because we are
sinners and have merited nothing but wrath. Against such thoughts (I
say) we should regard this commandment and turn to God, that we may not
by such disobedience excite His anger still more. For by this
commandment He gives us plainly to understand that He will not cast us
from Him nor chase us away, although we are sinners, but rather draw
us to Himself, so that we might humble ourselves before Him, bewail
this misery and plight of ours, and pray for grace and help. Therefore
we read in the Scriptures that He is angry also with those who were
smitten for their sin, because they did not return to Him and by their
prayers assuage His wrath and seek His grace.

Now, from the fact that it is so solemnly commanded to pray, you are to
conclude and think, that no one should by any means despise his prayer,
but rather set great store by it, and always seek an illustration from
the other commandments. A child should by no means despise his
obedience to father and mother, but should always think: This work is a
work of obedience, and what I do I do with no other intention than that
I may walk in the obedience and commandment of God, on which I can
settle and stand firm, and esteem it a great thing, not on account of
my worthiness, but on account of the commandment. So here also, what
and for what we pray we should regard as demanded by God and done in
obedience to Him, and should reflect thus: On my account it would
amount to nothing; but it shall avail, for the reason that God has
commanded it. Therefore everybody, no matter what he has to say in
prayer, should always come before God in obedience to this commandment.


We pray, therefore, and exhort every one most diligently to take this
to heart and by no means to despise our prayer. For hitherto it has
been taught thus in the devil's name that no one regarded these things,
and men supposed it to be sufficient to have done the work, whether God
would hear it or not. But that is staking prayer on a risk, and
murmuring it at a venture, and therefore it is a lost prayer. For we
allow such thoughts as these to lead us astray and deter us: I am not
holy or worthy enough; if I were as godly and holy as St. Peter or St.
Paul, then I would pray. But put such thoughts far away, for just the
same commandment which applied to St. Paul applies also to me; and the
Second Commandment is given as much on my account as on his account, so
that he can boast of no better or holier commandment.

Therefore you should say: My prayer is as precious, holy, and pleasing
to God as that of St. Paul or of the most holy saints. This is the
reason: For I will gladly grant that he is holier in his person, but
not on account of the commandment; since God does not regard prayer on
account of the person, but on account of His word and obedience
thereto. For on the commandment on which all the saints rest their
prayer I, too, rest mine. Moreover I pray for the same thing for which
they all pray and ever have prayed; besides, I have just as great a
need of it as those great saints, yea, even a greater one than they.

Let this be the first and most important point, that all our prayers
must be based and rest upon obedience to God, irrespective of our
person, whether we be sinners or saints, worthy or unworthy. And we
must know that God will not have it treated as a jest, but be angry,
and punish all who do not pray, as surely as He punishes all other
disobedience; next, that He will not suffer our prayers to be in vain
or lost. For if He did not intend to answer your prayer, He would not
bid you pray and add such a severe commandment to it.

In the second place, we should be the more urged and incited to pray
because God has also added a promise, and declared that it shall surely
be done to us as we pray, as He says Ps. 50, 15: Call upon Me in the
day of trouble: I will deliver thee. And Christ in the Gospel of St.
Matthew, 7, 7: Ask, and it shall be given you. For every one that
asketh receiveth. Such promises ought certainly to encourage and kindle
our hearts to pray with pleasure and delight, since He testifies with
His [own] word that our prayer is heartily pleasing to Him, moreover,
that it shall assuredly be heard and granted, in order that we may not
despise it or think lightly of it, and pray at a venture.

This you can hold up to Him and say: Here I come, dear Father, and
pray, not of my own purpose nor upon my own worthiness, but at Thy
commandment and promise, which cannot fail or deceive me. Whoever,
therefore, does not believe this promise must know again that he
excites God to anger as a person who most highly dishonors Him and
reproaches Him with falsehood.

Besides this, we should be incited and drawn to prayer because in
addition to this commandment and promise God anticipates us, and
Himself arranges the words and form of prayer for us, and places them
upon our lips as to how and what we should pray, that we may see how
heartily He pities us in our distress, and may never doubt that such
prayer is pleasing to Him and shall certainly be answered; which [the
Lord's Prayer] is a great advantage indeed over all other prayers that
we might compose ourselves. For in them the conscience would ever be in
doubt and say: I have prayed, but who knows how it pleases Him, or
whether I have hit upon the right proportions and form? Hence there is
no nobler prayer to be found upon earth than the Lord's Prayer which we
daily pray because it has this excellent testimony, that God loves to
hear it, which we ought not to surrender for all the riches of the
world.

And it has been prescribed also for this reason that we should see and
consider the distress which ought to urge and compel us to pray without
ceasing. For whoever would pray must have something to present, state,
and name which he desires; if not, it cannot be called a prayer.

Therefore we have rightly rejected the prayers of monks and priests,
who howl and growl day and night like fiends; but none of them think of
praying for a hair's breadth of anything. And if we would assemble all
the churches, together with all ecclesiastics, they would be obliged to
confess that they have never from the heart prayed for even a drop of
wine. For none of them has ever purposed to pray from obedience to God
and faith in His promise, nor has any one regarded any distress, but
(when they had done their best) they thought no further than this, to
do a good work, whereby they might repay God, as being unwilling to
take anything from Him, but wishing only to give Him something.

But where there is to be a true prayer there must be earnestness. Men
must feel their distress, and such distress as presses them and compels
them to call and cry out then prayer will be made spontaneously, as it
ought to be, and men will require no teaching how to prepare for it and
to attain to the proper devotion. But the distress which ought to
concern us most, both as regards ourselves and every one, you will find
abundantly set forth in the Lord's Prayer. Therefore it is to serve
also to remind us of the same, that we contemplate it and lay it to
heart, lest we become remiss in prayer. For we all have enough that we
lack, but the great want is that we do not feel nor see it. Therefore
God also requires that you lament and plead such necessities and wants,
not because He does not know them, but that you may kindle your heart
to stronger and greater desires, and make wide and open your cloak to
receive much.

Therefore, every one of us should accustom himself from his youth
daily to pray for all his wants, whenever he is sensible of anything
affecting his interests or that of other people among whom he may live,
as for preachers, the government, neighbors, domestics, and always (as
we have said) to hold up to God His commandment and promise, knowing
that He will not have them disregarded. This I say because I would like
to see these things brought home again to the people that they might
learn to pray truly, and not go about coldly and indifferently, whereby
they become daily more unfit for prayer; which is just what the devil
desires, and for what he works with all his powers. For he is well
aware what damage and harm it does him when prayer is in proper
practice. For this we must know, that all our shelter and protection
rest in prayer alone. For we are far too feeble to cope with the devil
and all his power and adherents that set themselves against us, and
they might easily crush us under their feet. Therefore we must consider
and take up those weapons with which Christians must be armed in order
to stand against the devil. For what do you think has hitherto
accomplished such great things, has checked or quelled the counsels,
purposes, murder, and riot of our enemies, whereby the devil thought to
crush us, together with the Gospel, except that the prayer of a few
godly men intervened like a wall of iron on our side? They should else
have witnessed a far different tragedy, namely, how the devil would
have destroyed all Germany in its own blood. But now they may
confidently deride it and make a mock of it, however, we shall
nevertheless be a match both for themselves and the devil by prayer
alone, if we only persevere diligently and not become slack. For
whenever a godly Christian prays: Dear Father let Thy will be done, God
speaks from on high and says: Yes, dear child, it shall be so, in spite
of the devil and all the world.

Let this be said as an exhortation, that men may learn, first of all,
to esteem prayer as something great and precious, and to make a proper
distinction between babbling and praying for something. For we by no
means reject prayer, but the bare, useless howling and murmuring we
reject, as Christ Himself also rejects and prohibits long palavers. Now
we shall most briefly and clearly treat of the Lord's Prayer. Here
there is comprehended in seven successive articles, or petitions, every
need which never ceases to relate to us, and each so great that it
ought to constrain us to keep praying it all our lives.

 The First Petition.

Hallowed be Thy name.

This is, indeed, somewhat obscure, and not expressed in good German,
for in our mother-tongue we would say: Heavenly Father, help that by
all means Thy name may be holy. But what is it to pray that His name
may be holy? Is it not holy already? Answer: Yes, it is always holy in
its nature, but in our use it is not holy. For God's name was given us
when we became Christians and were baptized, so that we are called
children of God and have the Sacraments by which He so incorporates us
in Himself that everything which is God's must serve for our use.

Here now the great need exists for which we ought to be most
concerned, that this name have its proper honor, be esteemed holy and
sublime as the greatest treasure and sanctuary that we have; and that
as godly children we pray that the name of God, which is already holy
in heaven, may also be and remain holy with us upon earth and in all
the world.

But how does it become holy among us? Answer, as plainly as it can be
said: When both our doctrine and life are godly and Christian. For
since in this prayer we call God our Father, it is our duty always to
deport and demean ourselves as godly children, that He may not receive
shame, but honor and praise from us.

Now the name of God is profaned by us either in words or in works. (For
whatever we do upon the earth must be either words or works, speech or
act.) In the first place, then, it is profaned when men preach, teach,
and speak in the name of God what is false and misleading, so that His
name must serve to adorn and to find a market for falsehood. That is,
indeed, the greatest profanation and dishonor of the divine name.
Furthermore, also when men, by swearing, cursing, conjuring, etc.,
grossly abuse the holy name as a cloak for their shame. In the second
place also by an openly wicked life and works, when those who are
called Christians and the people of God are adulterers, drunkards,
misers, envious, and slanderers. Here again must the name of God come
to shame and be profaned because of us. For just as it is a shame and
disgrace to a natural father to have a bad perverse child that opposes
him in words and deeds, so that on its account he suffers contempt and
reproach, so also it brings dishonor upon God if we who are called by
His name and have all manner of goods from Him teach, speak, and live
in any other manner except as godly and heavenly children, so that
people say of us that we must be not God's, but the devil's children.

Thus you see that in this petition we pray just for that which God
demands in the Second Commandment; namely, that His name be not taken
in vain to swear, curse, lie, deceive, etc., but be usefully employed
to the praise and honor of God. For whoever employs the name of God for
any sort of wrong profanes and desecrates this holy name, as aforetime
a church was considered desecrated when a murder or any other crime had
been committed in it, or when a pyx or relic was desecrated, as being
holy in themselves, yet become unholy in use. Thus this point is easy
and clear if only the language is understood, that to hallow is the
same as in our idiom to praise, magnify, and honor both in word and
deed.

Here, now, learn how great need there is of such prayer. For because we
see how full the world is of sects and false teachers, who all wear the
holy name as a cover and sham for their doctrines of devils, we ought
by all means to pray without ceasing, and to cry and call upon God
against all such as preach and believe falsely and whatever opposes and
persecutes our Gospel and pure doctrine, and would suppress it, as
bishops, tyrants, enthusiasts, etc. Likewise also for ourselves who
have the Word of God, but are not thankful for it, nor live as we
ought according to the same. If now you pray for this with your heart,
you can be sure that it pleases God; for He will not hear anything more
dear to Him than that His honor and praise is exalted above everything
else, and His Word is taught in its purity and is esteemed precious and
dear.

The Second Petition.

 Thy kingdom come.

As we prayed in the First Petition concerning the honor and name of God
that He would prevent the world from adorning its lies and wickedness
with it, but cause it to be esteemed sublime and holy both in doctrine
and life, so that He may be praised and magnified in us, so here we
pray that His kingdom also may come. But just as the name of God is in
itself holy, and we pray nevertheless that it be holy among us, so also
His kingdom comes of itself, without our prayer, yet we pray
nevertheless that it may come to us, that is, prevail among us and with
us, so that we may be a part of those among whom His name is hallowed
and His kingdom prospers.

But what is the kingdom of God? Answer: Nothing else than what we
learned in the Creed, that God sent His Son Jesus Christ our Lord, into
the world to redeem and deliver us from the power of the devil, and to
bring us to Himself, and to govern us as a King of righteousness, life
and salvation against sin death, and an evil conscience, for which end
He has also bestowed His Holy Ghost, who is to bring these things home
to us by His holy Word, and to illumine and strengthen us in the faith
by His power.

Therefore we pray here in the first place that this may become
effective with us, and that His name be so praised through the holy
Word of God and a Christian life that both we who have accepted it may
abide and daily grow therein, and that it may gain approbation and
adherence among other people and proceed with power throughout the
world, that many may find entrance into the Kingdom of Grace, be made
partakers of redemption, being led thereto by the Holy Ghost, in order
that thus we may all together remain forever in the one kingdom now
begun.

For the coming of God's Kingdom to us occurs in two ways; first, here
in time through the Word and faith; and secondly, in eternity forever
through revelation. Now we pray for both these things, that it may come
to those who are not yet in it, and, by daily increase, to us who have
received the same, and hereafter in eternal life. All this is nothing
else than saying: Dear Father, we pray, give us first Thy Word, that
the Gospel be preached properly throughout the world; and secondly,
that it be received in faith, and work and live in us, so that through
the Word and the power of the Holy Ghost Thy kingdom may prevail among
us, and the kingdom of the devil be put down, that he may have no right
or power over us, until at last it shall be utterly destroyed, and sin,
death, and hell shall be exterminated, that we may live forever in
perfect righteousness and blessedness.

From this you perceive that we pray here not for a crust of bread or a
temporal, perishable good, but for an eternal inestimable treasure and
everything that God Himself possesses; which is far too great for any
human heart to think of desiring if He had not Himself commanded us to
pray for the same. But because He is God, He also claims the honor of
giving much more and more abundantly than any one can comprehend, --
like an eternal, unfailing fountain, which, the more it pours forth and
overflows, the more it continues to give, -- and He desires nothing
more earnestly of us than that we ask much and great things of Him, and
again is angry if we do not ask and pray confidently.

For just as when the richest and most mighty emperor would bid a poor
beggar ask whatever he might desire, and were ready to give great
imperial presents, and the fool would beg only for a dish of gruel, he
would be rightly considered a rogue and a scoundrel who treated the
command of his imperial majesty as a jest and sport, and was not worthy
of coming into his presence: so also it is a great reproach and
dishonor to God if we, to whom He offers and pledges so many
unspeakable treasures, despise the same, or have not the confidence to
receive them, but scarcely venture to pray for a piece of bread.

All this is the fault of the shameful unbelief which does not look to
God for as much good as will satisfy the stomach, much less expects
without doubt such eternal treasures of God. Therefore we must
strengthen ourselves against it, and let this be our first prayer;
then, indeed, we shall have all else in abundance, as Christ teaches
[Matt. 6, 33]: Seek ye first the kingdom of God and His righteousness
and all these things shall be added unto you. For how could He allow us
to suffer want and to be straitened in temporal things when He promises
that which is eternal and imperishable?

 The Third Petition.

Thy will be done on earth as it is in heaven.

Thus far we have prayed that God's name be honored by us, and that His
kingdom prevail among us; in which two points is comprehended all that
pertains to the honor of God and to our salvation, that we receive as
our own God and all His riches. But now a need just as great arises,
namely, that we firmly keep them, and do not suffer ourselves to be
torn therefrom. For as in a good government it is not only necessary
that there be those who build and govern well, but also those who make
defense, afford protection and maintain it firmly, so here likewise,
although we have prayed for the greatest need, for the Gospel, faith,
and the Holy Ghost, that He may govern us and redeem us from the power
of the devil, we must also pray that His will be done. For there will
be happenings quite strange if we are to abide therein, as we shall
have to suffer many thrusts and blows on that account from everything
that ventures to oppose and prevent the fulfillment of the two
petitions that precede.

For no one believes how the devil opposes and resists them, and cannot
suffer that any one teach or believe aright. And it hurts him beyond
measure to suffer his lies and abominations, that have been honored
under the most specious pretexts of the divine name, to be exposed, and
to be disgraced himself, and, besides, be driven out of the heart, and
suffer such a breach to be made in his kingdom. Therefore he chafes and
rages as a fierce enemy with all his power and might, and marshals all
his subjects, and, in addition enlists the world and our own flesh as
his allies. For our flesh is in itself indolent and inclined to evil,
even though we have accepted and believe the Word of God. The world,
however, is perverse and wicked; this he incites against us, fans and
stirs the fire, that he may hinder and drive us back, cause us to
fall, and again bring us under his power. Such is all his will, mind,
and thought, for which he strives day and night, and never rests a
moment, employing all arts, wiles, ways, and means whichever he can
invent.

If we would be Christians, therefore, we must surely expect and reckon
upon having the devil with all his angels and the world as our enemies,
who will bring every possible misfortune and grief upon us. For where
the Word of God is preached, accepted, or believed, and produces fruit,
there the holy cross cannot be wanting. And let no one think that he
shall have peace; but he must risk what whatever he has upon earth --
possessions, honor. house and estate, wife and children, body and life.
Now, this hurts our flesh and the old Adam; for the test is to be
steadfast and to suffer with patience in whatever way we are assailed,
and to let go whatever is taken from us.

Hence there is just as great need, as in all the others, that we pray
without ceasing: "Dear Father, Thy will be done, not the will of the
devil and of our enemies, nor of anything that would persecute and
suppress Thy holy Word or hinder Thy kingdom; and grant that we may
bear with patience and overcome whatever is to be endured on that
account, lest our poor flesh yield or fall away from weakness or
sluggishness."

Behold, thus we have in these three petitions, in the simplest manner,
the need which relates to God Himself, yet all for our sakes. For
whatever we pray concerns only us, namely, as we have said, that what
must be done anyway without us, may also be done in us. For as His name
must be hallowed and His kingdom come without our prayer, so also His
will must be done and succeed although the devil with all his adherents
raise a great tumult, are angry and rage against it, and undertake to
exterminate the Gospel utterly. But for our own sakes we must pray that
even against their fury His will be done without hindrance also among
us, that they may not be able to accomplish anything and we remain firm
against all violence and persecution, and submit to such will of God.

Such prayer, then, is to be our protection and defense now, is to
repel and put down all that the devil, Pope, bishops, tyrants, and
heretics can do against our Gospel. Let them all rage and attempt their
utmost, and deliberate and resolve how they may suppress and
exterminate us, that their will and counsel may prevail: over and
against this one or two Christians with this petition alone shall be
our wall against which they shall run and dash themselves to pieces.
This consolation and confidence we have, that the will and purpose of
the devil and of all our enemies shall and must fail and come to
naught, however proud, secure, and powerful they know themselves to be.
For if their will were not broken and hindered, the kingdom of God
could not abide on earth nor His name be hallowed.

 The Fourth Petition.

Give us this day our daily bread.

Here, now, we consider the poor breadbasket, the necessaries of our
body and of the temporal life. It is a brief and simple word, but it
has a very wide scope. For when you mention and pray for daily bread,
you pray for everything that is necessary in order to have and enjoy
daily bread and, on the other hand, against everything which interferes
with it. Therefore you must open wide and extend your thoughts not only
to the oven or the flour-bin but to the distant field and the entire
land, which bears and brings to us daily bread and every sort of
sustenance. For if God did not cause it to grow, and bless and preserve
it in the field, we could never take bread from the oven or have any to
set upon the table.

To comprise it briefly, this petition includes everything that belongs
to our entire life in the world, because on that account alone do we
need daily bread. Now for our life it is not only necessary that our
body have food and covering and other necessaries, but also that we
spend our days in peace and quiet among the people with whom we live
and have intercourse in daily business and conversation and all sorts
of doings, in short, whatever pertains both to the domestic and to the
neighborly or civil relation and government. For where these two things
are hindered [intercepted and disturbed] that they do not prosper as
they ought, the necessaries of life also are impeded, so that
ultimately life cannot be maintained. And there is, indeed, the
greatest need to pray for temporal authority and government, as that by
which most of all God preserves to us our daily bread and all the
comforts of this life. For though we have received of God all good
things in abundance we are not able to retain any of them or use them
in security and happiness, if He did not give us a permanent and
peaceful government. For where there are dissension, strife, and war,
there the daily bread is already taken away, or at least checked.

Therefore it would be very proper to place in the coat-of-arms of
every pious prince a loaf of bread instead of a lion, or a wreath of
rue, or to stamp it upon the coin, to remind both them and their
subjects that by their office we have protection and peace, and that
without them we could not eat and retain our daily bread. Therefore
they are also worthy of all honor, that we give to them for their
office what we ought and can, as to those through whom we enjoy in
peace and quietness what we have, because otherwise we would not keep a
farthing; and that, in addition, we also pray for them that through
them God may bestow on us the more blessing and good.

Let this be a very brief explanation and sketch, showing how far this
petition extends through all conditions on earth. Of this any one might
indeed make a long prayer, and with many words enumerate all the things
that are included therein, as that we pray God to give us food and
drink, clothing, house, and home, and health of body; also that He
cause the grain and fruits of the field to grow and mature well;
furthermore, that He help us at home towards good housekeeping, that He
give and preserve to us a godly wife, children, and servants, that He
cause our work, trade, or whatever we are engaged in to prosper and
succeed, favor us with faithful neighbors and good friends, etc.
Likewise, that He give to emperors, kings, and all estates, and
especially to the rulers of our country and to all counselors,
magistrates, and officers, wisdom, strength, and success that they may
govern well and vanquish the Turks and all enemies; to subjects and the
common people, obedience, peace, and harmony in their life with one
another, and on the other hand, that He would preserve us from all
sorts of calamity to body and livelihood, as lightning, hail, fire,
flood, poison, pestilence, cattle-plague, war and bloodshed, famine,
destructive beasts, wicked men, etc. All this it is well to impress
upon the simple, namely, that these things come from God, and must be
prayed for by us.

But this petition is especially directed also against our chief enemy,
the devil. For all his thought and desire is to deprive us of all that
we have from God, or to hinder it; and he is not satisfied to obstruct
and destroy spiritual government in leading souls astray by his lies
and bringing them under his power, but he also prevents and hinders the
stability of all government and honorable, peaceable relations on
earth.
There he causes so much contention, murder, sedition, and war also
lightning and hail to destroy grain and cattle, to poison the air, etc.
In short, he is sorry that any one has a morsel of bread from God and
eats it in peace; and if it were in his power, and our prayer (next to
God) did not prevent him, we would not keep a straw in the field, a
farthing in the house, yea, not even our life for an hour, especially
those who have the Word of God and would like to be Christians.

Behold, thus God wishes to indicate to us how He cares for us in all
our need, and faithfully provides also for our temporal support. and
although He abundantly grants and preserves these things even to the
wicked and knaves, yet He wishes that we pray for them, in order that
we may recognize that we receive them from His hand, and may feel His
paternal goodness toward us therein. For when He withdraws His hand,
nothing can prosper nor be maintained in the end, as, indeed, we daily
see and experience. How much trouble there is now in the world only on
account of bad coin, yea, on account of daily oppression and raising of
prices in common trade, bargaining and labor on the part of those who
wantonly oppress the poor and deprive them of their daily bread! This
we must suffer indeed; but let them take care that they do not lose the
common intercession, and beware lest this petition in the Lord's Prayer
be against them.

 The Fifth Petition.

And forgive us our trespasses, as we forgive those who trespass
against us.

This part now relates to our poor miserable life, which, although we
have and believe the Word of God, and do and submit to His will, and
are supported by His gifts and blessings is nevertheless not without
sin. For we still stumble daily and transgress because we live in the
world among men who do us much harm and give us cause for impatience,
anger, revenge, etc. Besides, we have Satan at our back, who sets upon
us on every side, and fights (as we have heard) against all the
foregoing petitions, so that it is not possible always to stand firm in
such a persistent conflict.

Therefore there is here again great need to call upon God and to pray:
Dear Father, forgive us our trespasses. Not as though He did not
forgive sin without and even before our prayer (for He has given us the
Gospel, in which is pure forgiveness before we prayed or ever thought
about it). But this is to the intent that we may recognize and accept
such forgiveness. For since the flesh in which we daily live is of such
a nature that it neither trusts nor believes God, and is ever active in
evil lusts and devices, so that we sin daily in word and deed, by
commission and omission by which the conscience is thrown into unrest,
so that it is afraid of the wrath and displeasure of God, and thus
loses the comfort and confidence derived from the Gospel; therefore it
is ceaselessly necessary that we run hither and obtain consolation to
comfort the conscience again.

But this should serve God's purpose of breaking our pride and keeping
us humble. For in case any one should boast of his godliness and
despise others, God has reserved this prerogative to Himself, that the
person is to consider himself and place this prayer before his eyes,
and he will find that he is no better than others, and that in the
presence of God all must lower their plumes, and be glad that they can
attain forgiveness. And let no one think that as long as we live here
he can reach such a position that he will not need such forgiveness. In
short, if God does not forgive without ceasing, we are lost.

It is therefore the intent of this petition that God would not regard
our sins and hold up to us what we daily deserve, but would deal
graciously with us, and forgive, as He has promised, and thus grant us
a joyful and confident conscience to stand before Him in prayer. For
where the heart is not in right relation towards God, nor can take such
confidence, it will nevermore venture to pray. But such a confident and
joyful heart can spring from nothing else than the [certain] knowledge
of the forgiveness of sin.

But there is here attached a necessary, yet consolatory addition: As we
forgive. He has promised that we shall be sure that everything is
forgiven and pardoned, yet in the manner that we also forgive our
neighbor. For just as we daily sin much against God and yet He forgives
everything through grace, so we, too, must ever forgive our neighbor
who does us injury, violence, and wrong, shows malice toward us, etc.
If, therefore you do not forgive, then do not think that God forgives
you; but if you forgive, you have this consolation and assurance, that
you are forgiven in heaven, not on account of your forgiving, -- for
God forgives freely and without condition, out of pure grace, because
He has so promised, as the Gospel teaches, -- but in order that He may
set this up for our confirmation and assurance for a sign alongside of
the promise which accords with this prayer, Luke 6, 37: Forgive, and ye
shall be forgiven. Therefore Christ also repeats it soon after the
Lord's Prayer, and says, Matt. 6,14: For if ye forgive men their
trespasses, your heavenly Father will also forgive you, etc.

This sign is therefore attached to this petition, that, when we pray,
we remember the promise and reflect thus: Dear Father, for this reason
I come and pray Thee to forgive me, not that I can make satisfaction,
or can merit anything by my works, but because Thou hast promised and
attached the seal thereto that I should be as sure as though I had
absolution pronounced by Thyself. For as much as Baptism and the Lord's
Supper appointed as external signs, effect, so much also this sign can
effect to confirm our consciences and cause them to rejoice. And it is
especially given for this purpose, that we might use and practice it
every hour, as a thing that we have with us at all times.

 The Sixth Petition.

And lead us not into temptation.

We have now heard enough what toil and labor is required to retain all
that for which we pray, and to persevere therein, which, however, is
not achieved without infirmities and stumbling. Besides, although we
have received forgiveness and a good conscience and are entirely
acquitted, yet is our life of such a nature that one stands to-day and
to-morrow falls. Therefore, even though we be godly now and stand
before God with a good conscience, we must pray again that He would not
suffer us to relapse and yield to trials and temptations.

Temptation, however, or (as our Saxons in olden times used to call it)
Bekoerunge, is of three kinds, namely, of the flesh, of the world and
of the devil. For in the flesh we dwell and carry the old Adam about
our neck, who exerts himself and incites us daily to inchastity,
laziness, gluttony and drunkenness, avarice and deception, to defraud
our neighbor and to overcharge him, and, in short, to all manner of
evil lusts which cleave to us by nature, and to which we are incited by
the society, example and what we hear and see of other people, which
often wound and inflame even an innocent heart.

Next comes the world, which offends us in word and deed, and impels us
to anger and impatience. In short, there is nothing but hatred and
envy, enmity, violence and wrong, unfaithfulness, vengeance, cursing,
raillery slander, pride and haughtiness, with superfluous finery,
honor, fame, and power, where no one is willing to be the least, but
every one desires to sit at the head and to be seen before all.

Then comes the devil, inciting and provoking in all directions, but
especially agitating matters that concern the conscience and spiritual
affairs, namely, to induce us to despise and disregard both the Word
and works of God to tear us away from faith, hope, and love and bring
us into misbelief, false security, and obduracy, or, on the other hand,
to despair, denial of God, blasphemy, and innumerable other shocking
things. These are indeed snares and nets, yea, real fiery darts which
are shot most venomously into the heart, not by flesh and blood, but by
the devil.

Great and grievous, indeed, are these dangers and temptations which
every Christian must bear, even though each one were alone by himself,
so that every hour that we are in this vile life where we are attacked
on all sides, chased and hunted down, we are moved to cry out and to
pray that God would not suffer us to become weary and faint and to
relapse into sin, shame, and unbelief. For otherwise it is impossible
to overcome even the least temptation.

This, then, is leading us not into temptation, to wit, when He gives us
power and strength to resist, the temptation, however, not being taken
away or removed. For while we live in the flesh and have the devil
about us, no one can escape temptation and allurements; and it cannot
be otherwise than that we must endure trials, yea, be engulfed in them;
but we pray for this, that we may not fall and be drowned in them.

To feel temptation is therefore a far different thing from consenting
or yielding to it. We must all feel it, although not all in the same
manner, but some in a greater degree and more severely than others; as,
the young suffer especially from the flesh, afterwards, they that
attain to middle life and old age, from the world, but others who are
occupied with spiritual matters, that is, strong Christians, from the
devil. But such feeling, as long as it is against our will and we would
rather be rid of it, can harm no one. For if we did not feel it, it
could not be called a temptation. But to consent thereto is when we
give it the reins and do not resist or pray against it.

Therefore we Christians must be armed and daily expect to be
incessantly attacked, in order that no one may go on in security and
heedlessly, as though the devil were far from us, but at all times
expect and parry his blows. For though I am now chaste, patient, kind,
and in firm faith, the devil will this very hour send such an arrow
into my heart that I can scarcely stand. For he is an enemy that never
desists nor becomes tired, so that when one temptation ceases, there
always arise others and fresh ones.

Accordingly, there is no help or comfort except to run hither and to
take hold of the Lord's Prayer, and thus speak to God from the heart:
Dear Father, Thou hast bidden me pray; let me not relapse because of
temptations. Then you will see that they must desist, and finally
acknowledge themselves conquered. Else if you venture to help yourself
by your own thoughts and counsel, you will only make the matter worse
and give the devil more space. For he has a serpent's head, which if it
gain an opening into which he can slip, the whole body will follow
without check. But prayer can prevent him and drive him back.

The Seventh and Last Petition.

But deliver us from evil. Amen. In the Greek text this petition reads
thus: Deliver or preserve us from the Evil One, or the Malicious One;
and it looks as if He were speaking of the devil, as though He would
comprehend everything in one so that the entire substance of all our
prayer is directed against our chief enemy. For it is he who hinders
among us everything that we pray for: the name or honor of God, God's
kingdom and will, our daily bread, a cheerful good conscience, etc.

Therefore we finally sum it all up and say: Dear Father pray, help that
we be rid of all these calamities. But there is nevertheless also
included whatever evil may happen to us under the devil's kingdom --
poverty, shame, death, and, in short, all the agonizing misery and
heartache of which there is such an unnumbered multitude on the earth.
For since the devil is not only a liar, but also a murderer, he
constantly seeks our life, and wreaks his anger whenever he can afflict
our bodies with misfortune and harm. Hence it comes that he often
breaks men's necks or drives them to insanity, drowns some, and incites
many to commit suicide, and to many other terrible calamities.
Therefore there is nothing for us to do upon earth but to pray against
this arch enemy without ceasing. For unless God preserved us, we would
not be safe from him even for an hour.

Hence you see again how God wishes us to pray to Him also for all the
things which affect our bodily interests, so that we seek and expect
help nowhere else except in Him. But this matter He has put last; for
if we are to be preserved and delivered from all evil, the name of God
must first be hallowed in us, His kingdom must be with us, and His will
be done. After that He will finally preserve us from sin and shame,
and, besides, from everything that may hurt or injure us.

Thus God has briefly placed before us all the distress which may ever
come upon us, so that we might have no excuse whatever for not praying.
But all depends upon this, that we learn also to say Amen, that is,
that we do not doubt that our prayer is surely heard and [what we pray]
shall be done. For this is nothing else than the word of undoubting
faith, which does not pray at a venture, but knows that God does not
lie to him, since He has promised to grant it. Therefore, where there
is no such faith, there cannot be true prayer either.

It is, therefore, a pernicious delusion of those who pray in such a
manner that they dare not from the heart say yea and positively
conclude that God hears them, but remain in doubt and say, How should I
be so bold as to boast that God hears my prayer? For I am but a poor
sinner, etc.

The reason for this is, they regard not the promise of God, but their
own work and worthiness, whereby they despise God and reproach Him with
lying, and therefore they receive nothing. As St. James says [1, 6]:
But let him ask in faith, nothing wavering; for he that wavereth is
like a wave of the sea, driven with the wind and tossed. For let not
that man think that he shall receive anything of the Lord. Behold, such
importance God attaches to the fact that we are sure we do not pray in
vain, and that we do not in any way despise our prayer.

Part Fourth.

OF BAPTISM.

We have now finished the three chief parts of the common Christian
doctrine. Besides these we have yet to speak of our two Sacraments
instituted by Christ, of which also every Christian ought to have at
least an ordinary, brief instruction, because without them there can be
no Christian; although, alas! hitherto no instruction concerning them
has been given. But, in the first place, we take up Baptism, by which
we are first received into the Christian Church. However, in order that
it may be readily understood we will treat of it in an orderly manner,
and keep only to that which it is necessary for us to know. For how it
is to be maintained and defended against heretics and sects we will
commend to the learned.

In the first place, we must above all things know well the words upon
which Baptism is founded, and to which everything refers that is to be
said on the subject, namely, where the Lord Christ speaks in the last
chapter of Matthew, v. 19:

Go ye therefore and teach all nations, baptizing them in the name of
the Father, and of the Son, and of the Holy Ghost.

Likewise in St. Mark, the last chapter, v. 16:

He that believeth and is baptized shall be saved; but he that
believeth not shall be damned .

In these words you must note, in the first place, that here stand
God's commandment and institution, lest we doubt that Baptism is
divine, not devised nor invented by men. For as truly as I can say, No
man has spun the Ten Commandments, the Creed, and the Lord's Prayer out
of his head, but they are revealed and given by God Himself, so also I
can boast that Baptism is no human trifle, but instituted by God
Himself, moreover, that it is most solemnly and strictly commanded that
we must be baptized or we cannot be saved, lest any one regard it as a
trifling matter, like putting on a new red coat. For it is of the
greatest importance that we esteem Baptism excellent, glorious, and
exalted, for which we contend and fight chiefly, because the world is
now so full of sects clamoring that Baptism is an external thing, and
that external things are of no benefit. But let it be ever so much an
external thing here stand God's Word and command which institute,
establish, and confirm Baptism. But what God institutes and commands
cannot be a vain, but must be a most precious thing, though in
appearance it were of less value than a straw. If hitherto people could
consider it a great thing when the Pope with his letters and bulls
dispensed indulgences and confirmed altars and churches, solely because
of the letters and seals, we ought to esteem Baptism much more highly
and more precious, because God has commanded it, and, besides, it is
performed in His name. For these are the words, Go ye baptize; however,
not in your name, but in the name of God.

For to be baptized in the name of God is to be baptized not by men, but
by God Himself. Therefore although it is performed by human hands, it
is nevertheless truly God's own work. From this fact every one may
himself readily infer that it is a far higher work than any work
performed by a man or a saint. For what work greater than the work of
God can we do?

But here the devil is busy to delude us with false appearances, and
lead us away from the work of God to our own works. For there is a much
more splendid appearance when a Carthusian does many great and
difficult works and we all think much more of that which we do and
merit ourselves. But the Scriptures teach thus: Even though we collect
in one mass the works of all the monks, however splendidly they may
shine, they would not be as noble and good as if God should pick up a
straw. Why? Because the person is nobler and better. Here, then, we
must not estimate the person according to the works, but the works
according to the person, from whom they must derive their nobility. But
insane reason will not regard this, and because Baptism does not shine
like the works which we do, it is to be esteemed as nothing.

From this now learn a proper understanding of the subject, and how to
answer the question what Baptism is, namely thus, that it is not mere
ordinary water, but water comprehended in God's Word and command, and
sanctified thereby, so that it is nothing else than a divine water; not
that the water in itself is nobler than other water, but that God's
Word and command are added.

Therefore it is pure wickedness and blasphemy of the devil that now our
new spirits, to mock at Baptism, omit from it God's Word and
institution, and look upon it in no other way than as water which is
taken from the well, and then blather and say: How is a handful of
water to help the soul? Aye, my friend, who does not know that water is
water if tearing things asunder is what we are after? But how dare you
thus interfere with God's order, and tear away the most precious
treasure with which God has connected and enclosed it, and which He
will not have separated? For the kernel in the water is God's Word or
command and the name of God which is a treasure greater and nobler than
heaven and earth.

Comprehend the difference, then, that Baptism is quite another thing
than all other water; not on account of the natural quality, but
because something more noble is here added; for God Himself stakes His
honor His power and might on it. Therefore it is not only natural
water, but a divine, heavenly, holy, and blessed water, and in whatever
other terms we can praise it, -- all on account of the Word, which is a
heavenly, holy Word, that no one can sufficiently extol, for it has,
and is able to do, all that God is and can do [since it has all the
virtue and power of God comprised in it]. Hence also it derives its
essence as a Sacrament, as St. Augustine also taught: Accedat verbum ad
elementum et fit sacramentum. That is, when the Word is joined to the
element or natural substance, it becomes a Sacrament, that is, a holy
and divine matter and sign.

Therefore we always teach that the Sacraments and all external things
which God ordains and institutes should not be regarded according to
the coarse, external mask, as we regard the shell of a nut, but as the
Word of God is included therein. For thus we also speak of the parental
estate and of civil government. If we propose to regard them in as far
as they have noses, eyes, skin, and hair flesh and bones, they look
like Turks and heathen, and some one might start up and say: Why should
I esteem them more than others? But because the commandment is added:
Honor thy father and thy mother, I behold a different man, adorned and
clothed with the majesty and glory of God. The commandment (I say) is
the chain of gold about his neck, yea, the crown upon his head which
shows to me how and why one must honor this flesh and blood.

Thus, and much more even, you must honor Baptism and esteem it
glorious on account of the Word, since He Himself has honored it both
by words and deeds; moreover, confirmed it with miracles from heaven.
For do you think it was a jest that, when Christ was baptized, the
heavens were opened and the Holy Ghost descended visibly, and
everything was divine glory and majesty?

Therefore I exhort again that these two the water and the Word, by no
means be separated from one another and parted. For if the Word is
separated from it, the water is the same as that with which the servant
cooks, and may indeed be called a bath-keeper's baptism. But when it is
added, as God has ordained, it is a Sacrament, and is called
Christ-baptism. Let this be the first part regarding the essence and
dignity of the holy Sacrament.

In the second place, since we know now what Baptism is, and how it is
to be regarded, we must also learn why and for what purpose it is
instituted; that is, what it profits, gives and works. And this also we
cannot discern better than from the words of Christ above quoted: He
that believeth and is baptized shall be saved. Therefore state it most
simply thus, that the power, work, profit, fruit, and end of Baptism is
this, namely, to save. For no one is baptized in order that he may
become a prince, but, as the words declare, that he be saved. But to be
saved. we know. is nothing else than to be delivered from sin, death,
and the devil, and to enter into the kingdom of Christ, and to live
with Him forever.

Here you see again how highly and precious we should esteem Baptism,
because in it we obtain such an unspeakable treasure, which also
indicates sufficiently that it cannot be ordinary mere water. For mere
water could not do such a thing, but the Word does it, and (as said
above) the fact that the name of God is comprehended therein. But where
the name of God is, there must be also life and salvation, that it may
indeed be called a divine, blessed, fruitful, and gracious water; for
by the Word such power is imparted to Baptism that it is a laver of
regeneration, as St. Paul also calls it, Titus 3, 5.

But as our would-be wise, new spirits assert that faith alone saves,
and that works and external things avail nothing, we answer: It is
true, indeed, that nothing in us is of any avail but faith, as we shall
hear still further. But these blind guides are unwilling to see this,
namely, that faith must have something which it believes, that is, of
which it takes hold, and upon which it stands and rests. Thus faith
clings to the water, and believes that it is Baptism, in which there is
pure salvation and life; not through the water (as we have sufficiently
stated), but through the fact that it is embodied in the Word and
institution of God, and the name of God inheres in it. Now, if I
believe this, what else is it than believing in God as in Him who has
given and planted His Word into this ordinance, and proposes to us this
external thing wherein we may apprehend such a treasure?

Now, they are so mad as to separate faith and that to which faith
clings and is bound though it be something external. Yea, it shall and
must be something external, that it may be apprehended by the senses,
and understood and thereby be brought into the heart, as indeed the
entire Gospel is an external, verbal preaching. In short, what God does
and works in us He proposes to work through such external ordinances.
Wherever, therefore, He speaks, yea, in whichever direction or by
whatever means He speaks, thither faith must look, and to that it must
hold. Now here we have the words: He that believeth and is baptized
shall be saved. To what else do they refer than to Baptism, that is, to
the water comprehended in God's ordinance? Hence it follows that
whoever rejects Baptism rejects the Word of God, faith, and Christ, who
directs us thither and binds us to Baptism.

In the third place since we have learned the great benefit and power of
Baptism, let us see further who is the person that receives what
Baptism gives and profits. This is again most beautifully and clearly
expressed in the words: He that believeth and is baptized shall be
saved. That is, faith alone makes the person worthy to receive
profitably the saving, divine water. For, since these blessings are
here presented and promised in the words in and with the water, they
cannot be received in any other way than by believing them with the
heart. Without faith it profits nothing, notwithstanding it is in
itself a divine superabundant treasure. Therefore this single word (He
that believeth) effects this much that it excludes and repels all
works which we can do, in the opinion that we obtain and merit
salvation by them. For it is determined that whatever is not faith
avails nothing nor receives anything.

But if they say, as they are accustomed: Still Baptism is itself a
work, and you say works are of no avail for salvation; what then,
becomes of faith? Answer: Yes, our works, indeed, avail nothing for
salvation; Baptism, however, is not our work, but God's (for, as was
stated, you must put Christ-baptism far away from a bath-keeper's
baptism). God's works, however, are saving and necessary for salvation,
and do not exclude, but demand, faith; for without faith they could not
be apprehended. For by suffering the water to be poured upon you, you
have not yet received Baptism in such a manner that it benefits you
anything; but it becomes beneficial to you if you have yourself
baptized with the thought that this is according to God's command and
ordinance, and besides in God's name, in order that you may receive in
the water the promised salvation. Now, this the fist cannot do, nor the
body; but the heart must believe it.

Thus you see plainly that there is here no work done by us, but a
treasure which He gives us, and which faith apprehends; just as the
Lord Jesus Christ upon the cross is not a work, but a treasure
comprehended in the Word, and offered to us and received by faith.
Therefore they do us violence by exclaiming against us as though we
preach against faith; while we alone insist upon it as being of such
necessity that without it nothing can be received nor enjoyed.

Thus we have these three parts which it is necessary to know
concerning this Sacrament especially that the ordinance of God is to be
held in all honor, which alone would be sufficient, though it be an
entirely external thing like the commandment, Honor thy father and thy
mother, which refers to bodily flesh and blood. Therein we regard not
the flesh and blood, but the commandment of God in which they are
comprehended, and on account of which the flesh is called father and
mother; so also, though we had no more than these words, Go ye and
baptize, etc., it would be necessary for us to accept and do it as the
ordinance of God. Now there is here not only God's commandment and
injunction, but also the promise, on account of which it is still far
more glorious than whatever else God has commanded and ordained, and
is, in short, so full of consolation and grace that heaven and earth
cannot comprehend it. But it requires skill to believe this, for the
treasure is not wanting, but this is wanting that men apprehend it and
hold it firmly.

Therefore every Christian has enough in Baptism to learn and to
practice all his life; for he has always enough to do to believe
firmly what it promises and brings: victory over death and the devil,
forgiveness of sin, the grace of God, the entire Christ, and the Holy
Ghost with His gifts. In short, it is so transcendent that if timid
nature could realize it, it might well doubt whether it could be true.
For consider, if there were somewhere a physician who understood the
art of saving men from dying, or, even though they died, of restoring
them speedily to life, so that they would thereafter live forever, how
the world would pour in money like snow and rain, so that because of
the throng of the rich no one could find access! But here in Baptism
there is brought free to every one's door such a treasure and medicine
as utterly destroys death and preserves all men alive.

Thus we must regard Baptism and make it profitable to ourselves, that
when our sins and conscience oppress us, we strengthen ourselves and
take comfort and say: Nevertheless I am baptized; but if I am baptized,
it is promised me that I shall be saved and have eternal life, both in
soul and body. For that is the reason why these two things are done in
Baptism namely, that the body, which can apprehend nothing but the
water, is sprinkled, and, in addition, the word is spoken for the soul
to apprehend. Now, since both, the water and the Word, are one Baptism,
therefore body and soul must be saved and live forever: the soul
through the Word which it believes, but the body because it is united
with the soul and also apprehends Baptism as it is able to apprehend
it. We have, therefore, no greater jewel in body and soul, for by it we
are made holy and are saved, which no other kind of life, no work upon
earth, can attain.

Let this suffice respecting the nature, blessing, and use of Baptism,
for it answers the present purpose.

 [Part Fifth.]

OF THE SACRAMENT OF THE ALTAR.

In the same manner as we have heard regarding Holy Baptism, we must
speak also concerning the other Sacrament, namely, these three points:
What is it? What are its benefits? and, Who is to receive it? And all
these are established by the words by which Christ has instituted it,
and which every one who desires to be a Christian and go to the
Sacrament should know. For it is not our intention to admit to it and
to administer it to those who know not what they seek, or why they
come. The words, however, are these:

Our Lord Jesus Christ, the same night in which He was betrayed, took
bread; and when He had given thanks, He brake it, and gave it to His
disciples, and said, Take, eat; this is My body, which is given for
you:
this do in remembrance of Me.

After the same manner also He took the cup when He had supped, gave
thanks, and gave it to them, saying, Drink ye all of it; this cup is
the new testament in My blood, which is shed for you for the remission
of sins: this do ye, as oft as ye drink it, in remembrance of Me.

Here also we do not wish to enter into controversy and contend with the
traducers and blasphemers of this Sacrament, but to learn first (as we
did regarding Baptism) what is of the greatest importance, namely that
the chief point is the Word and ordinance or command of God. For it has
not been invented nor introduced by any man, but without any one's
counsel and deliberation it has been instituted by Christ. Therefore,
just as the Ten Commandments, the Lord's Prayer, and the Creed retain
their nature and worth although you never keep, pray, or believe them,
so also does this venerable Sacrament remain undisturbed, so that
nothing is detracted or taken from it, even though we employ and
dispense it unworthily. What do you think God cares about what we do or
believe, so that on that account He should suffer His ordinance to be
changed? Why, in all worldly matters every thing remains as God has
created and ordered it, no matter how we employ or use it. This must
always be urged, for thereby the prating of nearly all the fanatical
spirits can be repelled. For they regard the Sacraments, aside from the
Word of God, as something that we do.

Now, what is the Sacrament of the Altar!

Answer: It is the true body and blood of our Lord Jesus Christ, in and
under the bread and wine which we Christians are commanded by the Word
of Christ to eat and to drink. And as we have said of Baptism that it
is not simple water, so here also we say the Sacrament is bread and
wine, but not mere bread and wine, such as are ordinarily served at the
table, but bread and wine comprehended in, and connected with, the Word
of God.

It is the Word (I say) which makes and distinguishes this Sacrament, so
that it is not mere bread and wine, but is, and is called, the body and
blood of Christ. For it is said: Accedat verbum ad elementum, et At
sacramentum. If the Word be joined to the element it becomes a
Sacrament. This saying of St. Augustine is so properly and so well put
that he has scarcely said anything better. The Word must make a
Sacrament of the element, else it remains a mere element. Now, it is
not the word or ordinance of a prince or emperor, but of the sublime
Majesty, at whose feet all creatures should fall, and affirm it is as
He says, and accept it with all reverence fear, and humility.

With this Word you can strengthen your conscience and say: If a
hundred thousand devils, together with all fanatics, should rush
forward, crying, How can bread and wine be the body and blood of
Christ? etc., I know that all spirits and scholars together are not as
wise as is the Divine Majesty in His little finger. Now here stands the
Word of Christ: Take, eat; this is My body; Drink ye all of it; this is
the new testament in My blood, etc. Here we abide, and would like to
see those who will constitute themselves His masters, and make it
different from what He has spoken. It is true, indeed, that if you take
away the Word or regard it without the words, you have nothing but mere
bread and wine. But if the words remain with them as they shall and
must, then, in virtue of the same, it is truly the body and blood of
Christ. For as the lips of Christ say and speak, so it is, as He can
never lie or deceive.

Hence it is easy to reply to all manner of questions about which men
are troubled at the present time, such as this one: Whether even a
wicked priest can minister at, and dispense, the Sacrament, and
whatever other questions like this there may be. For here we conclude
and say: Even though a knave takes or distributes the Sacrament, he
receives the true Sacrament, that is, the true body and blood of
Christ, just as truly as he who [receives or] administers it in the
most worthy manner. For it is not founded upon the holiness of men, but
upon the Word of God. And as no saint upon earth, yea, no angel in
heaven, can make bread and wine to be the body and blood of Christ, so
also can no one change or alter it, even though it be misused. For the
Word by which it became a Sacrament and was instituted does not become
false because of the person or his unbelief. For He does not say: If
you believe or are worthy, you receive My body and blood, but: Take,
eat and drink; this is By body and blood. Likewise: Do this (namely,
what I now do, institute, give, and bid you take) . That is as much as
to say, No matter whether you are worthy or unworthy, you have here His
body and blood by virtue of these words which are added to the bread
and wine. Only note and remember this well; for upon these words rest
all our foundation, protection, and defense against all errors and
deception that have ever come or may yet come.

Thus we have briefly the first point which relates to the essence of
this Sacrament. Now examine further the efficacy and benefits on
account of which really the Sacrament was instituted; which is also its
most necessary part, that we may know what we should seek and obtain
there. Now this is plain and clear from the words just mentioned: This
is My body and blood, given and shed FOR YOU, for the remission of
sins. Briefly that is as much as to say: For this reason we go to the
Sacrament because there we receive such a treasure by and in which we
obtain forgiveness of sins. Why so? Because the words stand here and
give us this; for on this account He bids me eat and drink, that it may
be my own and may benefit me, as a sure pledge and token, yea, the very
same treasure that is appointed for me against my sins, death, and
every calamity.

On this account it is indeed called a food of souls, which nourishes
and strengthens the new man. For by Baptism we are first born anew; but
(as we said before) there still remains, besides, the old vicious
nature of flesh and blood in man, and there are so many hindrances and
temptations of the devil and of the world that we often become weary
and faint, and sometimes also stumble.

Therefore it is given for a daily pasture and sustenance, that faith
may refresh and strengthen itself so as not to fall back in such a
battle, but become ever stronger and stronger. For the new life must be
so regulated that it continually increase and progress, but it must
suffer much opposition. For the devil is such a furious enemy that when
he sees that we oppose him and attack the old man, and that he cannot
topple us over by force, he prowls and moves about on all sides, tries
all devices, and does not desist until he finally wearies us, so that
we either renounce our faith or yield hands and feet and become
listless or impatient. Now to this end the consolation is here given
when the heart feels that the burden is becoming too heavy, that it may
here obtain new power and refreshment.

But here our wise spirits contort themselves with their great art and
wisdom, crying out and bawling: How can bread and wine forgive sins or
strengthen faith? Although they hear and know that we do not say this
of bread and wine, because in itself bread is bread, but of such bread
and wine as is the body and blood of Christ, and has the words attached
to it. That, we say, is verily the treasure, and nothing else, through
which such forgiveness is obtained. Now the only way in which it is
conveyed and appropriated to us is in the words (Given and shed for
you). For herein you have both truths, that it is the body and blood of
Christ, and that it is yours as a treasure and gift. Now the body of
Christ can never be an unfruitful, vain thing, that effects or profits
nothing. Yet however great is the treasure in itself, it must be
comprehended in the Word and administered to us, else we should never
be able to know or seek it.

Therefore also it is vain talk when they say that the body and blood of
Christ are not given and shed for us in the Lord's Supper, hence we
could not have forgiveness of sins in the Sacrament. For although the
work is accomplished and the forgiveness of sins acquired on the cross,
yet it cannot come to us in any other way than through the Word. For
what would we otherwise know about it, that such a thing was
accomplished or was to be given us if it were not presented by
preaching or the oral Word? Whence do they know of it, or how can they
apprehend and appropriate to themselves the forgiveness, except they
lay hold of and believe the Scriptures and the Gospel? But now the
entire Gospel and the article of the Creed: I believe a holy Christian
Church, the forgiveness of sin, etc., are by the Word embodied in this
Sacrament and presented to us. Why, then, should we allow this treasure
to be torn from the Sacrament when they must confess that these are the
very words which we hear everywhere in the Gospel, and they cannot say
that these words in the Sacrament are of no use, as little as they dare
say that the entire Gospel or Word of God, apart from the Sacrament, is
of no use?

Thus we have the entire Sacrament, both as to what it is in itself and
as to what it brings and profits. Now we must also see who is the
person that receives this power and benefit. That is answered briefly,
as we said above of Baptism and often elsewhere: Whoever believes it
has what the words declare and bring. For they are not spoken or
proclaimed to stone and wood, but to those who hear them, to whom He
says: Take and eat, etc. And because He offers and promises forgiveness
of sin, it cannot be received otherwise than by faith. This faith He
Himself demands in the Word when He says: Given and shed for you. As if
He said: For this reason I give it, and bid you eat and drink, that you
may claim it as yours and enjoy it. Whoever now accepts these words,
and believes that what they declare is true, has it. But whoever does
not believe it has nothing, as he allows it to be offered to him in
vain, and refuses to enjoy such a saving good. The treasure, indeed, is
opened and placed at every one's door, yea upon his table, but it is
necessary that you also claim it, and confidently view it as the words
suggest to you.

This, now, is the entire Christian preparation for receiving this
Sacrament worthily. For since this treasure is entirely presented in
the words, it cannot be apprehended and appropriated in any other way
than with the heart. For such a gift and eternal treasure cannot be
seized with the fist. Fasting and prayer, etc., may indeed be an
external preparation and discipline for children, that the body may
keep and bear itself modestly and reverently towards the body and blood
of Christ; yet what is given in and with it the body cannot seize and
appropriate. But this is done by the faith of the heart, which discerns
this treasure and desires it. This may suffice for what is necessary
as a general instruction respecting this Sacrament; for what is
further to be said of it belongs to another time.

Conclusion

In conclusion, since we have now the true understanding and doctrine of
the Sacrament, there is indeed need of some admonition and exhortation,
that men may not let so great a treasure which is daily administered
and distributed among Christians pass by unheeded, that is, that those
who would be Christians make ready to receive this venerable Sacrament
often. For we see that men seem weary and lazy with respect to it; and
there is a great multitude of such as hear the Gospel, and, because the
nonsense of the Pope has been abolished, and we are freed from his laws
and coercion, go one, two, three years, or even longer without the
Sacrament, as though they were such strong Christians that they have no
need of it; and some allow themselves to be prevented and deterred by
the pretense that we have taught that no one should approach it except
those who feel hunger and thirst, which urge them to it. Some pretend
that it is a matter of liberty and not necessary, and that it is
sufficient to believe without it; and thus for the most part they go so
far that they become quite brutish, and finally despise both the
Sacrament and the Word of God.

Now, it is true, as we have said, that no one should by any means be
coerced or compelled, lest we institute a new murdering of souls.
Nevertheless, it must be known that such people as deprive themselves
of, and withdraw from, the Sacrament so long a time are not to be
considered Christians. For Christ has not instituted it to be treated
as a show, but has commanded His Christians to eat and drink it, and
thereby remember Him.

And, indeed, those who are true Christians and esteem the Sacrament
precious and holy will urge and impel themselves unto it. Yet that the
simple-minded and the weak who also would like to be Christians be the
more incited to consider the cause and need which ought to impel them,
we will treat somewhat of this point. For as in other matters
pertaining to faith, love, and patience, it is not enough to teach and
instruct only, but there is need also of daily exhortation, so here
also there is need of continuing to preach that men may not become
weary and disgusted, since we know and feel how the devil always
opposes this and every Christian exercise, and drives and deters
therefrom as much as he can.

And we have, in the first place, the clear text in the very words of
Christ: Do this in remembrance of Me. These are bidding and commanding
words by which all who would be Christians are enjoined to partake of
this Sacrament. Therefore, whoever would be a disciple of Christ, with
whom He here speaks, must also consider and observe this, not from
compulsion, as being forced by men, but in obedience to the Lord Jesus
Christ, and to please Him. However, if you say: But the words are
added, As oft as ye do it; there He compels no one, but leaves it to
our free choice, answer: That is true, yet it is not written that we
should never do so. Yea, just because He speaks the words, As oft as ye
do it, it is nevertheless implied that we should do it often; and it is
added for the reason that He wishes to have the Sacrament free, not
limited to special times, like the Passover of the Jews, which they
were obliged to eat only once a year, and that just upon the fourteenth
day of the first full moon in the evening, and which they must not vary
a day. As if He would say by these words: I institute a Passover or
Supper for you which you shall enjoy not only once a year, just upon
this evening, but often, when and where you will, according to every
one's opportunity and necessity, bound to no place or appointed time;
although the Pope afterwards perverted it, and again made a Jewish
feast of it.

Thus, you perceive, it is not left free in the sense that we may
despise it. For that I call despising it if one allow so long a time to
elapse and with nothing to hinder him yet never feels a desire for it.
If you wish such liberty, you may just as well have the liberty to be
no Christian, and neither have to believe nor pray; for the one is just
as much the command of Christ as the other. But if you wish to be a
Christian, you must from time to time render satisfaction and obedience
to this commandment. For this commandment ought ever to move you to
examine yourself and to think: See, what sort of a Christian I am! If I
were one, I would certainly have some little longing for that which my
Lord has commanded [me] to do.

And, indeed, since we act such strangers to it, it is easily seen what
sort of Christians we were under the Papacy, namely, that we went from
mere compulsion and fear of human commandments, without inclination and
love, and never regarded the commandment of Christ. But we neither
force nor compel any one; nor need any one do it to serve or please us.
But this should induce and constrain you by itself, that He desires it
and that it is pleasing to Him. You must not suffer men to coerce you
unto faith or any good work. We are doing no more than to say and
exhort you as to what you ought to do, not for our sake, but for your
own sake. He invites and allures you; if you despise it, you must
answer for it yourself.

Now, this is to be the first point, especially for those who are cold
and indifferent, that they may reflect upon and rouse themselves. For
this is certainly true, as I have found in my own experience, and as
every one will find in his own case, that if a person thus withdraw
from this Sacrament, he will daily become more and more callous and
cold, and will at last disregard it altogether. To avoid this, we must,
indeed, examine heart and conscience, and act like a person who desires
to be right with God. Now, the more this is done, the more will the
heart be warmed and enkindled, that it may not become entirely cold.

But if you say: How if I feel that I am not prepared? Answer: That is
also my scruple, especially from the old way under the Pope, in which a
person tortured himself to be so perfectly pure that God could not find
the least blemish in us. On this account we became so timid that every
one was instantly thrown into consternation and said to himself: Alas!
you are unworthy! For then nature and reason begin to reckon our
unworthiness in comparison with the great and precious good; and then
it appears like a dark lantern in contrast with the bright sun, or as
filth in comparison with precious stones. Because nature and reason see
this, they refuse to approach and tarry until they are prepared so long
that one week trails another, and one half year the other. But if you
are to regard how good and pure you are, and labor to have no
compunctions, you must never approach.

We must, therefore, make a distinction here among men. For those who
are wanton and dissolute must be told to stay away; for they are not
prepared to receive forgiveness of sin since they do not desire it and
do not wish to be godly. But the others, who are not such callous and
wicked people, and desire to be godly, must not absent themselves, even
though otherwise they be feeble and full of infirmities, as St. Hilary
also has said: If any one have not committed sin for which he can
rightly be put out of the congregation and esteemed as no Christian, he
ought not stay away from the Sacrament, lest he may deprive himself of
life. For no one will make such progress that he will not retain many
daily infirmities in flesh and blood.

Therefore such people must learn that it is the highest art to know
that our Sacrament does not depend upon our worthiness. For we are not
baptized because we are worthy and holy, nor do we go to confession
because we are pure and without sin, but the contrary because we are
poor miserable men and just because we are unworthy; unless it be some
one who desires no grace and absolution nor intends to reform.

But whoever would gladly obtain grace and consolation should impel
himself, and allow no one to frighten him away, but say: I, indeed,
would like to be worthy, but I come, not upon any worthiness, but upon
Thy Word, because Thou hast commanded it, as one who would gladly be
Thy disciple, no matter what becomes of my worthiness. But this is
difficult; for we always have this obstacle and hindrance to encounter,
that we look more upon ourselves than upon the Word and lips of Christ.
For nature desires so to act that it can stand and rest firmly on
itself, otherwise it refuses to make the approach. Let this suffice
concerning the first point.

In the second place, there is besides this command also a promise, as
we heard above, which ought most strongly to incite and encourage us.
For here stand the kind and precious words: This is My body, given for
you. This is My blood, shed for you, for the remission of sins. These
words, I have said, are not preached to wood and stone, but to me and
you; else He might just as well be silent and not institute a
Sacrament. Therefore consider, and put yourself into this YOU, that He
may not speak to you in vain.

For here He offers to us the entire treasure which He has brought for
us from heaven, and to which He invites us also in other places with
the greatest kindness, as when He says in St. Matthew 11, 28: Come unto
Me, all ye that labor and are heavy laden, and I will give you rest.
Now it is surely a sin and a shame that He so cordially and faithfully
summons and exhorts us to our highest and greatest good, and we act so
distantly with regard to it, and permit so long a time to pass [without
partaking of the Sacrament] that we grow quite cold and hardened, so
that we have no inclination or love for it. We must never regard the
Sacrament as something injurious from which we had better flee but as a
pure wholesome, comforting remedy imparting salvation and comfort,
which will cure you and give you life both in soul and body. For where
the soul has recovered, the body also is relieved. Why, then, is it
that we act as if it were a poison, the eating of which would bring
death?

To be sure, it is true that those who despise it and live in an
unchristian manner receive it to their hurt and damnation; for nothing
shall be good or wholesome to them, just as with a sick person who from
caprice eats and drinks what is forbidden him by the physician. But
those who are sensible of their weakness, desire to be rid of it and
long for help, should regard and use it only as a precious antidote
against the poison which they have in them. For here in the Sacrament
you are to receive from the lips of Christ forgiveness of sin which
contains and brings with it the grace of God and the Spirit with all
His gifts, protection, shelter, and power against death and the devil
and all misfortune.

Thus you have, on the part of God, both the command and the promise of
the Lord Jesus Christ. Besides this, on your part, your own distress
which is about your neck, and because of which this command, invitation
and promise are given, ought to impel you. For He Himself says: They
that be whole need not a physician, but they that be sick; that is,
those who are weary and heavy-laden with their sins, with the fear of
death temptations of the flesh and of the devil. If therefore, you are
heavy-laden and feel your weakness, then go joyfully to this Sacrament
and obtain refreshment, consolation, and strength. For if you would
wait until you are rid of such burdens, that you might come to the
Sacrament pure and worthy, you must forever stay away. For in that case
He pronounces sentence and says: If you are pure and godly, you have no
need of Me, and I, in turn, none of thee. Therefore those alone are
called unworthy who neither feel their infirmities nor wish to be
considered sinners.

But if you say: What, then, shall I do if I cannot feel such distress
or experience hunger and thirst for the Sacrament? Answer: For those
who are so minded that they do not realize their condition I know no
better counsel than that they put their hand into their bosom to
ascertain whether they also have flesh and blood. And if you find that
to be the case, then go, for your good, to St. Paul's Epistle to the
Galatians, and hear what sort of a fruit your flesh is: Now the works
of the flesh (he says [chap. 5, 19ff.]) are manifest, which are these:
Adultery fornication uncleanness, lasciviousness, idolatry, witchcraft,
hatred, variance, emulations, wrath, strife, seditions, heresies,
envyings, murders, drunkenness, revelings, and such like.

Therefore, if you cannot feel it, at least believe the Scriptures, they
will not lie to you and they know your flesh better than you yourself.
Yea, St. Paul further concludes in Rom. 7, 18: l know that in me, that
is, in my flesh, dwelleth no good thing. If St. Paul may speak thus of
his flesh, we do not propose to be better nor more holy. But that we do
not feel it is so much the worse; for it is a sign that there is a
leprous flesh which feels nothing, and yet [the leprosy] rages and
keeps spreading. Yet as we have said, if you are quite dead to all
sensibility, still believe the Scriptures, which pronounce sentence
upon you. And, in short, the less you feel your sins and infirmities,
the more reason have you to go to the Sacrament to seek help and a
remedy.

In the second place, look about you and see whether you are also in the
world, or if you do not know it, ask your neighbors about it. If you
are in the world, do not think that there will be lack of sins and
misery. For only begin to act as though you would be godly and adhere
to the Gospel, and see whether no one will become your enemy, and,
moreover, do you harm, wrong, and violence, and likewise give you cause
for sin and vice. If you have not experienced it, then let the
Scriptures tell you, which everywhere give this praise and testimony to
the world.

Besides this, you will also have the devil about you, whom you will not
entirely tread under foot, because our Lord Christ Himself could not
entirely avoid him. Now, what is the devil? Nothing else than what the
Scriptures call him, a liar and murderer. A liar, to lead the heart
astray from the Word of God, and to blind it, that you cannot feel your
distress or come to Christ. A murderer, who cannot bear to see you live
one single hour. If you could see how many knives, darts, and arrows
are every moment aimed at you, you would be glad to come to the
Sacrament as often as possible. But there is no reason why we walk so
securely and heedlessly, except that we neither think nor believe that
we are in the flesh, and in this wicked world or in the kingdom of the
devil.

Therefore, try this and practice it well, and do but examine yourself,
or look about you a little, and only keep to the Scriptures. If even
then you still feel nothing, you have so much the more misery to lament
both to God and to your brother. Then take advice and have others pray
for you, and do not desist until the stone be removed from your heart.
Then, indeed, the distress will not fail to become manifest, and you
will find that you have sunk twice as deep as any other poor sinner,
and are much more in need of the Sacrament against the misery which
unfortunately you do not see, so that, with the grace of God, you may
feel it more and become the more hungry for the Sacrament, especially
since the devil plies his force against you, and lies in wait for you
without ceasing, to seize and destroy you, soul and body, so that you
are not safe from him one hour. How soon can he have brought you
suddenly into misery and distress when you least expect it!

Let this, then, be said for exhortation, not only for those of us who
are old and grown, but also for the young people, who ought to be
brought up in the Christian doctrine and understanding. For thereby the
Ten Commandments, the Creed, and the Lord's Prayer might be the more
easily inculcated to our youth, so that they would receive them with
pleasure and earnestness, and thus would practice them from their youth
and accustom themselves to them. For the old are now well-nigh done
for, so that these and other things cannot be attained, unless we train
the people who are to come after us and succeed us in our office and
work, in order that they also may bring up their children successfully
that the Word of God and the Christian Church may be preserved.
Therefore let every father of a family know that it is his duty by the
injunction and command of God, to teach these things to his children,
or have them learn what they ought to know. For since they are baptized
and received into the Christian Church, they should also enjoy this
communion of the Sacrament, in order that they may serve us and be
useful to us; for they must all indeed help us to believe, love, pray,
and fight against the devil.
