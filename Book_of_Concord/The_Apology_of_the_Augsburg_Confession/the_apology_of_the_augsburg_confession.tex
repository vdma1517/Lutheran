The Apology of the Augsburg Confession (1531)

Table of Contents

Philip Melanchthon's Introduction to the Apology
Part One: On Articles I-II of the Augustana
Part Two: On Articles III-IV of the Augustana
Part Three: What is Justifying Faith?
Part Four: That Faith in Christ Justifies
Part Five: That We Obtain Remission of Sins by Faith Alone in Christ
Part Six: On Article III: Love and the Fulfilling of the Law
Part Seven: Reply to the Arguments of the Adversaries
Part Eight: Continuation of: Reply to the Arguments...
Part Nine: Second Continuation of: Reply to the Arguments...
Part Ten: Third Continuation of: Reply to the Arguments...
Part Eleven: Articles Seven and Eight of the Augustana
Part Twelve: Article Nine of the Augustana
Part Thirteen: Article Ten of the Augustana
Part Fourteen: Article Eleven of the Augustana
Part Fifteen: Article Twelve of the Augustana
Part Sixteen: Article Six of the Augustana (Pt. 1)
Part Seventeen: Article Six of the Augustana (Pt. 2)
Part Eighteen: Article Seven of the Augustana
Part Nineteen: Article Fourteen of the Augustana
Part Twenty: Article Fifteen of the Augustana
Part Twenty-One: Article Sixteen of the Augustana
Part Twenty-Two: Article Seventeen of the Augustana
Part Twenty-Three: Article Eighteen of the Augustana
Part Twenty-Four: Article Nineteen of the Augustana
Part Twenty-Five: Article Twenty of the Augustana
Part Twenty-Six: Article Twenty-One of the Augustana
Part Twenty-Seven: Article Twenty-Two of the Augustana
Part Twenty-Eight: Article Twenty-Three of the Augustana
Part Twenty-Nine: Article Twenty-Four of the Augustana
Part Thirty: A Definition of the term "Sacrifice"
Part Thirty-One: What the Fathers Thought About Sacrifice
Part Thirty-Two: Of the Use of the Sacrament and Sacrifice
Part Thirty-Three: Of the Term "Mass"
Part Thirty-Four:Of the Mass for the Dead
Part Thirty-Five: Of Monastic Vows
Part Thirty-Six: Of Ecclesiatical Power
Part Thirty-Seven: End


INTRODUCTION


THE APOLOGY OF THE CONFESSION.

Philip Melanchthon Presents His Greeting to the Reader.  Wherefore we
believe that troubles and dangers for the glory of Christ and the
good of the Church should be endured, and we are confident that this
our fidelity to duty is approved of God, and we hope that the
judgment of posterity concerning us will be more just.

For it is undeniable that many topics of Christian doctrine whose
existence in the Church is of the greatest moment have been brought
to view by our theologians and explained; in reference to which we
are not disposed here to recount under what sort of opinions, and how
dangerous, they formerly lay covered in the writings of the monks,
canonists, and sophistical theologians.  [This may have to be done
later.]

We have the public testimonials of many good men, who give God thanks
for this greatest blessing, namely, that concerning many necessary
topics it has taught better things than are read everywhere in the
books of our adversaries.

We shall commend our cause, therefore, to Christ, who some time will
judge these controversies, and we beseech Him to look upon the
afflicted and scattered churches, and to bring them back to godly and
perpetual concord.  [Therefore, if the known and clear truth is
trodden under foot, we will resign this cause to God and Christ in
heaven, who is the Father of orphans and the Judge of widows and of
all the forsaken, who (as we certainly know) will judge and pass
sentence upon this cause aright.  Lord Jesus Christ, it is Thy holy
Gospel, it is Thy cause; look Thou upon the many troubled hearts and
consciences, and maintain and strengthen in Thy truth Thy churches
and little flocks, who suffer anxiety and distress from the devil.
Confound all hypocrisy and lies, and grant peace and unity, so that
Thy glory may advance, and Thy kingdom, strong against all the gates
of hell, may continually grow and increase.]




Part 1


Article I: _Of God._

The First Article of our Confession our adversaries approve, in which
we declare that we believe and teach that there is one divine essence,
undivided, etc., and yet, that there are three distinct persons, of
the same divine essence, and coeternal, Father, Son, and Holy Ghost.
This article we have always taught and defended, and we believe that
it has, in Holy Scripture, sure and firm testimonies that cannot be
overthrown.  And we constantly affirm that those thinking otherwise
are outside of the Church of Christ, and are idolaters, and insult
God.


Article II (I): _Of Original Sin._

The Second Article, Of Original Sin, the adversaries approve, but in
such a way that they, nevertheless, censure the definition of
original sin, which we incidentally gave.  Here, immediately at the
very threshold, His Imperial Majesty will discover that the writers
of the _Confutation_ were deficient not only in judgment, but also in
candor.  For whereas we, with a simple mind, desired, in passing, to
recount those things which original sin embraces, these men, by
framing an invidious interpretation, artfully distort a proposition
that has in it nothing which of itself is wrong.  Thus they say: "To
be without the fear of God, to be without faith, is actual guilt";
and therefore they deny that it is original guilt.

It is quite evident that such subtilties have originated in the
schools, not in the council of the Emperor.  But although this
sophistry can be very easily refuted; yet, in order that all good men
may understand that we teach in this matter nothing that is absurd,
we ask first of all that the German Confession be examined.  This
will free us from the suspicion of novelty.  For there it is written:
_Weiter wird gelehrt, dass nach dem Fall Adams alle Menschen, so
natuerlich geboren werden, in Suenden empfangen und geboren werdenen,
das ist, dass sie alle von Mutterleibe an voll boeser Lueste und
Neigung sind, keine wahre Gottesfurcht, keinen wahren Glauben an Gott
von Natur haben koennen._ [It is further taught that since the Fall
of Adam all men who are naturally born are conceived and born in sin,
i.e., that they all, from their mother's womb, are full of evil
desire and inclination, and can have by nature no true fear of God,
no true faith in God.] This passage testifies that we deny to those
propagated according to carnal nature not only the acts, but also the
power or gifts of producing fear and trust in God.  For we say that
those thus born have concupiscence, and cannot produce true fear and
trust in God.  What is there here with which fault can be found?  To
good men, we think, indeed, that we have exculpated ourselves
sufficiently.  For in this sense the Latin description denies to
nature [even to innocent infants] the power, i.e., it denies the
gifts and energy by which to produce fear and trust in God, and, in
adults [over and above this innate evil disposition of the heart,
also] the acts, so that, when we mention concupiscence, we understand
not only the acts or fruits, but the constant inclination of the
nature [the evil inclination within, which does not cease as long as
we are not born anew through the Spirit and faith].

But hereafter we will show more fully that our description agrees
with the usual and ancient definition.  For we must first show our
design in preferring to employ these words in this place.  In their
schools the adversaries confess that "the material," as they call it,
"of original sin is concupiscence." Wherefore, in framing the
definition, this should not have been passed by, especially at this
time, when some are philosophizing concerning it in a manner
unbecoming teachers of religion [are speaking concerning this innate,
wicked desire more after the manner of heathen from philosophy than
according to God's Word, or Holy Scripture].

For some contend that original sin is not a depravity or corruption
in the nature of man, but only servitude, or a condition of mortality
[not an innate evil nature, but only a blemish or imposed load, or
burden], which those propagated from Adam bear because of the guilt
of another [namely, Adam's sin], and without any depravity of their
own.  Besides, they add that no one is condemned to eternal death on
account of original sin, just as those who are born of a bond-woman
are slaves, and bear this condition without any natural blemish, but
because of the calamity of their mother [while, of themselves, they
are born without fault, like other men: thus original sin is not an
innate evil but a defect and burden which we bear since Adam, but we
are not on that account personally in sin and inherited disgrace].
To show that this impious opinion is displeasing to us, we made
mention of "concupiscence," and, with the best intention, have termed
and explained it as "diseases," that "the nature of men is born
corrupt and full of faults" [not a part of man, but the entire person
with its entire nature is born in sin as with a hereditary disease].

Nor, indeed, have we only made use of the term concupiscence, but we
have also said that "the fear of God and faith are wanting." This we
have added with the following design: The scholastic teachers also,
not sufficiently understanding the definition of original sin, which
they have received from the Fathers, extenuate the sin of origin.
They contend concerning the fomes [or evil inclination] that it is a
quality of [blemish in the] body, and, with their usual folly, ask
whether this quality be derived from the contagion of the apple or
from the breath of the serpent, and whether it be increased by
remedies.  With such questions they have suppressed the main point.
Therefore, when they speak of the sin of origin, they do not mention
the more serious faults of human nature, to wit, ignorance of God,
contempt for God, being destitute of fear and confidence in God,
hatred of God's judgment, flight from God [as from a tyrant] when He
judges, anger toward God, despair of grace, putting one's trust in
present things [money, property, friends], etc. These diseases, which
are in the highest degree contrary to the Law of God, the scholastics
do not notice; yea, to human nature they meanwhile ascribe unimpaired
strength for loving God above all things, and for fulfilling God's
commandments according to the substance of the acts; nor do they see
that they are saying things that are contradictory to one another.
For what else is the being able in one's own strength to love God
above all things, and to fulfil His commandments, than to have
original righteousness [to be a new creature in Paradise, entirely
pure and holy]?  But if human nature have such strength as to be able
of itself to love God above all things, as the scholastics
confidently affirm, what will original sin be?  For what will there
be need of the grace of Christ if we can be justified by our own
righteousness [powers]?  For what will there be need of the Holy
Ghost if human strength can by itself love God above all things, and
fulfil God's commandments?  Who does not see what preposterous
thoughts our adversaries entertain?  The lighter diseases in the
nature of man they acknowledge, the more severe they do not
acknowledge; and yet of these, Scripture everywhere admonishes us,
and the prophets constantly complain [as the 13th Psalm, and some
other psalms say Ps. 14, 1-3; 5, 9; 140, 3; 36, 1], namely, of carnal
security, of the contempt of God, of hatred toward God, and of
similar faults born with us.  [For Scripture clearly says that all
these things are not blown at us, but born with us.] But after the
scholastics mingled with Christian doctrine philosophy concerning the
perfection of nature [light of reason], and ascribed to the free will
and the acts springing therefrom more than was sufficient, and taught
that men are justified before God by philosophic or civil
righteousness (which we also confess to be subject to reason, and in
a measure, within our power), they could not see the inner
uncleanness of the nature of men.  For this cannot be judged except
from the Word of God, of which the scholastics, in their discussions,
do not frequently treat.

These were the reasons why, in the description of original sin, we
made mention of concupiscence also, and denied to man's natural
strength the fear of God and trust in Him.  For we wished to indicate
that original sin contains also these diseases, namely, ignorance of
God, contempt for God, the being destitute of the fear of God and
trust in Him, inability to love God.  These are the chief faults of
human nature, conflicting especially with the first table of the
Decalog.

Neither have we said anything new.  The ancient definition understood
aright expresses precisely the same thing when it says: "Original sin
is the absence of original righteousness" [a lack of the first purity
and righteousness in Paradise].  But what is righteousness?  Here the
scholastics wrangle about dialectic questions, they do not explain
what original righteousness is.  Now, in the Scriptures,
righteousness comprises not only the second table of the Decalog
[regarding good works in serving our fellow-man], but the first also,
which teaches concerning the fear of God, concerning faith,
concerning the love of God.  Therefore original righteousness was to
embrace not only an even temperament of the bodily qualities [perfect
health and, in all respects, pure blood, unimpaired powers of the
body, as they contend], but also these gifts, namely, a quite certain
knowledge of God, fear of God, confidence in God, or certainly the
rectitude and power to yield these affections [but the greatest
feature in that noble first creature was a bright light in the heart
to know God and His work, etc.].  And Scripture testifies to this,
when it says, Gen. 1, 27, that man was fashioned in the image and
likeness of God.  What else is this than that there were embodied in
man such wisdom and righteousness as apprehended God, and in which
God was reflected, i.e., to man there were given the gifts of the
knowledge of God, the fear of God, confidence in God, and the like?
For thus Irenaeus and Ambrose interpret the likeness to God, the
latter of whom not only says many things to this effect, but
especially declares: That soul is not, therefore, in the image of God,
in which God is not at all times.  And Paul shows in the Epistles to
the Ephesians, 5, 9, and Colossians, 3,10, that the image of God is
the knowledge of God, righteousness, and truth.  Nor does Longobard
fear to say that original righteousness is the very likeness to God
which God implanted in man.  We recount the opinions of the ancients,
which in no way interfere with Augustine's interpretation of the
image.

Therefore the ancient definition, when it says that sin is the lack
of righteousness, not only denies obedience with respect to man's
lower powers [that man is not only corrupt in his body and its
meanest and lowest faculties], but also denies the knowledge of God,
confidence in God, the fear and love of God, or certainly the power
to produce these affections [the light in the heart which creates a
love and desire for these matters].  For even the theologians
themselves teach in their schools that these are not produced without
certain gifts and the aid of grace.  In order that the matter may be
understood, we term these very gifts the knowledge of God, and fear
and confidence in God.  From these facts it appears that the ancient
definition says precisely the same thing that we say, denying fear
and confidence toward God, to wit, not only the acts, but also the
gifts and power to produce these acts [that we have no good heart
toward God, which truly loves God, not only that we are unable to do
or achieve any perfectly good work].

Of the same import is the definition which occurs in the writings of
Augustine, who is accustomed to define original sin as concupiscence
[wicked desire].  For he means that when righteousness had been lost,
concupiscence came in its place.  For inasmuch as diseased nature
cannot fear and love God and believe God, it seeks and loves carnal
things.  God's judgment it either contemns when at ease, or hates,
when thoroughly terrified.  Thus Augustine includes both the defect
and the vicious habit which has come in its place.  Nor indeed is
concupiscence only a corruption of the qualities of the body, but
also, in the higher powers, a vicious turning to carnal things.  Nor
do those persons see what they say who ascribe to man at the same
time concupiscence that is not entirely destroyed by the Holy Ghost,
and love to God above all things.

We, therefore, have been right in expressing, in our description of
original sin, both namely, these defects: the not being able to
believe God, the not being able to fear and love God; and, likewise:
the having concupiscence, which seeks carnal things contrary to God's
Word, i.e., seeks not only the pleasure of the body, but also carnal
wisdom and righteousness, and, contemning God, trusts in these as god
things.  Nor only the ancients [like Augustine and others], but also
the more recent [teachers and scholastics], at least the wiser ones
among them, teach that original sin is at the same time truly these
namely, the defects which I have recounted and concupiscence.  For
Thomas says thus: Original sin comprehends the loss of original
righteousness, and with this an inordinate disposition of the parts
of the soul; whence it is not pure loss, but a corrupt habit
[something positive].  And Bonaventura: When the question is asked,
What is original sin?  The correct answer is, that it is immoderate
[unchecked] concupiscence.  The correct answer is also, that it is
want of the righteousness that is due.  And in one of these replies
the other is included.  The same is the opinion of Hugo, when he says
that original sin is ignorance in the mind and concupiscence in the
flesh.  For he thereby indicates that when we are born, we bring with
us ignorance of God unbelief, distrust, contempt, and hatred of God.
For when he mentions ignorance, he includes these.  And these
opinions [even of the most recent teachers] also agree with Scripture.
For Paul sometimes expressly calls it a defect [a lack of divine
light], as 1 Cor. 2, 14: The natural man receiveth not the things of
the Spirit of God.  In another place, Rom. 7, 5, he calls it
concupiscence working in our members to bring forth fruit unto death.
We could cite more passages relating to both parts, but in regard to
a manifest fact there is no need of testimonies.  And the intelligent
reader will readily be able to decide that to be without the fear of
God and without faith are more than actual guilt.  For they are
abiding defects in our unrenewed nature.

In reference to original sin we therefore hold nothing differing
either from Scripture or from the Church catholic, but cleanse from
corruptions and restore to light most important declarations of
Scripture and of the Fathers, that had been covered over by the
sophistical controversies of modern theologians.  For it is manifest
from the subject itself that modern theologians have not noticed what
the Fathers meant when they spake of defect [lack of original
righteousness].  But the knowledge of original sin is necessary.  For
the magnitude of the grace of Christ cannot be understood [no one can
heartily long and have a desire for Christ for the inexpressibly
great treasure of divine favor and grace which the Gospel offers],
unless our diseases be recognized.  [As Christ says Matt. 9, 12; Mark
2, 17: They that are whole need not a physician.] The entire
righteousness of man is mere hypocrisy [and abomination] before God,
unless we acknowledge that our heart is naturally destitute of love,
fear, and confidence in God [that we are miserable sinners who are in
disgrace with God].  For this reason the prophet Jeremiah, 31, 19,
says: After that I was instructed, I smote upon my thigh.  Likewise
Ps. 116, 11: I said in my haste, All men are liars, i.e., not
thinking aright concerning God.

Here our adversaries inveigh against Luther also because he wrote
that, "Original sin remains after Baptism." They add that this
article was justly condemned by Leo X. But His Imperial Majesty will
find on this point a manifest slander.  For our adversaries know in
what sense Luther intended this remark that original sin remains
after Baptism.  He always wrote thus, namely, that Baptism removes
the guilt of original sin, although the material, as they call it, of
the sin, i.e., concupiscence, remains.  He also added in reference to
the material that the Holy Ghost, given through Baptism, begins to
mortify the concupiscence, and creates new movements [a new light, a
new sense and spirit] in man.  In the same manner, Augustine also
speaks who says: Sin is remitted in Baptism, not in such a manner
that it no longer exists, but so that it is not imputed.  Here he
confesses openly that sin exists, i.e., that it remains although it
is not imputed.  And this judgment was so agreeable to those who
succeeded him that it was recited also in the decrees.  Also against
Julian, Augustine says: The Law, which is in the members, has been
annulled by spiritual regeneration, and remains in the mortal flesh.
It has been annulled because the guilt has been remitted in the
Sacrament, by which believers are born again; but it remains, because
it produces desires against which believers contend.  Our adversaries
know that Luther believes and teaches thus, and while they cannot
reject the matter, they nevertheless pervert his words, in order by
this artifice to crush an innocent man.

But they contend that concupiscence is a penalty, and not a sin [a
burden and imposed penalty, and is not such a sin as is subject to
death and condemnation].  Luther maintains that it is a sin.  It has
been said above that Augustine defines original sin as concupiscence.
If there be anything disadvantageous in this opinion, let them
quarrel with Augustine.  Besides Paul says, Rom. 7, 7. 23: I had not
known lust (concupiscence), except the Law had said, Thou shalt not
covet.  Likewise: I see another law in my members, warring against
the law of my mind, and bringing me into captivity to the law of sin
which is in my members.  These testimonies can be overthrown by no
sophistry.  [All devils, all men cannot overthrow them.] For they
clearly call concupiscence sin, which, nevertheless, is not imputed
to those who are in Christ although by nature it is a matter worthy
of death where it is not forgiven.  Thus, beyond all controversy, the
Fathers believe.  For Augustine, in a long discussion refutes the
opinion of those who thought that concupiscence in man is not a fault
but an adiaphoron, as color of the body or ill health is said to be
an adiaphoron [as to have a black or a white body is neither good nor
evil].

But if the adversaries will contend that the fomes [or evil
inclination] is an adiaphoron, not only many passages of Scripture
but simply the entire Church [and all the Fathers] will contradict
them.  For [even if not entire consent, but only the inclination and
desire be there] who ever dared to say that these matters, even
though perfect agreement could not be attained, were adiaphora,
namely, to doubt concerning God's wrath,: concerning God's grace,
concerning God's Word, to be angry at the judgments of God, to be
provoked because God does not at once deliver one from afflictions,
to murmur because the wicked enjoy a better fortune than the good, to
be urged on by wrath, lust, the desire for glory, wealth, etc.?  And
yet godly men acknowledge these in themselves, as appears in the
Psalms and the prophets.  [For all tried, Christian hearts know, alas!
that these evils are wrapped up in man's skin, namely to esteem
money, goods, and all other matters more highly than God, and to
spend our lives in security; again, that after the manner of our
carnal security we always imagine that God's wrath against sin is not
as serious and great as it verily is.  Again, that we murmur against
the doing and will of God, when He does not succor us speedily in our
tribulations, and arranges our affairs to please us.  Again, we
experience every day that it hurts us to see wicked people in good
fortune in this world, as David and all the saints have complained.
Over and above this, all men feel that their hearts are easily
inflamed, now with ambition, now with anger and wrath, now with
lewdness.] But in the schools they transferred hither from philosophy
notions entirely different, that, because of passions, we are neither
good nor evil, we are neither deserving of praise nor blame.
Likewise, that nothing is sin, unless it be voluntary [inner desires
and thoughts are not sins, if I do not altogether consent thereto].
These notions were expressed among philosophers with respect to civil
righteousness, and not with respect to God's judgment.  [For there it
is true, as the jurists say, L. cogitationis, thoughts are exempt
from custom and punishment.  But God searches the hearts; in God's
court and judgment it is different.] With no greater prudence they
add also other notions, such as, that [God's creature and] nature is
not [cannot in itself be] evil.  In its proper place we do not
censure this; but it is not right to twist it into an extenuation of
original sin.  And, nevertheless, these notions are read in the works
of scholastics, who inappropriately mingle philosophy or civil
doctrine concerning ethics with the Gospel.  Nor were these matters
only disputed in the schools, but, as is usually the case, were
carried from the schools to the people.  And these persuasions
[godless, erroneous, dangerous, harmful teachings] prevailed, and
nourished confidence in human strength, and suppressed the knowledge
of Christ's grace.  Therefore, Luther wishing to declare the
magnitude of original sin and of human infirmity [what a grievous
mortal guilt original sin is in the sight of God], taught that these
remnants of original sin [after Baptism] are not, by their own nature,
adiaphora in man, but that, for their non-imputation, they need the
grace of Christ and, likewise for their mortification, the Holy Ghost.

Although the scholastics extenuate both sin and punishment when they
teach that man by his own strength, can fulfil the commandments of
God; in Genesis the punishment, imposed on account of original sin,
is described otherwise.  For there human nature is subjected not only
to death and other bodily evils, but also to the kingdom of the devil.
For there, Gen. 3, 16, this fearful sentence is proclaimed: I will
put enmity between thee and the woman, and between thy seed and her
seed.  The defects and the concupiscence are punishments and sins.
Death and other bodily evils and the dominion of the devil, are
properly punishments.  For human nature has been delivered into
slavery, and is held captive by the devil, who infatuates it with
wicked opinions and errors, and impels it to sins of every kind.  But
just as the devil cannot be conquered except by the aid of Christ, so
by our own strength we cannot free ourselves from this slavery.  Even
the history of the world shows how great is the power of the devil's
kingdom.  The world is full of blasphemies against God and of wicked
opinions, and the devil keeps entangled in these bands those who are
wise and righteous [many hypocrites who appear holy] in the sight of
the world.  In other persons grosser vices manifest themselves.  But
since Christ was given to us to remove both these sins and these
punishments, and to destroy the kingdom of the devil, sin and death,
it will not be possible to recognize the benefits of Christ unless we
understand our evils.  For this reason our preachers have diligently
taught concerning these subjects, and have delivered nothing that is
new but have set forth Holy Scripture and the judgments of the holy
Fathers.

We think that this will satisfy His Imperial Majesty concerning the
puerile and trivial sophistry with which the adversaries have
perverted our article.  For we know that we believe aright and in
harmony with the Church catholic of Christ.  But if the adversaries
will renew this controversy, there will be no want among us of those
who will reply and defend the truth.  For in this case our
adversaries, to a great extent, do not understand what they say.
They often speak what is contradictory, and neither explain correctly
and logically that which is essential to [i.e., that which is or is
not properly of the essence of] original sin, nor what they call
defects.  But we have been unwilling at this place to examine their
contests with any very great subtlety.  We have thought it worth
while only to recite, in customary and well-known words, the belief
of the holy Fathers, which we also follow.




PART 2


Article III: _Of Christ._

The Third Article the adversaries approve, in which we confess that
there are in Christ two natures, namely, a human nature, assumed by
the Word into the unity of His person; and that the same Christ
suffered and died to reconcile the Father to us; and that He was
raised again to reign, and to justify and sanctify believers, etc.,
according to the Apostles' Creed and the Nicene Creed.


Article IV (II): _Of Justification._

In the Fourth, Fifth, Sixth, and, below, in the Twentieth Article,
they condemn us, for teaching that men obtain remission of sins, not
because of their own merits, but freely for Christ's sake, through
faith in Christ.  [They reject quite stubbornly both these statements.
] For they condemn us both for denying that men obtain remission of
sins because of their own merits, and for affirming that, through
faith, men obtain remission of sins, and through faith in Christ are
justified.  But since in this controversy the chief topic of
Christian doctrine is treated, which, understood aright, illumines
and amplifies the honor of Christ [which is of especial service for
the clear, correct understanding of the entire Holy Scriptures, and
alone shows the way to the unspeakable treasure and right knowledge
of Christ, and alone opens the door to the entire Bible], and brings
necessary and most abundant consolation to devout consciences, we ask
His Imperial Majesty to hear us with forbearance in regard to matters
of such importance.  For since the adversaries understand neither
what the remission of sins, nor what faith, nor what grace, nor what
righteousness is, they sadly corrupt this topic, and obscure the
glory and benefits of Christ and rob devout consciences of the
consolations offered in Christ.  But that we may strengthen the
position of our Confession, and also remove the charges which the
adversaries advance against us, certain things are to be premised in
the beginning, in order that the sources of both kinds of doctrine, i.
e., both that of our adversaries and our own, may be known.

All Scripture ought to be distributed into these two principal topics,
the Law and the promises.  For in some places it presents the Law,
and in others the promise concerning Christ, namely, either when [in
the Old Testament] it promises that Christ will come, and offers, for
His sake, the remission of sins justification, and life eternal, or
when, in the Gospel [in the New Testament], Christ Himself, since He
has appeared, promises the remission of sins, justification, and life
eternal.  Moreover, in this discussion, by Law we designate the Ten
Commandments, wherever they are read in the Scriptures.  Of the
ceremonies and judicial laws of Moses we say nothing at present.

Of these two parts the adversaries select the Law, because human
reason naturally understands, in some way, the Law (for it has the
same judgment divinely written in the mind); [the natural law agrees
with the law of Moses, or the Ten Commandments] and by the Law they
seek the remission of sins and justification.  Now, the Decalog
requires not only outward civil works, which reason can in some way
produce, but it also requires other things placed far above reason,
namely, truly to fear God, truly to love God, truly to call upon God,
truly to be convinced that God hears us, and to expect the aid of God
in death and in all afflictions; finally, it requires obedience to
God, in death and all afflictions, so that we may not flee from these,
or refuse them when God imposes them.

Here the scholastics, having followed the philosophers, teach only a
righteousness of reason, namely, civil works, and fabricate besides
that without the Holy Ghost reason can love God above all things.
For, as long as the human mind is at ease, and does not feel the
wrath or judgment of God, it can imagine that it wishes to love God,
that it wishes to do good for God's sake.  [But it is sheer hypocrisy.
] In this manner they teach that men merit the remission of sins by
doing what is in them, i.e., if reason, grieving over sin, elicit an
act of love to God, or for God's sake be active in that which is good.
And because this opinion naturally flatters men, it has brought
forth and multiplied in the Church many services, monastic vows,
abuses of the mass; and, with this opinion the one has, in the course
of time, devised this act of worship and observances, the other that.
And in order that they might nourish and increase confidence in such
works, they have affirmed that God necessarily gives grace to one
thus working, by the necessity not of constraint, but of immutability
[not that He is constrained, but that this is the order which God
will not transgress or alter].

In this opinion there are many great and pernicious errors, which it
would be tedious to enumerate.  Let the discreet reader think only of
this: If this be Christian righteousness, what difference is there
between philosophy and the doctrine of Christ?  If we merit the
remission of sins by these elicit acts [that spring from our mind],
of what benefit is Christ?  If we can be justified by reason and the
works of reason, wherefore is there need of Christ or regeneration
[as Peter declares, 1 Pet. 1, 18 ff.]?  And from these opinions the
matter has now come to such a pass that many ridicule us because we
teach that an other than the philosophic righteousness must be sought
after.  [Alas! it has come to this, that even great theologians at
Louvain, Paris, etc., have known nothing of any other godliness or
righteousness (although every letter and syllable in Paul teaches
otherwise) than the godliness which philosophers teach.  And although
we ought to regard this as a strange teaching, and ought to ridicule
it, they rather ridicule us, yea, make a jest of Paul himself.] We
have heard that some, after setting aside the Gospel, have, instead
of a sermon, explained the ethics of Aristotle.  [I myself have heard
a great preacher who did not mention Christ and the Gospel, and
preached the ethics of Aristotle.  Is this not a childish, foolish
way to preach to Christians?] Nor did such men err if those things
are true which the adversaries defend [if the doctrine of the
adversaries be true, the Ethics is a precious book of sermons, and a
fine new Bible].  For Aristotle wrote concerning civil morals so
learnedly that nothing further concerning this need be demanded.  We
see books extant in which certain sayings of Christ are compared with
the sayings of Socrates, Zeno, and others, as though Christ had come
for the purpose of delivering certain laws through which we might
merit the remission of sins, as though we did not receive this
gratuitously, because of His merits.  Therefore, if we here receive
the doctrine of the adversaries, that by the works of reason we merit
the remission of sins and justification, there will be no difference
between philosophic, or certainly pharisaic, and Christian
righteousness.

Although the adversaries, not to pass by Christ altogether, require a
knowledge of the history concerning Christ, and ascribe to Him that
it is His merit that a habit is given us, or, as they say, _prima
gratia_, "first grace," which they understand as a habit, inclining
us the more readily to love God; yet what they ascribe to this habit
is of little importance [is a feeble, paltry, small, poor operation,
that would be ascribed to Christ], because they imagine that the acts
of the will are of the same kind before and after this habit.  They
imagine that the will can love God; but nevertheless this habit
stimulates it to do the same the more cheerfully.  And they bid us
first merit this habit by preceding merits; then they bid us merit by
the works of the Law an increase of this habit and life eternal.
Thus they bury Christ, so that men may not avail themselves of Him as
a Mediator, and believe that for His sake they freely receive
remission of sins and reconciliation, but may dream that by their own
fulfilment of the Law they merit the remission of sins, and that by
their own fulfilment of the Law they are accounted righteous before
God; while, nevertheless, the Law is never satisfied, since reason
does nothing except certain civil works, and, in the mean time
neither [in the heart] fears God, nor truly believes that God cares
for it.  And although they speak of this habit, yet, without the
righteousness of faith, neither the love of God can exist in man, nor
can it be understood what the love of God is.

Their feigning a distinction between _meritum congrui_ and _meritum
condigni_ [due merit and true, complete merit] is only an artifice in
order not to appear openly to Pelagianize, For, if God necessarily
gives grace for the _meritum congrui_ [due merit], it is no longer
_meritum congrui_, but _meritum condigni_ [a true duty and complete
merit].  But they do not know what they are saying.  After this habit
of love [is there], they imagine that man can acquire _merit de
condigno_.  And yet they bid us doubt whether there be a habit
present.  How, therefore, do they know whether they acquire merit _de
congruo_ or _de condigno_ [in full, or half]?  But this whole matter
was fabricated by idle men [But, good God! these are mere inane ideas
and dreams of idle, wretched, inexperienced men who do not much
reduce the Bible to practise], who did not know how the remission of
sins occurs, and how, in the judgment of God and terrors of
conscience, trust in works is driven out of us.  Secure hypocrites
always judge that they acquire _merit de condigno_, whether the habit
be present or be not present, because men naturally trust in their
own righteousness, but terrified consciences waver and hesitate, and
then seek and accumulate other works in order to find rest.  Such
consciences never think that they acquire merit _de condigno_, and
they rush into despair unless they hear, in addition to the doctrine
of the Law, the Gospel concerning the gratuitous remission of sins
and the righteousness of faith.  [Thus some stories are told that
when the Barefooted monks had in vain praised their order and good
works to some good consciences in the hour of death, they at last had
to be silent concerning their order and St. Franciscus, and to say:
"Dear man, Christ has died for you." This revived and refreshed in
trouble, and alone gave peace and comfort.]

Thus the adversaries teach nothing but the righteousness of reason,
or certainly of the Law, upon which they look just as the Jews upon
the veiled face of Moses, and, in secure hypocrites who think that
they satisfy the Law, they excite presumption and empty confidence in
works [they place men on a sand foundation, their own works] and
contempt of the grace of Christ.  On the contrary, they drive timid
consciences to despair, which, laboring with doubt, never can
experience what faith is, and how efficacious it is; thus, at last
they utterly despair.

Now, we think concerning the righteousness of reason thus, namely,
that God requires it, and that, because of God's commandment, the
honorable works which the Decalog commands must necessarily be
performed, according to the passage Gal. 3, 24: The Law was our
schoolmaster; likewise 1 Tim. 1, 9: The Law is made for the ungodly.
For God wishes those who are carnal [gross sinners] to be restrained
by civil discipline, and to maintain this, He has given laws, letters,
doctrine, magistrates, penalties.  And this righteousness reason, by
its own strength, can, to a certain extent, work, although it is
often overcome by natural weakness, and by the devil impelling it to
manifest crimes.  Now, although we cheerfully assign this
righteousness of reason the praises that are due it (for this corrupt
nature has no greater good [in this life and in a worldly nature,
nothing is ever better than uprightness and virtue], and Aristotle
says aright: Neither the evening star nor the morning star is more
beautiful than righteousness, and God also honors it with bodily
rewards), yet it ought not to be praised with reproach to Christ.

For it is false [I thus conclude, and am certain that it is a fiction,
and not true] that we merit the remission of sins by our works.

False also is this, that men are accounted righteous before God
because of the righteousness of reason [works and external piety].

False also is this that reason, by its own strength, is able to love
God above all things, and to fulfil God's Law, namely, truly to fear
God to be truly confident that God hears prayer, to be willing to
obey God in death and other dispensations of God, not to covet what
belongs to others, etc.; although reason can work civil works.

False also and dishonoring Christ is this, that men do not sin who,
without grace, do the commandments of God [who keep the commandments
of God merely in an external manner, without the Spirit and grace in
their hearts].  We have testimonies for this our belief, not only
from the Scriptures, but also from the Fathers.  For in opposition to
the Pelagians, Augustine contends at great length that grace is not
given because of our merits.  And in _De Natura et Gratia_ he says:
If natural ability, through the free will, suffice both for learning
to know how one ought to live and for living aright, then Christ has
died in vain, then the offense of the Cross is made void.  Why may I
not also here cry out?  Yea I will cry out, and, with Christian grief,
will chide them: Christ has become of no effect unto you whosoever
of you are justified by the Law; ye are fallen from grace.  Gal. 5, 4;
cf. 2, 21. For they, being ignorant of God's righteousness, and
going about to establish their own righteousness, have not submitted
themselves unto the righteousness of God.  For Christ is the end of
the Law for righteousness to every one that believeth.  Rom. 10 3. 4.
And John 8, 36: If the Son therefore shall make you free, ye shall be
free indeed.  Therefore by reason we cannot be freed from sins and
merit the remission of sins.  And in John 3, 5 it is written: Except
a man be born of water and of the Spirit, he cannot enter into the
kingdom of God.  But if it is necessary to be born again of the Holy
Ghost the righteousness of reason does not justify us before God, and
does not fulfil the Law, Rom. 3, 23: All have come short of the glory
of God, i.e., are destitute of the wisdom and righteousness of God,
which acknowledges and glorifies God.  Likewise Rom. 8, 7. 8: The
carnal mind is enmity against God; for it is not subject to the Law
of God, neither indeed can be.  So then they that are in the flesh
cannot please God.  These testimonies are so manifest that, to use
the words of Augustine which he employed in this case, they do not
need an acute understanding, but only an attentive hearer.  If the
carnal mind is enmity against God, the flesh certainly does not love
God; if it cannot be subject to the Law of God, it cannot love God.
If the carnal mind is enmity against God, the flesh sins even when we
do external civil works.  If it cannot be subject to the Law of God,
it certainly sins even when, according to human judgment, it
possesses deeds that are excellent and worthy of praise.  The
adversaries consider only the precepts of the Second Table which
contain civil righteousness that reason understands.  Content with
this, they think that they satisfy the Law of God.  In the mean time
they do not see the First Table which commands that we love God, that
we declare as certain that God is angry with sin, that we truly fear
God, that we declare as certain that God hears prayer.  But the human
heart without the Holy Ghost either in security despises God's
judgment, or in punishment flees from, and hates, God when He judges.
Therefore it does not obey the First Table.  Since, therefore,
contempt of God, and doubt concerning the Word of God and concerning
the threats and promises, inhere in human nature, men truly sin, even
when, without the Holy Ghost, they do virtuous works, because they do
them with a wicked heart, according to Rom. 14, 23: Whatsoever is not
of faith is sin.  For such persons perform their works with contempt
of God, just as Epicurus does not believe that God cares for him, or
that he is regarded or heard by God.  This contempt vitiates works
seemingly virtuous, because God judges the heart.

Lastly, it was very foolish for the adversaries to write that men who
are under eternal wrath merit the remission of sins by an act of love,
which springs from their mind, since it is impossible to love God,
unless the remission of sins be apprehended first by faith.  For the
heart, truly feeling that God is angry, cannot love God, unless He be
shown to have been reconciled.  As long as He terrifies us, and seems
to cast us into eternal death, human nature is not able to take
courage, so as to love a wrathful, judging, and punishing God [poor,
weak nature must lose heart and courage, and must tremble before such
great wrath, which so fearfully terrifies and punishes, and can never
feel a spark of love before God Himself comforts].  It is easy for
idle men to feign such dreams concerning love as, that a person
guilty of mortal sin can love God above all things, because they do
not feel what the wrath or judgment of God is.  But in agony of
conscience and in conflicts [with Satan] conscience experiences the
emptiness of these philosophical speculations.  Paul says, Rom. 4,15:
The Law worketh wrath.  He does not say that by the Law men merit the
remission of sins.  For the Law always accuses and terrifies
consciences.  Therefore it does not justify, because conscience
terrified by the Law flees from the judgment of God.  Therefore they
err who trust that by the Law, by their own works, they merit the
remission of sins.  It is sufficient for us to have said these things
concerning the righteousness of reason or of the Law, which the
adversaries teach.  For after a while, when we will declare our
belief concerning the righteousness of faith, the subject itself will
compel us to adduce more testimonies, which also will be of service
in overthrowing the errors of the adversaries which we have thus far
reviewed.

Because, therefore, men by their own strength cannot fulfil the Law
of God, and all are under sin, and subject to eternal wrath and death,
on this account we cannot be freed by the Law from sin and be
justified but the promise of the remission of sins and of
justification has been given us for Christ's sake, who was given for
us in order that He might make satisfaction for the sins of the world,
and has been appointed as the [only] Mediator and Propitiator.  And
this promise has not the condition of our merits [it does not read
thus: Through Christ you have grace salvation, etc., if you merit it],
but freely offers the remission of sins and justification, as Paul
says, Rom. 11, 6: If it be of works, then is it no more grace.  And
in another place, Rom. 3, 21: The righteousness of God without the
Law is manifested, i.e., the remission of sins is freely offered.
Nor does reconciliation depend upon our merits.  Because, if the
remission of sins were to depend upon our merits, and reconciliation
were from the Law, it would be useless.  For, as we do not fulfil the
Law, it would also follow that we would never obtain the promise of
reconciliation.  Thus Paul reasons, Rom. 4, 14: For if they which are
of the Law be heirs, faith is made void, and the promise made of none
effect.  For if the promise would require the condition of our merits
and the Law, which we never fulfil, it would follow that the promise
would be useless.

But since justification is obtained through the free promise, it
follows that we cannot justify ourselves.  Otherwise, wherefore would
there be need to promise?  [And why should Paul so highly extol and
praise grace?] For since the promise cannot be received except by
faith, the Gospel, which is properly the promise of the remission of
sins and of justification for Christ's sake, proclaims the
righteousness of faith in Christ, which the Law does not teach.  Nor
is this the righteousness of the Law.  For the Law requires of us our
works and our perfection.  But the Gospel freely offers, for Christ's
sake, to us, who have been vanquished by sin and death,
reconciliation, which is received, not by works, but by faith alone.
This faith brings to God not confidence in one's own merits, but only
confidence in the promise, or the mercy promised in Christ.  This
special faith, therefore, by which an individual believes that for
Christ's sake his sins are remitted him, and that for Christ's sake
God is reconciled and propitious, obtains remission of sins and
justifies us.  And because in repentance, i.e. in terrors, it
comforts and encourages hearts it regenerates us, and brings the Holy
Ghost that then we may be able to fulfil God's Law, namely, to love
God, truly to fear God, truly to be confident that God hears prayer,
and to obey God in all afflictions; it mortifies concupiscence, etc.
Thus, because faith, which freely receives the remission of sins,
sets Christ, the Mediator and Propitiator, against God's wrath, it
does not present our merits or our love [which would be tossed aside
like a little feather by a hurricane].  This faith is the true
knowledge of Christ, and avails itself of the benefits of Christ, and
regenerates hearts, and precedes the fulfilling of the Law.  And of
this faith not a syllable exists in the doctrine of our adversaries.
Hence we find fault with the adversaries, equally because they teach
only the righteousness of the Law and because they do not teach the
righteousness of the Gospel, which proclaims the righteousness of
faith in Christ.




Part 3


_What Is Justifying Faith?_

The adversaries feign that faith is only a knowledge of the history,
and therefore teach that it can coexist with mortal sin.  Hence they
say nothing concerning faith, by which Paul so frequently says that
men are justified, because those who are accounted righteous before
God do not live in mortal sin.  But that faith which justifies is not
merely a knowledge of history, [not merely this, that I know the
stories of Christ's birth, suffering, etc. (that even the devils know,
)] but it is to assent to the promise of God, in which for Christ's
sake, the remission of sins and justification are freely offered.
[It is the certainty or the certain trust in the heart, when, with my
whole heart, I regard the promises of God as certain and true,
through which there are offered me, without my merit, the forgiveness
of sins, grace, and all salvation, through Christ the Mediator.] And
that no one may suppose that it is mere knowledge we will add further:
it is to wish and to receive the offered promise of the remission of
sins and of justification.  [Faith is that my whole heart takes to
itself this treasure.  It is not my doing, not my presenting or
giving, not my work or preparation, but that a heart comforts itself,
and is perfectly confident with respect to this, namely, that God
makes a present and gift to us, and not we to Him, that He sheds upon
us every treasure of grace in Christ.]

And the difference between this faith and the righteousness of the
Law can be easily discerned.  Faith is the _latreia_ [divine service],
which receives the benefits offered by God; the righteousness of the
Law is the _latreia_ [divine service] which offers to God our merits.
By faith God wishes to be worshiped in this way, that we receive
from Him those things which He promises and offers.

Now, that faith signifies, not only a knowledge of the history, but
such faith as assents to the promise, Paul plainly testifies when he
says, Rom. 4, 16: Therefore it is of faith, to the end the promise
might be sure.  For he judges that the promise cannot be received
unless by faith.  Wherefore he puts them together as things that
belong to one another, and connects promise and faith.  [There Paul
fastens and binds together these two, thus: Wherever there is a
promise faith is required and conversely, wherever faith is required
there must be a promise.] Although it will be easy to decide what
faith is if we consider the Creed where this article certainly stands:
The forgiveness of sins.  Therefore it is not enough to believe that
Christ was born, suffered, was raised again, unless we add also this
article, which is the purpose of the history: The forgiveness of sins.
To this article the rest must be referred, namely, that for
Christ's sake, and not for the sake of our merits, forgiveness of
sins is given us.  For what need was there that Christ was given for
our sins if for our sins our merits can make satisfaction?

As often, therefore, as we speak of justifying faith, we must keep in
mind that these three objects concur: the promise, and that, too,
gratuitous, and the merits of Christ, as the price and propitiation.
The promise is received by faith; the "gratuitous" excludes our
merits, and signifies that the benefit is offered only through mercy;
the merits of Christ are the price, because there must be a certain
propitiation for our sins.  Scripture frequently implores mercy, and
the holy Fathers often say that we are saved by mercy.  As often,
therefore, as mention is made of mercy, we must keep in mind that
faith is there required, which receives the promise of mercy.  And,
again, as often as we speak of faith, we wish an object to be
understood, namely, the promised mercy.  For faith justifies and
saves, not on the ground that it is a work in itself worthy, but only
because it receives the promised mercy.

And throughout the prophets and the psalms this worship, this
_latreia_, is highly praised, although the Law does not teach the
gratuitous remission of sins.  But the Fathers knew the promise
concerning Christ that God for Christ's sake wished to remit sins.
Therefore, since they understood that Christ would be the price for
our sins, they knew that our works are not a price for so great a
matter [could not pay so great a debt].  Accordingly, they received
gratuitous mercy and remission of sins by faith, just as the saints
in the New Testament.  Here belong those frequent repetitions
concerning mercy and faith, in the psalms and the prophets, as this,
Ps. 130, 3 sq.: If Thou Lord, shouldest mark iniquities, O Lord, who
shall stand?  Here David confesses his sins and does not recount his
merits.  He adds; But there is forgiveness with Thee.  Here he
comforts himself by his trust in God's mercy, and he cites the
promise: My soul doth wait and in His Word do I hope, i.e., because
Thou hast promised the remission of sins, I am sustained by this Thy
promise.  Therefore the fathers also were justified, not by the Law
but by the promise and faith.  And it is amazing that the adversaries
extenuate faith to such a degree, although they see that it is
everywhere praised as an eminent service, as in Ps. 50, 15: Call upon
Me in the day of trouble: I will deliver thee.  Thus God wishes
Himself to be known, thus He wishes Himself to be worshiped, that
from Him we receive benefits, and receive them, too, because of His
mercy, and not because of our merits.  This is the richest
consolation in all afflictions [physical or spiritual, in life or in
death as all godly persons know].  And such consolations the
adversaries abolish when they extenuate and disparage faith, and
teach only that by means of works and merits men treat with God [that
we treat with God, the great Majesty, by means of our miserable,
beggarly works and merits].




Part 4


_That Faith in Christ Justifies._

In the first place, lest any one may think that we speak concerning
an idle knowledge of the history, we must declare how faith is
obtained [how the heart begins to believe].  Afterward we will show
both that it justifies, and how this ought to be understood, and we
will explain the objections of the adversaries.  Christ, in the last
chapter of Luke 24, 47, commands that repentance and remission of
sins should be preached in His name.  For the Gospel convicts all men
that they are under sin, that they all are subject to eternal wrath
and death, and offers for Christ's sake remission of sin and
justification, which is received by faith.  The preaching of
repentance, which accuses us, terrifies consciences with true and
grave terrors.  [For the preaching of repentance, or this declaration
of the Gospel: Amend your lives!  Repent!  When it truly penetrates
the heart, terrifies the conscience, and is no jest, but a great
terror, in which the conscience feels its misery and sin and the
wrath of God.] In these, hearts ought again to receive consolation.
This happens if they believe the promise of Christ, that for His sake
we have remission of sins.  This faith, encouraging and consoling in
these fears, receives remission of sins, justifies and quickens.  For
this consolation is a new and spiritual life [a new birth and a new
life].  These things are plain and clear, and can be understood by
the pious, and have testimonies of the Church [as is to be seen in
the conversion of Paul and Augustine].  The adversaries nowhere can
say how the Holy Ghost is given.  They imagine that the Sacraments
confer the Holy Ghost _ex opere operato_, without a good emotion in
the recipient, as though, indeed, the gift of the Holy Ghost were an
idle matter.

But since we speak of such faith as is not an idle thought, but of
that which liberates from death and produces a new life in hearts
[which is such a new light, life, and force in the heart as to renew
our heart, mind, and spirit, makes new men of us and new creatures,]
and is the work of the Holy Ghost; this does not coexist with mortal
sin [for how can light and darkness coexist?], but as long as it is
present, produces good fruits as we will say after a while.  For
concerning the conversion of the wicked, or concerning the mode of
regeneration, what can be said that is more simple and more clear?
Let them, from so great an array of writers, adduce a single
commentary upon the Sententiae that speaks of the mode of
regeneration.  When they speak of the habit of love, they imagine
that men merit it through works and they do not teach that it is
received through the Word, precisely as also the Anabaptists teach at
this time.  But God cannot be treated with, God cannot be apprehended,
except through the Word.  Accordingly, justification occurs through
the Word, just as Paul says, Rom. 1, 16: The Gospel is the power of
God unto salvation to every one that believeth.  Likewise 10, 17:
Faith cometh by hearing.  And proof can be derived even from this
that faith justifies, because, if justification occurs only through
the Word, and the Word is apprehended only by faith, it follows that
faith justifies.  But there are other and more important reasons.  We
have said these things thus far in order that we might show the mode
of regeneration, and that the nature of faith [what is, or is not,
faith], concerning which we speak, might be understood.

Now we will show that faith [and nothing else] justifies.  Here, in
the first place readers must be admonished of this, that just as it
is necessary to maintain this sentence: Christ is Mediator, so is it
necessary to defend that faith justifies, [without works].  For how
will Christ be Mediator if in justification we do not use Him as
Mediator; if we do not hold that for His sake we are accounted
righteous?  But to believe is to trust in the merits of Christ, that
for His sake God certainly wishes to be reconciled with us.  Likewise,
just as we ought to maintain that, apart from the Law, the promise
of Christ is necessary, so also is it needful to maintain that faith
justifies.  [For the Law does not preach the forgiveness of sin by
grace.] For the Law cannot be performed unless the Holy Ghost be
first received.  It is, therefore, needful to maintain that the
promise of Christ is necessary.  But this cannot be received except
by faith.  Therefore, those who deny that faith justifies, teach
nothing but the Law, both Christ and the Gospel being set aside.

But when it is said that faith justifies, some perhaps understand it
of the beginning, namely, that faith is the beginning of
justification or preparation for justification, so that not faith
itself is that through which we are accepted by God, but the works
which follow; and they dream, accordingly, that faith is highly
praised, because it is the beginning.  For great is the importance of
the beginning, as they commonly say, _Archae aemioy pantos_, The
beginning is half of everything; just as if one would say that
grammar makes the teachers of all arts, because it prepares for other
arts, although in fact it is his own art that renders every one an
artist.  We do not believe thus concerning faith, but we maintain
this, that properly and truly, by faith itself, we are for Christ's
sake accounted righteous, or are acceptable to God.  And because "to
be justified" means that out of unjust men just men are made, or born
again, it means also that they are pronounced or accounted just.  For
Scripture speaks in both ways.  [The term "to be justified" is used
in two ways: to denote, being converted or regenerated; again, being
accounted righteous.] Accordingly we wish first to show this, that
faith alone makes of an unjust, a just man, i.e., receives remission
of sins.

The particle alone offends some, although even Paul says, Rom. 3, 28:
We conclude that a man is justified by faith, without the deeds of
the Law.  Again, Eph. 2, 8: It is the gift of God; not of works, lest
any man should boast.  Again, Rom. 3, 24: Being justified freely.  If
the exclusive alone displeases, let them remove from Paul also the
exclusives freely, not of works, it is the gift, etc. For these also
are [very strong] exclusives.  It is, however, the opinion of merit
that we exclude.  We do not exclude the Word or Sacraments, as the
adversaries falsely charge us.  For we have said above that faith is
conceived from the Word, and we honor the ministry of the Word in the
highest degree.  Love also and works must follow faith.  Wherefore,
they are not excluded so as not to follow, but confidence in the
merit of love or of works is excluded in justification.  And this we
will clearly show.




Part 5


_That We Obtain Remission of Sins by Faith Alone in Christ._

We think that even the adversaries acknowledge that, in justification,
the remission of sins is necessary first.  For we all are under sin.
Wherefore we reason thus:-To attain the remission of sins is to be
justified, according to Ps. 32, 1: Blessed is he whose transgression
is forgiven.  By faith alone in Christ, not through love, not because
of love or works, do we acquire the remission of sins, although love
follows faith.  Therefore by faith alone we are justified,
understanding justification as the making of a righteous man out of
an unrighteous, or that he be regenerated.

It will thus become easy to declare the minor premise [that we obtain
forgiveness of sin by faith, not by love] if we know how the
remission of sins occurs.  The adversaries with great indifference
dispute whether the remission of sins and the infusion of grace are
the same change [whether they are one change or two].  Being idle men,
they did not know what to answer [cannot speak at all on this
subject].  In the remission of sins, the terrors of sin and of
eternal death, in the heart, must be overcome, as Paul testifies, 1
Cor. 15, 56 sq.: The sting of death is sin, and the strength of sin
is the Law.  But thanks be to God, which giveth us the victory
through our Lord Jesus Christ.  That is, sin terrifies consciences,
this occurs through the Law, which shows the wrath of God against sin;
but we gain the victory through Christ.  How?  By faith, when we
comfort ourselves by confidence in the mercy promised for Christ's
sake.  Thus, therefore we prove the minor proposition.  The wrath of
God cannot be appeased if we set against it our own works, because
Christ has been set forth as a Propitiator, so that, for His sake,
the Father may become reconciled to us.  But Christ is not
apprehended as a Mediator except by faith.  Therefore, by faith alone
we obtain remission of sins when we comfort our hearts with
confidence in the mercy promised for Christ's sake.  Likewise Paul,
Rom. 5, 2, says: By whom also we have access, and adds, by faith.
Thus, therefore, we are reconciled to the Father, and receive
remission of sins when we are comforted with confidence in the mercy
promised for Christ's sake.  The adversaries regard Christ as
Mediator and Propitiator for this reason, namely, that He has merited
the habit of love; they do not urge us to use Him now as Mediator,
but, as though Christ were altogether buried, they imagine that we
have access through our own works, and, through these, merit this
habit and afterwards, by this love, come to God.  Is not this to bury
Christ altogether, and to take away the entire doctrine of faith?
Paul, on the contrary, teaches that we have access, i.e.,
reconciliation, through Christ.  And to show how this occurs, he adds
that we have access by faith.  By faith, therefore, for Christ's sake,
we receive remission of sins.  We cannot set our own love and our
own works over against God's wrath.

Secondly.  It is certain that sins are forgiven for the sake of
Christ, as Propitiator, Rom. 3, 25: Whom God hath set forth to be a
propitiation.  Moreover, Paul adds: through faith.  Therefore this
Propitiator thus benefits us, when by faith we apprehend the mercy
promised in Him, and set it against the wrath and judgment of God.
And to the same effect it is written, Heb. 4, 14. 16: Seeing, then,
that we have a great High Priest, etc., let us therefore come with
confidence.  For the Apostle bids us come to God, not with confidence
in our own merits, but with confidence in Christ as a High Priest;
therefore he requires faith.

Thirdly.  Peter, in Acts 10, 43, says: To Him give all the prophets
witness that through His name, whosoever believeth on Him, shall
receive remission of sins.  How could this be said more clearly?  We
receive remission of sins, he says, through His name i.e., for His
sake; therefore, not for the sake of our merits, not for the sake of
our contrition, attrition, love, worship, works.  And he adds: When
we believe in Him.  Therefore he requires faith.  For we cannot
apprehend the name of Christ except by faith.  Besides he cites the
agreement of all the prophets.  This is truly to cite the authority
of the Church.  [For when all the holy prophets bear witness, that is
certainly a glorious, great excellent, powerful decretal and
testimony.] But of this topic we will speak again after a while, when
treating of "Repentance."

Fourthly.  Remission of sins is something promised for Christ's sake.
Therefore it cannot be received except by faith alone.  For a
promise cannot be received except by faith alone.  Rom. 4, 16:
Therefore it is of faith that it might be by grace, to the end that
the promise might be sure; as though he were to say: "If the matter
were to depend upon our merits, the promise would be uncertain and
useless, because we never could determine when we would have
sufficient merit." And this, experienced consciences can easily
understand [and would not, for a thousand worlds, have our salvation
depend upon ourselves].  Accordingly, Paul says, Gal. 3, 22: But the
Scripture hath concluded all under sin, that the promise by faith of
Jesus Christ might be given to them that believe.  He takes merit
away from us, because he says that all are guilty and concluded under
sin; then he adds that the promise, namely, of the remission of sins
and of justification, is given, and adds how the promise can be
received, namely, by faith.  And this reasoning, derived from the
nature of a promise, is the chief reasoning [a veritable rock] in
Paul, and is often repeated.  Nor can anything be devised or imagined
whereby this argument of Paul can be overthrown.  Wherefore let not
good minds suffer themselves to be forced from the conviction that we
receive remission of sins for Christ's sake, only through faith.  In
this they have sure and firm consolation against the terrors of sin,
and against eternal death and against all the gates of hell.
[Everything else is a foundation of sand that sinks in trials.]

But since we receive remission of sins and the Holy Ghost by faith
alone, faith alone justifies, because those reconciled are accounted
righteous and children of God, not on account of their own purity,
but through mercy for Christ's sake, provided only they by faith
apprehend this mercy.  Accordingly, Scripture testifies that by faith
we are accounted righteous, Rom. 3, 26. We, therefore, will add
testimonies which clearly declare that faith is that very
righteousness by which we are accounted righteous before God, namely,
not because it is a work that is in itself worthy, but because it
receives the promise by which God has promised that for Christ's sake
He wishes to be propitious to those believing in Him, or because He
knows that Christ of God is made unto us wisdom, and righteousness,
and sanctification, and redemption, 1 Cor. 1, 30.

In the Epistle to the Romans, Paul discusses this topic especially,
and declares that, when we believe that God, for Christ's sake is
reconciled to us, we are justified freely by faith.  And this
proposition, which contains the statement of the entire discussion
[the principal matter of all Epistles, yea, of the entire Scriptures],
he maintains in the third chapter: We conclude that a man is
justified by faith, without the deeds of the Law, Rom. 3, 28. Here
the adversaries interpret that this refers to Levitical ceremonies
[not to other virtuous works].  But Paul speaks not only of the
ceremonies, but of the whole Law.  For he quotes afterward (7, 7)
from the _Decalog_: Thou shalt not covet.  And if moral works [that
are not Jewish ceremonies] would merit the remission of sins and
justification, there would also be no need of Christ and the promise,
and all that Paul speaks of the promise would be overthrown.  He
would also have been wrong in writing to the Ephesians, 2, 8: By
grace are ye saved through faith, and that not of yourselves; it is
the gift of God, not of works.  Paul likewise refers to Abraham and
David, Rom. 4, 1. 6. But they had the command of God concerning
circumcision.  Therefore, if any works justified these works must
also have justified at the time that they had a command.  But
Augustine teaches correctly that Paul speaks of the entire Law, as he
discusses at length in his book, Of the Spirit and Letter, where he
says finally: These matters, therefore, having been considered and
treated, according to the ability that the Lord has thought worthy to
give us, we infer that man is not justified by the precepts of a good
life, but by faith in Jesus Christ.

And lest we may think that the sentence that faith justifies, fell
from Paul inconsiderately, he fortifies and confirms this by a long
discussion in the fourth chapter to the Romans, and afterwards
repeats it in all his epistles.  Thus he says, Rom. 4, 4. 5: To him
that worketh is the reward not reckoned of grace, but of debt.  But
to him that worketh not, but believeth on Him that justifieth the
ungodly, his faith is counted for righteousness.  Here he clearly
says that faith itself is imputed for righteousness.  Faith,
therefore, is that thing which God declares to be righteousness, and
he adds that it is imputed freely, and says that it could not be
imputed freely, if it were due on account of works.  Wherefore he
excludes also the merit of moral works [not only Jewish ceremonies,
but all other good works].  For if justification before God were due
to these, faith would not be imputed for righteousness without works.
And afterwards, Rom. 4, 9: For we say that faith was reckoned to
Abraham for righteousness.  Chapter 5, 1 says: Being justified by
faith, we have peace with God, i.e., we have consciences that are
tranquil and joyful before God.  Rom. 10, 10: With the heart man
believeth unto righteousness.  Here he declares that faith is the
righteousness of the heart.  Gal. 2, 15: We have believed in Christ
Jesus that we might be justified by the faith of Christ, and not by
the works of the Law.  Eph. 2, 8. For by grace are ye saved through
faith, and that not of yourselves; it is the gift of God; not of
works, lest any man should boast.

John 1, 12: To them gave He power to become the sons of God, even to
them that believe on His name; which were born, not of blood, nor of
the will of the flesh nor of the will of man, but of God.  John 3, 14.
15: As Moses lifted up the serpent in the wilderness, even so must
the Son of Man be lifted up, that whosoever believeth in Him should
not perish.  Likewise, v. 17: For God sent not His Son into the world
to condemn the world, but that the world through Him might be saved.
He that believeth on Him is not condemned.

Acts 13, 38. 39: Be it known unto you therefore, men and brethren,
that through this Man is preached unto you the forgiveness of sins;
and by Him all that believe are justified from all things from which
ye could not be justified by the Law of Moses.  How could the office
of Christ and justification be declared more clearly?  The Law, he
says, did not justify.  Therefore Christ was given, that we may
believe that for His sake we are justified.  He plainly denies
justification to the Law.  Hence, for Christ's sake we are accounted
righteous when we believe that God, for His sake, has been reconciled
to us.  Acts 4, 11. 12: This is the stone which was set at naught of
you builders, which is become the head of the corner.  Neither is
there salvation in any other; for there is none other name under
heaven given among men whereby we must be saved.  But the name of
Christ is apprehended only by faith.  [I cannot believe in the name
of Christ in any other way than when I hear His merit preached, and
lay hold of that.] Therefore, by confidence in the name of Christ,
and not by confidence in our works, we are saved.  For "the name"
here signifies the cause which is mentioned because of which
salvation is attained.  And to call upon the name of Christ is to
trust in the name of Christ, as the cause or price because of which
we are saved.  Acts 15, 9: Purifying their hearts by faith.
Wherefore that faith of which the Apostles speak is not idle
knowledge, but a reality, receiving the Holy Ghost and justifying us
[not a mere knowledge of history, but a strong powerful work of the
Holy Ghost, which changes hearts].

Hab. 2, 4: The just shall live by his faith.  Here he says, first
that men are just by faith by which they believe that God is
propitious and he adds that the same faith quickens, because this
faith produces in the heart peace and joy and eternal life [which
begins in the present life].

Is. 53, 11: By His knowledge shall He justify many.  But what is the
knowledge of Christ unless to know the benefits of Christ, the
promises which by the Gospel He has scattered broadcast in the world?
And to know these benefits is properly and truly to believe in
Christ, to believe that that which God has promised for Christ's sake
He will certainly fulfil.

But Scripture is full of such testimonies, since, in some places, it
presents the Law, and in others the promises concerning Christ, and
the remission of sins, and the free acceptance of the sinner for
Christ's sake.

Here and there among the Fathers similar testimonies are extant.  For
Ambrose says in his letter to a certain Irenaeus: Moreover, the world
was subject to him by the Law for the reason that, according to the
command of the Law, all are indicted, and yet, by the works of the
Law, no one is justified, i.e., because, by the Law, sin is perceived,
but guilt is not discharged.  The Law, which made all sinners,
seemed to have done injury, but when the Lord Jesus Christ came, He
forgave to all sin which no one could avoid, and, by the shedding of
His own blood, blotted out the handwriting which was against us.
This is what he says in Rom. 5, 20: "The Law entered that the offense
might abound.  But where sin abounded, grace did much more abound."
Because after the whole world become subject, He took away the sin of
the whole world, as he [John] testified, saying, John 1, 29: "Behold
the Lamb of God, which taketh away the sin of the world." And on this
account let no one boast of works, because no one is justified by his
deeds.  But he who is righteous has it given him because he was
justified after the laver [of Baptism].  Faith, therefore, is that
which frees through the blood of Christ, because he is blessed "whose
transgression is forgiven, whose sin is covered," Ps. 32, 1. These
are the words of Ambrose, which clearly favor our doctrine; he denies
justification to works, and ascribes to faith that it sets us free
through the blood of Christ.  Let all the Sententiarists, who are
adorned with magnificent titles, be collected into one heap.  For
some are called angelic; others, subtile; and others irrefragable
[that is, doctors who cannot err].  When all these have been read and
reread, they will not be of as much aid for understanding Paul as is
this one passage of Ambrose.

To the same effect, Augustine writes many things against the
Pelagians.  In f the Spirit and Letter he says: The righteousness of
the Law, namely, that he who has fulfilled it shall live in it, is
set forth for this reason that when any one has recognized his
infirmity he may attain and work the same and live in it,
conciliating the Justifier not by his own strength nor by the letter
of the Law itself (which cannot be done), but by faith.  Except in a
justified man, there is no right work wherein he who does it may live.
But justification is obtained by faith.  Here he clearly says that
the Justifier is conciliated by faith, and that justification is
obtained by faith.  And a little after: By the Law we fear God; by
faith we hope in God.  But to those fearing punishment grace is
hidden; and the soul laboring, etc., under this fear betakes itself
by faith to God's mercy, in order that He may give what lie commands.
Here he teaches that by the Law hearts are terrified, but by faith
they receive consolation.  He also teaches us to apprehend, by faith,
mercy, before we attempt to fulfil the Law.  We will shortly cite
certain other passages.

Truly, it is amazing that the adversaries are in no way moved by so
many passages of Scripture, which clearly ascribe justification to
faith, and, indeed, deny it to works.  Do they think that the same is
repeated so often for no purpose?  Do they think that these words
fell inconsiderately from the Holy Ghost?  But they have also devised
sophistry whereby they elude them.  They say that these passages of
Scripture, (which speak of faith,) ought to be received as referring
to a _fides formata_, i.e., they do not ascribe justification to
faith except on account of love.  Yea, they do not, in any way,
ascribe justification to faith, but only to love, because they dream
that faith can coexist with mortal sin.  Whither does this tend,
unless that they again abolish the promise and return to the Law?  If
faith receive the remission of sins on account of love, the remission
of sins will always be uncertain, because we never love as much as we
ought, yea, we do not love unless our hearts are firmly convinced
that the remission of sins has been granted us.  Thus the adversaries,
while they require in the remission of sins and justification
confidence in one's own love, altogether abolish the Gospel
concerning the free remission of sins; although at the same time,
they neither render this love nor understand it, unless they believe
that the remission of sins is freely received.

We also say that love ought to follow faith as Paul also says, Gal. 5,
6: For in Jesus Christ neither circumcision availeth anything, nor
uncircumcision, but faith which worketh by love.  And yet we must not
think on that account that by confidence in this love or on account
of this love we receive the remission of sins and reconciliation just
as we do not receive the remission of sins because of other works
that follow.  But the remission of sins is received by faith alone,
and, indeed, by faith properly so called, because the promise cannot
be received except by faith.  But faith, properly so called, is that
which assents to the promise [is when my heart, and the Holy Ghost in
the heart, says: The promise of God is true and certain].  Of this
faith Scripture speaks.  And because it receives the remission of
sins, and reconciles us to God, by this faith we are [like Abraham]
accounted righteous for Christ's sake before we love and do the works
of the Law, although love necessarily follows.  Nor, indeed, is this
faith an idle knowledge, neither can it coexist with mortal sin, but
it is a work of the Holy Ghost, whereby we are freed from death, and
terrified minds are encouraged and quickened.  And because this faith
alone receives the remission of sins, and renders us acceptable to
God, and brings the Holy Ghost, it could be more correctly called
_gratia gratum faciens_, grace rendering one pleasing to God, than an
effect following, namely, love.

Thus far, in order that the subject might be made quite clear, we
have shown with sufficient fulness, both from testimonies of
Scripture, and arguments derived from Scripture, that by faith alone
we obtain the remission of sins for Christ's sake, and that by faith
alone we are justified, i.e., of unrighteous men made righteous, or
regenerated.  But how necessary the knowledge of this faith is, can
be easily judged, because in this alone the office of Christ is
recognized, by this alone we receive the benefits of Christ; this
alone brings sure and firm consolation to pious minds.  And in the
Church [if there is to be a church, if there is to be a Christian
Creed], it is necessary that there should be the [preaching and]
doctrine [by which consciences are not made to rely on a dream or to
build on a foundation of sand, but] from which the pious may receive
the sure hope of salvation.  For the adversaries give men bad advice
[therefore the adversaries are truly unfaithful bishops, unfaithful
preachers and doctors; they have hitherto given evil counsel to
consciences, and still do so by introducing such doctrine] when they
bid them doubt whether they obtain remission of sins.  For how will
such persons sustain themselves in death who have heard nothing of
this faith, and think that they ought to doubt whether they obtain
the remission of sins?  Besides it is necessary that in the Church of
Christ the Gospel be retained, i.e., the promise that for Christ's
sake sins are freely remitted.  Those who teach nothing of this faith,
concerning which we speak, altogether abolish the Gospel.  But the
scholastics mention not even a word concerning this faith.  Our
adversaries follow them, and reject this faith.  Nor do they see that,
by rejecting this faith, they abolish the entire promise concerning
the free remission of sins and the righteousness of Christ.




Part 6


Article III: _Of Love and the Fulfilling of the Law._

Here the adversaries urge against us: If thou wilt enter into life,
keep the commandments, Matt. 19, 17; likewise: The doers of the Law
shall be justified, Rom. 2, 13, and many other like things concerning
the Law and works.  Before we reply to this, we must first declare
what we believe concerning love and the fulfilling of the Law.

It is written in the prophet, Jer. 31, 33: I will put My Law in their
inward parts, and write it in their hearts.  And in Rom. 3, 31 Paul
says: Do we, then, make void the Law through faith?  God forbid!  Yea,
we establish the Law.  And Christ says, Matt. 19, 17: If thou wilt
enter into life, keep the commandments.  Likewise, 1 Cor. 13, 3: If I
have not charity, it profiteth me nothing.  These and similar
sentences testify that the Law ought to be begun in us, and be kept
by us more and more [that we are to keep the Law when we have been
justified by faith, and thus increase more and more in the Spirit].
Moreover, we speak not of ceremonies, but of that Law which gives
commandment concerning the movements of the heart, namely, the
_Decalog_.  Because, indeed, faith brings the Holy Ghost, and
produces in hearts a new life, it is necessary that it should produce
spiritual movements in hearts.  And what these movements are, the
prophet, Jer. 31, 33, shows, when he says: I will put My Law into
their inward parts, and write it in their hearts.  Therefore, when we
have been justified by faith and regenerated, we begin to fear and
love God, to pray to Him, to expect from Him aid, to give thanks and
praise Him and to obey Him in afflictions.  We begin also to love our
neighbors, because our hearts have spiritual and holy movements
[there is now, through the Spirit of Christ a new heart mind, and
spirit within].

These things cannot occur until we have been justified by faith, and,
regenerated, we receive the Holy Ghost: first, because the Law cannot
be kept without [the knowledge of] Christ; and likewise the Law
cannot be kept without the Holy Ghost.  But the Holy Ghost is
received by faith, according to the declaration of Paul, Gal. 3, 14:
That we might receive the promise of the Spirit through faith.  Then,
too, how can the human heart love God while it knows that He is
terribly angry, and is oppressing us with temporal and perpetual
calamities?  But the Law always accuses us, always shows that God is
angry.  [Therefore, what the scholastics say of the love of God is a
dream.] God therefore is not loved until we apprehend mercy by faith.
Not until then does He become a lovable object.

Although, therefore, civil works, i.e., the outward works of the Law,
can be done, in a measure, without Christ and without the Holy Ghost
[from our inborn light], nevertheless it appears from what we have
said that those things which belong peculiarly to the divine Law, i.e.,
the affections of the heart towards God, which are commanded in the
first table, cannot be rendered without the Holy Ghost.  But our
adversaries are fine theologians; they regard the second table and
political works; for the first table [in which is contained the
highest theology, on which all depends] they care nothing, as though
it were of no matter; or certainly they require only outward
observances.  They in no way consider the Law that is eternal, and
placed far above the sense and intellect of all creatures [which
concerns the very Deity, and the honor of the eternal Majesty], Deut.
6, 5: Thou shalt love the Lord, thy God with all thine heart.  [This
they treat as such a paltry small matter as if it did not belong to
theology.]

But Christ was given for this purpose, namely, that for His sake
there might be bestowed on us the remission of sins, and the Holy
Ghost to bring forth in us new and eternal life, and eternal
righteousness [to manifest Christ in our hearts, as it is written
John 16, 15: He shall take of the things of Mine, and show them unto
you.  Likewise, He works also other gifts, love, thanksgiving,
charity, patience, etc.].  Wherefore the Law cannot be truly kept
unless the Holy Ghost be received through faith.  Accordingly, Paul
says that the Law is established by faith, and not made void; because
the Law can only then be thus kept when the Holy Ghost is given.  And
Paul teaches 2 Cor. 3, 15 sq., the veil that covered the face of
Moses cannot be removed except by faith in Christ, by which the Holy
Ghost is received.  For he speaks thus: But even unto this day, when
Moses is read, the veil is upon their heart.  Nevertheless, when it
shall turn to the Lord, the veil shall be taken away.  Now the Lord
is that Spirit, and where the Spirit of the Lord is, there is liberty.
Paul understands by the veil the human opinion concerning the
entire Law, the _Decalog_ and the ceremonies, namely, that hypocrites
think that external and civil works satisfy the Law of God and that
sacrifices and observances justify before God _ex opere operato_.
But then this veil is removed from us, i.e., we are freed from this
error, when God shows to our hearts our uncleanness and the
heinousness of sin.  Then, for the first time, we see that we are far
from fulfilling the Law.  Then we learn to know how flesh, in
security and indifference, does not fear God, and is not fully
certain that we are regarded by God, but imagines that men are born
and die by chance.  Then we experience that we do not believe that
God forgives and hears us.  But when, on hearing the Gospel and the
remission of sins, we are consoled by faith, we receive the Holy
Ghost, so that now we are able to think aright concerning God, and to
fear and believe God, etc. From these facts it is apparent that the
Law cannot be kept without Christ and the Holy Ghost.

We, therefore, profess that it is necessary that the Law be begun in
us, and that it be observed continually more and more.  And at the
same time we comprehend both spiritual movements and external good
works [the good heart within and works without].  Therefore the
adversaries falsely charge against us that our theologians do not
teach good works, while they not only require these, but also show
how they can be done [that the heart must enter into these works,
lest they be mere lifeless, cold works of hypocrites].  The result
convicts hypocrites, who by their own powers endeavor to fulfil the
Law, that they cannot accomplish what they attempt.  [For are they
free from hatred, envy, strife, anger, wrath, avarice, adultery, etc.?
Why, these vices were nowhere greater than in the cloisters and
sacred institutes.] For human nature is far too weak to be able by
its own powers to resist the devil, who holds as captives all who
have not been freed through faith.  There is need of the power of
Christ against the devil, namely, that, inasmuch as we know that for
Christ's sake we are heard, and have the promise, we may pray for the
governance and defense of the Holy Ghost, that we may neither be
deceived and err, nor be impelled to undertake anything contrary to
God's will.  [Otherwise we should, every hour, fall into error and
abominable vices.] Just as Ps. 68, 18 teaches: Thou hast led
captivity captive; Thou hast received gifts for man.  For Christ has
overcome the devil, and has given to us the promise and the Holy
Ghost, in order that, by divine aid, we ourselves also may overcome.
And 1 John 3, 8: For this purpose the Son of God was manifested, that
He might destroy the works of the devil.  Again, we teach not only
how the Law can be observed, but also how God is pleased if anything
be done, namely, not because we render satisfaction to the Law, but
because we are in Christ, as we shall say after a little.  It is,
therefore, manifest that we require good works.  Yea, we add also
this, that it is impossible for love to God, even though it be small,
to be sundered from faith, because through Christ we come to the
Father, and, the remission of sins having been received, we now are
truly certain that we have a God, i.e., that God cares for us; we
call upon Him, we give Him thanks, we fear Him, we love Him as John
teaches in his first Epistle, 4, 19: We love Him he says, because He
first loved us, namely, because He gave His Son for us, and forgave
us our sins.  Thus he indicates that faith precedes and love follows.
Likewise the faith of which we speak exists in repentance i.e., it
is conceived in the terrors of conscience, which feels the wrath of
God against our sins, and seeks the remission of sins, and to be
freed from sin.  And in such terrors and other afflictions this faith
ought to grow and be strengthened.  Wherefore it cannot exist in
those who live according to the flesh, who are delighted by their own
lusts and obey them.  Accordingly, Paul says, Rom. 8, 1: There is,
therefore, now no condemnation to them that are in Christ Jesus, who
walk not after the flesh, but after the Spirit.  So, too, vv. 12. 13:
We are debtors, not to the flesh, to live after the flesh.  For if ye
live after the flesh, ye shall die; but if ye, through the Spirit, do
mortify the deeds of the body, ye shall live.  Wherefore, the faith
which receives remission of sins in a heart terrified and fleeing
from sin does not remain in those who obey their desires, neither
does it coexist with mortal sin.

From these effects of faith the adversaries select one, namely, love,
and teach that love justifies.  Thus it is clearly apparent that they
teach only the Law.  They do not teach that remission of sins through
faith is first received.  They do not teach of Christ as Mediator,
that for Christ's sake we have a gracious God; but because of our
love.  And yet, what the nature of this love is they do not say,
neither can they say.  They proclaim that they fulfil the Law,
although this glory belongs properly to Christ; and they set against
the judgment of God confidence in their own works; for they say that
they _merit de condigno_ (according to righteousness) grace and
eternal life.  This confidence is absolutely impious and vain.  For
in this life we cannot satisfy the Law, because carnal nature does
not cease to bring forth wicked dispositions [evil inclination and
desire], even though the Spirit in us resists them.

But some one may ask: Since we also confess that love is a work of
the Holy Ghost, and since it is righteousness, because it is the
fulfilling of the Law, why do we not teach that it justifies?  To
this we must reply: In the first place, it is certain that we receive
remission of sins, neither through our love nor for the sake of our
love, but for Christ's sake, by faith alone.  Faith alone, which
looks upon the promise, and knows that for this reason it must be
regarded as certain that God forgives, because Christ has not died in
vain, etc., overcomes the terrors of sin and death.  If any one
doubts whether sins are remitted him, he dishonors Christ, since he
judges that his sin is greater or more efficacious than the death and
promise of Christ although Paul says, Rom. 5, 20: Where sin abounded,
grace did much more abound, i.e., that mercy is more comprehensive
[more powerful, richer, and stronger] than sin.  If any one thinks
that he obtains the remission of sins because he loves, he dishonors
Christ, and will discover in God's judgment that this confidence in
his own righteousness is wicked and vain.  Therefore it is necessary
that faith [alone] reconciles and justifies.  And as we do not
receive remission of sins through other virtues of the Law, or on
account of these namely, on account of patience, chastity, obedience
towards magistrates, etc., and nevertheless these virtues ought to
follow, so, too, we do not receive remission of sins because of love
to God although it is necessary that this should follow.  Besides,
the custom of speech is well known that by the same word we sometimes
comprehend by synecdoche the cause and effects.  Thus in Luke 7, 47
Christ says: Her sins, which are many, are forgiven for she loved
much.  For Christ interprets Himself [this very passage] when He adds:
Thy faith hath saved thee.  Christ, therefore, did not mean that the
woman, by that work of love, had merited the remission of sins.  For
that is the reason He says: Thy faith hath sated thee.  But faith is
that which freely apprehends God's mercy on account of God's Word
[which relies upon God's mercy and Word, and not upon one's own work].
If any one denies that this is faith [if any one imagines that he
can rely at the same time upon God and his own works], he does not
understand at all what faith is.  [For the terrified conscience is
not satisfied with its own works, but must cry after mercy, and is
comforted and encouraged alone by God's Word.] And the narrative
itself shows in this passage what that is which He calls love.  The
woman came with the opinion concerning Christ that with Him the
remission of sins should be sought.  This worship is the highest
worship of Christ.  Nothing greater could she ascribe to Christ.  To
seek from Him the remission of sins was truly to acknowledge the
Messiah.  Now, thus to think of Christ, thus to worship Him, thus to
embrace Him, is truly to believe.  Christ, moreover, employed the
word "love" not towards the woman, but against the Pharisee, because
He contrasted the entire worship of the Pharisee with the entire
worship of the woman.  He reproved the Pharisee because he did not
acknowledge that He was the Messiah, although he rendered Him the
outward offices due to a guest and a great and holy man.  He points
to the woman and praises her worship, ointment, tears, etc., all of
which were signs of faith and a confession, namely, that with Christ
she sought the remission of sins.  It is indeed a great example which,
not without reason, moved Christ to reprove the Pharisee, who was a
wise and honorable man, but not a believer.  He charges him with
impiety, and admonishes him by the example of the woman, showing
thereby that it is disgraceful to him, that, while an unlearned woman
believes God, he, a doctor of the Law, does not believe, does not
acknowledge the Messiah, and does not seek from Him remission of sins
and salvation.  Thus, therefore, He praises the entire worship [faith
with its fruits, but towards the Pharisee He names only the fruits
which prove to men that there is faith in the heart] as it often
occurs in the Scriptures that by one word we embrace many things; as
below we shall speak at greater length in regard to similar passages,
such as Luke 11, 41: Give alms of such things as ye have; and, behold,
all things are clean unto you.  He requires not only alms, but also
the righteousness of faith.  Thus He here says: Her sins, which are
many, are forgiven, for she loved much i.e., because she has truly
worshiped Me with faith and the exercises and signs of faith.  He
comprehends the entire worship.  Meanwhile He teaches this, that the
remission of sins is properly received by faith, although love,
confession, and other good fruits ought to follow.  Wherefore He does
not mean this, that these fruits are the price, or are the
propitiation, because of which the remission of sins, which
reconciles us to God, is given.  We are disputing concerning a great
subject, concerning the honor of Christ, and whence good minds may
seek for sure and firm consolation whether confidence is to be placed
in Christ or in our works.  Now, if it is to be placed in our works,
the honor of Mediator and Propitiator will be withdrawn from Christ.
And yet we shall find, in God's judgment, that this confidence is
vain, and that consciences rush thence into despair.  But if the
remission of sins and reconciliation do not occur freely for Christ's
sake, but for the sake of our love, no one will have remission of
sins, unless when he has fulfilled the entire Law, because the Law
does not justify as long as it can accuse us.  Therefore it is
manifest that, since justification is reconciliation for Christ's
sake we are justified by faith, because it is very certain that by
faith alone the remission of sins is received.

Now, therefore, let us reply to the objection which we have above
stated: [Why does love not justify anybody before God?] The
adversaries are right in thinking that love is the fulfilling of the
Law, and obedience to the Law is certainly righteousness.  [Therefore
it would be true that love justifies us if we would keep the Law.
But who in truth can say or boast that he keeps the Law, and loves
God as the Law has commanded?  We have shown above that God has made
the promise of grace, because we cannot observe the Law.  Therefore
Paul says everywhere that we cannot be justified before God by the
Law.] But they make a mistake in this that they think that we are
justified by the Law.  [The adversaries have to fail at this point,
and miss the main issue, for in this business they only behold the
Law.  For all men's reason and wisdom cannot but hold that we must
become pious by the Law, and that a person externally observing the
Law is holy and pious.  But the Gospel faces us about, directs us
away from the Law to the divine promises, and teaches that we are not
justified, etc.] Since, however, we are not justified by the Law
[because no person can keep it], but receive remission of sins and
reconciliation by faith for Christ's sake, and not for the sake of
love or the fulfilling of the Law, it follows necessarily that we are
justified by faith in Christ.  [For before we fulfil one tittle of
the Law, there must be faith in Christ by which we are reconciled to
God and first obtain the remission of sin.  Good God, how dare people
call themselves Christians or say that they once at least looked into
or read the books of the Gospel when they still deny that we obtain
remission of sins by faith in Christ?  Why, to a Christian it is
shocking merely to hear such a statement.]

Again, [in the second place,] this fulfilling of the Law or obedience
towards the Law, is indeed righteousness, when it is complete; but in
us it is small and impure.  [For, although they have received the
first-fruits of the Spirit, and the new, yea the eternal life has
begun in them, there still remains a remnant of sin and evil lust,
and the Law still finds much of which it must accuse us.] Accordingly,
it is not pleasing for its own sake, and is not accepted for its own
sake.  But although from those things which have been said above it
is evident that justification signifies not the beginning of the
renewal, but the reconciliation by which also we afterwards are
accepted, nevertheless it can now be seen much more clearly that the
inchoate fulfilling of the Law does not justify, because it is
accepted only on account of faith.  [Trusting in our own fulfilment
of the Law is sheer idolatry and blaspheming Christ, and in the end
it collapses and causes our consciences to despair.  Therefore, this
foundation shall stand forever, namely, that for Christ's sake we are
accepted with God, and justified by faith, not on account of our love
and works.  This we shall make so plain and certain that anybody may
grasp it.  As long as the heart is not at peace with God, it cannot
be righteous, for it flees from the wrath of God, despairs, and would
have God not to judge it.  Therefore the heart cannot be righteous
and accepted with God while it is not at peace with God.  Now, faith
alone makes the heart to be content, and obtains peace and life Rom.
5, 1, because it confidently and frankly relies on the promise of God
for Christ's sake.  But our works do not make the heart content, for
we always find that they are not pure.  Therefore it must follow that
we are accepted with God, and justified by faith alone, when in our
hearts we conclude that God desires to be gracious to us, not on
account of our works and fulfilment of the Law, but from pure grace,
for Christ's sake.  What can our opponents bring forward against this
argument?  What can they invent and devise against the plain truth?
For this is quite certain, and experience teaches forcibly enough,
that when we truly feel the judgment and wrath of God, or become
afflicted, our works and worship cannot set the heart at rest.
Scripture indicates this often enough as in Ps. 143, 2: Enter not
into judgment with Thy servant; for in Thy sight shall no man living
be justified.  Here he clearly shows that all the saints, all the
pious children of God, who have the Holy Ghost, if God would not by
grace forgive them their sin, still have remnants of sin in the flesh.
For when David in another place, Ps. 7, 8, says: Judge me O Lord,
according to my righteousness, he refers to his cause, and not to his
righteousness, and asks God to protect his cause and word, for he
says: Judge, O Lord, my cause.  Again, in Ps. 130, 3 he clearly
states that no person, not even the greatest saints, can bear God's
judgment, if He were to observe our iniquity, as he says: If Thou,
Lord, shouldest mark iniquity, O Lord, who shall stand!  And thus
says Job, 9, 28: I was afraid of all my works (Engl. vers., sorrows).
Likewise chap. 9, 30: If I wash myself with snow-water, and make my
hands never so clean, yet shalt Thou plunge me in the ditch.  And
Prov. 20, 9: Who can say, I have made my heart clean?  And 1 John 1,
8: If we say that we have no sin, we deceive ourselves and the truth
is not in us.  And in the Lord's Prayer the saints ask for the
forgiveness of sins.  Therefore even the saints have guilt and sins.
Again in Num. 14, 18: The innocent will not be innocent.  And
Zechariah, 2, 13, says: Be silent O all flesh, before the Lord.  And
Isaiah 40, 6 sqq.: All flesh is grass, i.e., flesh and righteousness
of the flesh cannot endure the judgment of God.  And Jonah says, 2, 9:
They that observe lying vanities forsake their own mercy.  Therefore,
pure mercy preserves us, our own works, merits, endeavors, cannot
preserve us.  These and similar declarations in the Scriptures
testify that our works are unclean, and that we need mercy.
Wherefore works do not render consciences pacified but only mercy
apprehended by faith does.] Nor must we trust that we are accounted
righteous before God by our own perfection and fulfilling of the Law,
but rather for Christ's sake.

First [in the third place], because Christ does not cease to be
Mediator after we have been renewed.  They err who imagine that He
has merited only a first grace, and that afterwards we please God and
merit eternal life by our fulfilling of the Law.  Christ remains
Mediator, and we ought always to be confident that for His sake we
have a reconciled God even although we are unworthy.  As Paul clearly
teaches when he says [By whom also we have access to God, Rom. 5, 2.
For our best works, even after the grace of the Gospel has been
received, as I stated, are still weak and not at all pure.  For sin
and Adam's fall are not such a trifling thing as reason holds or
imagines, it exceeds the reason and thought of all men to understand
what a horrible wrath of God has been handed on to us by that
disobedience.  There occurred a shocking corruption of the entire
human nature, which no work of man, but only God Himself, can
restore], 1 Cor. 4, 4: I know nothing by myself, yet am I not hereby
justified, but he knows that by faith he is accounted righteous for
Christ's sake, according to the passage: Blessed are they whose
iniquities are forgiven, Ps. 32, 1; Rom. 4, 7. [Therefore we need
grace, and the gracious goodness of God, and the forgiveness of sin,
although we have done many good works.] But this remission is always
received by faith.  Likewise, the imputation of the righteousness of
the Gospel is from the promise; therefore it is always received by
faith, and it always must be regarded certain that by faith we are
for Christ's sake, accounted righteous.  If the regenerate ought
afterwards to think that they will be accepted on account of the
fulfilling of the Law, when would conscience be certain that it
pleased God, since we never satisfy the Law?  Accordingly, we must
always recur to the promise; by this our infirmity must be sustained,
and we must regard it as certain that we are accounted righteous for
the sake of Christ, who is ever at the right hand of God, who also
maketh intercession for us, Rom. 8, 34. If any one think that he is
righteous and accepted on account of his own fulfilment of the Law,
and not on account of Christ's promise, he dishonors this High Priest.
Neither can it be understood how one could imagine that man is
righteous before God when Christ is excluded as Propitiator and
Mediator.

Again [in the fourth place], what need is there of a long discussion?
[If we were to think that, after we have come to the Gospel and are
born again, we were to merit by our works that God be gracious to us,
not by faith, conscience would never find rest, but would be driven
to despair.  For the Law unceasingly accuses us, since we never can
satisfy the Law.] All Scripture, all the Church cries out that the
Law cannot be satisfied.  Therefore this inchoate fulfilment of the
Law does not please on its own account, but on account of faith in
Christ.  Otherwise the Law always accuses us.  For who loves or fears
God sufficiently?  Who with sufficient patience bears the afflictions
imposed by God?  Who does not frequently doubt whether human affairs
are ruled by God's counsel or by chance?  Who does not frequently
doubt whether he be heard by God?  Who is not frequently enraged
because the wicked enjoy a better lot than the pious, because the
pious are oppressed by the wicked?  Who does satisfaction to his own
calling?  Who loves his neighbor as himself?  Who is not tempted by
lust?  Accordingly Paul says, Rom. 7, 19: The good that I would I do
not; but the evil which I would not that I do.  Likewise v. 25: With
the mind I myself serve the Law of God, but with the flesh, the law
of sin.  Here he openly declares that he serves the law of sin.  And
David says, Ps. 143, 2: Enter not into judgment with Thy servant; for
in Thy sight shall no man living be justified.  Here even a servant
of God prays for the averting of judgment.  Likewise Ps. 32, 2:
Blessed is the man unto whom the Lord imputeth not iniquity.
Therefore, in this our infirmity there is always present sin, which
could be imputed, and of which he says a little while after, v. 6:
For this shall every one that is godly pray unto Thee.  Here he shows
that even saints ought to seek remission of sins.  More than blind
are those who do not perceive that wicked desires in the flesh are
sins, of which Paul, Gal. 5, 17, says: The flesh lusteth against the
Spirit, and the Spirit against the flesh.  The flesh distrusts God,
trusts in present things, seeks human aid in calamities, even
contrary to God's will, flees from afflictions, which it ought to
bear because of God's commands, doubts concerning God's mercy, etc.
The Holy Ghost in our hearts contends with such dispositions [with
Adam's sin] in order to suppress and mortify them [this poison of the
old Adam, this desperately wicked disposition], and to produce new
spiritual movements.  But concerning this topic we will collect more
testimonies below, although they are everywhere obvious not only in
the Scriptures, but also in the holy Fathers.

Well does Augustine say: All the commandments of God are fulfilled
when whatever is not done, is forgiven.  Therefore he requires faith
even in good works [which the Holy Spirit produces in us], in order
that we may believe that for Christ's sake we please God, and that
even the works are not of themselves worthy and pleasing.  And Jerome,
against the Pelagians, says: Then, therefore, we are righteous when
we confess that we are sinners, and that our righteousness consists
not in our own merit, but in God's mercy.  Therefore, in this
inchoate fulfilment of the Law, faith ought to be present, which is
certain that for Christ's sake we have a reconciled God.  For mercy
cannot be apprehended unless by faith, as has been repeatedly said
above.  [Therefore those who teach that we are not accepted by faith
for Christ's sake but for the sake of our own works, lead consciences
into despair.] Wherefore, when Paul says, Rom. 3, 31: We establish
the Law through faith, by this we ought to understand, not only that
those regenerated by faith receive the Holy Ghost, and have movements
agreeing with God's Law, but it is by far of the greatest importance
that we add also this, that we ought to perceive that we are far
distant from the perfection of the Law.  Wherefore we cannot conclude
that we are accounted righteous before God because of our fulfilling
of the Law, but in order that the conscience may become tranquil,
justification must be sought elsewhere.  For we are not righteous
before God as long as we flee from God's judgment, and are angry with
God.  Therefore we must conclude that, being reconciled by faith, we
are accounted righteous for Christ's sake, not for the sake of the
Law or our works, but that this inchoate fulfilling of the Law
pleases on account of faith, and that, on account of faith, there is
no imputation of the imperfection of the fulfilling of the Law, even
though the sight of our impurity terrifies us.  Now, if justification
is to be sought elsewhere, our love and works do not therefore
justify.  Far above our purity, yea, far above the Law itself ought
to be placed the death and satisfaction of Christ, presented to us
that we might be sure that because of this satisfaction, and not
because of our fulfilling of the Law, we have a gracious God.

Paul teaches this in Gal. 3, 13, when he says: Christ hath redeemed
us from the curse of the Law, being made a curse for us, i.e. the Law
condemns all men, but Christ, because without sin He has borne the
punishment of sin, and been made a victim for us has removed that
right of the Law to accuse and condemn those who believe in Him,
because He Himself is the propitiation for them for whose sake we are
now accounted righteous.  But since they are accounted righteous, the
Law cannot accuse or condemn them, even though they have not actually
satisfied the Law.  To the same purport he writes to the Colossians,
2, 10: Ye are complete in Him, as though he were to say: Although ye
are still far from the perfection of the Law, yet the remnants of sin
do not condemn you, because for Christ's sake we have a sure and firm
reconciliation, if you believe, even though sin inhere in your flesh.

The promise ought always to be in sight that God, because of His
promise, wishes for Christ's sake, and not because of the Law or our
works, to be gracious and to justify.  In this promise timid
consciences ought to seek reconciliation and justification, by this
promise they ought to sustain themselves, and be confident that for
Christ's sake, because of His promise, they have a gracious God.
Thus works can never render a conscience pacified, but only the
promise can.  If, therefore, justification and peace of conscience
must be sought elsewhere than in love and works, love and works do
not justify, although they are virtues and pertain to the
righteousness of the Law, in so far as they are a fulfilling of the
Law.  So far also this obedience of the Law justifies by the
righteousness of the Law.  But this imperfect righteousness of the
Law is not accepted by God, unless on account of faith.  Accordingly
it does not justify, i.e., it neither reconciles, nor regenerates,
nor by itself renders us accepted before God.

From this it is evident that we are justified before God by faith
alone [i.e., it obtains the remission of sins and grace for Christ's
sake and regenerates us.  Likewise, it is quite clear that by faith
alone the Holy Ghost is received; again, that our works and this
inchoate fulfilling of the Law do not by themselves please God.  Now,
even if I abound in good works like Paul or Peter, I must seek my
righteousness elsewhere, namely, in the promise of the grace of
Christ, again, if only faith calms the conscience, it must, indeed be
certain that only faith justifies before God.  For, if we wish to
teach correctly, we must adhere to this, that we are accepted with
God not on account of the Law, not on account of works, but for
Christ's sake.  For the honor, due Christ, must not be given to the
Law or our-miserable works.] because by faith alone we receive
remission of sins and reconciliation, because reconciliation or
justification is a matter promised for Christ's sake, and not for the
sake of the Law.  Therefore it is received by faith alone, although,
when the Holy Ghost is given, the fulfilling of the Law follows.




Part 7


_Reply to the Arguments of the Adversaries._

Now, when the grounds of this case have been understood, namely, the
distinction between the Law and the promises, or the Gospel, it will
be easy to resolve the objections of the adversaries.  For they cite
passages concerning the Law and works, and omit passages concerning
the promises.  But a reply can once for all be made to all opinions
concerning the Law, namely, that the Law cannot be observed without
Christ, and that if civil works are wrought without Christ, they do
not please God.  [God is not pleased with the person.] Wherefore,
when works are commended, it is necessary to add that faith is
required, that they are commended on account of faith, that they are
the fruits and testimonies of faith.  [This our doctrine is, indeed,
plain; it need not fear the light, and may be held against the Holy
Scriptures.  We have also clearly and correctly presented it here, if
any will receive instruction and not knowingly deny the truth.  For
rightly to understand the benefit of Christ and the great treasure of
the Gospel (which Paul extols so greatly), we must separate, on the
one hand, the promise of God and the grace that is offered, and, on
the other hand the Law, as far as the heavens are from the earth.  In
shaky matters many explanations are needed, but in a good matter one
or two thoroughgoing explanations dissolve all objections which men
think they can raise.] Ambiguous and dangerous cases produce many and
various solutions.  For the judgment of the ancient poet is true:

"An unjust cause, being In Itself sick, requires skilfully applied
remedies."

But in just and sure cases one or two explanations derived from the
sources correct all things that seem to offend.  This occurs also in
this case of ours.  For the rule which I have just recited, explains
all the passages that are cited concerning the Law and works [namely,
that without Christ the Law cannot be truly observed, and although
external works may be performed, still the person doing them does not
please God outside of Christ].  For we acknowledge that Scripture
teaches in some places the Law, and in other places the Gospel, or
the gratuitous promise of the remission of sins for Christ's sake.
But our adversaries absolutely abolish the free promise when they
deny that faith justifies, and teach that for the sake of love and of
our works we receive remission of sins and reconciliation.  If the
remission of sins depends upon the condition of our works, it is
altogether uncertain.  [For we can never be certain whether we do
enough works, or whether our works are sufficiently holy and pure.
Thus, too, the forgiveness of sins is made uncertain, and the promise
of God perishes, as Paul says, Rom. 4, 14: The promise is made of
none effect, and everything is rendered uncertain.] Therefore the
promise will be abolished.  Hence we refer godly minds to the
consideration of the promises, and we teach concerning the free
remission of sins and concerning reconciliation, which occurs through
faith in Christ.  Afterwards we add also the doctrine of the Law.
[Not that by the Law we merit the remission of sins, or that for the
sake of the Law we are accepted with God, but because God requires
good works.] And it is necessary to divide these things aright, as
Paul says, 2 Tim. 2, 15. We must see what Scripture ascribes to the
Law, and what to the promises.  For it praises works in such a way as
not to remove the free promise [as to place the promise of God and
the true treasure, Christ, a thousand leagues above it].

For good works are to be done on account of God's command, likewise
for the exercise of faith [as Paul says, Eph. 2, 10: We are His
workmanship, created in Christ Jesus unto good works], and on account
of confession and giving of thanks.  For these reasons good works
ought necessarily to be done, which, although they are done in the
flesh not as yet entirely renewed, that retards the movements of the
Holy Ghost, and imparts some of its uncleanness, yet, on account of
Christ, are holy, divine works, sacrifices, and acts pertaining to
the government of Christ, who thus displays His kingdom before this
world.  For in these He sanctifies hearts and represses the devil,
and, in order to retain the Gospel among men, openly opposes to the
kingdom of the devil the confession of saints, and, in our weakness,
declares His power.  The dangers, labors, and sermons of the Apostle
Paul, of Athanasius, Augustine, and the like, who taught the churches,
are holy works, are true sacrifices acceptable to God, are contests
of Christ through which He repressed the devil, and drove him from
those who believed.  David's labors, in waging wars and in his home
government, are holy works, are true sacrifices, are contests of God,
defending the people who had the Word of God against the devil, in
order that the knowledge of God might not be entirely extinguished on
earth.  We think thus also concerning every good work in the humblest
callings and in private affairs.  Through these works Christ
celebrates His victory over the devil, just as the distribution of
alms by the Corinthians, 1 Cor. 16, 1, was a holy work and a
sacrifice and contest of Christ against the devil, who labors that
nothing may be done for the praise of God.  To disparage such works,
the confession of doctrine, affliction, works of love, mortifications
of the flesh would be indeed to disparage the outward government of
Christ's kingdom among men.  Here also we add something concerning
rewards and merits.  We teach that rewards have been offered and
promised to the works of believers.  We teach that good works are
meritorious, not for the remission of sins, for grace or
justification (for these we obtain only by faith), but for other
rewards, bodily and spiritual, in this life and after this life
because Paul says, 1 Cor. 3, 8: Every man shall receive his own
reward, according to his own labor.  There will, therefore, be
different rewards according to different labors.  But the remission
of sins is alike and equal to all, just as Christ is one, and is
offered freely to all who believe that for Christ's sake their sins
are remitted.  Therefore the remission of sins and justification are
received only by faith, and not on account of any works, as is
evident in the terrors of conscience, because none of our works can
be opposed to God's wrath, as Paul clearly says, Rom. 5, 1: Being
justified by faith, toe have peace with God through our Lord Jesus
Christ, by whom also we have access by faith, etc. But because faith
makes sons of God, it also makes coheirs with Christ.  Therefore,
because by our works we do not merit justification, through which we
are made sons of God, and coheirs with Christ, we do not by our works
merit eternal life; for faith obtains this, because faith justifies
us and has a reconciled God.  But eternal life is due the justified,
according to the passage Rom. 8, 30: Whom He justified, them He also
glorified.  Paul, Eph. 6, 2, commends to us the commandment
concerning honoring parents, by mention of the reward which is added
to that commandment where he does not mean that obedience to parents
justifies us before God, but that, when it occurs in those who have
been justified, it merits other great rewards.  Yet God exercises His
saints variously, and often defers the rewards of the righteousness
of works in order that they may learn not to trust in their own
righteousness, and may learn to seek the will of God rather than the
rewards, as appears in Job, in Christ, and other saints.  And of this,
many psalms teach us, which console us against the happiness of the
wicked, as Ps. 37, 1: Neither be thou envious.  And Christ says, Matt.
5, 10: Blessed are they which are persecuted for righteousness' sake;
for theirs is the kingdom of heaven.  By these praises of good works,
believers are undoubtedly moved to do good works.  Meanwhile, the
doctrine of repentance is also proclaimed against the godless, whose
works are wicked; and the wrath of God is displayed, which He has
threatened all who do not repent.  We therefore praise and require
good works, and show many reasons why they ought to be done.

Thus of works Paul also teaches when he says, Rom. 4, 9 sq., that
Abraham received circumcision, not in order that by this work he
might be justified; for by faith he had already attained it that he
was accounted righteous.  But circumcision was added in order that he
might have in his body a written sign, admonished by which he might
exercise faith, and by which also he might confess his faith before
others, and by his testimony might invite others to believe.  By
faith Abel offered unto God a more excellent sacrifice, Heb. 11, 4.
Because, therefore, he was just by faith, the sacrifice which he made
was pleasing to God, not that by this work he merited the remission
of sins and grace, but that he exercised his faith and showed it to
others, in order to invite them to believe.

Although in this way good works ought to follow faith, men who cannot
believe and be sure that for Christ's sake they are freely forgiven,
and that freely for Christ's sake they have a reconciled God, employ
works far otherwise.  When they see the works of saints, they judge
in a human manner that saints have merited the remission of sins and
grace through these works.  Accordingly, they imitate them, and think
that through similar works they merit the remission of sins and grace;
they think that through these works they appease the wrath of God,
and attain that for the sake of these works they are accounted
righteous.  This godless opinion concerning works we condemn.  In the
first place, because it obscures the glory of Christ when men offer
to God these works as a price and propitiation.  This honor, due to
Christ alone, is ascribed to our works.  Secondly, they nevertheless
do not find, in these works, peace of conscience, but in true terrors,
heaping up works upon works, they at length despair because they
find no work sufficiently pure [sufficiently important and precious
to propitiate God, to obtain with certainty eternal life, in a word,
to tranquilize and pacify the conscience].  The Law always accuses,
and produces wrath.  Thirdly, such persons never attain the knowledge
of God [nor of His will]; for, as in anger they flee from God, who
judges and afflicts them, they never believe that they are heard.
But faith manifests the presence of God, since it is certain that God
freely forgives and hears us.

Moreover, this godless opinion concerning works always has existed in
the world [sticks to the world quite tightly].  The heathen had
sacrifices, derived from the fathers.  They imitated their works.
Their faith they did not retain, but thought that the works were a
propitiation and price on account of which God would be reconciled to
them.  The people in the law [the Israelites] imitated sacrifices
with the opinion that by means of these works they would appease God,
so to say, _ex opere operato_.  We see here how earnestly the
prophets rebuke the people: Ps. 50, 8: I will not reprove thee for
thy sacrifices, and Jer. 7, 22: I spake not unto your fathers
concerning burnt offerings.  Such passages condemn not works, which
God certainly had commanded as outward exercises in this government,
but they condemn the godless opinion according to which they thought
that by these works they appeased the wrath of God, and thus cast
away faith.  And because no works pacify the conscience, new works,
in addition to God's commands, were from time to time devised [the
hypocrites nevertheless used to invent one work after another, one
sacrifice after another, by a blind guess and in reckless wantonness,
and all this without the word and command of God, with wicked
conscience as we have seen in the Papacy].  The people of Israel had
seen the prophets sacrificing on high places [and in groves].
Besides, the examples of the saints very greatly move the minds of
those, hoping by similar works to obtain grace just as these saints
obtained it.  [But the saints believed.] Wherefore the people began,
with remarkable zeal, to imitate this work, in order that by such a
work [they might appease the wrath of God] they might merit remission
of sins, grace, and righteousness.  But the prophets had been
sacrificing on high places, not that by these works they might merit
the remission of sins and grace, but because on these places they
taught, and, accordingly, presented there a testimony of their faith.
The people had heard that Abraham had sacrificed his son.  Wherefore
they also, in order to appease God by a most cruel and difficult work,
put to death their sons.  But Abraham did not sacrifice his son with
the opinion that this work was a price and propitiatory work for the
sake of which he was accounted righteous.  Thus in the Church the
Lord's Supper was instituted that by remembrance of the promises of
Christ, of which we are admonished in this sign, faith might be
strengthened in us, and we might publicly confess our faith, and
proclaim the benefits of Christ, as Paul says, 1 Cor. 11, 26: As
often as ye eat this bread and drink this cup, ye do show the Lord's
death, etc. But our adversaries contend that the mass is a work that
justifies us _ex opere operato_, and removes the guilt and liability
to punishment in those for whom it is celebrated, for thus writes
Gabriel.

Anthony, Bernard, Dominicus, Franciscus, and other holy Fathers
selected a certain kind of life either for the sake of study [of more
readily reading the Holy Scriptures] or other useful exercises.  In
the mean time they believed that by faith they were accounted
righteous for Christ's sake, and that God was gracious to them, not
on account of those exercises of their own.  But the multitude since
then has imitated not the faith of the Fathers, but their example
without faith, in order that by such works they might merit the
remission of sins, grace, and righteousness: they did not believe
that they received these freely on account of Christ as Propitiator.
[Thus the human mind always exalts works too highly, and puts them in
the wrong place.  And this error the Gospel reproves which teaches
that men are accounted righteous not for the sake of the Law, but for
the sake of Christ alone.  Christ, however, is apprehended by faith
alone; wherefore we are accounted righteous by faith alone for
Christ's sake.] Thus the world judges of all works that they are a
propitiation by which God is appeased; that they are a price because
of which we are accounted righteous.  It does not believe that Christ
is Propitiator; it does not believe that by faith we freely attain
that we are accounted righteous for Christ's sake.  And, nevertheless,
since works cannot pacify the conscience, others are continually
chosen, new rites are performed, new vows made, and new orders of
monks formed beyond the command of God, in order that some great work
may be sought which may be set against the wrath and judgment of God.
Contrary to Scripture, the adversaries uphold these godless opinions
concerning works.  But to ascribe to our works these things, namely,
that they are a propitiation, that they merit the remission of sins
and grace that for the sake of these and not by faith for the sake of
Christ as Propitiator we are accounted righteous before God, what
else is this than to deny Christ the honor of Mediator and
Propitiator?  Although, therefore, we believe and teach that good
works must necessarily be done (for the inchoate fulfilling of the
Law ought to follow faith), nevertheless we give to Christ His own
honor.  We believe and teach that by faith, for Christ's sake, we are
accounted righteous before God, that we are not accounted righteous
because of works without Christ as Mediator, that by works we do not
merit the remission of sins, grace, and righteousness, that we cannot
set our works against the wrath and justice of God, that works cannot
overcome the terrors of sin, but that the terrors of sin are overcome
by faith alone, that only Christ the Mediator is to be presented by
faith against the wrath and judgment of God.  If any one think
differently, he does not give Christ due honor, who has been set
forth that He might be a Propitiator, that through Him we might have
access to the Father.  We are speaking now of the righteousness
through which we treat with God not with men, but by which we
apprehend grace and peace of conscience.  Conscience however, cannot
be pacified before God, unless by faith alone, which is certain that
God for Christ's sake is reconciled to us, according to Rom. 5, 1:
Being justified by faith, we have peace because justification is only
a matter freely promised for Christ's sake, and therefore is always
received before God by faith alone.

Now, then, we will reply to those passages which the adversaries cite,
in order to prove that we are justified by love and works.  From 1
Cor. 13, 2 they cite: Though I have all faith, etc., and hove not
charity, I am nothing.  And here they triumph greatly.  Paul
testifies to the entire Church, they say, that faith alone does not
justify.  But a reply is easy after we have shown above what we hold
concerning love and works.  This passage of Paul requires love.  We
also require this.  For we have said above that renewal and the
inchoate fulfilling of the Law must exist in us, according to Jer. 31,
33: 1 will put My Law in their inward parts, and write it in their
hearts.  If any one should cast away love, even though he have great
faith, yet he does not retain it, for he does not retain the Holy
Ghost [he becomes cold and is now again fleshly, without Spirit and
faith; for the Holy Ghost is not where Christian love and other
fruits of the Spirit are not].  Nor indeed does Paul in this passage
treat of the mode of justification, but he writes to those who, after
they had been justified, should be urged to bring forth good fruits
lest they might lose the Holy Ghost.  The adversaries, furthermore,
treat the matter preposterously: they cite this one passage, in which
Paul teaches concerning fruits, they omit very many other passages,
in which in a regular order he discusses the mode of justification.
Besides, they always add a correction to the other passages, which
treat of faith, namely, that they ought to be understood as applying
to _fides formata_.  Here they add no correction that there is also
need of the faith that holds that we are accounted righteous for the
sake of Christ as Propitiator.  Thus the adversaries exclude Christ
from justification, and teach only a righteousness of the Law.  But
let us return to Paul.  No one can infer anything more from this text
than that love is necessary.  This we confess.  So also not to commit
theft is necessary.  But the reasoning will not be correct if some
one would desire to frame thence an argument such as this: "Not to
commit theft is necessary.  Therefore, not to commit theft justifies."
Because justification is not the approval of a certain work, but of
the entire person.  Hence this passage from Paul does not harm us;
only the adversaries must not in imagination add to it whatever they
please.  For he does not say that love justifies, but: ["And if I
have not love"] "I am nothing," namely, that faith, however great it
may have been, is extinguished.  He does not say that love overcomes
the terrors of sin and of death that we can set our love against the
wrath and judgment of God, that our love satisfies God's Law, that
without Christ as Propitiator we have access, by our love, to God,
that by our love we receive the promised remission of sins.  Paul
says nothing of this.  He does not, therefore, think that love
justifies, because we are justified only when we apprehend Christ as
Propitiator, and believe that for Christ's sake God is reconciled to
us.  Neither is justification even to be dreamed of with the omission
of Christ as Propitiator.  If there be no need of Christ, if by our
love we can overcome death, if by our love, without Christ as
Propitiator' we have access to God, then let our adversaries remove
the promise concerning Christ, then let them abolish the Gospel
[which teaches that we have access to God through Christ as
Propitiator, and that we are accepted not for the sake of our
fulfilling of the Law, but for Christ's sake].  The adversaries
corrupt very many passages, because they bring to them their own
opinions, and do not derive the meaning from the passages themselves.
For what difficulty is there in this passage if we remove the
interpretation which the adversaries, who do not understand what
justification is or how it occurs [what faith is, what Christ is, or
how a man is justified before God], out of their own mind attach to
it?  The Corinthians, being justified before, had received many
excellent gifts.  In the beginning they glowed with zeal, just as is
generally the case.  Then dissensions [factions and sects] began to
arise among them as Paul indicates; they began to dislike good
teachers.  Accordingly, Paul reproves them, recalling them [to unity
and] to offices of love.  Although these are necessary, yet it would
be foolish to imagine that works of the Second Table, through which
we have to do with man and not properly with God, justify us.  But in
justification we have to treat with God; His wrath must be appeased,
and conscience must be pacified with respect to God.  None of these
occur through the works of the Second Table [by love, but only by
faith, which apprehends Christ and the promise of God.  However, it
is true that losing love involves losing the Spirit and faith.  And
thus Paul says: If I have not love, I am nothing.  But he does not
add the affirmative statement, that love justifies in the sight of
God].

But they object that love is preferred to faith and hope.  For Paul
says, 1 Cor. 13, 13: The greatest of these is charity.  Now, it is
reasonable that the greatest and chief virtue should justify,
although Paul, in this passage, properly speaks of love towards one's
neighbor, and indicates that love is the greatest, because it has
most fruits.  Faith and hope have to do only with God; but love has
infinite offices externally towards men.  [Love goes forth upon earth
among the people, and does much good, by consoling, teaching,
instructing, helping, counseling privately and publicly.]
Nevertheless, let us, indeed, grant to the adversaries that love
towards God and our neighbor is the greatest virtue, because the
chief commandment is this: Thou shalt love the Lord, thy God Matt. 22,
37. But how will they infer thence that love justifies?  The
greatest virtue, they say, justifies.  By no means.  [It would be
true if we had a gracious God because of our virtue.  Now, it was
proven above that we are accepted and justified for Christ's sake,
not because of our virtue, for our virtue is impure.] For just as
even the greatest or first Law does not justify, so also the greatest
virtue of the Law does not justify.  [For, as the Law and virtue is
higher, and our ability to do the same proportionately lower, we are
not righteous because of love.] But that virtue justifies which
apprehends Christ, which communicates to us Christ's merits, by which
we receive grace and peace from God.  But this virtue is faith.  For
as it has been often said, faith is not only knowledge, but much
rather willing to receive or apprehend those things which are offered
in the promise concerning Christ.  Moreover this obedience towards
God, namely, to wish to receive the offered promise, is no less a
divine service, _latreia_, than is love.  God wishes us to believe
Him, and to receive from Him blessings, and this He declares to be
true divine service.

But the adversaries ascribe justification to love because they
everywhere teach and require the righteousness of the Law.  For we
cannot deny that love is the highest work of the Law.  And human
wisdom gazes at the Law, and seeks in it justification.  Accordingly,
also the scholastic doctors, great and talented men, proclaim this as
the highest work of the Law, and ascribe to this work justification.
But deceived by human wisdom, they did not look upon the uncovered,
but upon the veiled face of Moses, just as the Pharisees,
philosophers, Mahometans.  But we preach the foolishness of the
Gospel, in which another righteousness is revealed, namely, that for
the sake of Christ, as Propitiator, we are accounted righteous, when
we believe that for Christ's sake God has been reconciled to us.
Neither are we ignorant how far distant this doctrine is from the
judgment of reason and of the Law.  Nor are we ignorant that the
doctrine of the Law concerning love makes a much greater show; for it
is wisdom.  But we are not ashamed of the foolishness of the Gospel.
For the sake of Christ's glory we defend this, and beseech Christ, by
His Holy Ghost, to aid us that we may be able to make this clear and
manifest.

The adversaries, in the Confutation, have also cited against us Col.
3, 14: Charity, which is the bond of perfectness.  From this they
infer that love justifies because it renders men perfect.  Although a
reply concerning perfection could here be made in many ways, yet we
will simply recite the meaning of Paul.  It is certain that Paul
spoke of love towards one's neighbor.  Neither must we indeed think
that Paul would ascribe either justification or perfection to the
works of the Second Table, rather than to those of the First.  And if
love render men perfect, there will then be no need of Christ as
Propitiator, [However, Paul teaches in all places that we are
accepted on account of Christ, and not on account of our love, or our
works, or of the Law; for no saint (as was stated before) perfectly
fulfils the Law.  Therefore since he in all places writes and teaches
that in this life there is no perfection in our works, it is not to
be thought that he speaks here of personal perfection.] for faith
apprehends Christ only as Propitiator.  This, however, is far distant
from the meaning of Paul, who never suffers Christ to be excluded as
Propitiator.  Therefore he speaks not of personal perfection, but of
the integrity common to the Church [concerning the unity of the
Church and the word which they interpret as perfection means nothing
else than to be not rent].  For on this account he says that love is
a bond or connection, to signify that he speaks of the binding and
joining together, with each other, of the many members of the Church.
For just as in all families and in all states concord should be
nourished by mutual offices, and tranquillity cannot be retained
unless men overlook and forgive certain mistakes among themselves; so
Paul commands that there should be love in the Church in order that
it may preserve concord, bear with the harsher manners of brethren as
there is need, overlook certain less serious mistakes, lest the
Church fly apart into various schisms, and enmities and factions and
heresies arise from the schisms.

For concord must necessarily he rent asunder whenever either the
bishops impose [without cause] upon the people heavier burdens, or
have no respect to weakness in the people.  And dissensions arise
when the people judge too severely [quickly censure and criticize]
concerning the conduct [walk and life] of teachers [bishops or
preachers], or despise the teachers because of certain less serious
faults; for then both another kind of doctrine and other teachers are
sought after.  On the other hand, perfection, i.e., the integrity of
the Church, is preserved, when the strong bear with the weak, when
the people take in good part some faults in the conduct of their
teachers [have patience also with their preachers], when the bishops
make some allowances for the weakness of the people [know how to
exercise forbearance to the people, according to circumstances, with
respect to all kinds of weaknesses and faults].  Of these precepts of
equity the books of all the wise are full, namely, that in every day
life we should make many allowances mutually for the sake of common
tranquillity.  And of this Paul frequently teaches both here and
elsewhere.  Wherefore the adversaries argue indiscreetly from the
term "perfection" that love justifies, while Paul speaks of common
integrity and tranquillity.  And thus Ambrose interprets this passage:
Just as a building is said to be perfect or entire when all its
parts are fitly joined together with one another.  Moreover, it is
disgraceful for the adversaries to preach so much concerning love
while they nowhere exhibit it.  What are they now doing?  They are
rending asunder churches, they are writing laws in blood, and are
proposing to the most clement prince, the Emperor, that these should
be promulgated; they are slaughtering priests and other good men, if
any one have [even] slightly intimated that he does not entirely
approve some manifest abuse.  [They wish all dead who say a single
word against their godless doctrine.] These things are not consistent
with those declamations of love, which if the adversaries would
follow, the churches would be tranquil and the state have peace.  For
these tumults would be quieted if the adversaries would not insist
with too much bitterness [from sheer vengeful spite and pharisaical
envy, against the truth which they have perceived] upon certain
traditions, useless for godliness, most of which not even those very
persons observe who most earnestly defend them.  But they easily
forgive themselves, and yet do not likewise forgive others, according
to the passage in the poet: I forgive myself, Maevius said.  But this
is very far distant from those encomiums of love which they here
recite from Paul, nor do they understand the word any more than the
walls which give it back.  From Peter they cite also this sentence, 1
Pet. 4, 8: Charity shall cover the multitude of sins.  It is evident
that also Peter speaks of love towards one's neighbor, because he
joins this passage to the precept by which he commands that they
should love one another.  Neither could it have come into the mind of
any apostle that our love overcomes sin and death; that love is the
propitiation on account of which to the exclusion of Christ as
Mediator, God is reconciled; that love is righteousness without
Christ as Mediator.  For this love, if there would be any, would be a
righteousness of the Law, and not of the Gospel, which promises to us
reconciliation and righteousness if we believe that, for the sake of
Christ as Propitiator, the Father has been reconciled, and that the
merits of Christ are bestowed upon us.  Peter, accordingly, urges us,
a little before, to come to Christ that we may be built upon Christ.
And he adds, 1 Pet. 2, 4-6: He that believeth on Him shall not be
confounded.  When God judges and convicts us, our love does not free
us from confusion [from our works and lives, we truly suffer shame].
But faith in Christ liberates us in these fears, because we know that
for Christ's sake we are forgiven.

Besides, this sentence concerning love is derived from Prov. 10,12,
where the antithesis clearly shows how it ought to be understood:
Hatred stirreth up strifes; but love covereth all sins.  It teaches
precisely the same thing as that passage of Paul taken from
Colossians, that if any dissensions would occur, they should be
moderated and settled by our equitable and lenient conduct.
Dissensions, it says, increase by means of hatred, as we often see
that from the most trifling offenses tragedies arise [from the
smallest sparks a great conflagration arises].  Certain trifling
offenses occurred between Caius Caesar and Pompey, in which, if the
one had yielded a very little to the other, civil war would not have
arisen.  But while each indulged his own hatred, from a matter of no
account the greatest commotions arose.  And many heresies have arisen
in the Church only from the hatred of the teachers.  Therefore it
does not refer to a person's own faults, but to the faults of others,
when it says: Charity covereth sins, namely, those of others, and
that, too, among men, i.e., even though these offenses occur, yet
love overlooks them, forgives, yields, and does not carry all things
to the extremity of justice.  Peter, therefore, does not mean that
love merits in God's sight the remission of sins, that it is a
propitiation to the exclusion of Christ as Mediator, that it
regenerates and justifies, but that it is not morose, harsh,
intractable towards men, that it overlooks some mistakes of its
friends, that it takes in good part even the harsher manners of
others, just as the well-known maxim enjoins: Know, but do rot hate,
the manners of a fiend.  Nor was it without design that the apostle
taught so frequently concerning this office what the philosophers
call epieicheia, leniency.  For this virtue is necessary for
retaining public harmony [in the Church and the civil government],
which cannot last unless pastors and Churches mutually overlook and
pardon many things [if they want to be extremely particular about
every defect, and do not allow many things to flow by without
noticing them].

From James they cite 2, 24: Ye see, then how by works a man is
justified, and not by faith alone.  Nor is any other passage supposed
to be more contrary to our belief.  But the reply is easy and plain.
If the adversaries do not attach their own opinions concerning the
merits of works, the words of James have in them nothing that is of
disadvantage.  But wherever there is mention of works, the
adversaries add falsely their own godless opinions, that by means of
good works we merit the remission of sins; that good works are a
propitiation and price on account of which God is reconciled to us;
that good works overcome the terrors of sin and of death; that good
works are accepted in God's sight on account of their goodness; and
that they do not need mercy and Christ as Propitiator.  None of all
these things came into the mind of James, which the adversaries
nevertheless, defend under the pretext of this passage of James.

In the first place, then, we must ponder this, namely, that the
passage is more against the adversaries than against us.  For the
adversaries teach that man is justified by love and works.  Of faith,
by which we apprehend Christ as Propitiator, they say nothing.  Yea
they condemn this faith; nor do they condemn it only in sentences and
writings, but also by the sword and capital punishments they endeavor
to exterminate it in the Church.  How much better does James teach,
who does not omit faith, or present love in preference to faith, but
retains faith, so that in justification Christ may not be excluded as
Propitiator!  Just as Paul also, when he treats of the sum of the
Christian life, includes faith and love, 1 Tim. 1, 5: The end of the
commandment is charity out of a pure heart, and of a good conscience,
and of faith unfeigned.

Secondly, the subject itself declares that here such works are spoken
of as follow faith, and show that faith is not dead, but living and
efficacious in the heart.  James, therefore, did not believe that by
good works we merit the remission of sins and grace.  For he speaks
of the works of those who have been justified, who have already been
reconciled and accepted, and have obtained remission of sins.
Wherefore the adversaries err when they infer that James teaches that
we merit remission of sins and grace by good works, and that by our
works we have access to God, without Christ as Propitiator.




Part 8


Thirdly, James has spoken shortly before concerning regeneration,
namely, that it occurs through the Gospel.  For thus he says 1, 18:
Of His own will begat He us with the Word of Truth, that we should be
a kind of first-fruits of His creatures.  When he says that we have
been born again by the Gospel, he teaches that we have been born
again and justified by faith.  For the promise concerning Christ is
apprehended only by faith, when we set it against the terrors of sin
and of death.  James does not, therefore, think that we are born
again by our works.

From these things it is clear that James does not contradict us, who,
when censuring idle and secure minds, that imagine that they have
faith, although they do not have it, made a distinction between dead
and living faith.  He says that that is dead which does not bring
forth good works [and fruits of the Spirit: obedience, patience,
chastity, love]; he says that that is living which brings forth good
works.  Furthermore, we have frequently already shown what we term
faith.  For we do not speak of idle knowledge [that merely the
history concerning Christ should be known], such as devils have, but
of faith which resists the terrors of conscience, and cheers and
consoles terrified hearts [the new light and power which the Holy
Ghost works in the heart, through which we overcome the terrors of
death, of sin, etc.].  Such faith is neither an easy matter, as the
adversaries dream [as they say: Believe, believe, how easy it is to
believe! etc.], nor a human power [thought which I can form for
myself], but a divine power, by which we are quickened, and by which
we overcome the devil and death.  Just as Paul says to the Colossians,
2, 12, that faith is efficacious through the power of God, and
overcomes death: Wherein also ye are risen with Him through the faith
of the operation of God.  Since this faith is a new life, it
necessarily produces new movements and works.  [Because it is a new
light and life in the heart, whereby we obtain another mind and
spirit, it is living, productive, and rich in good works.]
Accordingly, James is right in denying that we are justified by such
a faith as is without works.  But when he says that we are justified
by faith and works, he certainly does not say that we are born again
by works.  Neither does he say this, that partly Christ is our
Propitiator, and partly our works are our propitiation.  Nor does he
describe the mode of justification, but only of what nature the just
are, after they have been already justified and regenerated.  [For he
is speaking of works which should follow faith.  There it is well
said: He who has faith and good works is righteous; not, indeed, on
account of the works, but for Christ's sake, through faith.  And as a
good tree should bring forth good fruit, and yet the fruit does not
make the tree good, so good works must follow the new birth, although
they do not make man accepted before God; but as the tree must first
be good, so also must man be first accepted before God by faith for
Christ's sake.  The works are too insignificant to render God
gracious to us for their sake, if He were not gracious to us for
Christ's sake.  Therefore James does not contradict St. Paul, and
does not say that by our works we merit, etc.] And here to be
justified does not mean that a righteous man is made from a wicked
man, but to be pronounced righteous in a forensic sense, as also in
the passage Rom. 2, 13: The doers of the Law shall be justified.  As,
therefore, these words: The doers of the Law shall be justified,
contain nothing contrary to our doctrine, so, too, we believe
concerning the words of James: By works a man is justified, and not
by faith alone, because men having faith and good works are certainly
pronounced righteous.  For, as we have said, the good works of saints
are righteous, and please on account of faith.  For James commends
only such works as faith produces, as he testifies when he says of
Abraham, 2, 21: Faith wrought with his works.  In this sense it is
said: The doers of the Law are justified, i.e., they are pronounced
righteous who from the heart believe God, and afterwards have good
fruits which please Him on account of faith, and accordingly, are the
fulfilment of the Law.  These things, simply spoken, contain nothing
erroneous, but they are distorted by the adversaries who attach to
them godless opinions out of their mind.  For it does not follow
hence that works merit the remission of sins; that works regenerate
hearts; that works are a propitiation, that works please without
Christ as Propitiator; that works do not need Christ as Propitiator.
James says nothing of these things, which, nevertheless, the
adversaries shamelessly infer from the words of James.

Certain other passages concerning works are also cited against us.
Luke 6, 37: Forgive, and ye shall be forgiven.  Is. 58, 7 [9]: Is it
not to deal thy bread to the hungry?...Then shalt thou call, and the
Lord will answer.  Dan. 4, 24 [27]: Break off thy sins, by showing
mercy to the poor.  Matt. 5, 3: Blessed are the poor in spirit; for
theirs is the kingdom of heaven; and v. 7: Blessed are the merciful;
for they shall obtain mercy.  Even these passages would contain
nothing contrary to us if the adversaries would not falsely attach
something to them.  For they contain two things: The one is a
preaching either of the Law or of repentance, which not only convicts
those doing wrong, but also enjoins them to do what is right; the
other is a promise which is added.  But it is not added that sins are
remitted without faith, or that works themselves are a propitiation.
Moreover, in the preaching of the Law these two things ought always
to be understood, namely: First, that the Law cannot be observed
unless we have been regenerated by faith in Christ, just as Christ
says, John 15, 5: Without Me ye can do nothing.  Secondly, and though
some external works can certainly be done, this general judgment:
Without faith it is impossible to please God, which interprets the
whole Law, must be retained: and the Gospel must be retained, that
through Christ we have access to the Father, Heb. 10, 19, Rom. 5, 2.
For it is evident that we are not justified by the Law.  Otherwise,
why would there be need of Christ or the Gospel, if the preaching of
the Law alone would be sufficient?  Thus in the preaching of
repentance, the preaching of the Law, or the Word convicting of sin,
is not sufficient, because the Law works wrath, and only accuses,
only terrifies consciences, because consciences never are at rest,
unless they hear the voice of God in which the remission of sins is
clearly promised.  Accordingly, the Gospel must be added, that for
Christ's sake sins are remitted, and that we obtain remission of sins
by faith in Christ.  If the adversaries exclude the Gospel of Christ
from the preaching of repentance, they are judged aright to be
blasphemers against Christ.

Therefore, when Isaiah, 1, 16. 18, preaches repentance: Cease to do
evil; learn to do well; seek judgment, relieve the oppressed, judge
the fatherless, plead for the widow.  Come now and let us reason
together, saith the Lord; though your sine be as scarlet, they shall
be white as snow, the prophet thus both exhorts to repentance, and
adds the promise.  But it would be foolish to consider in such a
sentence only the words: Relieve the oppressed; judge the fatherless.
For he says in the beginning: Cease to do evil, where he censures
impiety of heart and requires faith.  Neither does the prophet say
that through the works: Relieve the oppressed, judge the fatherless,
they can merit the remission of sins _ex opere operato_, but he
commands such works as are necessary in the new life.  Yet, in the
mean time, he means that remission of sins is received by faith, and
accordingly the promise is added.  Thus we must understand all
similar passages.  Christ preaches repentance when He says: Forgive,
and He adds the promise: And ye shall be forgiven, Luke 6, 37. Nor,
indeed, does He say this, namely, that, when we forgive, by this work
of ours we merit the remission of sins _ex opere operato_, as they
term it, but He requires a new life, which certainly is necessary.
Yet, in the mean time He means that remission of sins is received by
faith.  Thus, when Isaiah says, 58, 7: Deal thy bread to the hungry,
he requires a new life.  Nor does the prophet speak of this work
alone, but, as the text indicates, of the entire repentance; yet, in
the mean time, he intends that remission of sins is received by faith.
For the position is sure, and none of the gates of hell can
overthrow it, that in the preaching of repentance the preaching of
the Law is not sufficient, because the Law works wrath and always
accuses.  But the preaching of the Gospel should be added, namely,
that in this way remission of sins is granted us, if we believe that
sins are remitted us for Christ's sake.  Otherwise, why would there
be need of the Gospel, why would there be need of Christ?  This
belief ought always to be in view, in order that it may be opposed to
those who, Christ being cast aside and the Gospel being blotted out,
wickedly distort the Scriptures to the human opinions, that by our
works we purchase remission of sins.

Thus also in the sermon of Daniel, 4, 24, faith is required.  [The
words of the prophet which were full of faith and spirit, we must not
regard as heathenish as those of Aristotle or any other heathen.
Aristotle also admonished Alexander that he should not use his power
for his own wantonness, but for the improvement of countries and men.
This was written correctly and well; concerning the office of king
nothing better can be preached or written.  But Daniel is speaking to
his king, not only concerning his office as king, but concerning
repentance, the forgiveness of sins, reconciliation to God, and
concerning sublime, great, spiritual subjects, which far transcend
human thoughts and works.] For Daniel did not mean that the king
should only bestow alms [which even a hypocrite can do], but embraces
repentance when he says: Break off [Redeem, Vulg.] thy iniquities by
showing mercy to the poor, i.e. break off thy sins by a change of
heart and works.  But here also faith is required.  And Daniel
proclaims to him many things concerning the worship of the only God,
the God of Israel, and converts the king not only to bestow alms, but
much more to faith.  For we have the excellent confession of the king
concerning the God of Israel: There is no other God that can deliver
after this sort Dan. 3, 29. Therefore, in the sermon of Daniel there
are two parts.  The one part is that which gives commandment
concerning the new life and the works of the new life.  The other
part is, that Daniel promises to the king the remission of sins.
[Now, where there is a promise, faith is required.  For the promise
cannot be received in any other way than by the heart's relying on
such word of God, and not regarding its own worthiness or
unworthiness.  Accordingly, Daniel also demands faith: for thus the
promise reads: There will be healing for thy offenses.] And this
promise of the remission of sins is not a preaching of the Law, but a
truly prophetical and evangelical voice, of which Daniel certainly
meant that it should be received in faith.  For Daniel knew that the
remission of sins in Christ was promised not only to the Israelites,
but also to all nations.  Otherwise he could not have promised to the
king the remission of sins.  For it is not in the power of man
especially amid the terrors of sin, to assert without a sure word of
God concerning God's will, that He ceases to be angry.  And the words
of Daniel speak in his own language still more clearly of repentance
and still more clearly bring out the promise.  Redeem thy sins by
righteousness and thy iniquities by favors toward the poor.  These
words teach concerning the whole of repentance.  [It is as much as to
say: Amend your life!  And it is true, when we amend our lives, we
become rid of sin.] For they direct him to become righteous, then to
do good works, to defend the miserable against injustice, as was the
duty of a king.  But righteousness is faith in the heart.  Moreover,
sins are redeemed by repentance, i.e. the obligation or guilt is
removed, because God forgives those who repent, as it is written in
Ezek. 18, 21. 22. Nor are we to infer from this that He forgives on
account of works that follow, on account of alms, but on account of
His promise He forgives those who apprehend His promise.  Neither do
any apprehend His promise, except those who truly believe, and by
faith overcome sin and death.  These, being regenerated, ought to
bring forth fruits worthy of repentance, just as John says, Matt. 3,
8. The promise, therefore, was added: So, there will be healing for
thy offenses, Dan. 4, 24. [Daniel does not only demand works, but
says: Redeem thy sins by righteousness.  Now, everybody knows that in
Scripture righteousness does not mean only external works, but
embraces faith, as Paul says: _Iustus ex fide vivet_?  The just shall
live by his faith, Heb. 10, 38. Hence, Daniel first demands faith
when he mentions righteousness and says: Redeem thy sins by
righteousness, that is, by faith toward God, by which thou art made
righteous.  In addition to this do good works, administer your office,
do not be a tyrant, but see that your government be profitable to
your country and people, preserve peace, and protect the poor against
unjust force.  These are princely alms.] Jerome here added a particle
expressing doubt, that is beside the matter, and in his commentaries
contends much more unwisely that the remission of sins is uncertain.
But let us remember that the Gospel gives a sure promise of the
remission of sins.  And to deny that there must be a sure promise of
the remission of sins would completely abolish the Gospel.  Let us
therefore dismiss Jerome concerning this passage.  Although the
promise is displayed even in the word redeem.  For it signifies that
the remission of sins is possible that sins can be redeemed, i.e.,
that their obligation or guilt can be removed, or the wrath of God
appeased.  But our adversaries, overlooking the promises, everywhere,
consider only the precepts, and attach falsely the human opinion that
remission occurs on account of works, although the text does not say
this, but much rather requires faith.  For wherever a promise is,
there faith is required.  For a promise cannot be received unless by
faith.  [The same answer must also be given in reference to the
passage from the Gospel: Forgive, and you will be forgiven.  For this
is just such a doctrine of repentance.  The first part in this
passage demands amendment of life and good works, the other part adds
the promise.  Nor are we to infer from this that our forgiving merits
for us _ex opere operato_ remission of sin.  For that is not what
Christ says, but as in other sacraments Christ has attached the
promise to an external sign, so He attaches the promise of the
forgiveness of sin in this place to external good works.  And as in
the Lord's Supper we do not obtain forgiveness of sin without faith,
_ex opere operato_, so neither in this when we forgive.  For, our
forgiving is not a good work, except it is performed by a person
whose sins have been previously forgiven by God in Christ.  If,
therefore, our forgiving is to please God, it must follow after the
forgiveness which God extends to us.  For, as a rule, Christ combines
these two, the Law and the Gospel, both faith and good works, in
order to indicate that, where good works do not follow, there is no
faith either that we may have external marks, which remind us of the
Gospel and the forgiveness of sin, for our comfort and that thus our
faith may be exercised in many ways.  In this manner we are to
understand such passages, otherwise they would directly contradict
the entire Gospel, and our beggarly works would be put in the place
of Christ, who alone is to be the propitiation, which no man is by
any means to despise.  Again, if these passages were to be understood
as relating to works, the remission of sins would be quite uncertain;
for it would rest on a poor foundation, on our miserable works.]

But works become conspicuous among men.  Human reason naturally
admires these, and because it sees only works, and does not
understand or consider faith, it dreams accordingly that these works
merit remission of sins and justify.  This opinion of the Law inheres
by nature in men's minds; neither can it be expelled, unless when we
are divinely taught.  But the mind must be recalled from such carnal
opinions to the Word of God.  We see that the Gospel and the promise
concerning Christ have been laid before us.  When, therefore, the Law
is preached, when works are enjoined, we should not spurn the promise
concerning Christ.  But the latter must first be apprehended, in
order that we may be able to produce good works, and our works may
please God, as Christ says, John 16; 5: With out Me ye can do nothing.
Therefore, if Daniel would have used such words as these: "Redeem
your sins by repentance," the adversaries would take no notice of
this passage.  Now, since he has actually expressed this thought in
apparently other words, the adversaries distort his words to the
injury of the doctrine of grace and faith, although Daniel meant most
especially to include faith.  Thus, therefore, we reply to the words
of Daniel, that, inasmuch as he is preaching repentance, he is
teaching not only of works, but also of faith, as the narrative
itself in the context testifies.  Secondly, because Daniel clearly
presents the promise, he necessarily requires faith which believes
that sins are freely remitted by God.  Although, therefore, in
repentance he mentions works, yet Daniel does not say that by these
works we merit remission of sins.  For Daniel speaks not only of the
remission of the punishment; because remission of the punishment is
sought for in vain unless the heart first receive the remission of
guilt.  Besides, if the adversaries understand Daniel as speaking
only of the remission of punishment, this passage will prove nothing
against us, because it will thus be necessary for even them to
confess that the remission of sin and free justification precede.
Afterwards even we concede that the punishments by which we are
chastised, are mitigated by our prayers and good works, and finally
by our entire repentance, according to 1 Cor. 11, 31: For if we would
judge ourselves, we should not be judged.  And Jer. 15, 19: If thou
return, then will I bring thee again.  And Zech. 1, 3: Turn ye unto
Me, and I will turn unto you.  And Ps. 50, 15: Call upon Me in the
day of trouble.

Let us, therefore, in all our encomiums upon works and in the
preaching of the Law retain this rule: that the Law is not observed
without Christ.  As He Himself has said: Without Me ye can do nothing.
Likewise that: Without faith it is impossible to please God, Heb.
11, 6. For it is very certain that the doctrine of the Law is not
intended to remove the Gospel, and to remove Christ as Propitiator.
And let the Pharisees, our adversaries, be cursed, who so interpret
the Law as to ascribe the glory of Christ to works namely, that they
are a propitiation, that they merit the remission of sins.  It
follows, therefore, that works are always thus praised, namely, that
they are pleasing on account of faith, as works do not please without
Christ as Propitiator.  By Him we have access to God, Rom. 5, 2, not
by works, without Christ as Mediator.  Therefore, when it is said,
Matt. 19, 17: If thou wilt enter into life, keep the commandments, we
must believe that without Christ the commandments are not kept, and
without Him cannot please.  Thus in the Decalog itself, in the First
Commandment Ex. 20, 6: Showing mercy unto thousands of them that love
Me and keep My commandments, the most liberal promise of the Law is
added.  But this Law is not observed without Christ.  For it always
accuses the conscience which does not satisfy the Law, and therefore
in terror, flies from the judgment and punishment of the Law.
Because the Law worketh wrath, Rom. 4, 15. Man observes the Law,
however, when he hears that for Christ's sake God is reconciled to us,
even though we cannot satisfy the Law.  When, by this faith, Christ
is apprehended as Mediator, the heart finds rest, and begins to love
God and observe the Law, and knows that now, because of Christ as
Mediator, it is pleasing to God, even though the inchoate fulfilling
of the Law be far from perfection and be very impure.  Thus we must
judge also concerning the preaching of repentance.  For although in
the doctrine of repentance the scholastics have said nothing at all
concerning faith, yet we think that none of our adversaries is so mad
as to deny that absolution is a voice of the Gospel.  And absolution
ought to be received by faith, in order that it may cheer the
terrified conscience.

Therefore the doctrine of repentance, because it not only commands
new works, but also promises the remission of sins, necessarily
requires faith.  For the remission of sins is not received unless by
faith.  Therefore, in those passages that refer to repentance, we
should always understand that not only works, but also faith is
required, as in Matt. 6, 14. For if ye forgive men their trespasses,
your heavenly Father will also forgive you.  Here a work is required,
and the promise of the remission of sins is added which does not
occur on account of the work, but through faith, on account of Christ.
Just as Scripture testifies in many passages: Acts 10, 43: To Him
give all the prophets witness that through His name, whosoever
believeth in Him, shall receive remission of sins; and 1 John 2, 12:
Your sins are forgiven you for His name's sake; Eph. 1, 7: In whom we
have redemption through His blood the forgiveness of sins.  Although
what need is there to recite testimonies?  This is the very voice
peculiar to the Gospel, namely, that for Christ's sake, and not for
the sake of our works, we obtain by faith remission of sins.  Our
adversaries endeavor to suppress this voice of the Gospel by means of
distorted passages which contain the doctrine of the Law, or of works.
For it is true that in the doctrine of repentance works are
required, because certainly a new life is required.  But here the
adversaries wrongly add that by such works we merit the remission of
sins, or justification.  And yet Christ often connects the promise of
the remission of sins to good works not because He means that good
works are a propitiation, for they follow reconciliation; but for two
reasons.  One is, because good fruits must necessarily follow.
Therefore He reminds us that, if good fruits do not follow the
repentance is hypocritical and feigned.  The other reason is, because
we have need of external signs of so great a promise, because a
conscience full of fear has need of manifold consolation.  As,
therefore, Baptism and the Lord's Supper are signs that continually
admonish, cheer, and encourage desponding minds to believe the more
firmly that their sins are forgiven, so the same promise is written
and portrayed in good works, in order that these works may admonish
us to believe the more firmly.  And those who produce no good works
do not excite themselves to believe, but despise these promises.  The
godly on the other hand, embrace them, and rejoice that they have the
signs and testimonies of so great a promise.  Accordingly, they
exercise themselves in these signs and testimonies.  Just as,
therefore, the Lord's Supper does not justify us _ex opere operato_,
without faith, so alms do not justify us without faith, _ex opere
operato_.

So also the address of Tobias, 4, 11, ought to be received: Alms free
from every sin and from death.  We will not say that this is
hyperbole, although it ought thus to be received, so as not to
detract from the praise of Christ, whose prerogative it is to free
from sin and death.  But we must come back to the rule that without
Christ the doctrine of the Law is of no profit.  Therefore those alms
please God which follow reconciliation or justification, and not
those which precede.  Therefore they free from sin and death, not _ex
opere operato_, but, as we have said above concerning repentance,
that we ought to embrace faith and its fruits, so here we must say
concerning alms that this entire newness of life saves [that they
please God because they occur in believers].  Alms also are the
exercises of faith, which receives the remission of sins and
overcomes death, while it exercises itself more and more, and in
these exercises receives strength.  We grant also this, that alms
merit many favors from God [but they cannot overcome death, hell, the
devil, sins, and give the conscience peace (for this must occur alone
through faith in Christ)], mitigate punishments, and that they merit
our defense in the dangers of sins and of death, as we have said a
little before concerning the entire repentance.  [This is the simple
meaning, which agrees also with other passages of Scripture.  For
wherever in the Scriptures good works are praised, we must always
understand them according to the rule of Paul, that the Law and works
must not be elevated above Christ, but that Christ and faith are as
far above all works as the heavens are above the earth.] And the
address of Tobias, regarded as a whole shows that faith is required
before alms, 4, 5: Be mindful of the Lord, thy God, all thy days And
afterwards, v. 19. Bless the Lord, thy God, always, and desire of Him
that thy ways be directed.  This, however, belongs properly to that
faith of which we speak, which believes that God is reconciled to it
because of His mercy, and which wishes to be justified, sanctified,
and governed by God.  But our adversaries, charming men, pick out
mutilated sentences, in order to deceive those who are unskilled.
Afterwards they attach something from their own opinions.  Therefore,
entire passages are to be required, because, according to the common
precept, it is unbecoming, before the entire Law is thoroughly
examined, to judge or reply when any single clause of it is presented.
And passages, when produced in their entirety, very frequently
bring the interpretation with them.

Luke 11, 41 is also cited in a mutilated form, namely: Give alms of
such things as ye have; and, behold, all things are clean unto you.
The adversaries are very stupid [are deaf, and have callous ears;
therefore, we must so often etc.].  For time and again we have said
that to the preaching of the Law there should be added the Gospel
concerning Christ, because of whom good works are pleasing, but they
everywhere teach [without shame] that, Christ being excluded,
justification is merited by the works of the Law.  When this passage
is produced unmutilated, it will show that faith is required.  Christ
rebukes the Pharisees who think that they are cleansed before God i.e.
, that they are justified by frequent ablutions [by all sorts of
_baptismata carnis_, that is, by all sorts of baths, washings, and
cleansings of the body, of vessels, of garments].  Just as some Pope
or other says of the water sprinkled with salt that it sanctifies and
cleanses the people; and the gloss says that it cleanses from venial
sins.  Such also were the opinions of the Pharisees which Christ
reproved, and to this feigned cleansing He opposes a double cleanness,
the one internal, the other external.  He bids them be cleansed
inwardly [(which occurs only through faith)], and adds concerning the
outward cleanness: Give alms of such things as ye have; and, behold,
all things are clean unto you.  The adversaries do not apply aright
the universal particle all things; for Christ adds this conclusion to
both members: "All things will be clean unto you, if you will be
clean within, and will outwardly give alms." For He indicates that
outward cleanness is to be referred to works commanded by God, and
not to human traditions, such as the ablutions were at that time, and
the daily sprinkling of water, the vesture of monks, the distinctions
of food, and similar acts of ostentation are now.  But the
adversaries distort the meaning by sophistically transferring the
universal particle to only one part: "All things will be clean to
those having given alms." [As if any one would infer: Andrew is
present; therefore all the apostles are present.  Wherefore in the
antecedent both members ought to be joined: Believe and give alms.
For to this the entire mission, the entire office of Christ points;
to this end He is come that we should believe in Him.  Now, if both
parts are combined, believing and giving alms, it follows rightly
that all things are clean: the heart by faith, the external
conversation by good works.  Thus we must combine the entire sermon,
and not invert the parts, and interpret the text to mean that the
heart is cleansed from sin by alms.  Moreover, there are some who
think that these words were spoken by Christ against the Pharisees
ironically, as if He meant to say: Aye, my dear lords, rob and steal,
and then go and give alms, and you will be promptly cleansed, so that
Christ would in a somewhat sarcastic and mocking way puncture their
pharisaical hypocrisy.  For, although they abounded in unbelief,
avarice, and every evil work, they still observed their purifications,
gave alms, and believed that they were quite pure, lovely saints.
This interpretation is not contrary to the text.] Yet Peter says,
Acts 15, 9, that hearts are purified by faith.  And when this entire
passage is examined, it presents a meaning harmonizing with the rest
of Scripture, that, if the hearts are cleansed and then outwardly
alms are added, i.e., all the works of love, they are thus entirely
clean i.e. not only within, but also without.  And why is not the
entire discourse added to it?  There are many parts of the reproof,
some of which give commandment concerning faith and others concerning
works.  Nor is it the part of a candid reader to pick out the
commands concerning works, while the passages concerning faith are
omitted.

Lastly, readers are to be admonished of this, namely, that the
adversaries give the worst advice to godly consciences when they
teach that by works the remission of sins is merited, because
conscience, in acquiring remission through works, cannot be confident
that the work will satisfy God.  Accordingly, it is always tormented,
and continually devises other works and other acts of worship until
it altogether despairs.  This course is described by Paul, Rom. 4, 6,
where he proves that the promise of righteousness is not obtained
because of our works, because we could never affirm that we had a
reconciled God.  For the Law always accuses.  Thus the promise would
be in vain and uncertain.  He accordingly concludes that this promise
of the remission of sins and of righteousness is received by faith,
not on account of works.  This is the true, simple, and genuine
meaning of Paul, in which the greatest consolation is offered godly
consciences, and the glory of Christ is shown forth, who certainly
was given to us for this purpose, namely, that through Him we might
have grace, righteousness, and peace.

Thus far we have reviewed the principal passages which the
adversaries cite against us, in order to show that faith does not
justify, and that we merit, by our works, remission of sins and grace.
But we hope that we have shown clearly enough to godly consciences
that these passages are not opposed to our doctrine; that the
adversaries wickedly distort the Scriptures to their opinions; that
the most of the passages which they cite have been garbled; that,
while omitting the clearest passages concerning faith, they only
select from the Scriptures passages concerning works, and even these
they distort; that everywhere they add certain human opinions to that
which the words of Scripture say; that they teach the Law in such a
manner as to suppress the Gospel concerning Christ.  For the entire
doctrine of the adversaries is, in part, derived from human reason,
and is, in part, a doctrine of the Law, not of the Gospel.  For they
teach two modes of justification, of which the one has been derived
from reason and the other from the Law, not from the Gospel, or the
promise concerning Christ.

The former mode of justification with them is, that they teach that
by good works men merit grace both _de congruo and de condigno_.
This mode is a doctrine of reason, because reason, not seeing the
uncleanness of the heart, thinks that it pleases God if it perform
good works, and for this reason other works and other acts of worship
are constantly devised, by men in great peril, against the terrors of
conscience.  The heathen and the Israelites slew human victims, and
undertook many other most painful works in order to appease God's
wrath.  Afterwards, orders of monks were devised, and these vied with
each other in the severity of their observances against the terrors
of conscience and God's wrath.  And this mode of justification,
because it is according to reason, and is altogether occupied with
outward works, can be understood, and to a certain extent be rendered.
And to this the canonists have distorted the misunderstood Church
ordinances, which were enacted by the Fathers for a far different
purpose, namely, not that by these works we should seek after
righteousness, but that, for the sake of mutual tranquillity among
men, there might be a certain order in the Church.  In this manner
they also distorted the Sacraments and most especially the Mass,
through which they seek _ex opere operato_ righteousness, grace, and
salvation.




Part 9


Another mode of justification is handed down by the scholastic
theologians when they teach that we are righteous through a habit
infused by God, which is love, and that, aided by this habit, we
observe the Law of God outwardly and inwardly and that this
fulfilling of the Law is worthy of grace and of eternal life.  This
doctrine is plainly the doctrine of the Law.  For that is true which
the Law says: Thou shalt love the Lord, thy God, etc., Deut. 6, 5.
Thou shalt love thy neighbor Lev. 19, 18. Love is, therefore, the
fulfilling of the Law.

But it is easy for a Christian to judge concerning both modes,
because both modes exclude Christ, and are therefore to be rejected.
In the former, which teaches that our works are a propitiation for
sin, the impiety is manifest.  The latter mode contains much that is
injurious.  It does not teach that, when we are born again, we avail
ourselves of Christ.  It does not teach that justification is the
remission of sins.  It does not teach that we attain the remission of
sins before we love but falsely represents that we rouse in ourselves
the act of love, through which we merit remission of sins.  Nor does
it teach that by faith in Christ we overcome the terrors of sin and
death.  It falsely represents that, by their own fulfilling of the
Law, without Christ as Propitiator, men come to God.  Finally, it
represents that this very fulfilling of the Law, without Christ as
Propitiator, is righteousness worthy of grace and eternal life, while
nevertheless scarcely a weak and feeble fulfilling of the Law occurs
even in saints.

But if any one will only reflect upon it that the Gospel has not been
given in vain to the world, and that Christ has not been promised,
set forth, has not been born, has not suffered, has not risen again
in vain, he will most readily understand that we are justified not
from reason or from the Law.  In regard to justification, we
therefore are compelled to dissent from the adversaries.  For the
Gospel shows another mode; the Gospel compels us to avail ourselves
of Christ in justification, it teaches that through Him we have
access to God by faith; it teaches that we ought to set Him as
Mediator and Propitiator against God's wrath; it teaches that by
faith in Christ the remission of sins and reconciliation are received,
and the terrors of sin and of death overcome.  Thus Paul also says
that righteousness is not of the Law, but of the promise, in which
the Father has promised that He wishes to forgive, that for Christ's
sake He wishes to be reconciled.  This promise, however, is received
by faith alone, as Paul testifies, Rom. 4,13. This faith alone
receives remission of sins, justifies, and regenerates.  Then love
and other good fruits follow.  Thus, therefore, we teach that man is
justified, as we have above said, when conscience, terrified by the
preaching of repentance, is cheered and believes that for Christ's
sake it has a reconciled God.  This faith is counted for
righteousness before God, Rom. 4, 3. 5. And when in this manner the
heart is cheered and quickened by faith, it receives the Holy Ghost,
who renews us, so that we are able to observe the Law; so that we are
able to love God and the Word of God, and to be submissive to God in
afflictions, so that we are able to be chaste, to love our neighbor,
etc. Even though these works are as yet far distant from the
perfection of the Law, yet they please on account of faith, by which
we are accounted righteous, because we believe that for Christ's sake
we have a reconciled God.  These things are plain and in harmony with
the Gospel, and can be understood by persons of sound mind.  And from
this foundation it can easily be decided why we ascribe justification
to faith, and not to love; although love follows faith, because love
is the fulfilling of the Law.  But Paul teaches that we are justified
not from the Law, but from the promise which is received only by
faith.  For we neither come to God without Christ as Mediator, nor
receive remission of sins for the sake of our love, but for the sake
of Christ.  Likewise we are not able to love God while He is angry,
and the Law always accuses us, always manifests to us an angry God.
Therefore, by faith we must first apprehend the promise that for
Christ's sake the Father is reconciled and forgives.  Afterwards we
begin to observe the Law.  Our eyes are to be cast far away from
human reason, far away from Moses upon Christ, and we are to believe
that Christ is given us, in order that for His sake we may be
accounted righteous.  In the flesh we never satisfy the Law.  Thus,
therefore, we are accounted righteous, not on account of the Law but
on account of Christ because His merits are granted us, if we believe
on Him.  If any one, therefore, has considered these foundations,
that we are not justified by the Law because human nature cannot
observe the Law of God and cannot love God, but that we are justified
from the promise, in which, for Christ's sake, reconciliation,
righteousness, and eternal life have been promised, he will easily
understand that justification must necessarily be ascribed to faith,
if he only will reflect upon the fact that it is not in vain that
Christ has been promised and set forth, that He has been born and has
suffered and been raised again; if he will reflect upon the fact that
the promise of grace in Christ is not in vain, that it was made
immediately from the beginning of the world apart from and beyond the
Law; if he will reflect upon the fact that the promise should be
received by faith, as John says, 1 Ep. 5, 10 sq.: He that believeth
not God hath made Him a liar, because he believeth not the record
that God gave of His Son.  And this is the record that God hath given
to us eternal life, and this life is in His Son.  He that hath the
Son hath life, and he that hath not the Son of God hath not life.
And Christ says John 8, 36: If the Son, therefore, shall make you
free, ye shall be free indeed.  And Paul, Rom. 5, 2: By whom also we
have access to God; and he adds: by faith.  By faith in Christ,
therefore, the promise of remission of sins and of righteousness is
received.  Neither are we justified before God by reason or by the
Law.

These things are so plain and so manifest that we wonder that the
madness of the adversaries is so great as to call them into doubt.
The proof is manifest that, since we are justified before God not
from the Law but from the promise, it is necessary to ascribe
justification to faith.  What can be opposed to this proof, unless
some one wish to abolish the entire Gospel and the entire Christ?
The glory of Christ becomes more brilliant when we teach that we
avail ourselves of Him as Mediator and Propitiator.  Godly
consciences see that in this doctrine the most abundant consolation
is offered to them, namely, that they ought to believe and most
firmly assert that they have a reconciled Father for Christ's sake,
and not for the sake of our righteousness, and that, nevertheless,
Christ aids us, so that we are able to observe also the Law.  Of such
great blessings as these the adversaries deprive the Church when they
condemn and endeavor to efface, the doctrine concerning the
righteousness of faith.  Therefore let all well-disposed minds beware
of consenting to the godless counsels of the adversaries.  In the
doctrine of the adversaries concerning justification no mention is
made of Christ, and how we ought to set Him against the wrath of God,
as though, indeed, we were able to overcome the wrath of God by love,
or to love an angry God.  In regard to these things, consciences are
left in uncertainty.  For if they are to think that they have a
reconciled God for the reason that they love, and that they observe
the Law, they must needs always doubt whether they have a reconciled
God, because they either do not feel this love, as the adversaries
acknowledge, or they certainly feel that it is very small; and much
more frequently do they feel that they are angry at the judgment of
God, who oppresses human nature with many terrible evils, with
troubles of this life, the terrors of eternal wrath, etc. When,
therefore, will conscience be at rest, when will it be pacified?
When, in this doubt and in these terrors, will it love God?  What
else is the doctrine of the Law than a doctrine of despair?  And let
any one of our adversaries come forward who can teach us concerning
this love, how he himself loves God.  They do not at all understand
what they say they only echo, just like the walls of a house, the
little word "love," without understanding it.  So confused and
obscure is their doctrine: it not only transfers the glory of Christ
to human works, but also leads consciences either to presumption or
to despair.  But ours, we hope, is readily understood by pious minds,
and brings godly and salutary consolation to terrified consciences.
For as the adversaries quibble that also many wicked men and devils
believe, we have frequently already said that we speak of faith in
Christ, i.e., of faith in the remission of sins, of faith which truly
and heartily assents to the promise of grace.  This is not brought
about without a great struggle in human hearts.  And men of sound
mind can easily judge that the faith which believes that we are cared
for by God, and that we are forgiven and heard by Him, is a matter
above nature.  For of its own accord the human mind makes no such
decision concerning God.  Therefore this faith of which we speak is
neither in the wicked nor in devils.

Furthermore, if any sophist cavils that righteousness is in the will,
and therefore it cannot be ascribed to faith, which is in the
intellect, the reply is easy, because in the schools even such
persons acknowledge that the will commands the intellect to assent to
the Word of God.  We say also quite clearly: Just as the terrors of
sin and death are not only thoughts of the intellect, but also
horrible movements of the will fleeing God's judgment, so faith is
not only knowledge in the intellect, but also confidence in the will,
i.e., it is to wish and to receive that which is offered in the
promise, namely, reconciliation and remission of sins.  Scripture
thus uses the term "faith," as the following sentence of Paul
testifies, Rom. 5, 1: Being justified by faith, we have peace with
God.  Moreover, in this passage, to justify signifies, according to
forensic usage, to acquit a guilty one and declare him righteous, but
on account of the righteousness of another, namely, of Christ, which
righteousness of another is communicated to us by faith.  Therefore,
since in this passage our righteousness is the imputation of the
righteousness of another, we must here speak concerning righteousness
otherwise than when in philosophy or in a civil court we seek after
the righteousness of one's own work which certainly is in the will.
Paul accordingly says, 1 Cor. 1, 30: Of Him are ye in Christ Jesus,
who of God is made unto us Wisdom and Righteousness, and
Sanctification, and Redemption.  And 2 Cor. 5, 21: He hath mode Him
to be sin for us who knew no sin, that we might be made the
righteousness of God in Him.  But because the righteousness of Christ
is given us by faith, faith is for this reason righteousness in us
imputatively, i.e., it is that by which we are made acceptable to God
on account of the imputation and ordinance of God, as Paul says, Rom.
4, 3. 5: Faith is reckoned for righteousness.  Although on account of
certain captious persons we must say technically: Faith is truly
righteousness, because it is obedience to the Gospel.  For it is
evident that obedience to the command of a superior is truly a
species of distributive justice.  And this obedience to the Gospel is
reckoned for righteousness, so that, only on account of this, because
by this we apprehend Christ as Propitiator, good works, or obedience
to the Law, are pleasing.  For we do not satisfy the Law, but for
Christ's sake this is forgiven us, as Paul says, Rom. 8, 1: There is
therefore now no condemnation to them which are in Christ Jesus.
This faith gives God the honor, gives God that which is His own, in
this, that, by receiving the promises, it obeys Him.  Just as Paul
also says, Rom. 4, 20: He staggered not at the promise of God through
unbelief, but was strong in faith, giving glory to God.  Thus the
worship and divine service of the Gospel is to receive from God gifts,
on the contrary, the worship of the Law is to offer and present our
gifts to God.  We can, however, offer nothing to God unless we have
first been reconciled and born again.  This passage too, brings the
greatest consolation, as the chief worship of the Gospel is to wish
to receive remission of sins, grace, and righteousness.  Of this
worship Christ says, John 6, 40: This is the will of Him that sent Me,
that every one which seeth the Son, and believeth on Him, may have
everlasting life.  And the Father says, Matt. 17, 5: This is My
beloved Son, in whom I am well pleased, hear ye Him.  The adversaries
speak of obedience to the Law; they do not speak of obedience to the
Gospel, and yet we cannot obey the Law, unless, through the Gospel,
we have been born again, since we cannot love God, unless the
remission of sins has been received.  For as long as we feel that He
is angry with us, human nature flees from His wrath and judgment.  If
any one should make a cavil such as this: If that be faith which
wishes those things that are offered in the promise, the habits of
faith and hope seem to be confounded, because hope is that which
expects promised things, to this we reply that these dispositions
cannot in reality be severed, in the manner that they are divided by
idle speculations in the schools.  For also in the Epistle to the
Hebrews faith is defined as the substance (_exspectatio_) of things
hoped for, Heb. 11, 1. Yet if any one wish a distinction to be made,
we say that the object of hope is properly a future event, but that
faith is concerned with future and present things, and receives in
the present the remission of sins offered in the promise.

From these statements we hope that it can be sufficiently understood
both what faith is and that we are compelled to hold that by faith we
are justified, reconciled, and regenerated, if, indeed, we wish to
teach the righteousness of the Gospel, and not the righteousness of
the Law.  For those who teach that we are justified by love teach the
righteousness of the Law, and do not teach us in justification to
avail ourselves of Christ as Mediator.  These things also are
manifest namely, that not by love, but by faith, we overcome the
terrors of sin and death, that we cannot oppose our love and
fulfilling of the Law to the wrath of God, because Paul says, Rom. 5,
2: By Christ we have access to God by faith.  We urge this sentence
so frequently for the sake of perspicuity.  For it shows most clearly
the state of our whole case, and, when carefully considered, can
teach abundantly concerning the whole matter, and can console
well-disposed minds.  Accordingly, it is of advantage to have it at
hand and in sight, not only that we may be able to oppose it to the
doctrine of our adversaries, who teach that we come to God not by
faith, but by love and merits, without Christ as Mediator; and also,
at the same time that, when in fear, we may cheer ourselves and
exercise faith.  This is also manifest, that without the aid of
Christ we cannot observe the Law, as He Himself says John 15, 5:
Without Me ye can do nothing.  Accordingly, before we observe the Law,
our hearts must be born again by faith.  [From the explanations
which we have made it can easily be inferred what answer must be
given to similar quotations.  For the rule so interprets all passages
that treat of good works that outside of Christ they are to be
worthless before God, and that the heart must first have Christ, and
believe that it is accepted with God for Christ's sake, not because
of its own works.  The adversaries also bring forward some arguments
of the schools, which are easily answered, if you know what faith is.
Tried Christians speak of faith quite differently from the sophists,
for we have shown before that to believe means to rely on the mercy
of God, that He desires to be gracious for Christ's sake, without our
merits.  That is what it means to believe the article of the
forgiveness of sin.  To believe this does not mean to know the
history only, which the devils also know.  Therefore we can easily
meet the argument of the schools when they say that the devils also
believe, therefore faith does not justify.  Aye, the devils know the
history, but they do not believe the forgiveness of sin.  Again, they
say: To be righteous is to be obedient.  Now, to perform works is
certainly obedience; therefore works must justify.  We should answer
this as follows: To be righteous is a kind of obedience which God
accepts as such.  Now God is not willing to accept our obedience in
works as righteousness; for it is not an obedience of the heart,
because none truly keep the Law.  For this reason He has ordained
that there should be another kind of obedience which He will accept
as righteousness, namely, that we are to acknowledge our disobedience,
and trust that we are pleasing to God for Christ's sake, not on
account of our obedience.  Accordingly, to be righteous in this case
means to be pleasing to God, not on account of our own obedience, but
from mercy for Christ's sake.  Again, to sin is to hate God;
therefore, to love God must be righteousness.  True, to love God is
the righteousness of the Law.  But nobody fulfils this Law.
Therefore the Gospel teaches a new kind of righteousness, namely,
that we are pleasing to God for Christ's sake, although we have not
fulfilled the Law; and yet, we are to begin to do the Law.  Again,
what is the difference between faith and hope?  Answer: Hope expects
future blessings and deliverance from tribulation; faith receives the
present reconciliation, and concludes in the heart that God has
forgiven my sin, and that He is now gracious to me.  And this is a
noble service of God, which serves God by giving Him the honor, and
by esteeming His mercy and promise so sure that without merit we can
receive and expect from Him all manner of blessings.  And in this
service of God the heart should be exercised and increase, of which
the foolish sophists know nothing.]

Hence it can also be understood why we find fault with the doctrine
of the adversaries concerning _meritum condigni_.  The decision is
very easy: because they do not make mention of faith, that we please
God by faith for Christ's sake, but imagine that good works, wrought
by the aid of the habit of love, constitute a righteousness worthy by
itself to please God, and worthy of eternal life, and that they have
no need of Christ as Mediator.  [This can in no wise be tolerated.]
What else is this than to transfer the glory of Christ to our works,
namely that we please God because of our works, and not because of
Christ?  But this is also to rob Christ of the glory of being the
Mediator who is Mediator perpetually, and not merely in the beginning
of Justification.  Paul also says, Gal. 2, 17, that If one justified
in Christ have need afterwards to seek righteousness elsewhere, he
affirms of Christ that He is a minister of sin, i.e., that He does
not fully justify.  [And this is what the holy, catholic, Christian
Church teaches, preaches, and confesses, namely, that we are saved by
mercy as we have shown above from Jerome.] And most absurd is that
which the adversaries teach, namely, that good works merit _grace de
condigno_, as though indeed after the beginning of justification, if
conscience is terrified, as is ordinarily the case, grace must be
sought through a good work, and not by faith in Christ.

Secondly, the doctrine of the adversaries leaves consciences in doubt,
so that they never can be pacified, because the Law always accuses
us, even in good works.  For always the flesh lusteth against the
Spirit, Gal. 5, 17. How, therefore, will conscience here have peace
without faith, if it believe that, not for Christ's sake, but for the
sake of one's own work, it ought now to please God?  What work will
it find, upon what will it firmly rely as worthy of eternal life, if,
indeed, hope ought to originate from merits?  Against these doubts
Paul says, Rom. 5, 1: Being justified by faith, we have peace with
God; we ought to be firmly convinced that for Christ's sake
righteousness and eternal life are granted us.  And of Abraham he
says Rom. 4, 18: Against hope he believed in hope.

Thirdly, how will conscience know when by the inclination of this
habit of love, a work has been done of which it may affirm that it
merits _grace de condigno_?  But it is only to elude the Scriptures
that this very distinction has been devised, namely, that men merit
at one time _de congruo_ and at another time _de condigno_, because,
as we have above said, the intention of the one who works does not
distinguish the kinds of merit; but hypocrites, in their security,
think simply their works are worthy, and that for this reason they
are accounted righteous.  On the other hand, terrified consciences
doubt concerning all works, and for this reason are continually
seeking other works.  For this is what it means to _merit de congruo_,
namely to doubt and, without faith, to work, until despair takes
place.  In a word, all that the adversaries teach in regard to this
matter is full of errors and dangers.

Fourthly, the entire [the holy, catholic, Christian] Church confesses
that eternal life is attained through mercy.  For thus Augustine
speaks On Grace and Free Will, when indeed, he is speaking of the
works of the saints wrought after justification: God leads us to
eternal life not by our merits, but according to His mercy.  And
Confessions, Book IX: Woe to the life of man, however much it may be
worthy of praise, if it be judged with mercy removed.  And Cyprian in
his treatise on the Lord's Prayer: Lest any one should flatter
himself that he is innocent, and by exalting himself, should perish
the more deeply, he is instructed and taught that he sins daily, in
that he is bidden to entreat daily for his sins.  But the subject is
well known, and has very many and very clear testimonies in Scripture,
and in the Church Fathers, who all with one mouth declare that, even
though we have good works yet in these very works we need mercy.
Faith looking upon this mercy cheers and consoles us.  Wherefore the
adversaries teach erroneously when they so extol merits as to add
nothing concerning this faith that apprehends mercy.  For just as we
have above said that the promise and faith stand in a reciprocal
relation, and that the promise is not apprehended unless by faith, so
we here say that the promised mercy correlatively requires faith, and
cannot be apprehended without faith.  Therefore we justly find fault
with the doctrine concerning _meritum condigni_, since it teaches
nothing of justifying faith, and obscures the glory and office of
Christ as Mediator.  Nor should we be regarded as teaching anything
new in this matter, since the Church Fathers have so clearly handed
down the doctrine that even in good works we need mercy.

Scripture also often inculcates the same.  In Ps. 143, 9: And enter
not into judgment with Thy servant; for in Thy sight shall no man
living be justified.  This passage denies absolutely, even to all
saints and servants of God, the glory of righteousness, if God does
not forgive, but judges and convicts their hearts.  For when David
boasts in other places of his righteousness, he speaks concerning his
own cause against the persecutors of God's Word, he does not speak of
his personal purity; and he asks that the cause and glory of God be
defended, as in Ps. 7, 8: Judge me, O Lord, according to Thy
righteousness, and according to mine integrity that is in me.
Likewise in Ps. 130, 3, he says that no one can endure God's judgment,
if God were to mark our sins: If Thou, Lord, shouldest mark
iniquities, O Lord, who shall stand?  Job 9, 28: I am afraid of all
my sorrows [Vulg., opera, works]; v. 30: If I wash myself with
snow-water, and make my hands never so clean, yet Thou shalt plunge
me in the ditch.  Prov. 20, 9: Who can say, I have made my heart
clean, I am pure from my sin? 1 John 1, 8: If we say that we have no
sin, we deceive ourselves and the truth is not in us, etc. And in the
Lord's Prayer the saints ask for the remission of sins.  Therefore
even the saints have sins.  Num. 14, 18: The innocent shall not be
innocent [cf.  Ex. 34, 7].  Deut. 4, 24: The Lord, thy God, is a
consuming fire.  Zechariah also says, 2, 13: Be silent, O all flesh,
before the Lord.  Is. 40, 6: All flesh is as grass, and all the
goodliness thereof is as the flower of the field; the grass withereth,
the flower fadeth, because the Spirit of the Lord bloweth upon it, i.
e., flesh and righteousness of the flesh cannot endure the judgment
of God.  Jonah also says, chap. 2, 8: They that observe lying
vanities forsake their own mercy, i.e., all confidence is vain,
except confidence in mercy; mercy delivers us; our own merits, our
own efforts, do not.  Accordingly, Daniel also prays, 9, 18 sq.: For
we do not present our supplications before Thee for our
righteousnesses but for Thy great mercies.  O Lord, hear; O Lord,
forgive; O Lord, hearken and do it; defer not for Thine own sake, O
my God; for Thy city and Thy people are called by Thy name.  Thus
Daniel teaches us in praying to lay hold upon mercy, i.e., to trust
in God's mercy, and not to trust in our own merits before God.  We
also wonder what our adversaries do in prayer, if, indeed, the
profane men ever ask anything of God.  If they declare that they are
worthy because they have love and good works, and ask for grace as a
debt, they pray precisely like the Pharisee in Luke 18, 11, who says:
I am not as other men are.  He who thus prays for grace and does not
rely upon God's mercy, treats Christ with dishonor, who, since He is
our High Priest, intercedes for us.  Thus, therefore, prayer relies
upon God's mercy, when we believe that we are heard for the sake of
Christ the High Priest, as He Himself says, John 14, 13: Whatsoever
ye shall ask the Father in My name, He will give it you.  In My name,
He says, because without this High Priest we cannot approach the
Father.

[All prudent men will see what follows from the opinion of the
adversaries.  For if we shall believe that Christ has merited only
the _prima gratia_, as they call it, and that we afterwards merit
eternal life by our works, hearts or consciences will be pacified
neither at the hour of death, nor at any other time, nor can they
ever build upon certain ground; they are never certain that God is
gracious.  Thus their doctrine unintermittingly leads to nothing but
misery of soul and, finally, to despair.  For God's Law is not a
matter of pleasantry; it ceaselessly accuses consciences outside of
Christ, as Paul says, Rom. 4, 15: The Law worketh wrath.  Thus it
will happen that if consciences feel the judgment of God, they have
no certain comfort and will rush into despair.

Paul says: Whatsoever is not of faith is sin, Rom. 14, 23. But those
persons can do nothing from faith who are first to attain to this
that God is gracious to them only when they have at length fulfilled
the Law.  They will always quake with doubt whether they have done
enough good works, whether the Law has been satisfied, yea, they will
keenly feel and understand that they are still under obligation to
the Law.  Accordingly, they will never be sure that they have a
gracious God, and that their prayer is heard.  Therefore they can
never truly love God, nor expect any blessing from Him, nor truly
worship God.  What else are such hearts and consciences than hell
itself, since there is nothing in them but despair, fainting away
grumbling, discontent, and hatred of God, and yet in this hatred they
invoke and worship God, just as Saul worshiped Him

Here we appeal to all Christian minds and to all that are experienced
in trials; they will be forced to confess and say that such great
uncertainty, such disquietude, such torture and anxiety, such
horrible fear and doubt follow from this teaching of the adversaries
who imagine that we are accounted righteous before God by our own
works or fulfilling of the Law which we perform, and point us to
Queer Street by bidding us trust not in the rich, blessed promises of
Grace, given us by Christ the Mediator, but in our own miserable
works!  Therefore, this conclusion stands like a rock, yea, like a
wall, namely, that, although we have begun to do the Law, still we
are accepted with God and at peace with Him, not on account of such
works of ours, but for Christ's sake by faith; nor does God owe us
everlasting life on account of these works.  But just as forgiveness
of sin and righteousness is imputed to us for Christ's sake, not on
account of our works, or the Law, so everlasting life, together with
righteousness, is offered us, not on account of our works, or of the
Law, but for Christ's sake as Christ says, John 6, 40: This is the
Father's will that sent He, that every one which seeth the Son, and
believeth on Him may have everlasting life.  Again, v. 47: He that
believeth on the Son hath everlasting life.  Now, the adversaries
should be asked at this point what advice they give to poor
consciences in the hour of death: whether they comfort consciences by
telling them that they will have a blessed departure, that they will
be saved, and have a propitiated God, because of their own merits or
because of God's grace and mercy for Christ's sake.  For St. Peter St.
Paul, and saints like them cannot boast that God owes them eternal
life for their martyrdom, nor have they relied on their works, but on
the mercy promised in Christ.

Nor would it be possible that a saint, great and high though he be,
could make a firm stand against the accusations of the divine Law,
the great might of the devil, the terror of death, and, finally,
against despair and the anguish of hell, if he would not grasp the
divine promises, the Gospel, as a tree or branch in the great flood
in the strong, violent stream, amidst the waves and billows of the
anguish of death; if he does not cling by faith to the Word, which
proclaims grace, and thus obtains eternal life without works, without
the Law, from pure grace.  For this doctrine alone preserves
Christian consciences in afflictions and anguish of death.  Of these
things the adversaries know nothing, and talk of them like a blind
man about color.

Here they will say: If we are to be saved by pure mercy, what
difference is there between those who are saved, and those who are
not saved?  If merit is of no account, there is no difference between
the evil and the good and it follows that both are saved alike.  This
argument has moved the scholastics to invent the _meritum condigni_;
for there must be (they think) a difference between those who are
saved and those who are damned.

We reply; in the first place, that everlasting life is accorded to
those whom God esteems just, and when they have been esteemed just,
they are become, by that act, the children of God and coheirs of
Christ, as Paul says, Rom. 8, 30: Whom He justified, them He also
glorified.  Hence nobody is saved except only those who believe the
Gospel.  But as our reconciliation with God is uncertain if it is to
rest on our works, and not on the gracious promise of God, which
cannot fail, so, too, all that we expect by hope would be uncertain
if it must be built on the foundation of our merits and works.  For
the Law of God ceaselessly accuses the conscience and men feel in
their hearts nothing but this voice from the fiery, flaming cloud: I
am the Lord, thy God; this thou shalt do; that thou art obliged to do;
this I require of thee.  Deut. 5, 6 ff.  No conscience can for a
moment be at rest when the Law and Moses assails the heart, before it
apprehends Christ by faith.  Nor can it truly hope for eternal life,
unless it be pacified before.  For a doubting conscience flees from
God, despairs and cannot hope.  However, hope of eternal life must be
certain.  Now, in order that it may not be fickle, but certain, we
must believe that we have eternal life, not by our works or merits,
but from pure grace, by faith in Christ.

In secular affairs and in secular courts we meet with both, mercy and
justice.  Justice is certain by the laws and the verdict rendered,
mercy is uncertain.  In this matter that relates to God the case is
different; for grace and mercy have been promised us by a certain
word, and the Gospel is the word which commands us to believe that
God is gracious and wishes to save us for Christ's sake, as the text
reads, John 3, 17: God sent not His Son into the world to condemn the
world, but that the world through Him might be saved.  He that
believeth on Him is not condemned.

Now, whenever we speak of mercy, the meaning is to be this, that
faith is required, and it is this faith that makes the difference
between those who are saved, and those who are damned, between those
who are worthy, and those who are unworthy.  For everlasting life has
been promised to none but those who have been reconciled by Christ.
Faith, however, reconciles and justifies before God the moment we
apprehend the promise by faith.  And throughout our entire life we
are to pray God and be diligent, to receive faith and to grow in
faith.  For, as stated before, faith is where repentance is, and it
is not in those who walk after the flesh.  This faith is to grow and
increase throughout our life by all manner of afflictions.  Those who
obtain faith are regenerated, so that they lead a new life and do
good works.

Now, just as we say that true repentance is to endure throughout our
entire life, we say, too, that good works and the fruits of faith
must be done throughout our life, although our works never become so
precious as to be equal to the treasure of Christ, or to merit
eternal life, as Christ says, Luke 17, 10: When ye shall have done
all those things which are commanded you, say, We are unprofitable
servants.  And St. Bernard truly says: There is need that you must
first believe that you cannot have forgiveness of sin except by the
grace of God; next, that thereafter you cannot have and do any good
work unless God grants it to you; lastly, that you cannot earn
eternal life with your works, though it is not given you without
merit.  A little further on he says: Let no one deceive himself; for
when you rightly consider the matter, you will undoubtedly find that
you cannot meet with ten thousand him who approaches you with twenty
thousand.  These are strong sayings of St. Bernard; let them believe
these if they will not believe us.

In order, then, that hearts may have a true certain comfort and hope,
we point them, with Paul, to the divine promise of grace in Christ,
and teach that we must believe that God gives us eternal life, not on
account of our works, but for Christ's sake, as the Apostle John says
in his Epistle, 1, 5, 12: He that hath the Son hath life, and he that
hath not the Son of God hath not life.]




Part 10


Here belongs also the declaration of Christ, Luke 17, 10: So likewise
ye, when ye shall have done all those things which are commanded you,
say, We are unprofitable servants.  These words clearly declare that
God saves by mercy and on account of His promise, not that it is due
on account of the value of our works.  But at this point the
adversaries play wonderfully with the words of Christ.  In the first
place, they make an antistrophe and turn it against us.  Much more,
they say, can it be said: "If we have believed all things, say, We
are unprofitable servants." Then they add that works are of no profit
to God, but are not without profit to us.  See how the puerile study
of sophistry delights the adversaries, and although these absurdities
do not deserve a refutation, nevertheless we will reply to them in a
few words.  The antistrophe is defective.  For, in the first place,
the adversaries are deceived in regard to the term faith; because, if
it would signify that knowledge of the history which is also in the
wicked and in devils, the adversaries would be correct in arguing
that faith is unprofitable when they say: "When we have believed all
things, say, We are unprofitable servants." But we are speaking, not
of the knowledge of the history, but of confidence in the promise and
mercy of God.  And this confidence in the promise confesses that we
are unprofitable servants; yea, this confession that our works are
unworthy is the very voice of faith, as appears in this example of
Daniel, 9, 18, which we cited a little above: We do not present our
supplications before Thee for our righteousnesses, etc. For faith
saves because it apprehends mercy, or the promise of grace, even
though our works are unworthy; and, thus understood, namely that our
works are unworthy, the antistrophe does not injure us: "When ye
shall have believed all things, say, We are unprofitable servants";
for that we are saved by mercy, we teach with the entire Church.  But
if they mean to argue from the similar: When you have done all things,
do not trust in your works, so also, when you have believed all
things, do not trust in the divine promise there is no connection.
[The inference is wrong: "Works do not help; therefore, faith also
does not help." We must give the uncultured men a homely illustration:
It does not follow that because a half-farthing does not help,
therefore a florin also does not help.  Just as the florins is of
much higher denomination and value than the half-farthing, so also
should it be understood that faith is much higher and more
efficacious than works.  Not that faith helps because of its worth,
but because it trusts in God's promises and mercy.  Faith is strong,
not because of its worthiness, but because of the divine promise.]
For they are very dissimilar, as the causes and objects of confidence
in the former proposition are far dissimilar to those of the latter.
In the former, confidence is confidence in our own works.  In the
latter, confidence is confidence in the divine promise.  Christ,
however, condemns confidence in our works; He does not condemn
confidence in His promise.  He does not wish us to despair of God's
grace and mercy.  He accuses our works as unworthy, but does not
accuse the promise which freely offers mercy.  And here Ambrose says
well: grace is to be acknowledged; but nature must not be disregarded.
We must trust in the promise of grace and not in our own nature.
But the adversaries act in accordance with their custom, and distort,
against faith, the judgments which have been given on behalf of faith.
[Hence, Christ in this place forbids men to trust in their own
works; for they cannot help them.  On the other hand, He does not
forbid to trust in God's promise.  Yea, He requires such trust in the
promise of God for the very reason that we are unprofitable servants
and works can be of no help.  Therefore, the knaves have improperly
applied to our trust in the divine promise the words of Christ which
treat of trust in our own worthiness.  This clearly reveals and
defeats their sophistry.  May the Lord Christ soon put to shame the
sophists who thus mutilate His holy Word!  Amen.] We leave, however,
these thorny points to the schools.  The sophistry is plainly puerile
when they interpret "unprofitable servant " as meaning that the works
are unprofitable to God, but are profitable to us.  Yet Christ speaks
concerning that profit which makes God a debtor of grace to us,
although it is out of place to discuss here concerning that which is
profitable or unprofitable.  For "unprofitable servants" means
"insufficient," because no one fears God as much, and loves God as
much, and believes God as much as he ought.  But let us dismiss these
frigid cavils of the adversaries, concerning which, if at any time
they are brought to the light, prudent men will easily decide what
they should judge.  They have found a flaw in words which are very
plain and clear.  But every one sees that in this passage confidence
in our own works is condemned.

Let us, therefore, hold fast to this which the Church confesses,
namely, that we are saved by mercy.  And lest any one may here think:
"If we are to be saved by mercy, hope will be uncertain, if in those
who obtain salvation nothing precedes by which they may be
distinguished from those who do not obtain it," we must give him a
satisfactory answer.  For the scholastics, moved by this reason, seem
to have devised the _meritum condigni_.  For this consideration can
greatly exercise the human mind.  We will therefore reply briefly.
For the very reason that hope may be sure, for the very reason that
there may be an antecedent distinction between those who obtain
salvation, and those who do not obtain it, it is necessary firmly to
hold that we are saved by mercy.  When this is expressed thus
unqualifiedly, it seems absurd.  For in civil courts and in human
judgment, that which is of right or of debt is certain, and mercy is
uncertain.  But the matter is different with respect to God's
judgment; for here mercy has a clear and certain promise and command
from God.  For the Gospel is properly that command which enjoins us
to believe that God is propitious to us for Christ's sake.  For God
sent not His Son into the world to condemn the world, but that the
world through Him might be saved, John 3, 17. 18. As often, therefore,
as mercy is spoken of, faith in the promise must be added; and this
faith produces sure hope, because it relies upon the Word and command
of God.  If hope would rely upon works, then, indeed, it would be
uncertain, because works cannot pacify the conscience, as has been
said above frequently.  And this faith makes a distinction between
those who obtain salvation, and those who do not obtain it.  Faith
makes the distinction between the worthy and the unworthy, because
eternal life has been promised to the justified; and faith justifies.

But here again the adversaries will cry out that there is no need of
good works if they do not merit eternal life.  These calumnies we
have refuted above.  Of course, it is necessary to do good works.  We
say that eternal life has been promised to the justified.  But those
who walk according to the flesh retain neither faith nor
righteousness.  We are for this very end justified, that, being
righteous we may begin to do good works and to obey God's Law.  We
are regenerated and receive the Holy Ghost for the very end that the
new life may produce new works, new dispositions, the fear and love
of God, hatred of concupiscence, etc. This faith of which we speak
arises in repentance, and ought to be established and grow in the
midst of good works, temptations, and dangers, so that we may
continually be the more firmly persuaded that God for Christ's sake
cares for us, forgives us, hears us.  This is not learned with out
many and great struggles.  How often is conscience aroused, how often
does it incite even to despair when it brings to view sins, either
old or new, or the impurity of our nature!  This handwriting is not
blotted out without a great struggle, in which experience testifies
what a difficult matter faith is.  And while we are cheered in the
midst of the terrors and receive consolation, other spiritual
movements at the same time grow, the knowledge of God, fear of God,
hope, love of God; and we are regenerated, as Paul says, Col. 3, 10
and 2 Cor. 3, 18, in the knowledge of God, and, beholding the glory
of the Lord, are changed into the same image, i.e., we receive the
true knowledge of God, so that we truly fear Him, truly trust that we
are cared for and that we are heard by Him.  This regeneration is, as
it were, the beginning of eternal life, as Paul says, Rom. 8, 10: If
Christ be in you, the body is dead because of sin; but the Spirit is
life because of righteousness.  And 2 Cor. 5, 2. 3: We are clothed
upon, if so be that, being clothed, we shall not be found naked.
From these statements the candid reader can judge that we certainly
require good works, since we teach that this faith arises in
repentance, and in repentance ought continually to increase; and in
these matters we place Christian and spiritual perfection, if
repentance and faith grow together in repentance.  This can be better
understood by the godly than those things which are taught by the
adversaries concerning contemplation or perfection.  Just as, however,
justification pertains to faith, so also life eternal pertains to
faith.  And Peter says, 1 Pet. 1, 9: Receiving the end, or fruit, of
your faith, the salvation of your souls.  For the adversaries confess
that the justified are children of God and coheirs of Christ.
Afterwards works, because on account of faith they please God, merit
other bodily and spiritual rewards.  For there will be distinctions
in the glory of the saints.

But here the adversaries reply that eternal life is called a reward,
and that therefore it is merited _de condigno_ by good works.  We
reply briefly and plainly: Paul, Rom. 6, 23, calls eternal life a
gift, because by the righteousness presented for Christ's sake, we
are made at the same time sons of God and coheirs of Christ, as John
says, 3, 36: He that believeth on the Son hath everlasting life.  And
Augustine says, as also do very many others who follow him: God
crowns His gifts in us.  Elsewhere indeed, Luke 5, 23, it is written:
Your reward is great in heaven.  If these passages seem to the
adversaries to conflict, they themselves may explain them.  But they
are not fair judges; for they omit the word gift.  They omit also the
sources of the entire matter [the chief part, how we are justified
before God, also that Christ remains at all times the Mediator], and
they select the word reward, and most harshly interpret this not only
against Scripture, but also against the usage of the language.  Hence
they infer that inasmuch as it is called a reward, our works,
therefore, are such that they ought to be a price for which eternal
life is due.  They are, therefore, worthy of grace and life eternal,
and do not stand in need of mercy, or of Christ as Mediator, or of
faith.  This logic is altogether new; we hear the term reward, and
therefore are to infer that there is no need of Christ as Mediator,
or of faith having access to God for Christ's sake, and not for the
sake of our works!  Who does not see that these are anacoluthons?  We
do not contend concerning the term reward.  We dispute concerning
this matter, namely, whether good works are of themselves worthy of
grace and of eternal life, or whether they please only on account of
faith, which apprehends Christ as Mediator.  Our adversaries not only
ascribe this to works, namely, that they are worthy of grace and of
eternal life, but they also state falsely that they have superfluous
merits, which they can grant to others, and by which they can justify
others, as when monks sell the merits of their orders to others.
These monstrosities they heap up in the manner of Chrysippus, where
this one word reward is heard, namely: "It is called a reward, and
therefore we have works which are a price for which a reward is due;
therefore works please by themselves, and not for the sake of Christ
as Mediator.  And since one has more merits than another, therefore
some have superfluous merits.  And those who merit them can bestow
these merits upon others." Stop, reader; you have not the whole of
this sorites.  For certain sacraments of this donation must be added;
the hood is placed upon the dead.  [As the Barefooted monks and other
orders have shamelessly done in placing the hoods of their orders
upon dead bodies.] By such accumulations the blessings brought us in
Christ, and the righteousness of faith have been obscured.  [These
are acute and strong arguments, all of which they can spin from the
single word reward, whereby they obscure Christ and faith.]

We are not agitating an idle logomachy concerning the term reward
[but this great, exalted, most important matter, namely, where
Christian hearts are to find true and certain consolation; again,
whether our works can give consciences rest and peace; again, whether
we are to believe that our works are worthy of eternal life, or
whether that is given us for Christ's sake.  These are the real
questions regarding these matters; if consciences are not rightly
instructed concerning these, they can have no certain comfort.
However, we have stated clearly enough that good works do not fulfil
the Law, that we need the mercy of God, that by faith we are accepted
with God, that good works, be they ever so precious, even if they
were the works of St. Paul himself, cannot bring rest to the
conscience.  From all this it follows that we are to believe that we
obtain eternal life through Christ by faith, not on account of our
works, or of the Law.  But what do we say of the reward which
Scripture mentions?] If the adversaries will concede that we are
accounted righteous by faith because of Christ, and that good works
please God because of faith, we will not afterwards contend much
concerning the term reward.  We confess that eternal life is a reward,
because it is something due on account of the promise, not on
account of our merits.  For the justification has been promised,
which we have above shown to be properly a gift of God; and to this
gift has been added the promise of eternal life, according to Rom. 8,
30: Whom He justified, them He also glorified.  Here belongs what
Paul says, 2 Tim. 4, 8: There is laid up for me a crown of
righteousness which the Lord, the righteous Judge, shall give me.
For the crown is due the justified because of the promise.  And this
promise saints should know, not that they may labor for their own
profit, for they ought to labor for the glory of God; but in order
that they may not despair in afflictions, they should know God's will,
that He desires to aid, to deliver, to protect them.  [Just as the
inheritance and all possessions of a father are given to the son, as
a rich compensation and reward for his obedience, and yet the son
receives the inheritance, not on account of his merit, but because
the father, for the reason that he is his father, wants him to have
it.  Therefore it is a sufficient reason why eternal life is called a
reward, because thereby the tribulations which we suffer, and the
works of love which we do, are compensated, although we have not
deserved it.  For there are two kinds of compensation: one, which we
are obliged, the other, which we are not obliged, to render.  I.e.,
when the emperor grants a servant a principality, he therewith
compensates the servant's work; and yet the work is not worth the
principality, but the servant acknowledges that he has received a
gracious lien.  Thus God does not owe us eternal life, still, when He
grants it to believers for Christ's sake, that is a compensation for
our sufferings and works.] Although the perfect hear the mention of
penalties and rewards in one way, and the weak hear it in another way;
for the weak labor for the sake of their own advantage.  And yet the
preaching of rewards and punishments is necessary.  In the preaching
of punishments the wrath of God is set forth, and therefore this
pertains to the preaching of repentance.  In the preaching of rewards,
grace is set forth.  And just as Scripture, in the mention of good
works, often embraces faith,--for it wishes righteousness of the
heart to be included with the fruits,--so sometimes it offers grace
together with other rewards as in Is. 58, 8 f., and frequently in
other places in the prophets.  We also confess what we have often
testified, that, although justification and eternal life pertain to
faith, nevertheless good works merit other bodily and spiritual
rewards [which are rendered both in this life and after this life;
for God defers most rewards until He glorifies saints after this life,
because He wishes them in this life to be exercised in mortifying
the old man] and degrees of rewards, according to 1 Cor. 3, 8: Every
man shall receive his own reward according to his own labor.  [For
the blessed will have reward, one higher than the other.  This
difference merit makes, according as it pleases God; and it is merit,
because they do these good works whom God has adopted as children and
heirs.  For thus they have merit which is their own and peculiar as
one child with respect to another.] For the righteousness of the
Gospel, which has to do with the promise of grace, freely receives
justification and quickening.  But the fulfilling of the Law, which
follows faith, has to do with the Law, in which a reward is offered
and is due, not freely, but according to our works.  But those who
merit this are justified before they do the Law.  Therefore as Paul
says, Col. 1, 13; Rom. 8, 17, they have before been translated into
the kingdom of God's Son, and been made joint-heirs with Christ.  But
as often as mention is made of merit, the adversaries immediately
transfer the matter from other rewards to justification, although the
Gospel freely offers justification on account of Christ's merits and
not of our own; and the merits of Christ are communicated to us by
faith.  But works and afflictions merit, not justification, but other
remunerations, as the reward is offered for the works in these
passages: He which soweth sparingly shall reap also sparingly, and he
which soweth bountifully shall reap also bountifully, 2 Cor. 9, 6.
Here clearly the measure of the reward is connected with the measure
of the work.  Honor thy father and thy mother, that thy days may be
long upon the land, Ex. 20, 12. Also here the Law offers a reward to
a certain work.  Although, therefore, the fulfilling of the Law
merits a reward, for a reward properly pertains to the Law, yet we
ought to be mindful of the Gospel, which freely offers justification
for Christ's sake.  We neither observe the Law nor can observe it,
before we have been reconciled to God, justified, and regenerated.
Neither would this fulfilling of the Law please God, unless we would
be accepted on account of faith.  And because men are accepted on
account of faith, for this very reason the inchoate fulfilling of the
Law pleases, and has a reward in this life and after this life.
Concerning the term reward, very many other remarks might here be
made derived from the nature of the Law, which as they are too
extensive, must be explained in another connection.

But the adversaries urge that it is the prerogative of good works to
merit eternal life, because Paul says, Rom. 2, 5: Who will render to
every one according to his works.  Likewise v. 10: Glory, honor, and
peace to every man that worketh good.  John 6, 29: They that have
done good [shall come forth] unto the resurrection of life.  Matt. 25
36: I was an hungred and ye gave Me meat etc. In these and all
similar passages in which works are praised in the Scriptures, it is
necessary to understand not only outward works, but also the faith of
the heart, because Scripture does not speak of hypocrisy, but of the
righteousness of the heart with its fruits.  Moreover, as often as
mention is made of the Law and of works, we must know that Christ as
Mediator is not to be excluded.  For He is the end of the Law, and He
Himself says, John 16, 5: Without Me ye can do nothing.  According to
this rule we have said above that all passages concerning works can
be judged.  Wherefore, when eternal life is granted to works, it is
granted to those who have been justified, because no men except
justified men, who are led by the Spirit of Christ, can do good works;
and without faith and Christ, as Mediator, good works do not please,
according to Heb. 11, 6: Without faith it is impossible to please God.
When Paul says: He will render to every one according to his works,
not only the outward work ought to be understood, but all
righteousness or unrighteousness.  So: Glory to him that worketh good,
i.e., to the righteous.  Ye gave Me meat, is cited as the fruit and
witness of the righteousness of the heart and of faith, and therefore
eternal life is rendered to righteousness.  [There it must certainly
be acknowledged that Christ means not only the works, but that He
desires to have the heart, which He wishes to esteem God aright, and
to believe correctly concerning Him, namely, that it is through mercy
that it is pleasing to God.  Therefore Christ teaches that
everlasting life will be given the righteous, as Christ says: The
righteous shall go into everlasting life.] In this way Scripture, at
the same time with the fruits, embraces the righteousness of the
heart.  And it often names the fruits, in order that it may be better
understood by the inexperienced, and to signify that a new life and
regeneration, and not hypocrisy, are required.  But regeneration
occurs, by faith, in repentance.

No sane man can judge otherwise, neither do we here affect any idle
subtilty, so as to separate the fruits from the righteousness of the
heart; if the adversaries would only have conceded that the fruits
please because of faith, and of Christ as Mediator, and that by
themselves they are not worthy of grace and of eternal life.  For in
the doctrine of the adversaries we condemn this, that in such
passages of Scripture, understood either in a philosophical or a
Jewish manner, they abolish the righteousness of faith, and exclude
Christ as Mediator.  From these passages they infer that works merit
grace, sometimes de congruo, and at other times _de condigno_, namely,
when love is added; i.e., that they justify, and because they are
righteousness they are worthy of eternal life.  This error manifestly
abolishes the righteousness of faith, which believes that we have
access to God for Christ's sake, not for the sake of our works, and
that through Christ, as Priest and Mediator, we are led to the Father,
and have a reconciled Father, as has been sufficiently said above.
And this doctrine concerning the righteousness of faith is not to be
neglected in the Church of Christ, because without it the office of
Christ cannot be considered, and the doctrine of justification that
is left is only a doctrine of the Law.  But we should retain the
Gospel, and the doctrine concerning the promise, granted for Christ's
sake.

[We are here not seeking an unnecessary subtilty, but there is a
great reason why we must have a reliable account as regards these
questions.  For as soon as we concede to the adversaries that works
merit eternal life, they spin from this concession the awkward
teaching that we are able to keep the Law of God, that we are not in
need of mercy, that we are righteous before God, that is, accepted
with God by our works, not for the sake of Christ, that we can also
do works of supererogations namely, more than the Law requires.  Thus
the entire teaching concerning faith is suppressed.  However, if
there is to be and abide a Christian Church, the pure teaching
concerning Christ, concerning the righteousness of faith, must surely
be preserved.  Therefore we must fight against these great
pharisaical errors, in order that we redeem the name of Christ and
the honor of the Gospel and of Christ, and preserve for Christian
hearts a true, permanent, certain consolation.  For how is it
possible that a heart or conscience can obtain rest, or hope for
salvation, when in afflictions and in the anguish of death our works
in the judgment and sight of God utterly become dust, unless it
becomes certain by faith that men are saved by mercy, for Christ's
sake, and not for the sake of their works, their fulfilling of the
Law?  And, indeed, St. Laurentius, when placed on the gridiron, and
being tortured for Christ's sake did not think that by this work he
was perfectly and absolutely fulfilling the Law, that he was without
sin, that he did not need Christ as Mediator and the mercy of God.
He rested his case, indeed, with the prophet, who says: Enter not
into judgment with Thy servant; for in Thy sight shall no man living
be justified, Ps. 143, 2. Nor did St. Bernard boast that his works
were worthy of eternal life, when he says: _Perdite vixi_, I have led
a sinful life, etc. But he boldly comforts himself, clings to the
promise of grace, and believes that he has remission of sins and life
eternal for Christ's sake, just as Psalm 32, 1 teaches: Blessed is he
whose transgression is forgiven, whose sin is covered.  And Paul says,
Rom. 4, 6: David also describeth the blessedness of the man to whom
God imputeth righteousness without works.  Paul, then, says that he
is blessed to whom righteousness is imputed through faith in Christ,
even though he have not performed any good works.  That is the true,
permanent consolation, by which hearts and consciences can be
confirmed and encouraged, namely that for Christ's sake, through
faith, the remission of sins, righteousness, and life eternal are
given us.  Now, if passages which treat of works are understood in
such a manner as to comprise faith, they are not opposed to our
doctrine.  And, indeed, it is necessary always to add faith, so as
not to exclude Christ as Mediator.  But the fulfilment of the Law
follows faith; for the Holy Ghost is present, who renews life.  Let
this suffice concerning this article.]

We are not, therefore, on this topic contending with the adversaries
concerning a small matter.  We are not seeking out idle subtilties
when we find fault with them for teaching that we merit eternal life
by works, while that faith is omitted which apprehends Christ as
Mediator.  For of this faith which believes that for Christ's sake
the Father is propitious to us there is not a syllable in the
scholastics.  Everywhere they hold that we are accepted and righteous
because of our works, wrought either from reason, or certainly
wrought by the inclination of that love concerning which they speak.
And yet they have certain sayings, maxims, as it were, of the old
writers, which they distort in interpreting.  In the schools the
boast is made that good works please on account of grace, and that
confidence must be put in God's grace.  Here they interpret grace as
a habit by which we love God, as though, indeed, the ancients meant
to say that we ought to trust in our love, of which we certainly
experience how small and how impure it is.  Although it is strange
how they bid us trust in love, since they teach us that we are not
able to know whether it be present.  Why do they not here set forth
the grace, the mercy of God toward us?  And as often as mention is
made of this, they ought to add faith.  For the promise of God's
mercy, reconciliation, and love towards us is not apprehended unless
by faith.  With this view they would be right in saying that we ought
to trust in grace, that good works please because of grace, when
faith apprehends grace.  In the schools the boast is also made that
our good works avail by virtue of Christ's passion.  Well said!  But
why add nothing concerning faith?  For Christ is a propitiation, as
Paul, Rom. 3, 25, says, through faith.  When timid consciences are
comforted by faith, and are convinced that our sins have been blotted
out by the death of Christ, and that God has been reconciled to us on
account of Christ's suffering, then, indeed, the suffering of Christ
profits us.  If the doctrine concerning faith be omitted, it is said
in vain that works avail by virtue of Christ's passion.

And very many other passages they corrupt in the schools because they
do not teach the righteousness of faith and because they understand
by faith merely a knowledge of the history or of dogmas, and do not
understand by it that virtue which apprehends the promise of grace
and of righteousness, and which quickens hearts in the terrors of sin
and of death.  When Paul says, Rom. 10, 10: With the heart man
believeth unto righteousness, and with the mouth confession is made
unto salvation, we think that the adversaries acknowledge here that
confession justifies or saves, not _ex opere operato_, but only on
account of the faith of the heart.  And Paul thus says that
confession saves, in order to show what sort of faith obtains eternal
life; namely, that which is firm and active.  That faith, however,
which does not manifest itself in confession is not firm.  Thus other
good works please on account of faith, as also the prayers of the
Church ask that all things may be accepted for Christ's sake.  They
likewise ask all things for Christ's sake.  For it is manifest that
at the close of prayers this clause is always added: Through Christ,
our Lord.  Accordingly, we conclude that we are justified before God,
are reconciled to God and regenerated by faith, which in repentance
apprehends the promise of grace, and truly quickens the terrified
mind, and is convinced that for Christ's sake God is reconciled and
propitious to us.  And through this faith, says Peter, 1 Ep. 1, 5, we
are kept unto salvation ready to be revealed.  The knowledge of this
faith is necessary to Christians, and brings the most abundant
consolation in all afflictions, and displays to us the office of
Christ because those who deny that men are justified by faith, and
deny that Christ is Mediator and Propitiator, deny the promise of
grace and the Gospel.  They teach only the doctrine either of reason
or of the Law concerning justification.  We have shown the origin of
this case, so far as can here be done, and have explained the
objections of the adversaries.  Good men, indeed, will easily judge
these things, if they will think, as often as a passage concerning
love or works is cited, that the Law cannot be observed without
Christ, and that we cannot be justified from the Law, but from the
Gospel, that is, from the promise of the grace promised in Christ.
And we hope that this discussion, although brief, will be profitable
to good men for strengthening faith, and teaching and comforting
conscience.  For we know that those things which we have said are in
harmony with the prophetic and apostolic Scriptures, with the holy
Fathers, Ambrose, Augustine and very many others, and with the whole
Church of Christ, which certainly confesses that Christ is
Propitiator and Justifier.

Nor are we immediately to judge that the Roman Church agrees with
everything that the Pope, or cardinals, or bishops, or some of the
theologians, or monks approve.  For it is manifest that to most of
the pontiffs their own authority is of greater concern than the
Gospel of Christ.  And it has been ascertained that most of them are
openly Epicureans.  It is evident that theologians have mingled with
Christian doctrine more of philosophy than was sufficient.  Nor ought
their influence to appear so great that it will never be lawful to
dissent from their disputations, because at the same time many
manifest errors are found among them, such as, that we are able from
purely natural powers to love God above all things.  This dogma,
although it is manifestly false, has produced many other errors.  For
the Scriptures the holy Fathers, and the judgments of all the godly
everywhere make reply.  Therefore, even though Popes, or some
theologians, and monks in the Church have taught us to seek remission
of sins, grace, and righteousness through our own works, and to
invent new forms of worship, which have obscured the office of Christ,
and have made out of Christ not a Propitiator and Justifier, but
only a Legislator, nevertheless the knowledge of Christ has always
remained with some godly persons.  Scripture, moreover, has predicted
that the righteousness of faith would be obscured in this way by
human traditions and the doctrine of works.  Just as Paul often
complains (cf.  Gal. 4, 9; 5, 7; Col. 2, 8, 16 sq.; 1 Tim. 4, 2 sq.,
etc.) that there were even at that time those who, instead of the
righteousness of faith, taught that men were reconciled to God and
justified by their own works and own acts of worship, and not by
faith for Christ's sake; because men judge by nature that God ought
to be appeased by works.  Nor does reason see a righteousness other
than the righteousness of the Law, understood in a civil sense.
Accordingly, there have always existed in the world some who have
taught this carnal righteousness alone to the exclusion of the
righteousness of faith; and such teachers will also always exist.
The same happened among the people of Israel.  The greater part of
the people thought that they merited remission of sins by their works
they accumulated sacrifices and acts of worship.  On the contrary,
the prophets, in condemnation of this opinion, taught the
righteousness of faith.  And the occurrences among the people of
Israel are illustrations of those things which were to occur in the
Church.  Therefore, let the multitude of the adversaries, who condemn
our doctrine, not disturb godly minds.  For their spirit can easily
be judged, because in some articles they have condemned truth that is
so clear and manifest that their godlessness appears openly.  For the
bull of Leo X condemned a very necessary article, which all
Christians should hold and believe, namely, that we ought to trust
that we have been absolved not because of our contrition, but because
of Christ's word, Matt. 16, 19: Whatsoever thou shalt bind, etc. And
now, in this assembly, the authors of the _Confutation_ have in clear
words condemned this, namely, that we have said that faith is a part
of repentance, by which we obtain remission of sins, and overcome the
terrors of sin, and conscience is rendered pacified.  Who, however,
does not see that this article that by faith we obtain the remission
of sins, is most true, most certain, and especially necessary to all
Christians?  Who to all posterity, hearing that such a doctrine has
been condemned, will judge that the authors of this condemnation had
any knowledge of Christ?

And concerning their spirit, a conjecture can be made from the
unheard-of cruelty, which it is evident that they have hitherto
exercised towards most good men.  And in this assembly we have heard
that a reverend father, when opinions concerning our Confession were
expressed, said in the senate of the Empire that no plan seemed to
him better than to make a reply written in blood to the Confession
which we had presented written in ink.  What more cruel would
Phalaris say?  Therefore some princes also have judged this
expression unworthy to be spoken in such a meeting.  Wherefore,
although the adversaries claim for themselves the name of the Church,
nevertheless we know that the Church of Christ is with those who
teach the Gospel of Christ, not with those who defend wicked opinions
contrary to the Gospel, as the Lord says, John 10, 21: My sheep hear
My voice.  And Augustine says: The question is, Where is the Church!
What, therefore, are we to do?  Are we to seek it in our own words or
in the words of its Head our Lord Jesus Christ?  I think that we
ought to seek it in the words of Him who is Truth, and who knows His
own body best.  Hence the judgments of our adversaries will not
disturb us, since they defend human opinions contrary to the Gospel,
contrary to the authority of the holy Fathers, who have written in
the Church, and contrary to the testimonies of godly minds.




Part 11


Articles VII and VIII: _Of the Church._

The Seventh Article of our Confession, in which we said that the
Church is the congregation of saints, they have condemned and have
added a long disquisition, that the wicked are not to be separated
from the Church, since John has compared the Church to a
threshing-floor on which wheat and chaff are heaped together, Matt. 3,
12, and Christ has compared it to a net in which there are both good
and bad fishes, Matt. 13, 47. It is, verily, a true saying, namely,
that there is no remedy against the attacks of the slanderer.
Nothing can be spoken with such care that it can escape detraction.
For this reason we have added the Eighth Article, lest any one might
think that we separate the wicked and hypocrites from the outward
fellowship of the Church, or that we deny efficacy to Sacraments
administered by hypocrites or wicked men.  Therefore there is no need
here of a long defense against this slander.  The Eighth Article is
sufficient to exculpate us.  For we grant that in this life
hypocrites and wicked men have been mingled with the Church, and that
they are members of the Church according to the outward fellowship of
the signs of the Church, i.e., of Word, profession, and Sacraments,
especially if they have not been excommunicated.  Neither are the
Sacraments without efficacy for the reason that they are administered
by wicked men; yea, we can even be right in using the Sacraments
administered by wicked men.  For Paul also predicts, 2 Thess. 2, 4,
that Antichrist will sit in the temple of God, i.e., he will rule and
bear office in the Church.  But the Church is not only the fellowship
of outward objects and rites, as other governments, but it is
originally a fellowship of faith and of the Holy Ghost in hearts.
[The Christian Church consists not alone in fellowship of outward
signs, but it consists especially in inward communion of eternal
blessings in the heart, as of the Holy Ghost, of faith, of the fear
and love of God]; which fellowship nevertheless has outward marks so
that it can be recognized, namely, the pure doctrine of the Gospel,
and the administration of the Sacraments in accordance with the
Gospel of Christ.  [Namely, where God's Word is pure, and the
Sacraments are administered in conformity with the same, there
certainly is the Church, and there are Christians.] And this Church
alone is called the body of Christ, which Christ renews [Christ is
its Head, and] sanctifies and governs by His Spirit, as Paul
testifies, Eph. 1, 22 sq., when he says: And gave Him to be the Head
over all things to the Church, which is His body, the fulness of Him
that filleth all in all.  Wherefore, those in whom Christ does not
act [through His Spirit] are not the members of Christ.  This, too,
the adversaries acknowledge, namely, that the wicked are dead members
of the Church.  Therefore we wonder why they have found fault with
our description [our conclusion concerning Church] which speaks of
living members.  Neither have we said anything new.  Paul has defined
the Church precisely in the same way, Eph. 6, 25 f., that it should
be cleansed in order to be holy.  And he adds the outward marks, the
Word and Sacraments.  For he says thus: Christ also loved the Church,
and gave himself for it, that He might sanctify and cleanse it with
the washing of water by the Word, that He might present it to Himself
a glorious Church, not having spot, or wrinkle, or any such thing,
but that it should be holy and without blemish.  In the Confession we
have presented this sentence almost in the very words.  Thus also the
Church is defined by the article in the Creed which teaches us to
believe that there is a holy Catholic Church.  The wicked indeed are
not a holy Church.  And that which follows, namely, the communion of
saints, seems to be added in order to explain what the Church
signifies, namely, the congregation of saints, who have with each
other the fellowship of the same Gospel or doctrine [who confess one
Gospel, have the same knowledge of Christ] and of the same Holy Ghost,
who renews, sanctifies, and governs their hearts.

And this article has been presented for a necessary reason.  [The
article of the Church Catholic or Universal, which is gathered
together from every nation under the sun, is very comforting and
highly necessary.] We see the infinite dangers which threaten the
destruction of the Church.  In the Church itself, infinite is the
multitude of the wicked who oppress it [despise, bitterly hate, and
most violently persecute the Word, as, e.g., the Turks, Mohammedans,
other tyrants, heretics, etc. For this reason the true teaching and
the Church are often so utterly suppressed and disappear, as if there
were no Church which has happened under the papacy, it often seems
that the Church has completely perished].  Therefore, in order that
we may not despair, but may know that the Church will nevertheless
remain [until the end of the world], likewise that we may know that,
however great the multitude of the wicked is, yet the Church [which
is Christ's bride] exists, and that Christ affords those gifts which
He has promised to the Church, to forgive sins, to hear prayer, to
give the Holy Ghost, this article in the Creed presents us these
consolations.  And it says church Catholic, in order that we may not
understand the Church to be an outward government of certain nations
[that the Church is like any other external polity, bound to this or
that land, kingdom, or nation, as the Pope of Rome will say], but
rather men scattered throughout the whole world [here and there in
the world, from the rising to the setting of the sun], who agree
concerning the Gospel, and have the same Christ, the same Holy Ghost,
and the same Sacraments, whether they have the same or different
human traditions.  And the gloss upon the Decrees says that the
Church in its wide sense embraces good and evil; likewise, that the
wicked are in the Church only in name, not in fact; but that the good
are in the Church both in fact and in name.  And to this effect there
are many passages in the Fathers.  For Jerome says: The sinner,
therefore, who Has been soiled with any blotch cannot be called a
member of the Church of Christ, neither can he be said to be subject
to Christ.

Although, therefore, hypocrites and wicked men are members of this
true Church according to outward rites [titles and offices], yet when
the Church is defined, it is necessary to define that which is the
living body of Christ, and which is in name and in fact the Church
[which is called the body of Christ, and has fellowship not alone in
outward signs, but has gifts in the heart, namely, the Holy Ghost and
faith].  And for this there are many reasons.  For it is necessary to
understand what it is that principally makes us members, and that,
living members, of the Church.  If we will define the Church only as
an outward polity of the good and wicked, men will not understand
that the kingdom of Christ is righteousness of heart and the gift of
the Holy Ghost [that the kingdom of Christ is spiritual, as
nevertheless it is, that therein Christ inwardly rules, strengthens,
and comforts hearts, and imparts the Holy Ghost and various spiritual
gifts], but they will judge that it is only the outward observance of
certain forms of worship and rites.  Likewise, what difference will
there be between the people of the Law and the Church if the Church
is an outward polity?  But Paul distinguishes the Church from the
people of the Law thus, that the Church is a spiritual people, i.e.,
that it has been distinguished from the heathen not by civil rites
[not in the polity and civil affairs], but that it is the true people
of God, regenerated by the Holy Ghost.  Among the people of the Law,
apart from the promise of Christ, also the carnal seed [all those who
by nature were born Jews and Abraham's seed] had promises concerning
corporeal things, of government, etc. And because of these even the
wicked among them were called the people of God, because God had
separated this carnal seed from other nations by certain outward
ordinances and promises; and yet, these wicked persons did not please
God.  But the Gospel [which is preached in the Church] brings not
merely the shadow of eternal things, but the eternal things
themselves, the Holy Ghost and righteousness, by which we are
righteous before God.  [But every true Christian is even here upon
earth partaker of eternal blessings, even of eternal comfort, of
eternal life, and of the Holy Ghost, and of righteousness which is
from God, until he will be completely saved in the world to come.]

Therefore, only those are the people, according to the Gospel, who
receive this promise of the Spirit.  Besides, the Church is the
kingdom of Christ, distinguished from the kingdom of the devil.  It
is certain, however, that the wicked are in the power of the devil,
and members of the kingdom of the devil, as Paul teaches, Eph. 2, 2,
when he says that the devil now worketh in the children of
disobedience.  And Christ says to the Pharisees, who certainly had
outward fellowship with the Church, i.e., with the saints among the
people of the Law (for they held office, sacrificed, and taught): Ye
are of your father, the devil, John 8, 44. Therefore, the Church,
which is truly the kingdom of Christ is properly the congregation of
saints.  For the wicked are ruled by the devil, and are captives of
the devil; they are not ruled by the Spirit of Christ.

But what need is there of words in a manifest matter?  [However, the
adversaries contradict the plain truth.] If the Church, which is
truly the kingdom of Christ, is distinguished from the kingdom of the
devil, it follows necessarily that the wicked, since they are in the
kingdom of the devil, are not the Church; although in this life,
because the kingdom of Christ has not yet been revealed; they are
mingled with the Church, and hold offices [as teachers, and other
offices] in the Church.  Neither are the wicked the kingdom of Christ,
for the reason that the revelation has not yet been made.  For that
is always the kingdom which He quickens by His Spirit, whether it be
revealed or be covered by the cross; just as He who has now been
glorified is the same Christ who was before afflicted.  And with this
clearly agree the parables of Christ, who says, Matt. 13, 38, that
the good seed are the children of the kingdom, but the tares are the
children of the Wicked One.  The field, He says, is the world, not
the Church.  Thus John [Matt. 3,12: He will throughly purge His floor,
and gather His wheat into the garner; but He will burn up the chaff]
speaks concerning the whole race of the Jews, and says that it will
come to pass that the true Church will be separated from that people.
Therefore, this passage is more against the adversaries than in
favor of them, because it shows that the true and spiritual people is
to be separated from the carnal people.  Christ also speaks of the
outward appearance of the Church when He says, Matt. 13, 47: The
kingdom of heaven is like unto a net, likewise, to ten virgins; and
He teaches that the Church has been covered by a multitude of evils,
in order that this stumbling-block may not offend the pious; likewise,
in order that we may know that the Word and Sacraments are
efficacious even when administered by the wicked.  And meanwhile He
teaches that these godless men, although they have the fellowship of
outward signs, are nevertheless not the true kingdom of Christ and
members of Christ; for they are members of the kingdom of the devil.
Neither, indeed, are we dreaming of a Platonic state, as some
wickedly charge, but we say that this Church exists, namely, the
truly believing and righteous men scattered throughout the whole
world [We are speaking not of an imaginary Church, which is to be
found nowhere; but we say and know certainly that this Church,
wherein saints live, is and abides truly upon earth; namely, that
some of God's children are here and there in all the world, in
various kingdoms, islands, lands, and cities, from the rising of the
sun to its setting, who have truly learned to know Christ and His
Gospel.] And we add the marks: the pure doctrine of the Gospel [the
ministry or the Gospel] and the Sacraments.  And this Church is
properly the pillar of the truth, 1 Tim. 3, 15. For it retains the
pure Gospel, and, as Paul says, 1 Cor. 3, 11 [: "Other foundation can
no man lay than that is laid, which is Jesus Christ"], the foundation,
i.e., the true knowledge of Christ and faith.  Although among these
[in the body which is built upon the true foundation, i.e., upon
Christ and faith] there are also many weak persons, who build upon
the foundation stubble that will perish, i.e., certain unprofitable
opinions [some human thoughts and opinions], which, nevertheless,
because they do not overthrow the foundation are both forgiven them
and also corrected.  And the writings of the holy Fathers testify
that sometimes even they built stubble upon the foundation, but that
this did not overthrow their faith.  But most of those errors which
our adversaries defend, overthrow faith, as, their condemnation of
the article concerning the remission of sins, in which we say that
the remission of sins is received by faith.  Likewise it is a
manifest and pernicious error when the adversaries teach that men
merit the remission of sins by love to God, prior to grace.  [In the
place of Christ they set up their works, orders, masses, just as the
Jews, the heathen, and the Turks intend to be saved by their works.]
For this also is to remove "the foundation," i.e., Christ.  Likewise,
what need will there be of faith if the Sacraments justify _ex opere
operato_, without a good disposition on the part of the one using
them? [without faith.  Now, a person that does not regard faith as
necessary has already lost Christ.  Again, they set up the worship of
saints, call upon them instead of Christ, the Mediator, etc.] But
just as the Church has the promise that it will always have the Holy
Ghost, so it has also the threatenings that there will be wicked
teachers and wolves.  But that is the Church in the proper sense
which has the Holy Ghost.  Although wolves and wicked teachers become
rampant [rage and do injury] in the Church, yet they are not properly
the kingdom of Christ.  Just as Lyra also testifies, when he says:
The Church does not consist of men with respect to power, or
ecclesiastical or secular dignity, because many princes and
archbishops and others of lower rank have been found to have
apostatized from the faith.  Therefore, the Church consists of those
persons in whom there is a true knowledge and confession of faith and
truth.  What else have we said in our Confession than what Lyra here
says [in terms so clear that he could not have spoken more clearly]?

But the adversaries perhaps require [a new Roman definition], that
the Church be defined thus, namely, that it is the supreme outward
monarchy of the whole world, in which the Roman pontiff necessarily
has unquestioned power, which no one is permitted to dispute or
censure [no matter whether he uses it rightly, or misuses it], to
frame articles of faith; to abolish, according to his pleasure, the
Scriptures [to pervert and interpret them contrary to all divine law,
contrary to his own decretals, contrary to all imperial rights, as
often, to as great an extent, and whenever it pleases him, to sell
indulgences and dispensations for money]; to appoint rites of worship
and sacrifices; likewise, to frame such laws as he may wish, and to
dispense and exempt from whatever laws he may wish, divine, canonical,
or civil; and that from him [as from the vicegerent of Christ] the
Emperor and all kings receive, according to the command of Christ,
the power and right to hold their kingdoms, from whom, since the
Father has subjected all things to Him, it must be understood, this
right was transferred to the Pope; therefore the Pope must
necessarily be [a God on earth, the supreme Majesty,] lord of the
whole world, of all the kingdoms of the world, of all things private
and public, and must have absolute power in temporal and spiritual
things, and both swords, the spiritual and temporal Besides this
definition, not of the Church of Christ but of the papal kingdom, has
as its authors not only the canonists, but also Daniel 11 36 ff.
[Daniel, the prophet, represents Antichrist in this way.]

Now, if we would define the Church in this way [that it is such pomp,
as is exhibited in the Pope's rule], we would perhaps have fairer
judges.  For there are many things extant written extravagantly and
wickedly concerning the power of the Pope of Rome on account of which
no one has ever been arraigned.  We alone are blamed, because we
proclaim the beneficence of Christ [and write and preach the clear
word and teaching of the apostles], that by faith in Christ we obtain
remission of sins, and not by [hypocrisy or innumerable] rites of
worship devised by the Pope.  Moreover, Christ, the prophets, and the
apostles define the Church of Christ far otherwise than as the papal
kingdom.  Neither must we transfer to the Popes what belongs to the
true Church, namely, that they are pillars of the truth, that they do
not err.  For how many of them care for the Gospel or judge that it
[one little page, one letter of it] is worth being read?  Many [in
Italy and elsewhere] even publicly ridicule all religions, or, if
they approve anything, they approve such things only as are in
harmony with human reason, and regard the rest fabulous and like the
tragedies of the poets.  Wherefore we hold, according Scriptures,
that the Church, properly so called, is the congregation of saints
[of those here and there in the world], who truly believe the Gospel
of Christ, and have the Holy Ghost.  And yet we confess that in this
life many hypocrites and wicked men, mingled with these, have the
fellowship of outward signs who are members of the Church according
to this fellowship of outward signs, and accordingly bear offices in
the Church [preach, administer the Sacraments, and bear the title and
name of Christians].  Neither does the fact that the sacraments are
administered by the unworthy detract from their efficacy, because, on
account of the call of the Church, they represent the person of
Christ, and do not represent their own persons, as Christ testifies,
Luke 10, 16: He that heareth you heareth Me.  [Thus even Judas was
sent to preach.] When they offer the Word of God, when they offer the
Sacraments, they offer them in the stead and place of Christ.  Those
words of Christ teach us not to be offended by the unworthiness of
the ministers.

But concerning this matter we have spoken with sufficient clearness
in the Confession that we condemn the Donatists and Wyclifites, who
thought that men sinned when they received the sacraments from the
unworthy in the Church.  These things seem, for the present, to be
sufficient for the defense of the description of the Church which we
have presented.  Neither do we see how, when the Church, properly so
called, is named the body of Christ, it should be described otherwise
than we have described it.  For it is evident that the wicked belong
to the kingdom and body of the devil, who impels and holds captive
the wicked.  These things are clearer than the light of noonday,
however, if the adversaries still continue to pervert them, we will
not hesitate to reply at greater length.

The adversaries condemn also the part of the Seventh Article in which
we said that "to the unity of the Church it is sufficient to agree
concerning the doctrine of the Gospel and the administration of the
Sacraments; nor is it necessary that human traditions rites or
ceremonies instituted by men should be alike everywhere." Here they
distinguish between universal and particular rites, and approve our
article if it be understood concerning particular rites, they do not
receive it concerning universal rites.  [That is a fine clumsy
distinction!] We do not sufficiently understand what the adversaries
mean.  We are speaking of true, i.e., of spiritual unity [we say that
those are one harmonious Church who believe in one Christ, who have
one Gospel, one Spirit, one faith, the same Sacraments; and we are
speaking, therefore, of spiritual unity], without which faith in the
heart, or righteousness of heart before God cannot exist.  For this
we say that similarity of human rites, whether universal or
particular, is not necessary, because the righteousness of faith is
not a righteousness bound to certain traditions [outward ceremonies
of human ordinances] as the righteousness of the Law was bound to the
Mosaic ceremonies, because this righteousness of the heart is a
matter that quickens the heart.  To this quickening, human traditions,
whether they be universal or particular, contribute nothing; neither
are they effects of the Holy Ghost, as are chastity, patience, the
fear of God, love to one's neighbor, and the works of love.

Neither were the reasons trifling why we presented this article.  For
it is evident that many [great errors and] foolish opinions
concerning traditions had crept into the Church.  Some thought that
human traditions were necessary services for meriting justification
[that without such human ordinances Christian holiness and faith are
of no avail before God; also that no one can be a Christian unless he
observe such traditions, although they are nothing but an outward
regulation].  And afterwards they disputed how it came to pass that
God was worshiped with such variety, as though, indeed, these
observances were acts of worship, and not rather outward and
political ordinances, pertaining in no respect to righteousness of
heart or the worship of God, which vary, according to the
circumstances, for certain probable reasons, sometimes in one way and
at other times in another [as in worldly governments one state has
customs different from another].  Likewise some Churches have
excommunicated others because of such traditions, as the observance
of Easter, pictures, and the like.  Hence the ignorant have supposed
that faith, or the righteousness of the heart before God, cannot
exist [and that no one can be a Christian] without these observances.
For many foolish writings of the Summists and of others concerning
this matter are extant.

But just as the dissimilar length of day and night does not injure
the unity of the Church, so we believe that the true unity of the
Church is not injured by dissimilar rites instituted by men; although
it is pleasing to us that, for the sake of tranquillity [unity and
good order], universal rites be observed just as also in the churches
we willingly observe the order of the Mass, the Lord's Day, and other
more eminent festival days.  And with a very grateful mind we embrace
the profitable and ancient ordinances, especially since they contain
a discipline by which it is profitable to educate and train the
people and those who are ignorant [the young people].  But now we are
not discussing the question whether it be of advantage to observe
them on account of peace or bodily profit.  Another matter is treated
of.  For the question at issue is, whether the observances of human
traditions are acts of worship necessary for righteousness before God.
This is the point to be judged in this controversy and when this is
decided, it can afterwards be judged whether to the true unity of the
Church it is necessary that human traditions should everywhere be
alike.  For if human traditions be not acts of worship necessary for
righteousness before God, it follows that also they can be righteous
and be the sons of God who have not the traditions which have been
received elsewhere.  F.i., if the style of German clothing is not
worship of God, necessary for righteousness before God, it follows
that men can be righteous and sons of God and the Church of Christ,
even though they use a costume that is not German, but French.

Paul clearly teaches this to the Colossians, 2,16.17: Let no man,
therefore, judge you in meat, or in drink, or in respect of an
holy-day, or of the new moon, or of the Sabbath days, which are a
shadow of things to come; but the body is of Christ.  Likewise, v. 20
sqq.: If ye be dead with Christ from the rudiments of the world, why,
as though living in the world, are ye subject to ordinances (touch
not; taste not; handle not; which are to perish with the using),
after the commandments and doctrines of men?  Which things have,
indeed, a show of wisdom in will-worship and humility.  For the
meaning is: Since righteousness of the heart is a spiritual matter,
quickening hearts, and it is evident that human traditions do not
quicken hearts and are not effects of the Holy Ghost, as are love to
one's neighbor, chastity, etc., and are not instruments through which
God moves hearts to believe, as are the divinely given Word and
Sacraments, but are usages with regard to matters that pertain in no
respect to the heart, which perish with the using, we must not
believe that they are necessary for righteousness before God.  [They
are nothing eternal, hence, they do not procure eternal life, but are
an external bodily discipline, which does not change the heart.] And
to the same effect he says, Rom. 14, 17: The kingdom of God is not
meat and drink, but righteousness and peace and joy in the Holy Ghost.
But there is no need to cite many testimonies, since they are
everywhere obvious in the Scriptures, and in our Confession we have
brought together very many of them, in the latter articles.  And the
point to be decided in this controversy must be repeated after a
while, namely, whether human traditions be acts of worship necessary
for righteousness before God.  There we will discuss this matter more
fully.

The adversaries say that universal traditions are to be observed
because they are supposed to have been handed down by the apostles.
What religious men they are!  They wish that the rites derived from
the apostles be retained, they do not wish the doctrine of the
apostles to be retained.  They must judge concerning these rites just
as the apostles themselves judge in their writings.  For the apostles
did not wish us to believe that through such rites we are justified,
that such rites are necessary for righteousness before God.  The
apostles did not wish to impose such a burden upon consciences; they
did not wish to place righteousness and sin in the observance of days,
food, and the like.  Yea, Paul calls such opinions doctrines of
devils, 1 Tim. 4, 1. Therefore the will and advice of the apostles
ought to be derived from their writings; it is not enough to mention
their example.  They observed certain days, not because this
observance was necessary for justification, but in order that the
people might know at what time they should assemble.  They observed
also certain other rites and orders of lessons whenever they
assembled.  The people [In the beginning of the Church the Jews who
had become Christians] retained also from the customs of the Fathers
[from their Jewish festivals and ceremonies], as is commonly the case,
certain things which, being somewhat changed, the apostles adapted
to the history of the Gospel as the Passover, Pentecost, so that not
only by teaching, but also through these examples they might hand
down to posterity the memory of the most important subjects.  But if
these things were handed down as necessary for justification, why
afterwards did the bishops change many things in these very matters?
For, if they were matters of divine right, it was not lawful to
change them by human authority.  Before the Synod of Nice some
observed Easter at one time and others at another time.  Neither did
this want of uniformity injure faith.  Afterward the plan was adopted
by which our Passover [Easter] did not fall at the same time as that
of the Jewish Passover.  But the apostles had commanded the Churches
to observe the Passover with the brethren who had been converted from
Judaism.  Therefore, after the Synod of Nice, certain nations
tenaciously held to the custom of observing the Jewish time.  But the
apostles, by this decree, did not wish to impose necessity upon the
Churches, the words of the decree testify.  For it bids no one to be
troubled, even though his brethren, in observing Easter, do not
compute the time aright.  The words of the decree are extant in
Epiphanius: Do not calculate, but celebrate it whenever your brethren
of the circumcision do; celebrate it at the same time with them, and
even though they may have erred, let not this be a care to you..
Epiphanius writes that these are the words of the apostles presented
in a decree concerning Easter, in which the discreet reader can
easily judge that the apostles wished to free the people from the
foolish opinion of a fixed time, when they prohibit them from being
troubled, even though a mistake should be made in the computation.
Some, moreover in the East, who were called, from the author of the
dogma, Audians, contended, on account of this decree of the apostles,
that the Passover should be observed with the Jews.  Epiphanius, in
refuting them, praises the decree and says that it contains nothing
which deviates from the faith or rule of the Church, and blames the
Audians because they do not understand aright the expression, and
interprets it in the sense in which we interpret it because the
apostles did not consider it of any importance at what time the
Passover should be observed, but because prominent brethren had been
converted from the Jews who observed their custom, and, for the sake
of harmony, wished the rest to follow their example And the apostles
wisely admonished the reader neither to remove the liberty of the
Gospel, nor to impose necessity upon consciences, because they add
that they should not be troubled even though there should be an error
in making the computation.

Many things of this class can be gathered from the histories, in
which it appears that a want of uniformity in human observances does
not injure the unity of faith [separate no one from the universal
Christian Church].  Although, what need is there of discussion?  The
adversaries do not at all understand what the righteousness of faith
is, what the kingdom of Christ is, when they judge that uniformity of
observances in food, days, clothing, and the like, which do not have
the command of God, is necessary.  But look at the religious men, our
adversaries.  For the unity of the Church they require uniform human
observances, although they themselves have changed the ordinance of
Christ in the use of the Supper, which certainly was a universal
ordinance before.  But if universal ordinances are so necessary, why
do they themselves change the ordinance of Christ's Supper, which is
not human, but divine?  But concerning this entire controversy we
shall have to speak at different times below.

The entire Eighth Article has been approved, in which we confess that
hypocrites and wicked persons have been mingled with the Church, and
that the Sacraments are efficacious even though dispensed by wicked
ministers, because the ministers act in the place of Christ, and do
not represent their own persons, according to Luke 10, 16: He that
heareth you heareth Me.  Impious teachers are to be deserted [are not
to be received or heard], because these do not act any longer in the
place of Christ, but are antichrists.  And Christ says Matt. 7, 15:
Beware of false prophets.  And Paul, Gal. 1, 9: If any man preach any
other gospel unto you, let him be accursed.

Moreover, Christ has warned us in His parables concerning the Church,
that when offended by the private vices, whether of priests or people,
we should not excite schisms, as the Donatists have wickedly done.
As to those, however, who have excited schisms, because they denied
that priests are permitted to hold possessions and property, we hold
that they are altogether seditious.  For to hold property is a civil
ordinance.  It is lawful, however, for Christians to use civil
ordinances, just as they use the air, the light, food, drink.  For as
this order of the world and fixed movements of the heavenly bodies
are truly God's ordinances and these are preserved by God, so lawful
governments are truly God's ordinances, and are preserved and
defended by God against the devil.




Part 12


Article IX: _Of Baptism._

The Ninth Article has been approved, in which we confess that Baptism
is necessary to salvation, and that children are to be baptized, and
that the baptism of children is not in vain, but is necessary and
effectual to salvation.  And since the Gospel is taught among us
purely and diligently, by God's favor we receive also from it this
fruit, that in our Churches no Anabaptists have arisen [have not
gained ground in our Churches], because the people have been
fortified by God's Word against the wicked and seditious faction of
these robbers.  And as we condemn quite a number of other errors of
the Anabaptists, we condemn this also, that they dispute that the
baptism of little children is unprofitable.  For it is very certain
that the promise of salvation pertains also to little children [that
the divine promises of grace and of the Holy Ghost belong not alone
to the old, but also to children].  It does not, however, pertain to
those who are outside of Christ's Church where there is neither Word
nor Sacraments because the kingdom of Christ exists only with the
Word and Sacraments.  Therefore it is necessary to baptize little
children, that the promise of salvation may be applied to them,
according to Christ's command, Matt. 28, 19: Baptize all nations.
Just as here salvation is offered to all, so Baptism is offered to
all, to men, women, children, infants.  It clearly follows, therefore,
that infants are to be baptized, because with Baptism salvation [the
universal grace and treasure of the Gospel] is offered.  Secondly, it
is manifest that God approves of the baptism of little children.
Therefore the Anabaptists, who condemn the baptism of little children,
believe wickedly.  That God, however, approves of the baptism of
little children is shown--by this, namely, that God gives the Holy
Ghost to those thus baptized [to many who have been baptized in
childhood].  For if this baptism would be in vain, the Holy Ghost
would be given to none, none would be saved, and finally there would
be no Church.  [For there have been many holy men in the Church who
have not been baptized otherwise.] This reason, even taken alone, can
sufficiently establish good and godly minds against the godless and
fanatical opinions of the Anabaptists.




Part 13


Article X: _Of the Holy Supper._

The Tenth Article has been approved, in which we confess that we
believe, that in the Lord's Supper the body and blood of Christ are
truly and substantially present, and are truly tendered, with those
things which are seen, bread and wine to those who receive the
Sacrament.  This belief we constantly defend as the subject has been
carefully examined and considered.  For since Paul says, 1 Cor. 10,
16, that the bread is the communion of the Lord's body, etc., it
would follow, if the Lord's body were not truly present, that the
bread is not a communion of the body, but only of the spirit of
Christ.  And we have ascertained that not only the Roman Church
affirms the bodily presence of Christ, but the Greek Church also both
now believes, and formerly believed, the same.  For the canon of the
Mass among them testifies to this, in which the priest clearly prays
that the bread may be changed and become the very body of Christ.
And Vulgarius, who seems to us to be not a silly writer, says
distinctly that bread is not a mere figure, but is truly changed into
flesh.  And there is a long exposition of Cyril on John 15, in which
he teaches that Christ is corporeally offered us in the Supper.  For
he says thus: Nevertheless, we do not deny that we are joined
spiritually to Christ by true faith and sincere love.  But that we
have no mode of connection with Him, according to the flesh, this
indeed we entirely deny.  And this, we say, is altogether foreign to
the divine Scriptures.  For who has doubted that Christ is in this
manner a vine, and we the branches, deriving thence life for
ourselves?  Hear Paul saying 1 Cor. 10, 17; Rom. 12, 5; Gal. 3, 28:
We are all one body in Christ; although we are many, we are,
nevertheless, one in Him; for we are all partakers of that one bread.
Does he perhaps think that the virtue of the mystical benediction is
unknown to us?  Since this is in us, does it not also, by the
communication of Christ's flesh, cause Christ to dwell in us bodily?
And a little after: Whence we must consider that Christ is in us not
only according to the habit, which we call love, but also by natural
participation, etc. We have cited these testimonies, not to undertake
a discussion here concerning this subject, for His Imperial Majesty
does not disapprove of this article, but in order that all who may
read them may the more clearly perceive that we defend the doctrine
received in the entire Church, that in the Lord's Supper the body and
blood of Christ are truly and substantially present, and are truly
tendered with those things which are seen, bread and wine.  And we
speak of the presence of the living Christ [living body]; for we know
that death hath no more dominion over Him, Rom. 6, 9.




Part 14


Article XI: _Of Confession._

The Eleventh Article, Of Retaining Absolutism in the Church, is
approved.  But they add a correction in reference to confession,
namely, that the regulation headed, _Omnis Utriusque_, be observed,
and that both annual confession be made, and, although all sins
cannot be enumerated, nevertheless diligence be employed in order
that they be recollected, and those which can be recalled be
recounted.  Concerning this entire article, we will speak at greater
length after a while, when we will explain our entire opinion
concerning repentance.  It is well known that we have so elucidated
and extolled [that we have preached, written, and taught in a manner
so Christian, correct, and pure] the benefit of absolution and the
power of the keys that many distressed consciences have derived
consolation from our doctrine, after they heard that it is the
command of God, nay, rather the very voice of the Gospel, that we
should believe the absolution, and regard it as certain that the
remission of sins is freely granted us for Christ's sake, and that we
should believe that by this faith we are truly reconciled to God [as
though we heard a voice from heaven].  This belief has encouraged
many godly minds, and, in the beginning, brought Luther the highest
commendation from all good men, since it shows consciences sure and
firm consolation because previously the entire power of absolution
[entire necessary doctrine of repentance] had been kept suppressed by
doctrines concerning works, since the sophists and monks taught
nothing of faith and free remission [but pointed men to their own
works, from which nothing but despair enters alarmed consciences].

But with respect to the time, certainly most men in our churches use
the Sacraments, absolution and the Lord's Supper, frequently in a
year.  And those who teach of the worth and fruits of the Sacraments
speak in such a manner as to invite the people to use the Sacraments
frequently.  For concerning this subject there are many things extant
written by our theologians in such a manner that the adversaries, if
they are good men, will undoubtedly approve and praise them.
Excommunication is also pronounced against the openly wicked [those
who live in manifest vices, fornication, adultery, etc.] and the
despisers of the Sacraments.  These things are thus done both
according to the Gospel and according to the old canons.  But a fixed
time is not prescribed, because all are not ready in like manner at
the same time.  Yea, if all are to come at the same time, they cannot
be heard and instructed in order [so diligently].  And the old canons
and Fathers do not appoint a fixed time.  The canon speaks only thus:
If any enter the Church and be found never to commune, let them be
admonished that, if they do not commune, they come to repentance.  If
they commune [if they wish to be regarded as Christians], let them
not be expelled; if they fail to do so, let them be excommunicated.
Christ [Paul] says, I Cor. 11, 29, that those who eat unworthily eat
judgment to themselves.  The pastors, accordingly, do not compel
those who are not qualified to use the Sacraments.

Concerning the enumeration of sins in confession, men are taught in
such a way as not to ensnare their consciences.  Although it is of
advantage to accustom inexperienced men to enumerate some things
[which worry them], in order that they may be the more readily taught,
yet we are now discussing what is necessary according to divine Law.
Therefore, the adversaries ought not to cite for us the regulation
_Omnis Utriusque_, which is not unknown to us, but they ought to show
from the divine Law that an enumeration of sins is necessary for
obtaining their remission.  The entire Church, throughout all Europe,
knows what sort of snares this point of the regulation, which
commands that all sins be confessed, has east upon consciences.
Neither has the text by itself as much disadvantage as was afterwards
added by the Summists, who collect the circumstances of the sins.
What labyrinths were there!  How great a torture for the best minds!
For the licentious and profane were in no way moved by these
instruments of terror.  Afterwards what tragedies [what jealousy and
hatred] did the questions concerning one's own priest excite among
the pastors and brethren [monks of various orders], who then were by
no means brethren when they were warring concerning jurisdiction of
confessions! [for all brotherliness, all friendship, ceased, when the
question was concerning authority and confessor's fees.] We,
therefore, believe that, according to divine Law, the enumeration of
sins is not necessary.  This also is pleasing to Panormitanus and
very many other learned jurisconsults.  Nor do we wish to impose
necessity upon the consciences of our people by the regulation _Omnis
Utriusque_, of which we judge, just as of other human traditions,
that they are not acts of worship necessary for justification.  And
this regulation commands an impossible matter, that we should confess
all sins.  It is evident, however, that most sins we neither remember
nor understand [nor do we indeed even see the greatest sins],
according to Ps. 19, 13: Who can understand his errors?

If the pastors are good men, they will know how far it is of
advantage to examine [the young and otherwise] inexperienced persons
but we do not wish to sanction the torture [the tyranny of
consciences] of the Summists, which notwithstanding would have been
less intolerable if they had added one word concerning faith, which
comforts and encourages consciences.  Now, concerning this faith
which obtains the remission of sins, there is not a syllable in so
great a mass of regulations, glosses, summaries, books of confession.
Christ is nowhere read there.  [Nobody will there read a word by
which he could learn to know Christ, or what Christ is.] Only the
lists of sins are read [to the end of gathering and accumulating sins,
and this would be of some value if they understood those sins which
God regards as such].  And the greater part is occupied with sins
against human traditions, and this is most vain.  This doctrine has
forced to despair many godly minds, which were not able to find rest,
because they believed that by divine Law an enumeration was necessary,
and yet they experienced that it was impossible.  But other faults
of no less moment inhere in the doctrine of the adversaries
concerning repentance, which we will now recount.




Part 15


Article XII (V): _Of Repentance._

In the Twelfth Article they approve of the first part, in which we
set forth that such as have fallen after baptism may obtain remission
of sins at whatever time, and as often as they are converted.  They
condemn the second part, in which we say that the parts of repentance
are contrition and faith [a penitent, contrite heart, and faith,
namely that I receive the forgiveness of sins through Christ].  [Hear,
now, what it is that the adversaries deny.] They [without shame]
deny that faith is the second part of repentance.  What are we to do
here, O Charles, thou most invincible Emperor?  The very voice of the
Gospel is this, that by faith we obtain the remission of sins.  [This
word is not our word but the voice and word of Jesus Christ, our
Savior.] This voice of the Gospel these writers of the _Confutation_
condemn.  We, therefore, can in no way assent to the _Confutation_.
We cannot condemn the voice of the Gospel, so salutary and abounding
in consolation.  What else is the denial that by faith we obtain
remission of sins than to treat the blood and death of Christ with
scorn?  We therefore beseech thee, O Charles most invincible Emperor,
patiently and diligently to hear and examine this most important
subject, which contains the chief topic of the Gospel, and the true
knowledge of Christ, and the true worship of God [these great, most
exalted and important matters which concern our own souls and
consciences yea, also the entire faith of Christians, the entire
Gospel, the knowledge of Christ, and what is highest and greatest,
not only in this perishable, but also in the future life: the
everlasting welfare or perdition of us all before God].  For all good
men will ascertain that especially on this subject we have taught
things that are true, godly, salutary, and necessary for the whole
Church of Christ [things of the greatest significance to all pious
hearts in the entire Christian Church on which their whole salvation
and welfare depends, and without instruction on which there can be or
remain no ministry, no Christian Church].  They will ascertain from
the writings of our theologians that very much light has been added
to the Gospel, and many pernicious errors have been corrected, by
which, through the opinions of the scholastics and canonists, the
doctrine of repentance was previously covered.

Before we come to the defense of our position, we must say this first:
All good men of all ranks, and also of the theological rank
undoubtedly confess that before the writings of Luther appeared, the
doctrine of repentance was very much confused.  The books of the
Sententiaries are extant, in which there are innumerable questions
which no theologians were ever able to explain satisfactorily.  The
people were able neither to comprehend the sum of the matter, nor to
see what things especially were required in repentance, where peace
of conscience was to be sought for.  Let any one of the adversaries
come and tell us when remission of sins takes place.  O good God,
what darkness there is!  They doubt whether it is in attrition or in
contrition that remission of sins occurs.  And if it occurs on
account of contrition, what need is there of absolution, what does
the power of the keys effect, if sins have been already remitted?
Here, indeed, they also labor much more, and wickedly detract from
the power of the keys.  Some dream that by the power of the keys
guilt is not remitted, but that eternal punishments are changed into
temporal.  Thus the most salutary power would be the ministry, not of
life and the Spirit, but only of wrath and punishments.  Others,
namely, the more cautious imagine that by the power of the keys sins
are remitted before the Church and not before God.  This also is a
pernicious error.  For if the power of the keys does not console us
before God, what, then, will pacify the conscience?  Still more
involved is what follows.  They teach that by contrition we merit
grace.  In reference to which, if any one should ask why Saul and
Judas and similar persons, who were dreadfully contrite, did not
obtain grace, the answer was to be taken from faith and according to
the Gospel, that Judas did not believe, that he did not support
himself by the Gospel and promise of Christ.  For faith shows the
distinction between the contrition of Judas and of Peter.  But the
adversaries take their answer from the Law, that Judas did not love
God, but feared the punishments.  [Is not this teaching uncertain and
improper things concerning repentance?] When, however, will a
terrified conscience, especially in those serious, true, and great
terrors which are described in the psalms and the prophets, and which
those certainly taste who are truly converted, be able to decide
whether it fears God for His own sake [out of love it fears God, as
its God], or is fleeing from eternal punishments?  [These people may
not have experienced much of these anxieties, because they juggle
words and make distinctions according to their dreams.  But in the
heart when the test is applied, the matter turns out quite
differently, and the conscience cannot be set at rest with paltry
syllables and words.] These great emotions can be distinguished in
letters and terms; they are not thus separated in fact, as these
sweet sophists dream.  Here we appeal to the judgments of all good
and wise men [who also desire to know the truth].  They undoubtedly
will confess that these discussions in the writings of the
adversaries are very confused and intricate.  And nevertheless the
most important subject is at stake, the chief topic of the Gospel,
the remission of sins.  This entire doctrine concerning these
questions which we have reviewed, is, in the writings of the
adversaries, full of errors and hypocrisy, and obscures the benefit
of Christ, the power of the keys, and the righteousness of faith [to
inexpressible injury of conscience].

These things occur in the first act.  What when they come to
confession?  What a work there is in the endless enumeration of sins
which is nevertheless, in great part, devoted to those against human
traditions!  And in order that good minds may by this means be the
more tortured, they falsely assert that this enumeration is of divine
right.  And while they demand this enumeration under the pretext of
divine right, in the mean time they speak coldly concerning
absolution which is truly of divine right.  They falsely assert that
the Sacrament itself confers grace _ex opere operato_ without a good
disposition on the part of the one using it; no mention is made of
faith apprehending the absolution and consoling the conscience.  This
is truly what is generally called _apienai pro tohn mustehriohn_
departing before the mysteries.  [Such people are called genuine Jews.]

The third act [of this play] remains, concerning satisfactions.  But
this contains the most confused discussions.  They imagine that
eternal punishments are commuted to the punishments of purgatory, and
teach that a part of these is remitted by the power of the keys, and
that a part is to be redeemed by means of satisfactions.  They add
further that satisfactions ought to be works of supererogation, and
they make these consist of most foolish observances, such as
pilgrimages, rosaries, or similar observances which do not have the
command of God.  Then, just as they redeem purgatory by means of
satisfactions, so a scheme of redeeming satisfactions which was most
abundant in revenue [which became quite a profitable, lucrative
business and a grand fair] was devised.  For they sell [without
shame] indulgences which they interpret as remissions of
satisfactions.  And this revenue [this trafficking, this fair,
conducted so shamelessly] is not only from the living, but is much
more ample from the dead.  Nor do they redeem the satisfactions of
the dead only by indulgences, but also by the sacrifice of the Mass.
In a word, the subject of satisfactions is infinite.  Among these
scandals (for we cannot enumerate all things) and doctrines of devils
lies buried the doctrine of the righteousness of faith in Christ and
the benefit of Christ.  Wherefore, all good men understand that the
doctrine of the sophists and canonists concerning repentance has been
censured for a useful and godly purpose.  For the following dogmas
are clearly false, and foreign not only to Holy Scripture, but also
to the Church Fathers:-I. That from the divine covenant we merit
grace by good works wrought without grace.

II. That by attrition we merit grace.

III. That for the blotting out of sin the mere detestation of the
crime is sufficient.

IV. That on account of contrition, and not by faith in Christ, we
obtain remission of sins.

V. That the power of the keys avails for the remission of sins, not
before God, but before the Church.

VI. That by the power of the keys sins are not remitted before God,
but that the power of the keys has been instituted to commute eternal
to temporal punishments, to impose upon consciences certain
satisfactions, to institute new acts of worship, and to obligate
consciences to such satisfactions and acts of worship.

VII. That according to divine right the enumeration of offenses in
confession, concerning which the adversaries teach, is necessary.

VIII. That canonical satisfactions are necessary for redeeming the
punishment of purgatory, or they profit as a compensation for the
blotting out of guilt.  For thus uninformed persons understand it.
[For, although in the schools satisfactions are made to apply only to
the punishment, everybody thinks that remission of guilt is thereby
merited.]

IX. That the reception of the sacrament of repentance _ex opere
operato_, without a good disposition on the part of the one using it,
i.e., without faith in Christ, obtains grace.

X. That by the power of the keys our souls are freed from purgatory
through indulgences

XI. That in the reservation of cases not only canonical punishment,
but the guilt also, ought to be reserved in reference to one who is
truly converted.

In order, therefore, to deliver pious consciences from these
labyrinths of the sophists, we have ascribed to repentance [or
conversion] these two parts, namely, contrition and faith.  If any
one desires to add a third namely, fruits worthy of repentance, i.e.,
a change of the entire life and character for the better [good works
which shall and must follow conversion], we will not make any
opposition.  From contrition we separate those idle and infinite
discussions, as to when we grieve from love of God, and when from
fear of punishment.  [For these are nothing but mere words and a
useless babbling of persons who have never experienced the state of
mind of a terrified conscience.] But we say that contrition is the
true terror of conscience, which feels that God is angry with sin,
and which grieves that it has sinned.  And this contrition takes
place in this manner when sins are censured by the Word of God,
because the sum of the preaching of the Gospel is this, namely, to
convict of sin, and to offer for Christ's sake the remission of sins
and righteousness, and the Holy Ghost, and eternal life, and that as
regenerate men we should do good works.  Thus Christ comprises the
sum of the Gospel when He says in the last chapter of Luke, v. 74:
That repentance and remission of sins should be preached in My name
among all nations.  And of these terrors Scripture speaks, as Ps. 38,
4. 8: For mine iniquities are gone over mine head, as a heavy burden
they are too heavy for me...I am feeble and sore broken; I have
roared by reason of the disquietness of My heart.  And Ps. 6, 2. 3:
Have mercy upon me, O Lord; for I am weak; O Lord, heal me; for my
bones are vexed.  My soul is also sore vexed; but Thou, O Lord how
long!  And Is. 38, 10.13: I said in the cutting off of my days, I
shall go to the gates of the grave: I am deprived of the residue of
my years....I reckoned till morning that, as a lion, so will He break
all my bones.  [Again, v. 14: Mine eyes fail with looking upward; 0
Lord, I am oppressed.] In these terrors, conscience feels the wrath
of God against sin, which is unknown to secure men walking according
to the flesh [as the sophists and their like].  It sees the turpitude
of sin, and seriously grieves that it has sinned; meanwhile it also
flees from the dreadful wrath of God, because human nature, unless
sustained by the Word of God, cannot endure it.  Thus Paul says, Gal.
2, 19: I through the Law am dead to the Law, For the Law only accuses
and terrifies consciences.  In these terrors our adversaries say
nothing of faith, they present only the Word, which convicts of sin.
When this is taught alone, it is the doctrine of the Law, not of the
Gospel.  By these griefs and terrors, they say, men merit grace,
provided they love God.  But how will men love God in true terrors
when they feel the terrible and inexpressible wrath of God What else
than despair do those teach who in these terrors, display only the
Law?

We therefore add as the second part of repentance, Of Faith in Christ,
that in these terrors the Gospel concerning Christ ought to be set
forth to consciences, in which Gospel the remission of sins is freely
promised concerning Christ.  Therefore, they ought to believe that
for Christ's sake sins are freely remitted to them.  This faith
cheers, sustains, and quickens the contrite, according to Rom. 5, 1:
Being justified by faith, we have peace with God.  This faith obtains
the remission of sins.  This faith justifies before God, as the same
passage testifies: Being justified by faith.  This faith shows the
distinction between the contrition of Judas and Peter, of Saul and of
David.  The contrition of Judas or Saul is of no avail, for the
reason that to this there is not added this faith which apprehends
the remission of sins, bestowed as a gift for Christ's sake.
Accordingly, the contrition of David or Peter avails because to it
there is added faith, which apprehends the remission of sins granted
for Christ's sake.  Neither is love present before reconciliation has
been made by faith.  For without Christ the Law [God's Law or the
First Commandment] is not performed, according to [Eph. 2, 18; 3,12]
Rom. 5, 2: By Christ we have access to God.  And this faith grows
gradually and throughout the entire life, struggles with sin [is
tested by various temptations] in order to overcome sin and death.
But love follows faith, as we have said above.  And thus filial fear
can be clearly defined as such anxiety as has been connected with
faith, i.e., where faith consoles and sustains the anxious heart.  It
is servile fear when faith does not sustain the anxious heart [fear
without faith, where there is nothing but wrath and doubt].

Moreover, the power of the keys administers and presents the Gospel
through absolution, which [proclaims peace to me and] is the true
voice of the Gospel.  Thus we also comprise absolution when we speak
of faith, because faith cometh by hearing, as Paul says Rom. 10, 17.
For when the Gospel is heard and the absolution [i.e., the promise of
divine grace] is heard, the conscience is encouraged and receives
consolation.  And because God truly quickens through the Word, the
keys truly remit sins before God [here on earth sins are truly
canceled in such a manner that they are canceled also before God in
heaven] according to Luke 10,10: He that heareth you heareth Me
Wherefore the voice of the one absolving must be believed not
otherwise than we would believe a voice from heaven.  And absolution
[that blessed word of comfort] properly can be called a sacrament of
repentance, as also the more learned scholastic theologians speak.
Meanwhile this faith is nourished in a manifold way in temptations,
through the declarations of the Gospel [the hearing of sermons,
reading] and the use of the Sacraments.  For these are [seals and]
signs of [the covenant and grace in] the New Testament, i.e., signs
of [propitiation and] the remission of sins.  They offer, therefore,
the remission of sins, as the words of the Lord's Supper clearly
testify, Matt. 26, 26. 28: This is My body, which is given for you.
This is the cup of the New Testament, etc. Thus faith is conceived
and strengthened through absolution, through the hearing of the
Gospel, through the use of the Sacraments, so that it may not succumb
while it struggles with the terrors of sin and death.  This method of
repentance is plain and clear, and increases the worth of the power
of the keys and of the Sacraments, and illumines the benefit of
Christ, and teaches us to avail ourselves of Christ as Mediator and
Propitiator.

But as the Confutation condemns us for having assigned these two
parts to repentance, we must show that [not we, but] Scripture
expresses these as the chief parts in repentance or conversion.  For
Christ says Matt. 11, 28: Come unto Me, all ye that labor and are
heavy laden, and I will give you rest.  Here there are two members.
The labor and the burden signify the contrition, anxiety, and terrors
of sin and of death.  To come to Christ is to believe that sins are
remitted for Christ's sake, when we believe, our hearts are quickened
by the Holy Ghost through the Word of Christ.  Here, therefore, there
are these two chief parts, contrition and faith.  And in Mark 1, 15
Christ says: Repent ye and believe the Gospel, where in the first
member He convicts of sins, in the latter He consoles us, and shows
the remission of sins.  For to believe the Gospel is not that general
faith which devils also have [is not only to believe the history of
the Gospel], but in the proper sense it is to believe that the
remission of sins has been granted for Christ's sake.  For this is
revealed in the Gospel.  You see also here that the two parts are
joined, contrition when sins are reproved and faith, when it is said:
Believe the Gospel.  If any one should say here that Christ includes
also the fruits of repentance or the entire new life, we shall not
dissent.  For this suffices us, that contrition and faith are named
as the chief parts.

Paul almost everywhere, when he describes conversion or renewal,
designates these two parts, mortification and quickening, as in Col.
2, 11: In whom also ye are circumcised with the circumcision made
without hands, namely, by putting off the body of the sins of the
flesh.  And afterward, v. 12: Wherein also ye are risen with Him
through the faith of the operation of God.  Here are two parts.  [Of
these two parts he speaks plainly Rom. 6, 2. 4. 11, that we are dead
to sin, which takes place by contrition and its terrors, and that we
should rise again with Christ, which takes place when by faith we
again obtain consolation and life.  And since faith is to bring
consolation and peace into the conscience, according to Rom. 5, 1:
Being justified by faith, we have peace, it follows that there is
first terror and anxiety in the conscience.  Thus contrition and
faith go side by side.] One is putting off the body of sins; the
other is the rising again through faith.  Neither ought these words,
mortification, quickening, putting off the body of sins, rising again,
to be understood in a Platonic way, concerning a feigned change; but
mortification signifies true terrors, such as those of the dying,
which nature could not sustain unless it were supported by faith.  So
he names that as the putting off of the body of sins which we
ordinarily call contrition, because in these griefs the natural
concupiscence is purged away.  And quickening ought not to be
understood as a Platonic fancy, but as consolation which truly
sustains life that is escaping in contrition.  Here, therefore, are
two parts: contrition and faith.  For as conscience cannot be
pacified except by faith, therefore faith alone quickens, according
to the declaration, Hab. 2, 4; Rom. 1, 17: The just shall live by
faith.

And then in Col. 2, 14 it is said that Christ blots out the
handwriting which through the Law is against us.  Here also there are
two parts, the handwriting and the blotting out of the handwriting.
The handwriting, however, is conscience, convicting and condemning us.
The Law, moreover, is the word which reproves and condemns sins.
Therefore, this voice which says, I have sinned against the Lord, as
David says, 2 Sam. 12, 13, is the handwriting.  And wicked and secure
men do not seriously give forth this voice.  For they do not see,
they do not read the sentence of the Law written in the heart.  In
true griefs and terrors this sentence is perceived.  Therefore the
handwriting which condemns us is contrition itself.  To blot out the
handwriting is to expunge this sentence by which we declare that we
shall be condemned, and to engrave the sentence according to which we
know that we have been freed from this condemnation.  But faith is
the new sentence, which reverses the former sentence, and gives peace
and life to the heart.

However, what need is there to cite many testimonies since they are
everywhere obvious in the Scriptures?  Ps. 118, 18: The Lord hath
chastened me sore, but He hath not given me over unto death.  Ps. 119,
28: My soul melteth for heaviness; strengthen Thou me according unto
Thy word.  Here, in the first member, contrition is contained, and in
the second the mode is clearly described how in contrition we are
revived, namely, by the Word of God which offers grace.  This
sustains and quickens hearts.  And 1 Sam. 2, 6 The Lord killeth and
maketh alive; He bringeth down to the grave and bringeth up.  By one
of these, contrition is signified, by the other, faith is signified.
And Is. 28, 21: The Lord shall be wroth that He may do His work, His
strange work, and bring to pass His act, His strange act.  He calls
it the strange work of the Lord when He terrifies because to quicken
and console is God's own work.  [Other works, as, to terrify and to
kill, are not God's own works, for God only quickens.] But He
terrifies, he says, for this reason, namely, that there may be a
place for consolation and quickening, because hearts that are secure
and do not feel the wrath of God loathe consolation.  In this manner
Scripture is accustomed to join these two the terrors and the
consolation, in order to teach that in repentance there are these
chief members, contrition, and faith that consoles and justifies.
Neither do we see how the nature of repentance can be presented more
clearly and simply.  [We know with certainty that God thus works in
His Christians in the Church.]

For the two chief works of God in men are these, to terrify, and to
justify and quicken those who have been terrified.  Into these two
works all Scripture has been distributed.  The one part is the Law,
which shows, reproves, and condemns sins.  The other part is the
Gospel, i.e., the promise of grace bestowed in Christ, and this
promise is constantly repeated in the whole of Scripture, first
having been delivered to Adam [I will put enmity, etc., Gen. 3, 15],
afterwards to the patriarchs; then, still more clearly proclaimed by
the prophets; lastly, preached and set forth among the Jews by Christ
and disseminated over the entire world by the apostles.  For all the
saints were justified by faith in this promise, and not by their own
attrition or contrition.

And the examples [how the saints became godly] show likewise these
two parts.  After his sin Adam is reproved and becomes terrified,
this was contrition.  Afterward God promises grace, and speaks of a
future seed (the blessed seed, i.e., Christ), by which the kingdom of
the devil, death, and sin will be destroyed, there He offers the
remission of sins.  These are the chief things.  For although the
punishment is afterwards added, yet this punishment does not merit
the remission of sin.  And concerning this kind of punishment we
shall speak after a while.

So David is reproved by Nathan, and, terrified, he says, 2 Sam. 12,
13: I have sinned against the Lord.  This is contrition.  Afterward
he hears the absolution: The Lord also hath put away thy sin; thou
shalt not die.  This voice encourages David, and by faith sustains,
justifies, and quickens him.  Here a punishment is also added, but
this punishment does not merit the remission of sins.  Nor are
special punishments always added, but in repentance these two things
ought always to exist, namely, contrition and faith, as Luke 7, 37.
38. The woman, who was a sinner, came to Christ weeping.  By these
tears the contrition is recognized.  Afterward she hears the
absolution: Thy sins are forgiven; thy faith hath saved thee; go in
peace.  This is the second part of repentance, namely, faith, which
encourages and consoles her.  From all these it is apparent to godly
readers that we assign to repentance those parts which properly
belong to it in conversion, or regeneration, and the remission of sin.
Worthy fruits and punishments [likewise, patience that we be
willing to bear the cross and punishments, which God lays upon the
old Adam] follow regeneration and the remission of sin.  For this
reason we have mentioned these two parts, in order that the faith
which we require in repentance [of which the sophists and canonists
have all been silent] might be the better seen.  And what that faith
is which the Gospel proclaims can be better understood when it is set
over against contrition and mortification.

But as the adversaries expressly condemn our statement that men
obtain the remission of sins by faith, we shall add a few proofs from
which it will be understood that the remission of sins is obtained
not _ex opere operato_ because of contrition, but by that special
faith by which an individual believes that sins are remitted to him.
For this is the chief article concerning which we are contending with
our adversaries, and the knowledge of which we regard especially
necessary to all Christians.  As, however, it appears that we have
spoken sufficiently above concerning the same subject, we shall here
be briefer.  For very closely related are the topics of the doctrine
of repentance and the doctrine of justification.

When the adversaries speak of faith, and say that it precedes
repentance, they understand by faith, not that which justifies, but
that which, in a general way, believes that God exists, that
punishments have been threatened to the wicked [that there is a hell],
etc. In addition to this faith we require that each one believe that
his sins are remitted to him.  Concerning this special faith we are
disputing, and we oppose it to the opinion which bids us trust not in
the promise of Christ, but in the _opus operatum_, of contrition,
confession, and satisfactions, etc. This faith follows terrors in
such a manner as to overcome them, and render the conscience pacified.
To this faith we ascribe justification and regeneration, inasmuch
as it frees from terrors, and brings forth in the heart not only
peace and joy, but also a new life.  We maintain [with the help of
God we shall defend to eternity and against all the gates of hell]
that this faith is truly necessary for the remission of sins, and
accordingly place it among the parts of repentance.  Nor does the
Church of Christ believe otherwise, although our adversaries [like
mad dogs] contradict us.

Moreover, to begin with, we ask the adversaries whether to receive
absolution is a part of repentance, or not.  But if they separate it
from confession as they are subtile in making the distinction, we do
not see of what benefit confession is without absolution.  If,
however, they do not separate the receiving of absolution from
confession, it is necessary for them to hold that faith is a part of
repentance, because absolution is not received except by faith.  That
absolution, however is not received except by faith can be proved
from Paul, who teaches Rom. 4, 16, that the promise cannot be
received except by faith.  But absolution is the promise of the
remission of sins [nothing else than the Gospel, the divine promise
of God's grace and favor].  Therefore, it necessarily requires faith.
Neither do we see how he who does not assent to it may be said to
receive absolution.  And what else is the refusal to assent to
absolution but charging God with falsehood, If the heart doubts, it
regards those things which God promises as uncertain and of no
account.  Accordingly, in 1 John 5, 10 it is written: He that
believeth not God hath made Him a liar, because he believeth not the
record that God gave of His Son.

Secondly, we think that the adversaries acknowledge that the
remission of sins is either a part, or the end, or, to speak in their
manner, the _terminus ad quem_ of repentance.  [For what does
repentance help if the forgiveness of sins be not obtained?]
Therefore that by which the remission of sins is received is
correctly added to the parts [must certainly be the most prominent
part] of repentance.  It is very certain, however, that even though
all the gates of hell contradict us, yet the remission of sins cannot
be received except by faith alone, which believes that sins are
remitted for Christ's sake, according to Rom. 3, 25: Whom God hath
set forth to be a propitiation through faith in His blood.  Likewise
Rom. 5, 2: By whom also we have access by faith unto grace, etc. For
a terrified conscience cannot set against God's wrath our works or
our love, but it is at length pacified when it apprehends Christ as
Mediator, and believes the promises given for His sake.  For those
who dream that without faith in Christ hearts become pacified, do not
understand what the remission of sins is, or how it came to us.
Peter, 1 Ep. 2, 6, cites from Is. 49, 23, and 28, 16: He that
believeth on Him shall not be confounded.  It is necessary, therefore,
that hypocrites be confounded, who are confident that they receive
the remission of sins because of their own works, and not because of
Christ.  Peter also says in Acts 10, 43: To Him give all the prophets
witness that through His name whosoever believeth in Him, shall
receive remission of sins.  What he says, through His name, could not
be expressed more clearly and he adds: Whosoever believeth in Him.
Thus, therefore, we receive the remission of sins only through the
name of Christ, i.e., for Christ's sake, and not for the sake of any
merits and works of our own.  And this occurs when we believe that
sins are remitted to us for Christ's sake.

Our adversaries cry out that they are the Church, that they are
following the consensus of the Church [what the Church catholic
universal, holds].  But Peter also here cites in our issue the
consensus of the Church: To Him give all the prophets witness, that
through His name, whosoever believeth in Him, shall receive remission
of sins, etc. The consensus of the prophets is assuredly to be judged
as the consensus of the Church universal.  [I verily think that if
all the holy prophets are unanimously agreed in a declaration ( since
God regards even a single prophet as an inestimable treasure), it
would also be a decree, a declaration, and a unanimous strong
conclusion of the universal, catholic, Christian, holy Church, and
would be justly regarded as such.] We concede neither to the Pope nor
to the Church the power to make decrees against this consensus of the
prophets.  But the bull of Leo openly condemns this article, Of the
Remission of Sins and the adversaries condemn it in the Confutation.
From which it is apparent what sort of a Church we must judge that of
these men to be, who not only by their decrees censure the doctrine
that we obtain the remission of sins by faith, not on account of our
works, but on account of Christ, but who also give the command by
force and the sword to abolish it, and by every kind of cruelty [like
bloodhounds] to put to death good men who thus believe.

But they have authors of a great name Scotus, Gabriel, and the like,
and passages of the Fathers which are cited in a mutilated form in
the decrees.  Certainly, if the testimonies are to be counted, they
win.  For there is a very great crowd of most trifling writers upon
the Sententiae, who, as though they had conspired, defend these
figments concerning the merit of attrition and of works, and other
things which we have above recounted.  [Aye, it is true, they are all
called teachers and authors, but by their singing you can tell what
sort of birds they are.  These authors have taught nothing but
philosophy, and have known nothing of Christ and the work of God,
their books show this plainly.] But lest any one be moved by the
multitude of citations, there is no great weight in the testimonies
of the later writers, who did not originate their own writings, but
only, by compiling from the writers before them, transferred these
opinions from some books into others.  They have exercised no
judgment, but just like petty judges silently have approved the
errors of their superiors, which they have not understood.  Let us
not, therefore, hesitate to oppose this utterance of Peter, which
cites the consensus of the prophets, to ever so many legions of the
Sententiaries.  And to this utterance of Peter the testimony of the
Holy Ghost is added.  For the text speaks thus, Acts 10, 44: While
Peter yet spake these words, the Holy Ghost fell on all them which
heard the Word.  Therefore, let pious consciences know that the
command of God is this that they believe that they are freely
forgiven for Christ's sake, and not for the sake of our works.  And
by this command of God let them sustain themselves against despair,
and against the terrors of sin and of death.  And let them know that
this belief has existed among saints from the beginning of the world.
[Of this the idle sophists know little; and the blessed proclamation,
the Gospel, which proclaims the forgiveness of sins through the
blessed Seed, that is, Christ, has from the beginning of the world
been the greatest consolation and treasure to all pious kings all
prophets, all believers.  For they have believed in the same Christ
in whom we believe; for from the beginning of the world no saint has
been saved in any other way than through the faith of the same Gospel.
] For Peter clearly cites the consensus of the prophets, and the
writings of the apostles testify that they believe the same thing.
Nor are testimonies of the Fathers wanting.  For Bernard says the
same thing in words that are in no way obscure: For it is necessary
first of all to believe that you cannot have remission of sins except
by the indulgence of God, but add yet that you believe also this,
namely, that through Him sins are forgiven thee.  This is the
testimony which the Holy Ghost asserts in your heart, saying: "Thy
sins are forgiven thee." For thus the apostle judges that man is
justified freely through faith.  These words of Bernard shed a
wonderful light upon our cause, because he not only requires that we
in a general way believe that sins are remitted through mercy but he
bids us add special faith, by which we believe that sins are remitted
even to us; and he teaches how we may be rendered certain concerning
the remission of sins, namely when our hearts are encouraged by faith,
and become tranquil through the Holy Ghost.  What more do the
adversaries require?  [But how now, ye adversaries?  Is St. Bernard
also a heretic?] Do they still dare deny that by faith we obtain the
remission of sins, or that faith is a part of repentance?

Thirdly, the adversaries say that sin is remitted; because an attrite
or contrite person elicits an act of love to God [if we undertake
from reason to love God], and by this act merits to receive the
remission of sins.  This is nothing but to teach the Law, the Gospel
being blotted out, and the promise concerning Christ being abolished.
For they require only the Law and our works, because the Law demands
love.  Besides they teach us to be confident that we obtain remission
of sins because of contrition and love.  What else is this than to
put confidence in our works, not in the Word and promise of God
concerning Christ?  But if the Law be sufficient for obtaining the
remission of sins, what need is there of the Gospel?  What need is
there of Christ if we obtain remission of sins because of our own
work?  We, on the other hand call consciences away from the Law to
the Gospel, and from confidence in their own works to confidence in
the promise and Christ, because the Gospel presents to us Christ, and
promises freely the remission of sins for Christ's sake.  In this
promise it bids us trust, namely, that for Christ's sake we are
reconciled to the Father, and not for the sake of our own contrition
or love.  For there is no other Mediator or Propitiator than Christ.
Neither can we do the works of the Law unless we have first been
reconciled through Christ.  And if we would do anything, yet we must
believe that not for the sake of these works, but for the sake of
Christ, as Mediator and Propitiator, we obtain the remission of sins.

Yea, it is a reproach to Christ and a repeal of the Gospel to believe
that we obtain the remission of sins on account of the Law, or
otherwise than by faith in Christ.  This method also we have
discussed above in the chapter Of Justification, where we declared
why we confess that men are justified by faith, not by love.
Therefore the doctrine of the adversaries, when they teach that by
their own contrition and love men obtain the remission of sins, and
trust in this contrition and love, is merely the doctrine of the Law
and of that, too, as not understood [which they do not understand
with respect to the kind of love towards God which it demands], just
as the Jews looked upon the veiled face of Moses.  For let us imagine
that love is present, let us imagine that works are present, yet
neither love nor works can a propitiation for sin [or be of as much
value as Christ].  And they cannot even be opposed to the wrath and
judgment of God, according to Ps. 143, 2: Enter not into judgment
with Thy servant; for in Thy sight shall no man living be justified.
Neither ought the honor of Christ to be transferred to our works.

For these reasons Paul contends that we are not justified by the Law,
and he opposes to the Law the promise of the remission of sins which
is granted for Christ's sake and teaches that we freely receive the
remission of sins for Christ's sake.  Paul calls us away from the Law
to this promise.  Upon this promise he bids us look [and regard the
Lord Christ our treasure], which certainly will be void if we are
justified by the Law before we are justified through the promise, or
if we obtain the remission of sins on account of our own
righteousness.  But it is evident that the promise was given us and
Christ was tendered to us for the very reason that we cannot do the
works of the Law.  Therefore it is necessary that we are reconciled
by the promise before we do the works of the Law.  The promise,
however, is received only by faith.  Therefore it is necessary for
contrite persons to apprehend by faith the promise of the remission
of sins granted for Christ's sake, and to be confident that freely
for Christ's sake they have a reconciled Father.  This is the meaning
of Paul, Rom. 4, 13, where he says: Therefore it is of faith that it
might be by grace, to the end the promise might be sure.  And Gal. 3,
22: The Scripture hath concluded all under sin, that the promise by
faith of Jesus Christ might be given them that believe, i.e., all are
under sin, neither can they be freed otherwise than by apprehending
by faith the promise of the remission of sins.  Therefore we must by
faith accept the remission of sins before we do the works of the Law,
although, as has been said above, love follows faith, because the
regenerate receive the Holy Ghost, and accordingly begin [to become
friendly to the Law and] to do the works of the Law.

We would cite more testimonies if they were not obvious to every
godly reader in the Scriptures.  And we do not wish to be too prolix,
in order that this ease may be the more readily seen through.
Neither, indeed, is there any doubt that the meaning of Paul is what
we are defending, namely, that by faith we receive the remission of
sins for Christ's sake, that by faith we ought to oppose to God's
wrath Christ as Mediator, and not our works.  Neither let godly minds
be disturbed, even though the adversaries find fault with the
judgments of Paul.  Nothing is said so simply that it cannot be
distorted by caviling.  We know that what we have mentioned is the
true and genuine meaning of Paul, we know that this our belief brings
to godly consciences [in agony of death and temptation] sure comfort,
without which no one can in God's judgment.

Therefore let these pharisaic opinions of the adversaries be rejected,
namely, that we do not receive by faith the remission of sins, but
that it ought to be merited by our love and works; that we ought to
oppose our love and our works to the wrath of God.  Not of the Gospel,
but of the Law is this doctrine, which feigns that man is justified
by the Law before he has been reconciled through Christ to God, since
Christ says, John 15, 5: With out Me, ye can do nothing; likewise: I
am the true Vine; ye are the branches.  But the adversaries feign
that we are branches, not of Christ, but of Moses.  For they wish to
be justified by the Law, and to offer their love and works to God
before they are reconciled to God through Christ, before they are
branches of Christ.  Paul, on the other hand [who is certainly a much
greater teacher than the adversaries], contends that the Law cannot
be observed without Christ.  Accordingly, in order that we [those who
truly feel and have experienced sin and anguish of conscience must
cling to the promise of grace, in order that they] may be reconciled
to God for Christ's sake, the promise must be received before we do
the works of the Law.  We think that these things are sufficiently
clear to godly consciences.  And hence they will understand why we
have declared above that men are justified by faith, not by love,
because we must oppose to God's wrath not our love or works (or trust
in our love and works), but Christ as Mediator [for all our ability,
all our deeds and works, are far too weak to remove and appease God's
wrath].  And we must apprehend the promise of the remission of sins
before we do the works of the Law.

Lastly, when will conscience be pacified if we receive remission of
sins on the ground that we love, or that we do the works of the Law?
For the Law will always accuse us, because we never satisfy God's Law.
Just as Paul says, Rom. 4, 15: The Law worketh wrath.  Chrysostom
asks concerning repentance, Whence are we made sure that our sins are
remitted us?  The adversaries also, in their "Sentences," ask
concerning the same subject.  [The question, verily, is worth asking
blessed the man that returns the right answer.] This cannot be
explained, consciences cannot be made tranquil, unless they know that
it is God's command and the very Gospel that they should be firmly
confident that for Christ's sake sins are remitted freely, and that
they should not doubt that these are remitted to them.  If any one
doubts, he charges, as John says, 1 Ep. 5, 10, the divine promise
with falsehood.  We teach that this certainty of faith is required in
the Gospel.  The adversaries leave consciences uncertain and wavering.
Consciences, however do nothing from faith when they perpetually
doubt whether they have remission.  [For it is not possible that
there should be rest, or a quiet and peaceful conscience, if they
doubt whether God be gracious.  For if they doubt whether they have a
gracious God, whether they are doing right, whether they have
forgiveness of sins, how can, etc.] How can they in this doubt call
upon God, how can they be confident that they are heard?  Thus the
entire life is without God [faith] and without the true worship of
God.  This is what Paul says, Rom. 14, 23: Whatsoever is not of faith
is sin.  And because they are constantly occupied with this doubt,
they never experience what faith [God or Christ] is.  Thus it comes
to pass that they rush at last into despair [die in doubt, without
God, without all knowledge of God].  Such is the doctrine of the
adversaries, the doctrine of the Law, the annulling of the Gospel,
the doctrine of despair.  [Whereby Christ is suppressed, men are led
into overwhelming sorrow and torture of conscience, and finally, when
temptation comes, into despair.  Let His Imperial Majesty graciously
consider and well examine this matter, it does not concern gold or
silver but souls and consciences.] Now we are glad to refer to all
good men the judgment concerning this topic of repentance (for it has
no obscurity), in order that they may decide whether we or the
adversaries have taught those things which are more godly and
healthful to consciences.  Indeed, these dissensions in the Church do
not delight us; wherefore, if we did not have great and necessary
reasons for dissenting from the adversaries, we would with the
greatest pleasure be silent.  But now, since they condemn the
manifest truth, it is not right for us to desert a cause which is not
our own, but is that of Christ and the Church.  [We cannot with
fidelity to God and conscience deny this blessed doctrine and divine
truth, from which we expect at last, when this poor temporal life
ceases and all help of creatures fails, the only eternal, highest
consolation: nor will we in anything recede from this cause, which is
not only ours, but that of all Christendom, and concerns the highest
treasure, Jesus Christ.]

We have declared for what reasons we assigned to repentance these two
parts, contrition and faith.  And we have done this the more readily
because many expressions concerning repentance are published which
are cited in a mutilated form from the Fathers [Augustine and the
other ancient Fathers], and which the adversaries have distorted in
order to put faith out of sight.  Such are: Repentance is to lament
past evils, and not to commit again deeds that ought to be lamented.
Again: Repentance is a kind of vengeance of him who grieves, thus
punishing in himself what he is sorry for having committed.  In these
passages no mention is made of faith.  And not even in the schools,
when they interpret, is anything added concerning faith.  Therefore,
in order that the doctrine of faith might be the more conspicuous, we
have enumerated it among the parts of repentance.  For the actual
fact shows that those passages which require contrition or good works,
and make no mention of justifying faith, are dangerous [as
experience proves].  And prudence can justly be desired in those who
have collected these centos of the "Sentences" and decrees.  For
since the Fathers speak in some places concerning one part, and in
other places concerning another part of repentance, it would have
been well to select and combine their judgments not only concerning
one part, but concerning both, i.e., concerning contrition and faith.

For Tertullian speaks excellently concerning faith, dwelling upon the
oath in the prophet, Ezek. 33, 11: As I live, saith the Lord God, I
have no pleasure in the death of the wicked, but that the wicked turn
from his way and live.  For as God swears that He does not wish the
death of a sinner, He shows that faith is required, in order that we
may believe the one swearing, and be firmly confident that He
forgives us.  The authority of the divine promises ought by itself to
be great in our estimation.  But this promise has also been confirmed
by an oath.  Therefore, if any one be not confident that he is
forgiven, he denies that God has sworn what is true, than which a
more horrible blasphemy cannot be imagined.  For Tertullian speaks
thus: He invites by reward to salvation, even swearing.  Saying, "I
live," He desires that He be believed.  Oh, blessed we, for whose
sake God swears!  Oh, most miserable if we believe not the Lord even
when He swears!  But here we must know that this faith ought to be
confident that God freely forgives us for the sake of Christ, for the
sake of His own promise, not for the sake of our works, contrition,
confession, or satisfactions.  For if faith relies upon these works,
it immediately becomes uncertain, because the terrified conscience
sees that these works are unworthy.  Accordingly, Ambrose speaks
admirably concerning repentance: Therefore it is proper for us to
believe both that we are to repent, and that we are to be pardoned,
but so as to expect pardon as from faith, which obtains it as from a
handwriting.  Again: It is faith which covers our sins.  Therefore
there are sentences extant in the Fathers, not only concerning
contrition and works, but also concerning faith.  But the adversaries,
since they understand neither the nature of repentance nor the
language of the Fathers, select passages concerning a part of
repentance, namely, concerning works; they pass over the declarations
made elsewhere concerning faith, since they do not understand them.




Part 16


Article VI: _Of Confession and Satisfaction._

Good men can easily judge that it is of the greatest importance that
the true doctrine concerning the abovementioned parts, namely,
contrition and faith, be preserved.  [For the great fraud of
indulgences, etc., and the preposterous doctrines of the sophists
have sufficiently taught us what great vexation and danger arise
therefrom if a foul stroke is here made.  How many a godly conscience
under the Papacy sought with great labor the true way, and in the
midst of such darkness did not find it!] Therefore, we have always
been occupied more with the elucidation of these topics, and have
disputed nothing as yet concerning confession and satisfaction.  For
we also retain confession, especially on account of the absolution,
as being the word of God which, by divine authority, the power of the
keys pronounces upon individuals.  Therefore it would be wicked to
remove private absolution from the Church.  Neither do they
understand what the remission of sins or the power of the keys is, if
there are any who despise private absolution.  But in reference to
the enumeration of offenses in confession, we have said above that we
hold that it is not necessary by divine right.  For the objection,
made by some, that a judge ought to investigate a ease before he
pronounces upon it, pertains in no way to this subject; because the
ministry of absolution is favor or grace, it is not a legal process,
or law.  [For God is the Judge, who has committed to the apostles,
not the office of judges, but the administration of grace namely, to
acquit those who desire, etc.] Therefore ministers in the Church have
the command to remit sin, they have not the command to investigate
secret sins.  And indeed, they absolve from those that we do not
remember; for which reason absolution, which is the voice of the
Gospel remitting sins and consoling consciences, does not require
judicial examination.

And it is ridiculous to transfer hither the saying of Solomon, Prov.
27, 23: Be thou diligent to know the state of thy flocks.  For
Solomon says nothing of confession, but gives to the father of a
family a domestic precept, that he should use what is his own, and
abstain from what is another's, and he commands him to take care of
his own property diligently, yet in such a way that, with his mind
occupied with the increase of his resources, he should not cast away
the fear of God, or faith or care in God's Word.  But our adversaries,
by a wonderful metamorphosis, transform passages of Scripture to
whatever meaning they please.  [They produce from the Scriptures
black and white, as they please, contrary to the natural meaning of
the clear words.] Here to know signifies with them to hear
confessions, the state, not the outward life, but the secrets of
conscience; and the flocks signify men.  [Sable, we think means a
school within which there are such doctors and orators.  But it has
happened aright to those who thus despise the Holy Scriptures and all
fine arts that they make gross mistakes in grammar.] The
interpretation is assuredly neat, and is worthy of these despisers of
the pursuits of eloquence.  But if any one desires by a similitude to
transfer a precept from a father of a family to a pastor of a Church,
he ought certainly to interpret "state" [V. _vultus_, countenance] as
applying to the outward life.  This similitude will be more
consistent.

But let us omit such matters as these.  At different times in the
Psalms mention is made of confession, as, Ps. 32, 5: I said, I will
confess my transgressions unto the Lord; and Thou forgavest the
iniquity of my sin.  Such confession of sin which is made to God is
contrition itself.  For when confession is made to God, it must be
made with the heart not alone with the voice, as is made on the stage
by actors.  Therefore, such confession is contrition, in which,
feeling God's wrath, we confess that God is justly angry, and that He
cannot be appeased by our works, and nevertheless we seek for mercy
because of God's promise.  Such is the following confession, Ps. 51,
4: Against Thee only have I sinned, that Thou mightest be justified
and be clear when Thou judgest, i.e., "I confess that I am a sinner,
and have merited eternal wrath, nor can I set my righteousnesses, my
merits, against Thy wrath; accordingly, I declare that Thou art just
when Thou condemnest and punishest us, I declare that Thou art clear
when hypocrites judge Thee to be unjust in punishing them or in
condemning the well-deserving.  Yea, our merits cannot be opposed to
Thy judgment but we shall thus be justified, namely, if Thou
justifiest us, if through Thy mercy Thou accountest us righteous."
Perhaps some one may also cite Jas. 5, 16: Confess your faults one to
another.  But here the reference is not to confession that is to be
made to the priests, but, in general, concerning the reconciliation
of brethren to each other.  For it commands that the confession be
mutual.

Again, our adversaries will condemn many most generally received
teachers if they will contend that in confession an enumeration of
offenses is necessary according to divine Law.  For although we
approve of confession, and judge that some examination is of
advantage in order that men may be the better instructed [young and
inexperienced persons be questioned], yet the matter must be so
controlled that snares are not cast upon consciences, which never
will be tranquil if they think that they cannot obtain the remission
of sins unless this precise enumeration be made.  That which the
adversaries have expressed in the _Confutation_ is certainly most
false, namely, that a full confession is necessary for salvation.
For this is impossible.  And what snares they here cast upon the
conscience when they require a full confession!  For when will
conscience be sure that the confession is complete?  In the
Church-writers mention is made of confession, but they do not speak
of this enumeration of secret offenses, but of the rite of public
repentance.  For as the fallen or notorious [those guilty of public
crimes] were not received without fixed satisfactions [without a
public ceremony or reproof], they made confession on this account to
the presbyters, in order that satisfactions might be prescribed to
them according to the measure of their offenses.  This entire matter
contained nothing similar to the enumeration concerning which we are
disputing.  This confession was made, not because the remission of
sins before God could not occur without it, but because satisfactions
could not be prescribed unless the kind of offense were first known.
For different offenses had different canons.

And from this rite of public repentance there has been left the word
"satisfaction." For the holy Fathers were unwilling to receive the
fallen or the notorious, unless as far as it was possible, their
repentance had been first examined into and exhibited publicly.  And
there seem to have been many causes for this.  For to chastise those
who had fallen served as an example, just as also the gloss upon the
degrees admonishes, and it was improper immediately to admit
notorious men to the communion [without their being tested].  These
customs have long since grown obsolete.  Neither is it necessary to
restore them, because they are not necessary for the remission of
sins before God.  Neither did the Fathers hold this, namely, that men
merit the remission of sins through such customs or such works,
although these spectacles [such outward ceremonies] usually lead
astray the ignorant to think that by these works they merit the
remission of sins before God.  But if any one thus holds, he holds to
the faith of a Jew and heathen.  For also the heathen had certain
expiations for offenses through which they imagined to be reconciled
to God.  Now, however, although the custom has become obsolete, the
name satisfaction still remains, and a trace of the custom also
remains of prescribing in confession certain satisfactions, which
they define as works that are not due.  We call them canonical
satisfactions.  Of these we hold, just as of the enumeration, that
canonical satisfactions [these public ceremonies] are not necessary
by divine Law for the remission of sins, just as those ancient
exhibitions of satisfactions in public repentance were not necessary
by divine Law for the remission of sins.  For the belief concerning
faith must be retained, that by faith we obtain remission of sins for
Christ's sake, and not for the sake of our works that precede or
follow [when we are converted or born anew in Christ].  And for this
reason we have discussed especially the question of satisfactions,
that by submitting to them the righteousness of faith be not obscured,
or men think that for the sake of these works they obtain remission
of sins.  And many sayings that are current in the schools aid the
error, such as that which they give in the definition of satisfaction,
namely, that it is wrought for the purpose of appeasing the divine
displeasure.

But, nevertheless, the adversaries acknowledge that satisfactions are
of no profit for the remission of guilt.  Yet they imagine that
satisfactions are of profit in redeeming from the punishments,
whether of purgatory or other punishments.  For thus they teach that
in the remission of sins, God [without means, alone] remits the guilt,
and yet, because it belongs to divine justice to punish sin, that He
commutes eternal into temporal punishment.  They add further that a
part of this temporal punishment is remitted by the power of the keys,
but that the rest is redeemed by means of satisfactions.  Neither
can it be understood of what punishments a part is remitted by the
power of the keys, unless they say that a part of the punishments of
purgatory is remitted, from which it would follow that satisfactions
are only punishments redeeming from purgatory.  And these
satisfactions, they say, avail even though they are rendered by those
who have relapsed into mortal sin, as though indeed the divine
displeasure could be appeased by those who are in mortal sin.  This
entire matter is fictitious, and recently fabricated without the
authority of Scripture and the old writers of the Church.  And not
even Longobardus speaks in this way of satisfactions.  The
scholastics saw that there were satisfactions in the Church; and they
did not notice that these exhibitions had been instituted both for
the purpose of example, and for testing those who desired to be
received by the Church.  In a word, they did not see that it was a
discipline, and entirely a secular matter.  Accordingly, they
superstitiously imagined that these avail not for discipline before
the Church, but for appeasing God.  And just as in other places they
frequently, with great inaptness, have confounded spiritual and civil
matters [the kingdom of Christ, which is spiritual, and the kingdom
of the world, and external discipline], the same happens also with
regard to satisfactions.  But the gloss on the canons at various
places testifies that these observances were instituted for the sake
of church discipline [should serve alone for an example before the
Church].

Let us see, moreover, how in the Confutation which they had the
presumption to obtrude upon His Imperial Majesty, they prove these
figments of theirs.  They cite many passages from the Scriptures, in
order to impose upon the inexperienced, as though this subject which
was unknown even in the time of Longobard, had authority from the
Scriptures.  They bring forward such passages as these: Bring forth,
therefore, fruits meet for repentance, Matt. 3, 8, Mark 1, 15. Again:
Yield your members servants to righteousness Rom. 6, 19. Again,
Christ preaches repentance, Matt. 4, 17: Repent.  Again, Christ Luke
24, 47, commands the apostles to preach repentance, and Peter
preaches repentance Acts 2, 38. Afterward they cite certain passages
of the Fathers and the canons, and conclude that satisfactions in the
Church are not to be abolished contrary to the plain Gospel and the
decrees of the Councils and Fathers [against the decision of the Holy
Church]; nay, even that those who have been absolved by the priest
ought to bring to perfection the repentance that has been enjoined,
following the declaration of Paul, Titus 2, 14: Who gave Himself for
us that He might redeem us from all iniquity, and purify unto Himself
a peculiar people, zealous of good works.

May God put to confusion these godless sophists who so wickedly
distort God's Word to their own most vain dreams!  What good man is
there who is not moved by such indignity?" Christ says, Repent, the
apostles preach repentance; therefore eternal punishments are
compensated by the punishments of purgatory; therefore the keys have
the power to remit part of the punishments of purgatory; therefore
satisfactions redeem the punishments of purgatory"!  Who has taught
these asses such logic?  Yet this is neither logic nor sophistry, but
cunning trickery.  Accordingly, they appeal to the expression repent
in such a way that, when the inexperienced hear such a passage cited
against us they may derive the opinion that we deny the entire
repentance.  By these arts they endeavor to alienate minds and to
enkindle hatred, so that the inexperienced may cry out against us
[Crucify! crucify!], that such pestilent heretics as disapprove of
repentance should be removed from their midst.  [Thus they are
publicly convicted of being liars in this matter.]

But we hope that among good men these calumnies [and
misrepresentations of Holy Scripture] may make little headway.  And
God will not long endure such impudence and wickedness.  [They will
certainly be consumed by the First and Second Commandments.] Neither
has the Pope of Rome consulted well for his own dignity in employing
such patrons, because he has entrusted a matter of the greatest
importance to the judgment of these sophists.  For since we include
in the Confession almost the sum of the entire Christian doctrine,
judges should have been appointed to make a declaration concerning
matters so important and so many and various, whose learning and
faith would have been more approved than that of these sophists who
have written this Confutation.  It was particularly becoming for you,
O Campegius, in accordance with your wisdom, to have taken care that
in regard to matters of such importance they should write nothing
which either at this time or with posterity might seem to be able to
diminish regard for the Roman See.  If the Roman See judges it right
that all nations should acknowledge her as mistress of the faith, she
ought to take pains that learned and uncorrupt men make investigation
concerning matters of religion.  For what will the world judge if at
any time the writing of the adversaries be brought to light?  What
will posterity judge concerning these reproachful judicial
investigations?  You see, O Campegius, that these are the last times,
in which Christ predicted that there would be the greatest danger to
religion.  You, therefore, who ought, as it were, to sit on the
watch-tower and control religious matters, should in these times
employ unusual wisdom and diligence.  There are many signs which,
unless you heed them, threaten a change to the Roman state.  And you
make a mistake if you think that Churches should be retained only by
force and arms.  Men ask to be taught concerning religion.  How many
do you suppose there are, not only in Germany, but also in England,
in Spain, in France, in Italy, and finally even in the city of Rome,
who, since they see that controversies have arisen concerning of the
greatest importance, are beginning here and there to doubt, and to be
silently indignant that you refuse to investigate and judge aright
subjects of such weight as these; that you do not deliver wavering
consciences; that you only bid us be overthrown and annihilated by
arms?  There are many good men to whom this doubt is more bitter than
death.  You do not consider sufficiently how great a subject religion
is, if you think that good men are in anguish for a slight cause
whenever they begin to doubt concerning any dogma.  And this doubt
can have no other effect than to produce the greatest bitterness of
hatred against those who, when they ought to heal consciences, plant
themselves in the way of the explanation of the subject.  We do not
here say that you ought to fear God's judgment.  For the hierarchs
think that they can easily provide against this, for since they hold
the keys, of course they can open heaven for themselves whenever they
wish.  We are speaking of the judgments of men and the silent desires
of all nations, which, indeed, at this time require that these
matters be investigated and decided in such a manner that good minds
may be healed and freed from doubt.  For, in accordance with your
wisdom, you can easily decide what will take place if at any time
this hatred against you should break forth.  But by this favor you
will be able to bind to yourself all nations, as all sane men regard
it as the highest and most important matter, if you heal doubting
consciences.  We have said these things not because we doubt
concerning our Confession.  For we know that it is true, godly, and
useful to godly consciences.  But it is likely that there are many in
many places who waver concerning matters of no light importance, and
yet do not hear such teachers as are able to heal their consciences.

But let us return to the main point.  The Scriptures cited by the
adversaries speak in no way of canonical satisfactions, and of the
opinions of the scholastics, since it is evident that the latter were
only recently born.  Therefore it is pure slander when they distort
Scripture to their own opinions.  We say that good fruits, good works
in every kind of life, ought to follow repentance, i.e., conversion
or regeneration [the renewal of the Holy Ghost in the heart].
Neither can there be true conversion or true contrition where
mortifications of the flesh and good fruits do not follow [if we do
not externally render good works and Christian patience].  True
terrors, true griefs of mind, do not allow the body to indulge in
sensual pleasures, and true faith is not ungrateful to God, neither
does it despise God's commandments.  In a word, there is no inner
repentance unless it also produces outwardly mortifications of the
flesh.  We say also that this is the meaning of John when he says,
Matt. 3, 8: Bring forth, therefore, fruits meet for repentance.
Likewise of Paul when he says Rom. 6, 19: Yield your members servants
to righteousness; just as he likewise says elsewhere, Rom. 12, 1:
Present your bodies a living sacrifice, etc. And when Christ says
Matt. 4, 17: Repent, He certainly speaks of the entire repentance, of
the entire newness of life and its fruits, He does not speak of those
hypocritical satisfactions which, the scholastics avail for
compensating the punishment of purgatory or other punishments when
they are made by those who are in mortal sin.

Many arguments, likewise, can be collected to show that these
passages of Scripture pertain in no way to scholastic satisfactions.
These men imagine that satisfactions are works that are not due
[which we are not obliged to do]; but Scripture, in these passages,
requires works that are due [which we are obliged to do].  For this
word of Christ, Repent, is the word of a commandment.  Likewise the
adversaries write that if any one who goes to confession should
refuse to undertake satisfactions, he does not sin, but will pay
these penalties in purgatory.  Now the following passages are,
without controversy, precepts pertaining to this life: Repent; Bring
forth fruits meet for repentance; Yield your members servants to
righteousness.  Therefore they cannot be distorted to the
satisfactions which it is permitted to refuse.  For to refuse God's
commandments is not permitted.  [For God's commands are not thus left
to our discretion.] Thirdly, indulgences remit these satisfactions,
as is taught by the Chapter, _De Poenitentiis et Remissione_,
beginning _Quum ex eo_, etc. But indulgences do not free us from the
commandments: Repent; Bring forth fruits meet for repentance.
Therefore it is manifest that these passages of Scripture have been
wickedly distorted to apply to canonical satisfactions.  See further
what follows.  If the punishments of purgatory are satisfactions, or
satispassions [sufferings sufficient], or if satisfactions are a
redemption of the punishments of purgatory, do these passages also
give commandment that souls be punished in purgatory?  [The
above-cited passages of Christ and Paul must also show and prove that
souls enter purgatory and there suffer pain.] Since this must follow
from the opinions of the adversaries, these passages should be
interpreted in a new way [these passages should put on new coats]:
Bring forth fruits meet for repentance; Repent, i.e., suffer the
punishments of purgatory after this life.  But we do not care about
refuting in more words these absurdities of the adversaries.  For it
is evident that Scripture speaks of works that are due, of the entire
newness of life, and not of these observances of works that are not
due, of which the adversaries speak.  And yet, by these figments they
defend orders [of monks], the sale of Masses and infinite observances,
namely, as works which, if they do not make satisfaction for guilt,
yet make satisfaction for punishment.

Since, therefore, the passages of Scripture cited do not say that
eternal punishments are to be compensated by works that are not due,
the adversaries are rash in affirming that these satisfactions are
compensated by canonical satisfactions.  Nor do the keys have the
command to commute some punishments, and likewise to remit a part of
the punishments.  For where are such things [dreams and lies] read in
the Scriptures?  Christ speaks of the remission of sins when He says
Matt. 18, 18: Whatsoever ye shall loose, etc. [i.e.], sin being
forgiven, death eternal is taken away, and life eternal bestowed.
Nor does Whatsoever ye shall bind speak of the imposing of
punishments, but of retaining the sins of those who are not converted.
Moreover, the declaration of Longobard concerning remitting a part
of the punishments has been taken from the canonical punishments; a
part of these the pastors remitted.  Although, we hold that
repentance ought to bring forth good fruits for the sake of God's
glory and command, and good fruits, true fastings, true prayers, true
alms, etc., have the commands of God, yet in the Holy Scriptures we
nowhere find this, namely, that eternal punishments are not remitted
except on account of the punishment of purgatory or canonical
satisfactions, i.e., on account of certain works not due, or that the
power of the keys has the command to commute their punishments or to
remit a portion.  These things the adversaries were to prove.  [This
they will not attempt.]

Besides, the death of Christ is a satisfaction not only for guilt,
but also for eternal death, according to Hos. 13, 14: 0 death, I will
be thy death.  How monstrous, therefore, it is to say that the
satisfaction of Christ redeemed from the guilt, and our punishments
redeem from eternal death, as the expression, I will be thy death,
ought then to be understood, not concerning Christ, but concerning
our works, and, indeed, not concerning the works commanded by God,
but concerning some frigid observances devised by men!  And these are
said to abolish death, even when they are wrought in mortal sin.  It
is incredible with what grief we recite these absurdities of the
adversaries, which cannot but cause one who considers them to be
enraged against such doctrines of demons, which the devil has spread
in the Church in order to suppress the knowledge of the Law and
Gospel, of repentance and quickening, and the benefits of Christ.
For of the Law they speak thus: "God, condescending to our weakness,
has given to man a measure of those things to which of necessity he
is bound and this is the observance of precepts, so that from what is
left, i.e., from works of supererogation, he can render satisfaction
with reference to offenses that have been committed." Here men
imagine that they can observe the Law of God in such a manner as to
be able to do even more than the Law exacts.  But Scripture
everywhere exclaims that we are far distant from the perfection which
the Law requires.  Yet these men imagine that the Law of God has been
comprised in outward and civil righteousness; they do not see that it
requires true love to God "with the whole heart," etc., and condemns
the entire concupiscence in the nature.  Therefore no one does as
much as the Law requires.  Hence their imagination that we can do
more is ridiculous.  For although we can perform outward works not
commanded by God's Law [which Paul calls beggarly ordinances], yet
the confidence that satisfaction is rendered God's Law [yea, that
more is done than God demands] is vain and wicked.  And true prayers,
true alms, true fastings, have God's command; and where they have
God's command, they cannot without sin be omitted.  But these works,
in so far as they have not been commanded by God's Law, but have a
fixed form derived from human rule are works of human traditions of
which Christ says, Matt. 15, 9: In vain they do worship Me with the
commandments of men, such as certain fasts appointed not for
restraining the flesh, but that, by this work, honor may be given to
God, as Scotus says, and eternal death be made up for; likewise, a
fixed number of prayers, a fixed measure of alms when they are
rendered in such a way that this measure is a worship _ex opere
operato_ giving honor to God, and making up for eternal death.  For
they ascribe satisfaction to these _ex opere operato_, because they
teach that they avail even in those who are in mortal sin.  There are
works which depart still farther from God's commands, as [rosaries
and] pilgrimages; and of these there is a great variety: one makes a
journey [to St. Jacob] clad in mail, and another with bare feet.
Christ calls these "vain acts of worship," and hence they do not
serve to appease God's displeasure, as the adversaries say.  And yet
they adorn these works with magnificent titles; they call them works
of supererogation, to them the honor is ascribed of being a price
paid instead of eternal death.  Thus they are preferred to the works
of God's commandments [the true works expressly mentioned in the Ten
Commandments].  In this way the Law of God is obscured in two ways,
one, because satisfaction is thought to be rendered God's Law by
means of outward and civil works, the other, because human traditions
are added whose works are preferred to the works of the divine Law.




Part 17


In the second place, repentance and grace are obscured.  For eternal
death is not atoned for by this compensation of works because it is
idle, and does not in the present life taste of death.  Something
else must be opposed to death when it tries us.  For just as the
wrath of God is overcome by faith in Christ, so death is overcome by
faith in Christ.  Just as Paul says, 1 Cor. 16, 67: But thanks be to
God which giveth us the victory through our Lord Jesus Christ.  He
does not say: "Who giveth us the victory if we oppose our
satisfactions against death." The adversaries treat of idle
speculations concerning the remission of guilt, and do not see how in
the remission of guilt, the heart is freed by faith in Christ from
God's anger and eternal death.  Since, therefore, the death of Christ
is a satisfaction for eternal death, and since the adversaries
themselves confess that these works of satisfactions are works that
are not due, but are works of human traditions, of which Christ says,
Matt. 16, 9, that they are vain acts of worship, we can safely affirm
that canonical satisfactions are not necessary by divine Law for the
remission of guilt, or eternal punishment, or the punishment of
purgatory.

But the adversaries object that vengeance or punishment is necessary
for repentance, because Augustine says that repentance is vengeance
punishing, etc..  We grant that vengeance or punishment is necessary
in repentance, yet not as merit or price, as the adversaries imagine
that satisfactions are.  But vengeance is in repentance formally, i.e.,
because regeneration itself occurs by a perpetual mortification of
the oldness of life.  The saying of Scotus may indeed be very
beautiful, that _poenitentia_ is so called because it is, as it were,
_poenae tenentia_, holding to punishment.  But of what punishment, of
what vengeance, does Augustine speak?  Certainly of true punishment,
of true vengeance, namely, of contrition, of true terrors.  Nor do we
here exclude the outward mortifications of the body, which follow
true grief of mind.  The adversaries make a great mistake if they
imagine that canonical satisfactions [their juggler's tricks,
rosaries, pilgrimages, and such like] are more truly punishments than
are true terrors in the heart.  It is most foolish to distort the
name of punishment to these frigid satisfactions, and not to refer
them to those horrible terrors of conscience of which David says, Ps.
18, 4; 2 Sam. 22, 5: The sorrows of death compassed me.  Who would
not rather, clad in mail and equipped, seek the church of James, the
basilica of Peter, etc., than bear that ineffable violence of grief
which exists even in persons of ordinary lives, if there be true
repentance?

But they say that it belongs to God's justice to punish sin.  He
certainly punishes it in contrition, when in these terrors He shows
His wrath.  Just as David indicates when he prays, Ps. 6, 1: 0 Lord,
rebuke me not in Thine anger.  And Jeremiah, 10, 24: 0 Lord, correct
me, but with judgment; not in Thine anger, lest Thou bring me to
nothing.  Here indeed the most bitter punishments are spoken of.  And
the adversaries acknowledge that contrition can be so great that
satisfaction is not required.  Contrition is therefore more truly a
punishment than is satisfaction.  Besides, saints are subject to
death, and all general afflictions, as Peter says, 1 Ep. 4, 17: For
the time is come that judgment must begin at the house of God; and if
it first begin at us, what shall the end be of them that obey not the
Gospel of God?  And although these afflictions are for the most part
the punishments of sin, yet in the godly they have a better end,
namely, to exercise them, that they may learn amidst trials to seek
God's aid, to acknowledge the distrust of their own hearts, etc., as
Paul says of himself, 2 Cor. 1, 9: But we had the sentence of death
in ourselves, that we should not trust in ourselves, but in God which
raiseth the dead.  And Isaiah says, 26, 16: They poured out prayer
when Thy chastening was upon them i.e., afflictions are a discipline
by which God exercises the saints.  Likewise afflictions are
inflicted because of present sin, since in the saints they mortify
and extinguish concupiscence, so that they may be renewed by the
Spirit, as Paul says, Rom. 8, 10: The body is dead because of sin, i.
e., it is mortified [more and more every day] because of present sin
which is still left in the flesh.  And death itself serves this
purpose, namely, to abolish this flesh of sin, that we may rise
absolutely new.  Neither is there now in the death of the believer,
since by faith he has overcome the terrors of death, that sting and
sense of wrath of which Paul speaks 1 Cor. 15, 56: The sting of death
is sin; and the strength of sin is the Law.  This strength of sin,
this sense of wrath, is truly a punishment as long as it is present;
without this sense of wrath, death is not properly a punishment.
Moreover, canonical satisfactions do not belong to these punishments;
as the adversaries say that by the power of the keys a part of the
punishments is remitted.  Likewise, according to these very men, the
keys remit the satisfactions, and the punishments on account of which
the satisfactions are made.  But it is evident that the common
afflictions are not removed by the power of the keys.  And if they
wish to be understood concerning punishments, why do they add that
satisfaction is to be rendered in purgatory?

They oppose the example of Adam, and also of David, who was punished
for his adultery.  From these examples they derive the universal rule
that peculiar temporal punishments in the remission of sins
correspond to individual sins.  It has been said before that saints
suffer punishments, which are works of God; they suffer contrition or
terrors, they also suffer other common afflictions.  Thus, for
example, some suffer punishments of their own that have been imposed
by God.  And these punishments pertain in no way to the keys because
the keys neither can impose nor remit them, but God, without the
ministry of the keys, imposes and remits them [as He will].

Neither does the universal rule follow: Upon David a peculiar
punishment was imposed, therefore, in addition to common afflictions,
there is another punishment of purgatory, in which each degree
corresponds to each sin.  Where does Scripture teach that we cannot
be freed from eternal death except by the compensation of certain
punishments in addition to common afflictions?  But, on the other
hand, it most frequently teaches that the remission of sins occurs
freely for Christ's sake, that Christ is the Victor of sin and death.
Therefore the merit of satisfaction is not to be patched upon this.
And although afflictions still remain, yet Scripture interprets these
as the mortifications of present sin [to kill and humble the old
Adam], and not as the compensations of eternal death or as prices for
eternal death.

Job is excused that he was not afflicted on account of past evil
deeds, therefore afflictions are not always punishments or signs of
wrath.  Yea, terrified consciences are to be taught that other ends
of afflictions are more important [that they should learn to regard
troubles far differently, namely, as signs of grace], lest they think
that they are rejected by God when in afflictions they see nothing
but God's punishment and anger.  The other more important ends are to
be considered namely, that God is doing His strange work so that He
may he able to do His own work, etc., as Isaiah teaches in a long
discourse, chap. 28. And when the disciples asked concerning the
blind man who sinned, John 9, 2. 3, Christ replies that the cause of
his blindness is not sin, but that the works of God should be made
manifest in him.  And in Jeremiah, 49, 12, it is said: They whose
judgment was not to drink of the cup have assuredly drunken.  Thus
the prophets and John the Baptist and other saints were killed.
Therefore afflictions are not always punishments for certain past
deeds, but they are the works of God, intended for our profit, and
that the power of God might be made more manifest in our weakness
[how He can help in the midst of death].

Thus Paul says, 2 Cor. 12, 5. 9: The strength of God is made perfect
in my weakness.  Therefore, because of God's will, our bodies ought
to be sacrifices, declare our obedience [and patience], and not to
compensate for eternal death, for which God has another price namely,
the death of His own Son.  And in this sense Gregory interprets even
the punishment of David when he says: If God on account of that sin
had threatened that he would thus be humbled by his son, why, when
the sin was forgiven, did He fulfil that which He had threatened
against him?  The reply is that this remission was made that man
might not be hindered from receiving eternal life, but that the
example of the threatening followed, in order that the piety of the
man might be exercised and tested even in this humility.  Thus also
God inflicted upon man death of body on account of sin, and after the
remission of sins He did not remove it, for the sake of exercising
justice namely, in order that the righteousness of those who are
sanctified might be exercised and tested.

Nor, indeed, are common calamities [as war, famine, and similar
calamities], properly speaking, removed by these works of canonical
satisfactions, i.e., by these works of human traditions, which, they
say, _avail ex opere operato_, in such a way that, even though they
are wrought in mortal sin, yet they redeem from the punishments.
[And the adversaries themselves confess that they impose
satisfactions, not on account of such common calamities but on
account of purgatory; hence, their satisfactions are pure
imaginations and dreams.] And when the passage of Paul, 1 Cor. 11, 31,
is cited against us: If we would judge ourselves, we should not be
judged by the Lord [they conclude therefrom that, if we impose
punishment upon ourselves, God will judge us the more graciously],
the word to judge ought to be understood of the entire repentance and
due fruits, not of works which are not due.  Our adversaries pay the
penalty for despising grammar when they understand to judge to be the
same as to make a pilgrimage clad in mail to the church of St. James,
or similar works.  To judge signifies the entire repentance, it
signifies to condemn sins.  This condemnation truly occurs in
contrition and the change of life.  The entire repentance, contrition,
faith, the good fruits, obtain the mitigation of public and private
punishments and calamities, as Isaiah teaches chap. 1, 17, 19: Cease
to do evil; learn to do well, etc. Though your sins be as scarlet,
they shall be white as snow.  If ye be willing and obedient, ye shall
eat the good of the land.  Neither should a most important and
salutary meaning be transferred from the entire repentance, and from
works due or commanded by God, to the satisfactions and works of
human traditions.  And this it is profitable to teach that common
evils are mitigated by our repentance and by the true fruits of
repentance, by good works wrought from faith, not, as these men
imagine, wrought in mortal sin.  And here belongs the example of the
Ninevites, Jonah 3, 10, who by their repentance (we speak of the
entire repentance) were reconciled to God, and obtained the favor
that their city was not destroyed.

Moreover, the making mention, by the Fathers, of satisfaction, and
the framing of canons by the councils, we have said above was a
matter of church-discipline instituted on account of the example.
Nor did they hold that this discipline is necessary for the remission
either of the guilt or of the punishment.  For if some of them made
mention of purgatory, they interpret it not as compensation for
eternal punishment [which only Christ makes], not as satisfaction,
but as purification of imperfect souls.  Just as Augustine says that
venial [daily] offenses are consumed i.e., distrust towards God and
other similar dispositions are mortified.  Now and then the writers
transfer the term satisfaction from the rite itself or spectacle, to
signify true mortification.  Thus Augustine says: True satisfaction
is to cut off the causes of sin, i.e., to mortify the flesh, likewise
to restrain the flesh, not in order that eternal punishments may be
compensated for but so that the flesh may not allure to sin.

Thus concerning restitution, Gregory says that repentance is false if
it does not satisfy those whose property we have taken.  For he who
still steals does not truly grieve that he has stolen or robbed.  For
he is a thief or robber, so long as he is the unjust possessor of the
property of another.  This civil satisfaction is necessary, because
it is written Eph. 4, 28: Let him that stole, steal no more.
Likewise Chrysostom says: In the heart, contrition; in the mouth,
confession; in the work, entire humility.  This amounts to nothing
against us.  Good works ought to follow repentance, it ought to be
repentance, not simulation, but a change of the entire life for the
better.

Likewise, the Fathers wrote that it is sufficient if once in life
this public or ceremonial penitence occur, about which the canons
concerning satisfactions have been made.  Therefore it can be
understood that they held that these canons are not necessary for the
remission of sins.  For in addition to this ceremonial penitence,
they frequently wish that penitence be rendered otherwise, where
canons of satisfactions were not required.

The composers of the Confutation write that the abolition of
satisfactions contrary to the plain Gospel is not to be endured.  We,
therefore, have thus far shown that these canonical satisfactions, i.
e., works not due and that are to be performed in order to compensate
for punishment, have not the command of the Gospel.  The subject
itself shows this.  If works of satisfaction are works which are not
due, why do they cite the plain Gospel?  For if the Gospel would
command that punishments be compensated for by such works, the works
would already be due.  But thus they speak in order to impose upon
the inexperienced, and they cite testimonies which speak of works
that are due, although they themselves in their own satisfactions
prescribe works that are not due.  Yea, in their schools they
themselves concede that satisfactions can be refused without [mortal]
sin.  Therefore they here write falsely that we are compelled by the
plain Gospel to undertake these canonical satisfactions.

But we have already frequently testified that repentance ought to
produce good fruits: and what the good fruits are the [Ten]
Commandments teach, namely, [truly and from the heart most highly to
esteem, fear, and love God, joyfully to call upon Him in need],
prayer, thanksgiving, the confession of the Gospel [hearing this
Word], to teach the Gospel, to obey parents and magistrates, to be
faithful to one's calling, not to kill, not to retain hatred, but to
be forgiving [to be agreeable and kind to one's neighbor], to give to
the needy, so far as we can according to our means, not to commit
fornication or adultery, but to restrain and bridle and chastise the
flesh, not for a compensation of eternal punishment, but so as not to
obey the devil, or offend the Holy Ghost, likewise, to speak the
truth.  These fruits have God's injunction, and ought to be brought
forth for the sake of God's glory and command; and they have their
rewards also.  But that eternal punishments are not remitted except
on account of the compensation rendered by certain traditions or by
purgatory, Scripture does not teach.  Indulgences were formerly
remission of these public observances, so that men should not be
excessively burdened.  But if, by human authority, satisfactions and
punishments can be remitted, this compensation, therefore, is not
necessary by divine Law, for a divine Law is not annulled by human
authority.  Furthermore, since the custom has now of itself become
obsolete and the bishops have passed it by in silence, there is no
necessity for these remissions.  And yet the name indulgences
remained.  And just as satisfactions were understood not with
reference to external discipline, but with reference to the
compensation of punishment, so indulgences were incorrectly
understood to free souls from purgatory.  But the keys have not the
power of binding and loosing except upon earth, according to Matt. 16,
19 : Whatsoever thou shalt bind on earth shall be bound in heaven,
and whatsoever thou shalt loose on earth shall be loosed in heaven.
Although as we have said above, the keys have not the power to impose
penalties, or to institute rites of worship, but only the command to
remit sins to those who are converted, and to convict and
excommunicate those who are unwilling to be converted.  For just as
to loose signifies to remit sins, so to bind signifies not to remit
sins.  For Christ speaks of a spiritual kingdom.  And the command of
God is that the ministers of the Gospel should absolve those who are
converted, according to 2 Cor. 10, 8: The authority which the Lord
hath given us for edification.  Therefore the reservation of eases is
a secular affair.  For it is a reservation of canonical punishment;
it is not a reservation of guilt before God in those who are truly
converted.  Therefore the adversaries judge aright when they confess
that in the article of death the reservation of eases ought not to
hinder absolution.

We have set forth the sum of our doctrine concerning repentance,
which we certainly know is godly and salutary to good minds [and
highly necessary].  And if good men will compare our [yea, Christ's
and His apostles'] doctrine with the very confused discussions of our
adversaries, they will perceive that the adversaries have omitted the
doctrine [without which no one can teach or learn anything that is
substantial and Christian] concerning faith justifying and consoling
godly hearts.  They will also see that the adversaries invent many
things concerning the merits of attrition, concerning the endless
enumeration of offenses, concerning satisfactions, they say things
[that touch neither earth nor heaven] agreeing neither with human nor
divine law, and which not even the adversaries themselves can
satisfactorily explain.




Part 18


Article XIII (VII): _Of the Number and Use of the Sacraments._

In the Thirteenth Article the adversaries approve our statement that
the Sacraments are not only marks of profession among men, as some
imagine, but that they are rather signs and testimonies of God's will
toward us, through which God moves hearts to believe [are not mere
signs whereby men may recognize each other, as the watchword in war,
livery, etc., but are efficacious signs and sure testimonies, etc.].
But here they bid us also count seven sacraments.  We hold that it
should be maintained that the matters and ceremonies instituted in
the Scriptures, whatever the number, be not neglected.  Neither do we
believe it to be of any consequence, though, for the purpose of
teaching, different people reckon differently, provided they still
preserve aright the matters handed down in Scripture.  Neither have
the ancients reckoned in the same manner.  [But concerning this
number of seven sacraments, the fact is that the Fathers have not
been uniform in their enumeration, thus also these seven ceremonies
are not equally necessary.]

If we call Sacraments rites which have the command of God and to
which the promise of grace has been added, it is easy to decide what
are properly Sacraments.  For rites instituted by men will not in
this way be Sacraments properly so called.  For it does not belong to
human authority to promise grace.  Therefore signs instituted without
God's command are not sure signs of grace, even though they perhaps
instruct the rude [children or the uncultivated], or admonish as to
something [as a painted cross].  Therefore Baptism, the Lord's Supper,
and Absolution, which is the Sacrament of Repentance, are truly
Sacraments.  For these rites have God's command and the promise of
grace, which is peculiar to the New Testament.  For when we are
baptized, when we eat the Lord's body, when we are absolved, our
hearts must be firmly assured that God truly forgives us for Christ's
sake.  And God, at the same time, by the Word and by the rite, moves
hearts to believe and conceive faith, just as Paul says, Rom. 10, 17:
Faith cometh by hearing.  But just as the Word enters the ear in
order to strike our heart, so the rite itself strikes the eye, in
order to move the heart.  The effect of the Word and of the rite is
the same, as it has been well said by Augustine that a Sacrament is a
visible word, because the rite is received by the eyes, and is, as it
were, a picture of the Word, signifying the same thing as the Word.
Therefore the effect of both is the same.

Confirmation and Extreme Unction are rites received from the Fathers
which not even the Church requires as necessary to salvation, they do
not have God's command.  Therefore it is not useless to distinguish
these rites from the former, which have God's express command and a
clear promise of grace.

The adversaries understand priesthood not of the ministry of the Word,
and administering the Sacraments to others, but they understand it
as referring to sacrifice, as though in the New Testament there ought
to be a priesthood like the Levitical, to sacrifice for the people,
and merit the remission of sins for others.  We teach that the
sacrifice of Christ dying on the cross has been sufficient for the
sins of the whole world, and that there is no need, besides, of other
sacrifices, as though this were not sufficient for our sins.  Men,
accordingly, are justified not because of any other sacrifices, but
because of this one sacrifice of Christ, if they believe that they
have been redeemed by this sacrifice.  They are accordingly called
priests, not in order to make any sacrifices for the people as in the
Law so that by these they may merit remission of sins for the people;
but they are called to teach the Gospel and administer the Sacraments
to the people.  Nor do we have another priesthood like the Levitical,
as the Epistle to the Hebrews sufficiently teaches.  But if
ordination be understood as applying to the ministry of the Word, we
are not unwilling to call ordination a sacrament.  For the ministry
of the Word has God's command and glorious promises, Rom. 1, 16: The
Gospel is the power of God unto salvation to every one that believeth.
Likewise, Is. 55, 11: So shall My Word be that goeth forth out of
My mouth; it shall not return unto Me void, but it shall accomplish
that which I please.  If ordination be understood in this way,
neither will we refuse to call the imposition of hands a sacrament.
For the Church has the command to appoint ministers, which should be
most pleasing to us, because we know that God approves this ministry
and is present in the ministry [that God will preach and work through
men and those who have been chosen by men].  And it is of advantage,
so far as can be done, to adorn the ministry of the Word with every
kind of praise against fanatical men, who dream that the Holy Ghost
is given not through the Word, but because of certain preparations of
their own, if they sit unoccupied and silent in obscure places,
waiting for illumination, as the Enthusiasts formerly taught, and the
Anabaptists now teach.

Matrimony was not first instituted in the New Testament, but in the
beginning, immediately on the creation of the human race.  It has,
moreover, God's command; it has also promises, not indeed properly
pertaining to the New Testament, but pertaining rather to the bodily
life.  Wherefore, if any one should wish to call it a sacrament, he
ought still to distinguish it from those preceding ones [the two
former ones], which are properly signs of the New Testament, and
testimonies of grace and the remission of sins.  But if marriage will
have the name of sacrament for the reason that it has God's command
other states or offices also, which have God's command, may be called
sacraments, as, for example, the magistracy.

Lastly, if among the Sacraments all things ought to be numbered which
have God's command, and to which promises have been added, why do we
not add prayer, which most truly can be called a sacrament?  For it
has both God's command and very many promises and if placed among the
Sacraments, as though in a more eminent place, it would invite men to
pray.  Alms could also be reckoned here, and likewise afflictions,
which are even themselves signs, to which God has added promises.
But let us omit these things.  For no prudent man will strive greatly
concerning the number or the term, if only those objects still be
retained which have God's command and promises.

It is still more needful to understand how the Sacraments are to be
used.  Here we condemn the whole crowd of scholastic doctors, who
teach that the Sacraments confer grace _ex opere operato_, without a
good disposition on the part of the one using them, provided he do
not place a hindrance in the way.  This is absolutely a Jewish
opinion, to hold that we are justified by a ceremony, without a good
disposition of the heart, i.e., without faith.  And yet this impious
and pernicious opinion is taught with great authority throughout the
entire realm of the Pope.  Paul contradicts this and denies, Rom. 4,
9, that Abraham was justified by circumcision, but asserts that
circumcision was a sign presented for exercising faith.  Thus we
teach that in the use of the Sacraments faith ought to be added,
which should believe these promises, and receive the promised things,
there offered in the Sacrament.  And the reason is plain and
thoroughly grounded.  [This is a certain and true use of the holy
Sacrament, on which Christian hearts and consciences may risk to rely.
] The promise is useless unless it is received by faith.  But the
Sacraments are the signs [and seals] of the promises.  Therefore, in
the use of the Sacraments faith ought to be added so that, if any one
use the Lord's Supper, he use it thus.  Because this is a Sacrament
of the New Testament, as Christ clearly says, he ought for this very
reason to be confident that what is promised in the New Testament
namely, the free remission of sins, is offered him.  And let him
receive this by faith, let him comfort his alarmed conscience, and
know that these testimonies are not fallacious, but as sure as though
[and still surer than if] God by a new miracle would declare from
heaven that it was His will to grant forgiveness.  But of what
advantage would these miracles and promises be to an unbeliever?  And
here we speak of special faith which believes the present promise,
not only that which in general believes that God exists, but which
believes that the remission of sins is offered.  This use of the
Sacrament consoles godly and alarmed minds.

Moreover, no one can express in words what abuses in the Church this
fanatical opinion concerning the opus operate, without a good
disposition on the part of the one using the Sacraments, has produced.
Hence the infinite profanation of the Masses, but of this we shall
speak below.  Neither can a single letter be produced from the old
writers which in this matter favors the scholastics.  Yea Augustine
says the contrary, that the faith of the Sacrament, and not the
Sacrament justifies.  And the declaration of Paul is well known, Rom.
10, 10: With the heart man believeth unto righteousness.




Part 19


Article XIV: _Of Ecclesiastical Order._

The Fourteenth Article, in which we say that in the Church the
administration of the Sacraments and Word ought to be allowed no one
unless he be rightly called, they receive, but with the proviso that
we employ canonical ordination.  Concerning this subject we have
frequently testified in this assembly that it is our greatest wish to
maintain church-polity and the grades in the Church [old
church-regulations and the government of bishops], even though they
have been made by human authority [provided the bishops allow our
doctrine and receive our priests].  For we know that
church-discipline was instituted by the Fathers, in the manner laid
down in the ancient canons with a good and useful intention.  But the
bishops either compel our priests to reject and condemn this kind of
doctrine which we have confessed, or, by a new and unheard-of cruelty,
they put to death the poor innocent men.  These causes hinder our
priests from acknowledging such bishops.  Thus the cruelty of the
bishops is the reason why the canonical government, which we greatly
desired to maintain, is in some places dissolved.  Let them see to it
how they will give an account to God for dispersing the Church.  In
this matter our consciences are not in danger, because since we know
that our Confession is true, godly, and catholic, we ought not to
approve the cruelty of those who persecute this doctrine.  And we
know that the Church is among those who teach the Word of God aright,
and administer the Sacraments aright and not with those who not only
by their edicts endeavor to efface God's Word, but also put to death
those who teach what is right and true towards whom, even though they
do something contrary to the canons, yet the very canons are milder.
Furthermore we wish here again to testify that we will gladly
maintain ecclesiastical and canonical government, provided the
bishops only cease to rage against our Churches.  This our desire
will clear us both before God and among all nations to all posterity
from the imputation against us that the authority of the bishops is
being undermined, when men read and hear that, although protesting
against the unrighteous cruelty of the bishops, we could not obtain
justice.




Part 20


Article XV (VIII): _Of Human Traditions in the Church._

In the Fifteenth Article they receive the first part, in which we say
that such ecclesiastical rites are to be observed as can be observed
without sin, and are of profit in the Church for tranquility and good
order.  They altogether condemn the second part, in which we say that
human traditions instituted to appease God, to merit grace, and make
satisfactions for sins are contrary to the Gospel.  Although in the
Confession itself, when treating of the distinction of meats, we have
spoken at sufficient length concerning traditions, yet certain things
should be briefly recounted here.

Although we supposed that the adversaries would defend human
traditions on other grounds, yet we did not think that this would
come to pass, namely, that they would condemn this article: that we
do not merit the remission of sins or grace by the observance of
human traditions.  Since, therefore, this article has been condemned,
we have an easy and plain case.  The adversaries are now openly
Judaizing, are openly suppressing the Gospel by the doctrines of
demons.  For Scripture calls traditions doctrines of demons when it
is taught that religious rites are serviceable to merit the remission
of sins and grace.  For they are then obscuring the Gospel, the
benefit of Christ, and the righteousness of faith.  [For they are
just as directly contrary to Christ and to the Gospel as are fire and
water to one another.] The Gospel teaches that by faith we receive
freely, for Christ's sake, the remission of sins and are reconciled.
The adversaries, on the other hand, appoint another mediator, namely
these traditions.  On account of these they wish to acquire remission
of sins; on account of these they wish to appease God's wrath.  But
Christ clearly says, Matt. 15, 9: In vain do they worship Me,
teaching for doctrines the commandments of men.

We have above discussed at length that men are justified by faith
when they believe that they have a reconciled God, not because of our
works, but gratuitously, for Christ's sake.  It is certain that this
is the doctrine of the Gospel, because Paul clearly teaches Eph. 2, 8.
9: By grace are ye saved, through faith; and that not of yourselves:
it is the gift of God; not of works.  Now these men say that men
merit the remission of sins by these human observances.  What else is
this than to appoint another justifier, a mediator other than Christ?
Paul says to the Galatians, 5, 4: Christ has become of no effect
unto you, whosoever of you are justified by the Law, i.e., if you
hold that by the observance of the Law you merit to be accounted
righteous before God, Christ will profit you nothing; for what need
of Christ have those who hold that they are righteous by their own
observance of the Law?  God has set forth Christ with the promise
that on account of this Mediator, and not on account of our
righteousness, He wishes to be propitious to us.  But these men hold
that God is reconciled and propitious because of the traditions, and
not because of Christ.  Therefore they take away from Christ the
honor of Mediator.  Neither, so far as this matter is concerned is
there any difference between our traditions and the ceremonies of
Moses.  Paul condemns the ceremonies of Moses, just as he condemns
traditions, for the reason that they were regarded as works which
merit righteousness before God.  Thus the office of Christ and the
righteousness of faith were obscured.  Therefore, the Law being
removed, and traditions being removed, he contends that the remission
of sins has been promised not because of our works, but freely,
because of Christ, if only by faith we receive it.  For the promise
is not received except by faith.  Since, therefore, by faith we
receive the remission of sins since by faith we have a propitious God
for Christ's sake, it is an error and impiety to declare that because
of these observances we merit the remission of sins.  If any one
should say here that we do not merit the remission of sins, but that
those who have already been justified by these traditions merit grace,
Paul again replies, Gal. 2, 17, that Christ would be the minister of
sin if after justification we must hold that henceforth we are not
accounted righteous for Christ's sake, but we ought first, by other
observances, to merit that we be accounted righteous.  Likewise Gal.
3, 15: Though it be but a man's covenant, no man addeth thereto.
Therefore, neither to God's covenant, who promises that for Christ's
sake He will be propitious to us ought we to add that we must first
through these observances attain such merit as to be regarded as
accepted and righteous.

However, what need is there of a long discussion?  No tradition was
instituted by the holy Fathers with the design that it should merit
the remission of sins, or righteousness, but they have been
instituted for the sake of good order in the Church and for the sake
of tranquillity.  And when any one wishes to institute certain works
to merit the remission of sins, or righteousness, how will he know
that these works please God since he has not the testimony of God's
Word?  How, without God's command and Word, will he render men
certain of God's will?  Does He not everywhere in the prophets
prohibit men from instituting, without His commandment, peculiar
rites of worship?  In Ezek. 20, 18. 19 it is written: Walk ye not in
the statutes of your fathers, neither observe their judgments, nor
defile yourselves with their idols: I Am the Lord, your God.  Walk in
My statutes, and keep My judgements, and do them.  If men are allowed
to institute religious rites and through these rites merit grace, the
religious rites of all the heathen will have to be approved, and the
rites instituted by Jeroboam, 1 Kings 12, 26 f., and by others,
outside of the Law, will have to be approved.  For what difference
does it make?  If we have been allowed to institute religious rites
that are profitable for meriting grace, or righteousness, why was the
same not allowed the heathen and the Israelites?  But the religious
rites of the heathen and the Israelites were rejected for the very
reason that they held that by these they merited remission of sins
and righteousness, and yet did not know [the highest service of God]
the righteousness of faith.  Lastly, whence are we rendered certain
that rites instituted by men without God's command justify, inasmuch
as nothing can be affirmed of God's will without God's Word?  What if
God does not approve these services?  How, therefore, do the
adversaries affirm that they justify?  Without God's Word and
testimony this cannot be affirmed.  And Paul says, Rom. 14, 23
Whatsoever is not of faith is sin.  But as these services have no
testimony of God's Word, conscience must doubt as to whether they
please God.

And what need is there of words on a subject so manifest?  If the
adversaries defend these human services as meriting justification,
grace, and the remission of sins, they simply establish the kingdom
of Antichrist.  For the kingdom of Antichrist is a new service of God,
devised by human authority rejecting Christ, just as the kingdom of
Mahomet has services and works through which it wishes to be
justified before God; nor does it hold that men are gratuitously
justified before God by faith for Christ's sake.  Thus the Papacy
also will be a part of the kingdom of Antichrist if it thus defends
human services as justifying.  For the honor is taken away from
Christ when they teach that we are not justified gratuitously by
faith, for Christ's sake, but by such services, especially when they
teach that such services are not only useful for justification, but
are also necessary, as they hold above in Art.  VII, where they
condemn us for saying that unto true unity of the Church it is not
necessary that rites instituted by men should everywhere be alike.
Daniel, 11, 38, indicates that new human services will be the very
form and constitution of the kingdom of Antichrist.  For he says thus:
But in his estate shall he honor the god of forges; and a god whom
his fathers knew not shall he honor with gold and silver and precious
stones.  Here he describes new services, because he says that such a
god shall be worshiped as the fathers were ignorant of.  For although
the holy Fathers themselves had both rites and traditions, yet they
did not hold that these matters are useful or necessary for
justification they did not obscure the glory and office Christ, but
taught that we are justified by faith for Christ's sake, and not for
the sake of these human services.  But they observed human rites for
the sake of bodily advantage, that the people might know at what time
they should assemble; that, for the sake of example, all things in
the churches might be done in order and becomingly; lastly, that the
common people might receive a sort of training.  For the distinctions
of times and the variety of rites are of service in admonishing the
common people.  The Fathers had these reasons for maintaining the
rites, and for these reasons we also judge it to be right that
traditions [good customs] be maintained.  And we are greatly
surprised that the adversaries [contrary to the entire Scriptures of
the Apostles, contrary to the Old and New Testaments] contend for
another design of traditions, namely, that they may merit the
remission of sins, grace, or justification.  What else is this than
to honor God with gold and silver and precious stones [as Daniel
says], i.e., to hold that God becomes reconciled by a variety in
clothing, ornaments, and by similar rites [many kinds of church
decorations, banners, tapers], as are infinite in human traditions?

Paul writes to the Colossians, 2, 23, that traditions have a show of
wisdom.  And they indeed have.  For this good order is very becoming
in the Church, and for this reason is necessary.  But human reason,
because it does not understand the righteousness of faith, naturally
imagines that such works justify men because they reconcile God, etc.
Thus the common people among the Israelites thought, and by this
opinion increased such ceremonies, just as among us they have grown
in the monasteries [as in our time one altar after another and one
church after another is founded].  Thus human reason judges also of
bodily exercises, of fasts, although the end of these is to restrain
the flesh, reason falsely adds that they are services which justify.
As Thomas writes: Fasting avails for the extinguishing and the
prevention of guilt.  These are the words of Thomas.  Thus the
semblance of wisdom and righteousness in such works deceives men.
And the examples of the saints are added [when they say: St. Francis
wore a cap, etc.]; and when men desire to imitate these, they imitate,
for the most part, the outward exercises; their faith they do not
imitate.

After this semblance of wisdom and righteousness has deceived men,
then infinite evils follow; the Gospel concerning the righteousness
of faith in Christ is obscured, and vain confidence in such works
succeeds.  Then the commandments of God are obscured; these works
arrogate to themselves the title of a perfect and spiritual life, and
are far preferred to the works of God's commandments [the true, holy,
good works], as, the works of one's own calling, the administration
of the state, the management of a family, married life, the bringing
up of children.  Compared with those ceremonies, the latter are
judged to be profane, so that they are exercised by many with some
doubt of conscience.  For it is known that many have abandoned the
administration of the state and married life, in order to embrace
these observances as better and holier [have gone into cloisters in
order to become holy and spiritual].

Nor is this enough.  When the persuasion has taken possession of
minds that such observances are necessary to justification,
consciences are in miserable anxiety because they cannot exactly
fulfil all observances.  For how many are there who could enumerate
all these observances?  There are immense books, yea whole libraries,
containing not a syllable concerning Christ, concerning faith in
Christ, concerning the good works of one's own calling, but which
only collect the traditions and interpretations by which they are
sometimes rendered quite rigorous and sometimes relaxed.  [They write
of such precepts as of fasting for forty days, the four canonical
hours for prayer, etc.] How that most excellent man, Gerson, is
tortured while he searches for the grades and extent of the precepts!
Nevertheless, he is not able to fix _epieicheian_ [mitigation] in a
definite grade [and yet cannot find any sure grade where he could
confidently promise the heart assurance and peace].  Meanwhile, he
deeply deplores the dangers to godly consciences which this rigid
interpretation of the traditions produces.

Against this semblance of wisdom and righteousness in human rites,
which deceives men, let us therefore fortify ourselves by the Word of
God, and let us know, first of all that these neither merit before
God the remission of sins or justification, nor are necessary for
justification.  We have above cited some testimonies.  And Paul is
full of them.  To the Colossians, 2, 16. 17, he clearly says: Let no
man, therefore, judge you in meat or in drink, or in respect of an
holy-day, or of the new moon, or of the Sabbath-days, which are a
shadow of things to come; but the body is of Christ.  Here now he
embraces at the same time both the Law of Moses and human traditions
in order that the adversaries may not elude these testimonies,
according to their custom, upon the ground that Paul is speaking only
of the Law of Moses.  But he clearly testifies here that he is
speaking of human traditions.  However, the adversaries do not see
what they are saying; if the Gospel says that the ceremonies of Moses,
which were divinely instituted, do not justify, how much less do
human traditions justify!

Neither have the bishops the power to institute services, as though
they justified, or were necessary for justification.  Yea, the
apostles, Acts 15, 10, say: Why tempt ye God to put a yoke, etc.,
where Peter declares this purpose to burden the Church a great sin.
And Paul forbids the Galatians, 5, 1, to be entangled again with the
yoke of bondage.  Therefore, it is the will of the apostles that this
liberty remain in the Church, that no services of the Law or of
traditions be judged as necessary (just as in the Law ceremonies were
for a time necessary), lest the righteousness of faith be obscured,
if men judge that these services merit justification, or are
necessary for justification.  Many seek in traditions various
_epieicheian_ [mitigations] in order to heal consciences, and yet
they do not find any sure grades by which to free consciences from
these chains.  But just as Alexander once for all solved the Gordian
knot by cutting it with his sword when he could not disentangle it,
so the apostles once for all free consciences from traditions,
especially if they are taught to merit justification.  The apostles
compel us to oppose this doctrine by teaching and examples.  They
compel us to teach that traditions do not justify; that they are not
necessary for justification; that no one ought to frame or receive
traditions with the opinion that they merit justification.  Then,
even though any one should observe them, let him observe them without
superstition as civil customs, just as without superstition soldiers
are clothed in one way and scholars in another [as I regard my
wearing of a German costume among the Germans and a French costume
among the French as an observance of the usage of the land, and not
for the purpose of being saved thereby].  The apostles violate
traditions and are excused by Christ for the example was to be shown
the Pharisees that these services are unprofitable.  And if our
people neglect some traditions that are of little advantage, they are
now sufficiently excused, when these are required as though they
merit justification.  For such an opinion with regard to traditions
is impious [an error not to be endured].

But we cheerfully maintain the old traditions [as, the three high
festivals, the observance of Sunday, and the like] made in the Church
for the sake of usefulness and tranquillity, and we interpret them in
a more moderate way, to the exclusion of the opinion which holds that
they justify.  And our enemies falsely accuse us of abolishing good
ordinances and churchdiscipline.  For we can truly declare that the
public form of the churches is more becoming with us than with the
adversaries [that the true worship of God is observed in our churches
in a more Christian, honorable way].  And if any one will consider it
aright, we conform to the canons more truly than do the adversaries.
[For the adversaries, without shame, tread under foot the most
honorable canons, just as they do Christ and the Gospel.] With the
adversaries, unwilling celebrants, and those hired for pay, and very
frequently only for pay, celebrate the Masses.  They sing psalms, not
that they may learn or pray [for the greater part do not understand a
verse in the psalms], but for the sake of the service as though this
work were a service, or at feast, for the sake of reward.  [All this
they cannot deny.  Some who are upright among them are even ashamed
of this baffle, and declare that the clergy is in need of reformation.
] With us many use the Lord's Supper [willingly and without
constraint] every Lord's Day, but after having been first instructed,
examined [whether they know and understand anything of the Lord's
Prayer, the Creed, and the Ten Commandments], and absolved.  The
children sing psalms in order that they may learn [become familiar
with passages of Scripture], the people also sing [Latin and German
psalms], in order that they may either learn or pray.  With the
adversaries there is no catechization of the children whatever,
concerning which even the canons give commands.  With us the pastors
and ministers of the churches are compelled publicly [and privately]
to instruct and hear the youth; and this ceremony produces the best
fruits.  [And the Catechism is not a mere childish thing, as is the
bearing of banners and tapers, but a very profitable instruction.]
Among the adversaries, in many regions [as in Italy and Spain],
during the entire year no sermons are delivered, except in Lent [Here
they ought to cry out and justly make grievous complaint, for this
means at one blow to overthrow completely all worship.  For of all
acts that is the greatest most holy, most necessary, and highest,
which God has required as the highest in the First and the Second
Commandment, namely, to preach the Word of God.  For the ministry is
the highest office in the Church.  Now, if this worship is omitted,
how can there be knowledge of God, the doctrine of Christ, or the
Gospel,] But the chief service of God is to teach the Gospel.  And
when the adversaries do preach, they speak of human traditions, of
the worship of saints [of consecrated water], and similar tripes,
which the people justly loathe, therefore they are deserted
immediately in the beginning, after the text of the Gospel has been
recited.  [This practise may have started because the people did not
wish to hear the other lies.] A few better ones begin now to speak of
good works, but of the righteousness of faith, of faith in Christ, of
the consolation of consciences, they say nothing; yea, this most
wholesome part of the Gospel they rail at with their reproaches.
[This blessed doctrine, the precious holy Gospel, they call Lutheran.
] On the contrary, in our churches all the sermons are occupied with
such topics as these: of repentance, of the fear of God, of faith in
Christ, of the righteousness of faith, of the consolation of
consciences by faith, of the exercises of faith; of prayer, what its
nature should be, and that we should be fully confident that it is
efficacious, that it is heard of the cross; of the authority of
magistrates and all civil ordinances [likewise, how each one in his
station should live in a Christian manner, and, out of obedience to
the command of the Lord God, should conduct himself in reference to
every worldly ordinance and law]; of the distinction between the
kingdom of Christ, or the spiritual kingdom and political affairs, of
marriage; of the education and instruction of children, of chastity;
of all the offices of love.  From this condition of the churches it
may be judged that we diligently maintain church-discipline and godly
ceremonies and good churchcustoms.

And of the mortification of the flesh and discipline of the body we
thus teach, just as the Confession states, that a true and not a
feigned mortification occurs through the cross and afflictions by
which God exercises us [when God breaks our will, inflicts the cross
and trouble].  In these we must obey God's will, as Paul says, Rom.
12, 1: Present your bodies a living sacrifice.  And these are the
spiritual exercises of fear and faith.  But in addition to this
mortification which occurs through the cross [which does not depend
upon our will] there is also a voluntary kind of exercise necessary,
of which Christ says Luke 21, 34: Take heed to yourselves lest at any
time your hearts be overcharged with surfeiting.  And Paul, 1 Cor. 9,
27: I keep under my body, and bring it into subjection, etc. And
these exercises are to be undertaken not because they are services
that justify, but in order to curb the flesh, lest satiety may
overpower us, and render us secure and indifferent, the result of
which is that men indulge and obey the dispositions of the flesh.
This diligence ought to be perpetual, because it has the perpetual
command of God.  And this prescribed form of certain meats and times
does nothing [as experience shows] towards curbing the flesh.  For it
is more luxurious and sumptuous than other feasts [for they were at
greater expense, and practised greater gluttony with fish and various
Lenten meats than when the fasts were not observed], and not even the
adversaries observe the form given in the canons.

This topic concerning traditions contains many and difficult
questions of controversy and we have actually experienced that
traditions are truly snares of consciences.  When they are exacted as
necessary, they torture in wonderful ways the conscience omitting any
observance [as godly hearts, indeed, experience when in the canonical
hours they have omitted a compline, or offended against them in a
similar way].  Again their abrogation has its own evils and its own
questions.  [On the other hand, to teach absolute freedom has also
its doubts and questions, because the common people need outward
discipline and instruction.] But we have an easy and plain case,
because the adversaries condemn us for teaching that human traditions
do not merit the remission of sins.  Likewise they require universal
traditions, as they call them, as necessary for justification [and
place them in Christ's stead].  Here we have Paul as a constant
champion, who everywhere contends that these observances neither
justify nor are necessary in addition to the righteousness of faith.
And nevertheless we teach that in these matters the use of liberty is
to be so controlled that the inexperienced may not be offended, and,
on account of the abuse of liberty, may not become more hostile to
the true doctrine of the Gospel, or that without a reasonable cause
nothing in customary rites be changed, but that, in order to cherish
harmony, such old customs be observed as can be observed without sin
or without great inconvenience.  And in this very assembly we have
shown sufficiently that for love's sake we do not refuse to observe
adiaphora with others, even though they should have some disadvantage;
but we have judged that such public harmony as could indeed be
produced without offense to consciences ought to be preferred to all
other advantages [all other less important matters].  But concerning
this entire subject we shall speak after a while, when we shall treat
of vows and ecclesiastical power.




Part 21


Article XVI: _Of Political Order._

The Sixteenth Article the adversaries receive without any exception,
in which we have confessed that it is lawful for the Christian to
bear civil office, sit in judgment, determine matters by the imperial
laws, and other laws in present force, appoint just punishments
engage in just wars, act as a soldier, make legal contracts, hold
property, take an oath when magistrates require it, contract marriage;
finally, that legitimate civil ordinances are good creatures of God
and divine ordinances, which a Christian can use with safety.  This
entire topic concerning the distinction between the kingdom of Christ
and a political kingdom has been explained to advantage [to the
remarkably great consolation of many consciences] in the literature
of our writers, [namely] that the kingdom of Christ is spiritual
[inasmuch as Christ governs by the Word and by preaching], to wit,
beginning in the heart the knowledge of God, the fear of God and
faith, eternal righteousness, and eternal life; meanwhile it permits
us outwardly to use legitimate political ordinances of every nation
in which we live, just as it permits us to use medicine or the art of
building, or food, drink, air.  Neither does the Gospel bring new
laws concerning the civil state, but commands that we obey present
laws, whether they have been framed by heathen or by others, and that
in this obedience we should exercise love.  For Carlstadt was insane
in imposing upon us the judicial laws of Moses.  Concerning these
subjects, our theologians have written more fully, because the monks
diffused many pernicious opinions in the Church.  They called a
community of property the polity of the Gospel; they said that not to
hold property, not to vindicate one's self at law [not to have wife
and child], were evangelical counsels.  These opinions greatly
obscure the Gospel and the spiritual kingdom [so that it was not
understood at all what the Christian or spiritual kingdom of Christ
is; they concocted the secular kingdom with the spiritual whence much
trouble and seditions, harmful teaching resulted], and are dangerous
to the commonwealth.  For the Gospel does not destroy the State or
the family [buying, selling, and other civil regulations], but much
rather approves them, and bids us obey them as a divine ordinance,
not only on account of punishment, but also on account of conscience.

Julian the Apostate, Celsus, and very many others made the objection
to Christians that the Gospel would rend asunder states, because it
prohibited legal redress, and taught certain other things not at all
suited to political association.  And these questions wonderfully
exercised Origen, Nazianzen, and others, although, indeed, they can
be most readily explained, if we keep in mind the fact that the
Gospel does not introduce laws concerning the civil state, but is the
remission of sins and the beginning of a new life in the hearts of
believers; besides, it not only approves outward governments, but
subjects us to them, Rom. 13, 1, just as we have been necessarily
placed under the laws of seasons, the changes of winter and summer,
as divine ordinances.  [This is no obstacle to the spiritual kingdom.
] The Gospel forbids private redress [in order that no one should
interfere with the office of the magistrate], and Christ inculcates
this so frequently with the design that the apostles should not think
that they ought to seize the governments from those who held
otherwise, just as the Jews dreamed concerning the kingdom of the
Messiah, but that they might know they ought to teach concerning the
spiritual kingdom that it does not change the civil state.  Therefore
private redress is prohibited not by advice, but by a command, Matt.
5, 39; Rom. 12, 19. Public redress which is made through the office
of the magistrate, is not advised against, but is commanded, and is a
work of God, according to Paul, Rom. 13, 1 sqq.  Now the different
kinds of public redress are legal decisions, capital punishment, wars,
military service.  It is manifest how incorrectly many writers have
judged concerning these matters [some teachers have taught such
pernicious errors that nearly all princes, lords, knights, servants
regarded their proper estate as secular, ungodly, and damnable, etc.
Nor can it be fully expressed in words what an unspeakable peril and
damage has resulted from this to souls and consciences], because they
were in the error that the Gospel is an external, new and monastic
form of government, and did not see that the Gospel brings eternal
righteousness to hearts [teaches how a person is redeemed, before God
and in his conscience, from sin, hell, and the devil], while it
outwardly approves the civil state.

It is also a most vain delusion that it is Christian perfection not
to hold property.  For Christian perfection consists not in the
contempt of civil ordinances, but in dispositions of the heart, in
great fear of God, in great faith, just as Abraham, David, Daniel,
even in great wealth and while exercising civil power, were no less
perfect than any hermits.  But the monks [especially the Barefoot
monks] have spread this outward hypocrisy before the eyes of men, so
that it could not be seen in what things true perfection exists.
With what praises have they brought forward this communion of
property, as though it were evangelical!  But these praises have the
greatest danger, especially since they differ much from the
Scriptures.  For Scripture does not command that property be common,
but the Law of the Decalog, when it says, Ex. 20, 15: Thor shalt not
steal, distinguishes rights of ownership, and commands each one to
hold what is his own.  Wyclif manifestly was raging when he said that
priests were not allowed to hold property.  There are infinite
discussions concerning contracts, in reference to which good
consciences can never be satisfied unless they know the rule that it
is lawful for a Christian to make use of civil ordinances and laws.
This rule protects consciences when it teaches that contracts are
lawful before God just to the extent that the magistrates or laws
approve them.

This entire topic concerning civil affairs has been so clearly set
forth by our theologians that very many good men occupied in the
state and in business have declared that they have been greatly
benefited, who before, troubled by the opinion of the monks, were in
doubt as to whether the Gospel allowed these civil offices and
business.  Accordingly, we have recounted these things in order that
those without also may understand that by the kind of doctrine which
we follow, the authority of magistrates and the dignity of all civil
ordinances are not undermined, but are all the more strengthened [and
that it is only this doctrine which gives true instruction as to how
eminently glorious an office, full of good Christian works, the
office of rulers is].  The importance of these matters was greatly
obscured previously by those silly monastic opinions, which far
preferred the hypocrisy of poverty and humility to the state and the
family, although these have God's command, while this Platonic
communion [monasticism] has not God's command.




Part 22


Article XVII: _Of Christ's Return to Judgment._

The Seventeenth Article the adversaries receive without exception, in
which we confess that at the consummation of the world Christ shall
appear, and shall raise up all the dead, and shall give to the godly
eternal life and eternal joys, but shall condemn the ungodly to be
punished with the devil without end.




Part 23


Article XVIII: _Of Free Will._

The Eighteenth Article, Of Free Will, the adversaries receive,
although they add some testimonies not at all adapted to this case.
They add also a declamation that neither, with the Pelagians, is too
much to be granted to the free will, nor, with the Manicheans, is all
freedom to be denied it.  Very well; but what difference is there
between the Pelagians and our adversaries, since both hold that
without the Holy Ghost men can love God and perform God's
commandments with respect to the substance of the acts, and can merit
grace and justification by works which reason performs by itself,
without the Holy Ghost?  How many absurdities follow from these
Pelagian opinions, which are taught with great authority in the
schools!  These Augustine, following Paul, refutes pith great
emphasis, whose judgment we have recounted above in the article Of
Justification.  (See p. 119 and 153.) Nor, indeed, do we deny liberty
to the human will.  The human will has liberty in the choice of works
and things which reason comprehends by itself.  It can to a certain
extent render civil righteousness or the righteousness of works; it
can speak of God, offer to God a certain service by an outward work,
obey magistrates, parents; in the choice of an outward work it can
restrain the hands from murder, from adultery, from theft.  Since
there is left in human nature reason and judgement concerning objects
subjected to the senses, choice between these things, and the liberty
and power to render civil righteousness, are also left.  For
Scripture calls this the righteousness of the flesh which the carnal
nature, i.e., reason renders by itself, without the Holy Ghost.
Although the power of concupiscence is such that men more frequently
obey evil dispositions than sound judgment.  And the devil, who is
efficacious in the godless, as Paul says Eph. 2, 2, does not cease to
incite this feeble nature to various offenses.  These are the reasons
why even civil righteousness is rare among men, as we see that not
even the philosophers themselves, who seem to have aspired after this
righteousness, attained it.  But it is false to say that he who
performs the works of the commandments without grace does not sin.
And they add further that such works also merit _de congruo_ the
remission of sins and justification.  For human hearts without the
Holy Ghost are without the fear of God; without trust toward God,
they do not believe that they are heard, forgiven, helped, and
preserved by God.  Therefore they are godless.  For neither can a
corrupt tree bring forth good fruit, Matt. 7, 18. And without faith
it is impossible to please God, Heb. 11, 6.

Therefore, although we concede free will the liberty and power to
perform the outward works of the Law, yet we do not ascribe to free
will these spiritual matters, namely, truly to fear God, truly to
believe God, truly to be confident and hold that God regards us,
hears us, forgives us, etc. These are the true works of the First
Table, which the heart cannot render without the Holy Ghost, as Paul
says, 1 Cor. 2, 14: The natural man, i.e., man using only natural
strength, receiveth not the things of the Spirit of God [That is a
person who is not enlightened by the Spirit of God does not, by his
natural reason, receive anything of God's will and divine matters.]
And this can be decided if men consider what their hearts believe
concerning God's will, whether they are truly confident that they are
regarded and heard by God.  Even for saints to retain this faith [and,
as Peter says (1 Ep. 1, 8), to risk and commit himself entirely to
God, whom he does not see, to love Christ, and esteem Him highly,
whom he does not see] is difficult, so far is it from existing in the
godless.  But it is conceived, as we have said above, when terrified
hearts hear the Gospel and receive consolation [when we are born anew
of the Holy Ghost].

Therefore such a distribution is of advantage in which civil
righteousness is ascribed to the free will and spiritual
righteousness to the governing of the Holy Ghost in the regenerate.
For thus the outward discipline is retained, because all men ought to
know equally, both that God requires this civil righteousness [God
will not tolerate indecent, wild, reckless conduct], and that, in a
measure, we can afford it.  And yet a distinction is shown between
human and spiritual righteousness, between philosophical doctrine and
the doctrine of the Holy Ghost and it can be understood for what
there is need of the Holy Ghost.  Nor has this distribution been
invented by us, but Scripture most clearly teaches it.  Augustine
also treats of it, and recently it has been well treated of by
William of Paris, but it has been wickedly suppressed by those who
have dreamt that men can obey God's Law without the Holy Ghost, but
that the Holy Ghost is given in order that, in addition, it may be
considered meritorious.




Part 24


Article XIX: _Of the Cause of Sin._

The Nineteenth Article the adversaries receive, in which we confess
that, although God only and alone has framed all nature, and
preserves all things which exist, yet [He is not the cause of sin,
but] the cause of sin is the will in the devil and men turning itself
away from God, according to the saying of Christ concerning the devil,
John 8, 44: When he speaketh a lie, he speaketh of his own.




Part 25


Article XX: _Of Good Works._

In the Twentieth Article they distinctly lay down these words, namely,
that they reject and condemn our statement that men do not merit the
remission of sins by good works.  [Mark this well!] They clearly
declare that they reject and condemn this article.  What is to be
said on a subject so manifest?  Here the framers of the _Confutation_
openly show by what spirit they are led.  For what in the Church is
more certain than that the remission of sins occurs freely for
Christ's sake, that Christ, and not our works, is the propitiation
for sins, as Peter says, Acts 10, 43: To Him give all the prophets
witness that through His name, whosoever believeth on Him, shall
receive remission of sins?  [This strong testimony of all the holy
prophets may duly be called a decree of the catholic Christian Church.
For even a single prophet is very highly esteemed by God and a
treasure worth the whole world.] To this Church of the prophets we
would rather assent than to these abandoned writers of the
Confutation, who so impudently blaspheme Christ.  For although there
were writers who held that after the remission of sins men are just
before God, not by faith, but by works themselves, yet they did not
hold this, namely, that the remission of sins itself occurs on
account of our works, and not freely for Christ's sake.

Therefore the blasphemy of ascribing Christ's honor to our works is
not to be endured.  These theologians are now entirely without shame
if they dare to bring such an opinion into the Church.  Nor do we
doubt that His Most Excellent Imperial Majesty and very many of the
princes would not have allowed this passage to remain in the
_Confutation_ if they had been admonished of it.  Here we could cite
infinite testimonies from Scripture and from the Fathers [that this
article is certainly divine and true, and this is the sacred and
divine truth.  For there is hardly a syllable, hardly a leaf in the
Bible, in the principal books of the Holy Scriptures where this is
not clearly stated.] But also above we have said enough on this
subject.  And there is no need of more testimonies for one who knows
why Christ has been given to us, who knows that Christ is the
propitiation for our sins.  [God-fearing, pious hearts that know well
why Christ has been given, who for all the possessions and kingdoms
of the world would not be without Christ as our only Treasure, our
only Mediator and Redeemer must here be shocked and terrified that
God's holy Word and Truth should be so openly despised and condemned
by poor men.] Isaiah says, 53, 6: The Lord hath laid on Him the
iniquities of us all.  The adversaries, on the other hand, [accuse
Isaiah and the entire Bible of lying and teach that God lays our
iniquities not on Christ, but on our [beggarly] works.  Neither are
we disposed to mention here the sort of works [rosaries, pilgrimages,
and the like] which they teach.  We see that a horrible decree has
been prepared against us, which would terrify us still more if we
were contending concerning doubtful or trifling subjects.  Now, since
our consciences understand that by the adversaries the manifest truth
is condemned, whose defense is necessary for the Church and increases
the glory of Christ, we easily despise the terrors of the world, and
with a strong spirit will bear whatever is to be suffered for the
glory of Christ and the advantage of the Church.  Who would not
rejoice to die in the confession of such articles as that we obtain
the remission of sins by faith freely for Christ's sake, that we do
not merit the remission of sins by our works?  [Experience shows--and
the monks themselves must admit it--that] The consciences of the
pious will have no sufficiently sure consolation against the terrors
of sin and of death, and against the devil soliciting to despair [and
who in a moment blows away all our works like dust], if they do not
know that they ought to be confident that they have the remission of
sins freely for Christ's sake.  This faith sustains and quickens
hearts in that most violent conflict with despair [in the great agony
of death, in the great anguish, when no creature can help, yea, when
we must depart from this entire visible creation into another state
and world, and must die].

Therefore the cause is one which is worthy that for its sake we
should refuse no danger.  Whosoever you are that has assented to our
Confession, "do not yield to the wicked, but, on the contrary, go
forward the more boldly," when the adversaries endeavor, by means of
terrors and tortures and punishments, to drive away from you that
consolation which has been tendered to the entire Church in this
article of ours [but with all cheerfulness rely confidently and
gladly on God and the Lord Jesus, and joyfully confess this manifest
truth in opposition to the tyranny, wrath, threatening, and terrors
of all the world, yea, in opposition to the daily murders and
persecution of tyrants.  For who would suffer to have taken from him
this great, yea, everlasting consolation on which the entire
salvation of the whole Christian Church depends?  Any one who picks
up the Bible and reads it earnestly will soon observe that this
doctrine has its foundation everywhere in the Bible].  Testimonies of
Scripture will not be wanting to one seeking them, which will
establish his mind.  For Paul at the top of his voice, as the saying
is, cries out, Rom. 3, 24 f., and 4, 16, that sins are freely
remitted for Christ's sake.  It is of faith, he says, that it might
be by grace, to the end the promise might be sure.  That is, if the
promise would depend upon our works, it would not be sure.  If
remission of sins would be given on account of our works, when would
we know that we had obtained it, when would a terrified conscience
find a work which it would consider sufficient to appease God's
wrath?  But we spoke of the entire matter above.  Thence let the
reader derive testimonies.  For the unworthy treatment of the subject
has forced from us the present, not discussion, but complaint that on
this topic they have distinctly recorded themselves as disapproving
of this article of ours, that we obtain remission of sins not on
account of our works, but by faith and freely on account of Christ.

The adversaries also add testimonies to their own condemnation, and
it is worth while to recite several of them.  They quote from Peter,
2. Ep. 1, 10: Give diligence to make your calling sure, etc..  Now
you see, reader, that our adversaries have not wasted labor in
learning logic, but have the art of inferring from the Scriptures
whatever pleases them [whether it is in harmony with the Scriptures
or out of harmony; whether it is correctly or incorrectly concluded.
For they conclude thus:] "Make your calling sure by good works."
Therefore works merit the remission of sins.  A very agreeable mode
of reasoning, if one would argue thus concerning a person sentenced
to capital punishment, whose punishment has been remitted: "The
magistrate commands that hereafter you abstain from that which
belongs to another.  Therefore you have merited the remission of the
penalty, because you are now abstaining from what belongs to another."
Thus to argue is to make a cause out of that which is not a cause.
For Peter speaks of works following the remission of sins, and
teaches why they should be done, namely, that the calling may be sure,
i.e., lest they may fall from their calling if they sin again.  Do
good works that you may persevere in your calling, that you [do not
fall away again, grow cold and] may not lose the gifts of your
calling, which were given you before, and not on account of works
that follow, and which now are retained by faith, for faith does not
remain in those who lose the Holy Ghost, who reject repentance, just
as we have said above (p. 253) that faith exists in repentance.

They add other testimonies cohering no better.  Lastly they say that
this opinion was condemned a thousand years before, in the time of
Augustine.  This also is quite false.  For the Church of Christ
always held that the remission of sins is obtained freely.  Yea, the
Pelagians were condemned, who contended that grace is given on
account of our works.  Besides, we have above shown sufficiently that
we hold that good works ought necessarily to follow faith.  For we do
not make void the Law, says Paul, Rom. 3, 31; yea, we establish the
Law, because when by faith we have received the Holy Ghost, the
fulfilling of the Law necessarily follows, by which love, patience,
chastity, and other fruits of the Spirit gradually grow.




Part 26


The Twenty-first Article they absolutely condemn, because we do not
require the invocation of saints.  Nor on any topic do they speak
more eloquently and with more prolixity.  Nevertheless they do not
effect anything else than that the saints should be honored; likewise,
that the saints who live pray for others; as though, indeed, the
invocation of dead saints were on that account necessary.  They cite
Cyprian, because he asked Cornelius while yet alive to pray for his
brothers when departing.  By this example they prove the invocation
of the dead.  They quote also Jerome against Vigilantius.  "On this
field" [in this matter], they say, "eleven hundred years ago, Jerome
overcame Vigilantius." Thus the adversaries triumph, as though the
war were already ended.  Nor do those asses see that in Jerome,
against Vigilantius, there is not a syllable concerning invocation.
He speaks concerning honors for the saints, not concerning invocation.
Neither have the rest of the ancient writers before Gregory made
mention of invocation.  Certainly this invocation, with these
opinions which the adversaries now teach concerning the application
of merits, has not the testimonies of the ancient writers.

Our Confession approves honors to the saints.  For here a threefold
honor is to be approved.  The first is thanksgiving.  For we ought to
give thanks to God because He has shown examples of mercy, because He
has shown that He wishes to save men; because He has given teachers
or other gifts to the Church.  And these gifts, as they are the
greatest, should be amplified, and the saints themselves should be
praised, who have faithfully used these gifts, just as Christ praises
faithful business-men, Matt. 25, 21. 23. The second service is the
strengthening of our faith when we see the denial forgiven Peter we
also are encouraged to believe the more that grace truly superabounds
over sin, Rom. 5, 20. The third honor is the imitation, first, of
faith, then of the other virtues which every one should imitate
according to his calling.  These true honors the adversaries do not
require.  They dispute only concerning invocation, which, even though
it would have no danger, nevertheless is not necessary.

Besides, we also grant that the angels pray for us.  For there is a
testimony in Zech. 1, 12, where an angel prays: O Lord of hosts, how
long wilt Thou not have mercy on Jerusalem?  Although concerning the
saints we concede that, just as, when alive, they pray for the Church
universal in general, so in heaven they pray for the Church in
general, albeit no testimony concerning the praying of the dead is
extant in the Scriptures, except the dream taken from the Second Book
of Maccabees, 15, 14.

Moreover, even supposing that the saints pray for the Church ever so
much, yet it does not follow that they are to be invoked; although
our Confession affirms only this, that Scripture does not teach the
invocation of the saints, or that we are to ask the saints for aid.
But since neither a command, nor a promise, nor an example can be
produced from the Scriptures concerning the invocation of saints, it
follows that conscience can have nothing concerning this invocation
that is certain.  And since prayer ought to be made from faith, how
do we know that God approves this invocation?  Whence do we know
without the testimony of Scripture that the saints perceive the
prayers of each one?  Some plainly ascribe divinity to the saints
namely, that they discern the silent thoughts of the minds in us.
They dispute concerning morning and evening knowledge, perhaps
because they doubt whether they hear us in the morning or the evening.
They invent these things, not in order to treat the saints with
honor, but to defend lucrative services.  Nothing can be produced by
the adversaries against this reasoning, that, since invocation does
not have a testimony from God's Word, it cannot be affirmed that the
saints understand our invocation, or, even if they understand it,
that God approves it.  Therefore the adversaries ought not to force
us to an uncertain matter, because a prayer without faith is not
prayer.  For when they cite the example of the Church, it is evident
that this is a new custom in the Church; for although the old prayers
make mention of the saints, yet they do not invoke the saints.
Although also this new invocation in the Church is dissimilar to the
invocation of individuals.

Again, the adversaries not only require invocation in the worship of
the saints, but also apply the merits of the saints to others, and
make of the saints not only intercessors, but also propitiators.
This is in no way to be endured.  For here the honor belonging only
to Christ is altogether transferred to the saints.  For they make
them mediators and propitiators, and although they make a distinction
between mediators of intercession and mediators [the Mediator] of
redemption, yet they plainly make of the saints mediators of
redemption.  But even that they are mediators of intercession they
declare without testimony of Scripture, which, be it said ever so
reverently, nevertheless obscures Christ's office, and transfers the
confidence of mercy due Christ to the saints.  For men imagine that
Christ is more severe and the saints more easily appeased, and they
trust rather to the mercy of the saints than to the mercy of Christ,
and fleeing from Christ [as from a tyrant], they seek the saints.
Thus they actually make of them mediators of redemption.

Therefore we shall show that they truly make of the saints, not only
intercessors, but propitiators, i.e., mediators of redemption.  Here
we do not as yet recite the abuses of the common people [how manifest
idolatry is practiced at pilgrimages].  We are still speaking of the
opinions of the Doctors.  As regards the rest, even the inexperienced
[common people] can judge.

In a propitiator these two things concur.  In the first place, there
ought to be a word of God from which we may certainly know that God
wishes to pity, and hearken to, those calling upon Him through this
propitiator.  There is such a promise concerning Christ, John 16 23:
Whatsoever ye shall ask the Father in My name, He will give it you.
Concerning the saints there is no such promise.  Therefore
consciences cannot be firmly confident that by the invocation of
saints we are heard.  This invocation, therefore, is not made from
faith.  Then we have also the command to call upon Christ, according
to Matt. 11, 28: Come unto Me, all ye that labor, etc., which
certainly is said also to us.  And Isaiah says, 11,10: In that day
there shall be a root of Jesse, which shall stand for an ensign to
the people; to it shall the Gentiles seek.  And Ps. 45, 12: Even the
rich among the people shall entreat Thy favor.  And Ps. 72, 11. 16:
Yea, all kings shall fall down before Him.  And shortly after: Prayer
also shall be made for Him continually.  And in John 6, 23 Christ
says: That all men should honor the Son even as they honor the Father.
And Paul, 2 Thess. 2, 16. 17, says, praying: Now our Lord Jesus
Christ Himself, and God, even our Father,... comfort your hearts and
stablish you.  [All these passages refer to Christ.] But concerning
the invocation of saints, what commandment, what example can the
adversaries produce from the Scriptures?  The second matter in a
propitiator is, that his merits have been presented as those which
make satisfaction for others, which are bestowed by divine imputation
on others, in order that through these, just as by their own merits,
they may be accounted righteous.  As when any friend pays a debt for
a friend, the debtor is freed by the merit of another, as though it
were by his own.  Thus the merits of Christ are bestowed upon us, in
order that, when we believe in Him, we may be accounted righteous by
our confidence in Christ's merits as though we had merits of our own.

And from both, namely, from the promise and the bestowment of merits,
confidence in mercy arises [upon both parts must a Christian prayer
be founded].  Such confidence in the divine promise, and likewise in
the merits of Christ, ought to be brought forward when we pray.  For
we ought to be truly confident, both that for Christ's sake we are
heard, and that by His merits we have a reconciled Father.

Here the adversaries first bid us invoke the saints, although they
have neither God's promise, nor a command, nor an example from
Scripture.  And yet they cause greater confidence in the mercy of the
saints to be conceived than in that of Christ, although Christ bade
us come to Him and not to the saints.  Secondly, they apply the
merits of the saints, just as the merits of Christ, to others, they
bid us trust in the merits of the saints as though we were accounted
righteous on account of the merits of the saints, in like manner as
we are accounted righteous by the merits of Christ.  Here we
fabricate nothing.  In indulgences they say that they apply the
merits of the saints [as satisfactions for our sins].  And Gabriel,
the interpreter of the canon of the Mass, confidently declares:
According to the order instituted by God we should betake ourselves
to the aid of the saints, in order that we may be saved by their
merits and vows.  These are the words of Gabriel.  And nevertheless
in the books and sermons of the adversaries still more absurd things
are read here and there.  What is it to make propitiators if this is
not?  They are altogether made equal to Christ if we must trust that
we are saved by their merits.

But where has this arrangement, to which he refers when he says that
we ought to resort to the aid of the saints, been instituted by God?
Let him produce an example or command from the Scriptures.  Perhaps
they derive this arrangement from the courts of kings, where friends
must be employed as intercessors.  But if a king has appointed a
certain intercessor, he will not desire that eases be brought to him
through others.  Thus, since Christ has been appointed Intercessor
and High Priest, why do we seek others?  [What can the adversaries
say in reply to this?]

Here and there this form of absolution is used: The passion of our
lord Jesus Christ the merits of the most blessed Virgin Mary and of
all the saints, be to thee for the remission of sins.  Here the
absolution is pronounced on the supposition that we are reconciled
and accounted righteous not only by the merits of Christ, but also by
the merits of the other saints.  Some of us have seen a doctor of
theology dying, for consoling whom a certain theologian, a monk, was
employed.  He pressed on the dying man nothing but this prayer:
Mother of grace, protect us from the enemy; receive us in the hour of
death.

Granting that the blessed Mary prays for the Church, does she receive
souls in death, does she conquer death [the great power of Satan],
does she quicken?  What does Christ do if the blessed Mary does these
things?  Although she is most worthy of the most ample honors,
nevertheless she does not wish to be made equal to Christ, but rather
wishes us to consider and follow her example [the example of her
faith and her humility].  But the subject itself declares that in
public opinion the blessed Virgin has succeeded altogether to the
place of Christ.  Men have invoked her, have trusted in her mercy,
through her have desired to appease Christ, as though He were not a
Propitiator, but only a dreadful judge and avenger.  We believe,
however, that we must not trust that the merits of the saints are
applied to us, that on account of these God is reconciled to us, or
accounts us just, or saves us.  For we obtain remission of sins only
by the merits of Christ, when we believe in Him.  Of the other saints
it has been said, 1 Cor. 3, 8: Every man shall receive his own reward
according to his own labor, i.e., they cannot mutually bestow their
own merits, the one upon the other, as the monks sell the merits of
their orders.  Even Hilary says of the foolish virgins: And as the
foolish virgins could not go forth with their lamps extinguished,
they besought those who were prudent to lend them oil; to whom they
replied that they could not give it because peradventure there might
not be enough for all; i.e., no one can be aided by the works and
merits of another, because it is necessary for every one to buy oil
for his own lamp.  [Here he points out that none of us can aid
another by other people's works or merits.]

Since, therefore, the adversaries teach us to place confidence in the
invocation of saints, although they have neither the Word of God nor
the example of Scripture [of the Old or of the New Testament]; since
they apply the merits of the saints on behalf of others, not
otherwise than they apply the merits of Christ, and transfer the
honor belonging only to Christ to the saints, we can receive neither
their opinions concerning the worship of the saints, nor the practise
of invocation.  For we know that confidence is to be placed in the
intercession of Christ, because this alone has God's promise.  We
know that the merits of Christ alone are a propitiation for us.  On
account of the merits of Christ we are accounted righteous when we
believe in Him, as the text says, Rom. 9, 33 (cf. 1 Pet. 2, 6 and Is.
28, 16): Whosoever believeth on Him shall not be confounded.  Neither
are we to trust that we are accounted righteous by the merits of the
blessed Virgin or of the other saints.

With the learned this error also prevails namely, that to each saint
a particular administration has been committed, that Anna bestows
riches [protects from poverty], Sebastian keeps off pestilence,
Valentine heals epilepsy, George protects horsemen.  These opinions
have clearly sprung from heathen examples.  For thus, among the
Romans Juno was thought to enrich, Febris to keep off fever, Castor
and Pollux to protect horsemen, etc. Even though we should imagine
that the invocation of saints were taught with the greatest prudence,
yet since the example is most dangerous, why is it necessary to
defend it when it has no command or testimony from God's Word?  Aye,
it has not even the testimony of the ancient writers.  First because,
as I have said above, when other mediators are sought in addition to
Christ, and confidence is put in others, the entire knowledge of
Christ is suppressed.  The subject shows this.  In the beginning,
mention of the saints seems to have been admitted with a design that
is endurable, as in the ancient prayers.  Afterwards invocation
followed, and abuses that are prodigious and more than heathenish
followed invocation.  From invocation the next step was to images;
these also were worshiped, and a virtue was supposed to exist in
these, just as magicians imagine that a virtue exists in images of
the heavenly bodies carved at a particular time.  In a certain
monastery we [some of us] have seen a statue of the blessed Virgin,
which moved automatically by a trick [within by a string], so as to
seem either to turn away from [those who did not make a large
offering] or nod to those making request.

Still the fabulous stories concerning the saints, which are publicly
taught with great authority, surpass the marvelous tales of the
statues and pictures.  Barbara, amidst her torments, asks for the
reward that no one who would invoke her should die without the
Eucharist.  Another, standing on one foot, recited daily the whole
psaltery.  Some wise man painted [for children] Christophorus [which
in German means Bearer of Christ], in order by the allegory to
signify that there ought to be great strength of mind in those who
would bear Christ, i.e., who would teach or confess the Gospel,
because it is necessary to undergo the greatest dangers [for they
must wade by night through the great sea, i.e., endure all kinds of
temptations and dangers].  Then the foolish monks taught among the
people that they ought to invoke Chistophorus, as though such a
Polyphemus [such a giant who bore Christ through the sea] had once
existed.  And although the saints performed very great deeds, either
useful to the state or affording private examples the remembrance of
which would conduce much both toward strengthening faith and toward
following their example in the administration of affairs, no one has
searched for these from true narratives.  [Although God Almighty
through His saints, as a peculiar people, has wrought many great
things in both realms, in the Church and in worldly transactions;
although there are many great examples in the lives of the saints
which would be very profitable to princes and lords, to true pastors
and guardians of souls, for the government both of the world and of
the Church, especially for strengthening faith in God, yet they have
passed these by, and preached the most insignificant matters
concerning the saints, concerning their hard beds their hair shirts,
etc., which, for the greater part, are falsehoods.] Yet indeed it is
of advantage to hear how holy men administered governments [as in the
Holy Scriptures it is narrated of the kings of Israel and Judah],
what calamities, what dangers they underwent, how holy men were of
aid to kings in great dangers, how they taught the Gospel, what
encounters they had with heretics.  Examples of mercy are also of
service, as when we see the denial forgiven Peter, when we see
Cyprian forgiven for having been a magician, when we see Augustine,
having experienced the power of faith in sickness steadily affirming
that God truly hears the prayers of believers.  It was profitable
that such examples as these, which contain admonitions for either
faith or fear or the administration of the state, be recited.  But
certain triflers, endowed with no knowledge either of faith or for
governing states, have invented stories in imitation of poems, in
which there are nothing but superstitious examples concerning certain
prayers, certain fastings, and certain additions of service for
bringing in gain [where there are nothing but examples as to how the
saints wore hair shirts, how they prayed at the seven canonical hours
how they lived upon bread and water].  Such are the miracles that
have been invented concerning rosaries and similar ceremonies.  Nor
is there need here to recite examples.  For the legends, as they call
them, and the mirrors of examples, and the rosaries, in which there
are very many things not unlike the true narratives of Lucian, are
extant.

The bishops, theologians, and monks applaud these monstrous and
wicked stories [this abomination set up against Christ, this
blasphemy, these scandalous, shameless lies, these lying preachers;
and they have permitted them so long, to the great injury of
consciences, that it is terrible to think of it] because they aid
them to their daily bread.  They do not tolerate us, who, in order
that the honor and office of Christ may be more conspicuous, do not
require the invocation of saints, and censure the abuses in the
worship of saints.  And although [even their own theologians], all
good men everywhere [a long time before Dr. Luther began to write] in
the correction of these abuses, greatly longed for either the
authority of the bishops or the diligence of the preachers,
nevertheless our adversaries in the _Confutation_ altogether pass
over vices that are even manifest, as though they wish, by the
reception of the Confutation, to compel us to approve even the most
notorious abuses.

Thus the _Confutation_ has been deceitfully written, not only on this
topic, but almost everywhere.  [They pretend that they are as pure as
gold, that they have never muddled the water.] There is no passage in
which they make a distinction between the manifest abuses and their
dogmas.  And nevertheless, if there are any of sounder mind among
them they confess that many false opinions inhere in the doctrine of
the scholastics and canonists, and, besides, that in such ignorance
and negligence of the pastors many abuses crept into the Church.  For
Luther was not [the only one nor] the first to complain of
[innumerable] public abuses.  Many learned and excellent men long
before these times deplored the abuses of the Mass, confidence in
monastic observances, services to the saints intended to yield a
revenue, the confusion of the doctrine concerning repentance
[concerning Christ], which ought to be as clear and plain in the
Church as possible [without which there cannot be nor remain a
Christian Church].  We ourselves have heard that excellent
theologians desire moderation in the scholastic doctrine which
contains much more for philosophical quarrels than for piety.  And
nevertheless, among these the older ones are generally nearer
Scripture than are the more recent.  Thus their theology degenerated
more and more.  Neither had many good men, who from the very first
began to be friendly to Luther, any other reason than that they saw
that he was freeing the minds of men from these labyrinths of most
confused and infinite discussions which exist among the scholastic
theologians and canonists, and was teaching things profitable for
godliness.

The adversaries, therefore, have not acted candidly in passing over
the abuses when they wished us to assent to the Confutation.  And if
they wished to care for the interests of the Church [and of Buffeted
consciences, and not rather to maintain their pomp and avarice]
especially on that topic, at this occasion they ought to exhort our
most excellent Emperor to take measures for the correction of abuses
[which furnish grounds for derision among the Turks, the Jews, and
all unbelievers], as we observe plainly enough that he is most
desirous of healing and well establishing the Church.  But the
adversaries do not act as to aid the most honorable and most holy
will of the Emperor, but so as in every way to crush [the truth and]
us.  Many signs show that they have little anxiety concerning the
state of the Church.  [They lose little sleep from concern that
Christian doctrine and the pure Gospel be preached.] They take no
pains that there should be among the people a summary of the dogmas
of the Church.  [The office of the ministry they permit to be quite
desolate.] They defend manifest abuses [they continue every day to
shed innocent blood] by new and unusual cruelty.  They allow no
suitable teachers in the churches.  Good men can easily judge whither
these things tend.  But in this way they have no regard to the
interest either of their own authority or of the Church.  For after
the good teachers have been killed and sound doctrine suppressed,
fanatical spirits will rise up, whom the adversaries will not be able
to restrain, who both will disturb the Church with godless dogmas,
and will overthrow the entire ecclesiastical government, which we are
very greatly desirous of maintaining.

Therefore, most excellent Emperor Charles for the sake of the glory
of Christ, which we have no doubt that you desire to praise and
magnify, we beseech you not to assent to the violent counsels of our
adversaries, but to seek other honorable ways of so establishing
harmony that godly consciences are not burdened, that no cruelty is
exercised against innocent men, as we have hitherto seen, and that
sound doctrine is not suppressed in the Church.  To God most of all
you owe the duty [as far as this is possible to man] to maintain
sound doctrine and hand it down to posterity, and to defend those who
teach what is right.  For God demands this when He honors kings with
His own name and calls them gods, saying, Ps. 82, 6: I have said, Ye
are gods, namely, that they should attend to the preservation and
propagation of divine things, i.e., the Gospel of Christ, on the
earth, and, as the vicars of God, should defend the life and safety
of the innocent [true Christian teachers and preachers].




Part 27


Article XXII (X): _Of Both Kinds in the Lord's Supper._

It cannot be doubted that it is godly and in accordance with the
institution of Christ and the words of Paul to use both parts in the
Lord's Supper.  For Christ instituted both parts, and instituted them
not for a part of the Church, but for the entire Church.  For not
only the presbyters, but the entire Church uses the Sacrament by the
authority of Christ, and not by human authority, and this, we suppose,
the adversaries acknowledge.  Now, if Christ has instituted it for
the entire Church, why is one kind denied to a part of the Church?
Why is the use of the other kind prohibited?  Why is the ordinance of
Christ changed, especially when He Himself calls it His testament?
But if it is not allowable to annul man's testament, much less will
it be allowable to annul the testament of Christ.  And Paul says, 1
Cor. 11, 23 ff., that he had received of the Lord that which he
delivered.  But he had delivered the use of both kinds, as the text,
1 Cor. 11, clearly shows.  This do [in remembrance of Me], he says
first concerning His body; afterwards he repeats the same words
concerning the cup [the blood of Christ].  And then: Let a man
examine himself, and so let him eat of that bread and drink of that
cup.  [Here he names both.] These are the words of Him who has
instituted the Sacrament.  And, indeed, he says before that those who
will use the Lord's Supper should use both.  It is evident, therefore,
that the Sacrament was instituted for the entire Church.  And the
custom still remains in the Greek churches, and also once obtained in
the Latin churches, as Cyprian and Jerome testify.  For thus Jerome
says on Zephaniah: The priests who administer the Eucharist, and
distribute the Lord's blood to the people, etc. The Council of Toledo
gives the same testimony.  Nor would it be difficult to accumulate a
great multitude of testimonies.  Here we exaggerate nothing; we but
leave the prudent reader to determine what should be held concerning
the divine ordinance [whether it is proper to prohibit and change an
ordinance and institution of Christ].

The adversaries in the _Confutation_ do not endeavor to [comfort the
consciences or] excuse the Church, to which one part of the Sacrament
has been denied.  This would have been becoming to good and religious
men.  For a strong reason for excusing the Church, and instructing
consciences to whom only a part of the Sacrament could be granted,
should have been sought.  Now these very men maintain that it is
right to prohibit the other part, and forbid that the use of both
parts be allowed.  First, they imagine that, in the beginning of the
Church, it was the custom at some places that only one part was
administered.  Nevertheless they are not able to produce any ancient
example of this matter.  But they cite the passages in which mention
is made of bread, as in Luke 24, 35 where it is written that the
disciples recognized Christ in the breaking of bread.  They quote
also other passages, Acts 2, 42. 46; 20, 7, concerning the breaking
of bread.  But although we do not greatly oppose if some receive
these passages as referring to the Sacrament, yet it does not follow
that one part only was given, because, according to the ordinary
usage of language, by the naming of one part the other is also
signified.  They refer also to Lay Communion which was not the use of
only one kind, but of both; and whenever priests are commanded to use
Lay Communion [for a punishment are not to consecrate themselves, but
to receive Communion, however, of both kinds from another], it is
meant that they have been removed from the ministry of consecration.
Neither are the adversaries ignorant of this, but they abuse the
ignorance of the unlearned, who, when they hear of Lay Communion,
immediately dream of the custom of our time, by which only a part of
the Sacrament is given to the laymen.

And consider their impudence.  Gabriel recounts among other reasons
why both parts are not given that a distinction should be made
between laymen and presbyters.  And it is credible that the chief
reason why the prohibition of the one part is defended is this,
namely, that the dignity of the order may be the more highly exalted
by a religious rite.  To say nothing more severe, this is a human
design; and whither this tends can easily be judged.  In the
_Confutation_ they also quote concerning the sons of Eli that after
the loss of the high-priesthood, they were to seek the one part
pertaining to the priests, 1 Sam. 2, 36 [the text reads: Every one
that is left in thine house shall come and crouch him for a piece of
silver and a morsel of bread, and shall say, Put me, I pray thee,
into one of the priest's offices (German: _Lieber, lass mich zu einem
Priesterteil_) that I may eat a piece of bread].  Here they say that
the use of one kind was signified.  And they add: "Thus, therefore,
our laymen ought also to be content, with one part pertaining to the
priests, with one kind." The adversaries [the masters of the
_Confutation_ are quite shameless, rude asses, and] are clearly
trifling when they are transferring the history of the posterity of
Eli to the Sacrament.  The punishment of Eli is there described.
Will they also say this, that as a punishment the laymen have been
removed from the other party [They are quite foolish and mad.] The
Sacrament was instituted to console and comfort terrified minds when
they believe that the flesh of Christ given for the life of the world,
is food, when they believe that, being joined to Christ [through
this food], they are made alive.  But the adversaries argue that
laymen are removed from the other part as a punishment.  "They ought,"
they say, "to be content." This is sufficient for a despot.  [That,
surely, sounds proud and defiant enough.] But [my lords, may we ask
the reason] why ought they?  "The reason must not be asked but let
whatever the theologians say be law." [Is whatever you wish and
whatever you say to be sheer truth?  See now and be astonished how
shameless and impudent the adversaries are: they dare to set up their
own words as sheer commands of lords, they frankly say: The laymen
must be content.  But what if they must not?] This is a concoction of
Eck.  For we recognize those vainglorious words, which if we would
wish to criticize, there would be no want of language.  For you see
how great the impudence is.  He commands, as a tyrant in the
tragedies: "Whether they wish or not, they must be content." Will the
reasons which he cites excuse, in the judgment of God, those who
prohibit a part of the Sacrament, and rage against men using an
entire Sacrament?  [Are they to take comfort in the fact that it is
recorded concerning the sons of Eli: They will go begging?  That will
be a shuffling excuse at the judgment-seat of God.] If they make the
prohibition in order that there should be a distinguishing mark of
the order, this very reason ought to move us not to assent to the
adversaries, even though we would be disposed in other respects to
comply with their custom.  There are other distinguishing marks of
the order of priests and of the people, but it is not obscure what
design they have for defending this distinction so earnestly.  That
we may not seem to detract from the true worth of the order, we will
not say more concerning this shrewd design.

They also allege the danger of spilling and certain similar things,
which do not have force sufficient to change the ordinance of Christ.
[They allege more dreams like these for the sake of which it would
be improper to change the ordinance of Christ.] And, indeed, if we
assume that we are free to use either one part or both, how can the
prohibition [to use both kinds] be defended?  Although the Church
does not assume to itself the liberty to convert the ordinances of
Christ into matters of indifference.  We indeed excuse the Church
which has borne the injury [the poor consciences which have been
deprived of one part by force], since it could not obtain both parts;
but the authors who maintain that the use of the entire Sacrament is
justly prohibited, and who now not only prohibit, but even
excommunicate and violently persecute those using an entire Sacrament,
we do not excuse.  Let them see to it how they will give an account
to God for their decisions.  Neither is it to be judged immediately
that the Church determines or approves whatever the pontiffs
determine, especially since Scripture prophesies concerning the
bishops and pastors to effect this as Ezekiel says, 7, 28: The Law
shall perish from the priest [there will be priests or bishops who
will know no command or law of God].




Part 28


Article XXIII (XI): _Of the Marriage of Priests._


Despite the great infamy of their defiled celibacy, the adversaries
have the presumption not only to defend the pontifical law by the
wicked and false pretext of the divine name, but even to exhort the
Emperor and princes, to the disgrace and infamy of the Roman Empire,
not to tolerate the marriage of priests.  For thus they speak.
[Although the great, unheard-of lewdness, fornication, and adultery
among priests, monks, etc., at the great abbeys, in other churches
and cloisters, has become so notorious throughout the world that
people sing and talk about it, still the adversaries who have
presented the _Confutation_ are so blind and without shame that they
defend the law of the Pope by which marriage is prohibited, and that,
with the specious claim that they are defending a spiritual state.
Moreover, although it would be proper for them to be heartily ashamed
of the exceedingly shameful, lewd, abandoned loose life of the
wretches in their abbeys and cloisters, although on this account
alone they should not have the courage to show their face in broad
daylight, although their evil, restless heart and conscience ought to
cause them to tremble, to stand aghast, and to be afraid to lift
their eyes to our excellent Emperor, who loves uprightness, still
they have the courage of the hangman, they act like the very devil
and like all reckless, wanton people, proceeding in blind defiance
and forgetful of all honor and decency.  And these pure chaste
gentlemen dare to admonish His Imperial Majesty, the Electors and
Princes not to tolerate the marriage of priests _ad infamiam et
ignominiam imperti_, that is, to ward off shame and disgrace from the
Roman Empire.  For these are their words, as if their shameful life
were a great honor and glory to the Church.]

What greater impudence has ever been read of in any history than this
of the adversaries?  [Such shameless advocates before a Roman Emperor
will not easily be found.  If all the world did not know them, if
many godly, upright people among them, their own canonical brethren,
had not complained long ago of their shameful, lewd, indecent conduct,
if their vile, abominable, ungodly, lewd, heathenish, Epicurean life,
and the dregs of all filthiness at Rome were not quite manifest, one
might think that their great purity and their inviolate virgin
chastity were the reason why they could not bear to hear the word
woman or marriage pronounced, and why they baptize holy matrimony,
which the Pope himself calls a sacrament, _infamiam imperil_.] For
the arguments which they use we shall afterwards review.  Now let the
wise reader consider this, namely, what shame these good-for-nothing
men have who say that marriages [which the Holy Scriptures praise
most highly and command] produce infamy and disgrace to the
government, as though, indeed, this public infamy of flagitious and
unnatural lusts which glow among these very holy fathers, who feign
that they are Curii and live like bacchanals, were a great ornament
to the Church!  And most things which these men do with the greatest
license cannot even be named without a breach of modesty.  And these
their lusts they ask you to defend with your chaste right hand,
Emperor Charles (whom even certain ancient predictions name as the
king of modest face, for the saying appears concerning you: "One
modest in face shall reign everywhere").  For they ask that, contrary
to divine law, contrary to the law of nations, contrary to the canons
of Councils you sunder marriages, in order to impose merely for the
sake of marriage atrocious punishments upon innocent men, to put to
death priests, whom even barbarians reverently spare, to drive into
exile banished women and fatherless children.  Such laws they bring
to you, most excellent and most chaste Emperor, to which no barbarity,
however monstrous and cruel, could lend its ear.  But because the
stain of no disgrace or cruelty falls upon your character, we hope
that you will deal with us mildly in this matter, especially when you
have learned that we have the weightiest reasons for our belief
derived from the Word of God to which the adversaries oppose the most
trifling and vain opinions.

And nevertheless they do not seriously defend celibacy.  For they are
not ignorant how few there are who practise chastity, but [they stick
to that comforting saying which is found in their treatise, _Si non
caste, tamen caue_ (If not chastely, at least cautiously) and] they
devise a sham of religion for their dominion, which they think that
celibacy profits, in order that we may understand Peter to have been
right in admonishing, 2 Ep. 2, 1, that there will be false teachers
who will deceive men with feigned words.  For the adversaries say,
write, or do nothing truly [their words are merely an argument _ad
hominem_], frankly, and candidly in this entire case, but they
actually contend only concerning the dominion which they falsely
think to be imperiled, and which they endeavor to fortify with a
wicked pretense of godliness [they support their case with nothing
but impious, hypocritical lies; accordingly, it will endure about as
well as butter exposed to the sun].

We cannot approve this law concerning celibacy which the adversaries
defend, because it conflicts with divine and natural law and is at
variance with the very canons of the Councils.  And that it is
superstitious and dangerous is evident.  For it produces infinite
scandals, sins, and corruption of public morals [as is seen in the
real towns of priests, or, as they are called, their residences].
Our other controversies need some discussion by the doctors; in this
the subject is so manifest to both parties that it requires no
discussion.  It only requires as judge a man that is honest and fears
God.  And although the manifest truth is defended by us, yet the
adversaries have devised certain reproaches for satirizing our
arguments.

First.  Gen. 1, 28 teaches that men were created to be fruitful, and
that one sex in a proper way should desire the other.  For we are
speaking not of concupiscence, which is sin, but of that appetite
which was to have been in nature in its integrity [which would have
existed in nature even if it had remained uncorrupted], which they
call physical love.  And this love of one sex for the other is truly
a divine ordinance.  But since this ordinance of God cannot be
removed without an extraordinary work of God, it follows that the
right to contract marriage cannot be removed by statutes or vows.

The adversaries cavil at these arguments; they say that in the
beginning the commandment was given to replenish the earth but that
now since the earth has been replenished, marriage is not commanded.
See how wisely they judge!  The nature of men is so formed by the
word of God that it is fruitful not only in the beginning of the
creation, but as long as this nature of our bodies will exist just as
the earth becomes fruitful by the word Gen. 1, 11: Let the earth
bring forth grass, yielding seed.  Because of this ordinance the
earth not only commenced in the beginning to bring forth plants, but
the fields are clothed every year as long as this natural order will
exist.  Therefore, just as by human laws the nature of the earth
cannot be changed, so, without a special work of God the nature of a
human being can be changed neither by vows nor by human law [that a
woman should not desire a man, nor a man a woman].

Secondly.  And because this creation or divine ordinance in man is a
natural right, jurists have accordingly said wisely and correctly
that the union of male and female belongs to natural right.  But
since natural right is immutable, the right to contract marriage must
always remain.  For where nature does not change, that ordinance also
with which God has endowed nature does not change, and cannot be
removed by human laws.  Therefore it is ridiculous for the
adversaries to prate that marriage was commanded in the beginning,
but is not now.  This is the same as if they would say: Formerly,
when men were born, they brought with them sex; now they do not.
Formerly, when they were born, they brought with them natural right,
now they do not.  No craftsman (Faber) could produce anything more
crafty than these absurdities, which were devised to elude a right of
nature.  Therefore let this remain in the case which both Scripture
teaches and the jurist says wisely, namely, that the union of male
and female belongs to natural right.  Moreover, a natural right is
truly a divine right, because it is an ordinance divinely impressed
upon nature.  But inasmuch as this right cannot be changed without an
extraordinary work of God, it is necessary that the right to contract
marriage remains, because the natural desire of sex for sex is an
ordinance of God in nature, and for this reason is a right; otherwise,
why would both sexes have been created?  And we are speaking, as it
has been said above, not of concupiscence, which is sin, but of that
desire which they call physical love [which would have existed
between man and woman even though their nature had remained pure],
which concupiscence has not removed from nature, but inflames, so
that now it has greater need of a remedy, and marriage is necessary
not only for the sake of procreation, but also as a remedy [to guard
against sins].  These things are clear, and so well established that
they can in no way be overthrown.

Thirdly.  Paul says, 1 Cor. 7, 2: To avoid fornication, let every man
have his own wife.  This now is an express command pertaining to all
who are not fit for celibacy.  The adversaries ask that a commandment
be shown them which commands priests to marry.  As though priests are
not men!  We judge indeed that the things which we maintain
concerning human nature in general pertain also to priests.  Does not
Paul here command those who have not the gift of continence to marry?
For he interprets himself a little after when he says, v. 9: It is
better to marry than to burn.  And Christ has clearly said Matt. 19,
11: All men cannot receive this saying, save they to whom it is given.
Because now, since sin [since the fall of Adam], these two things
concur, namely, natural appetite and concupiscence, which inflames
the natural appetite, so that now there is more need of marriage than
in nature in its integrity, Paul accordingly speaks of marriage as a
remedy, and on account of these flames commands to marry.  Neither
can any human authority, any law, any vows remove this declaration:
It is better to marry than to burn, because they do not remove the
nature or concupiscence.  Therefore all who burn, retain the right to
marry.  By this commandment of Paul: To avoid fornication, let every
man have his own wife, all are held bound who do not truly keep
themselves continent; the decision concerning which pertains to the
conscience of each one.

For as they here give the command to seek continence of God, and to
weaken the body by labors and hunger, why do they not proclaim these
magnificent commandments to themselves?  But, as we have said above,
the adversaries are only playing; they are doing nothing seriously.
If continence were possible to all, it would not require a peculiar
gift.  But Christ shows that it has need of a peculiar gift;
therefore it does not belong to all.  God wishes the rest to use the
common law of nature which He has instituted.  For God does not wish
His ordinances, His creations to be despised.  He wishes men to be
chaste in this way, that they use the remedy divinely presented, just
as He wishes to nourish our life in this way, that we use food and
drink.  Gerson also testifies that there have been many good men who
endeavored to subdue the body, and yet made little progress.
Accordingly, Ambrose is right in saying: Virginity is only a thing
that can be recommended, but not commanded; it is a matter of vow
rather than of precept.  If any one here would raise the objection
that Christ praises those which have made themselves eunuchs for the
kingdom of heaven's sake, Matt. 19, 12, let him also consider this,
that He is praising such as have the gift of continence, for on this
account He adds: He that is able to receive it, let him receive it.
For an impure continence [such as there is in monasteries and
cloisters] does not please Christ.  We also praise true continence.
But now we are disputing concerning the law and concerning those who
do not have the gift of continence.  The matter ought to be left free
and snares ought not to be cast upon the weak through this law.

Fourthly.  The pontifical law differs also from the canons of the
Councils.  For the ancient canons do not prohibit marriage, neither
do they dissolve marriages that have been contracted, even if they
remove from the administration of their office those who have
contracted them in the ministry.  At those times this dismissal was
an act of kindness [rather than a punishment].  But the new canons,
which have not been framed in the Synods, but have been made
according to the private judgment of the Popes, both prohibit the
contraction of marriages, and dissolve them when contracted; and this
is to be done openly, contrary to the command of Christ, Matt. 19, 6:
What God hath joined together, let not man put asunder.  In the
_Confutation_ the adversaries exclaim that celibacy has been
commanded by the Councils.  We do not find fault with the decrees of
the Councils; for under a certain condition these allow marriage, but
we find fault with the laws which, since the ancient Synods, the
Popes of Rome have framed contrary to the authority of the Synods.
The Popes despise the authority of the Synods, just as much as they
wish it to appear holy to others [under peril of God's wrath and
eternal damnation].  Therefore this law concerning perpetual celibacy
is peculiar to this new pontifical despotism.  Nor is it without a
reason.  For Daniel, 11, 37, ascribes to the kingdom of Antichrist
this mark, namely, the contempt of women.

Fifthly.  Although the adversaries do not defend the law because of
superstition, [not because of its sanctity, as from ignorance], since
they see that it is not generally observed, nevertheless they diffuse
superstitious opinions, while they give a pretext of religion.  They
proclaim that they require celibacy because it is purity.  As though
marriage were impurity and a sin, or as though celibacy merited
justification more than does marriage!  And to this end they cite the
ceremonies of the Mosaic Law, because, since under the Law, the
priests, at the time of ministering, were separated from their wives,
the priest in the New Testament, inasmuch as he ought always to pray,
ought always to practise continence.  This silly comparison is
presented as a proof which should compel priests to perpetual
celibacy, although, indeed, in this very comparison marriage is
allowed, only in the time of ministering its use is interdicted.  And
it is one thing to pray; another, to minister.  The saints prayed
even when they did not exercise the public ministry; nor did conjugal
intercourse hinder them from praying.

But we shall reply in order to these figments.  In the first place,
it is necessary for the adversaries to acknowledge this, namely, that
in believers marriage is pure because it has been sanctified by the
Word of God, i.e., it is a matter that is permitted and approved by
the Word of God, as Scripture abundantly testifies.  For Christ calls
marriage a divine union, when He says, Matt. 19, 6: What God hath
joined together [let not man put asunder.  Here Christ says that
married people are joined together by God.  Accordingly, it is a pure,
holy, noble, praiseworthy work of God].  And Paul says of marriage,
of meats and similar things, I Tim. 4, 6: It is sanctified by the
Word of God and prayer, i.e., by the Word, by which consciences
become certain that God approves; and by prayer, i.e., by faith,
which uses it with thanksgiving as a gift of God.  Likewise, 1 Cor. 7,
14: The unbelieving husband is sanctified by the wife, etc., i.e..
the use of marriage is permitted and holy on account of faith in
Christ, just as it is permitted to use meat, etc. Likewise, 1 Tim. 2,
16: She shall, be saved in childbearing [if they continue in faith],
etc. If the adversaries could produce such a passage concerning
celibacy, then indeed they would celebrate a wonderful triumph.  Paul
says that woman is saved by child-bearing.  What more honorable could
be said against the hypocrisy of celibacy than that woman is saved by
the conjugal works themselves, by conjugal intercourse, by bearing
children and the other duties?  But what does St. Paul mean?  Let the
reader observe that faith is added, and that domestic duties without
faith are not praised.  If they continue, he says, in faith.  For he
speaks of the whole class of mothers.  Therefore he requires
especially faith [that they should have God's Word and be believing],
by which woman receives the remission of sins and justification.
Then he adds a particular work of the calling, just as in every man a
good work of a particular calling ought to follow faith.  This work
pleases God on account of faith.  Thus the duties of the woman please
God on account of faith, and the believing woman is saved who in such
duties devoutly serves her calling.

These testimonies teach that marriage is a lawful [a holy and
Christian] thing.  If therefore purity signifies that which is
allowed and approved before God, marriages are pure, because they
have been approved by the Word of God.  And Paul says of lawful
things, Titus 1, 15: Unto the pure all things are pure, i.e., to
those who believe in Christ and are righteous by faith.  Therefore,
as virginity is impure in the godless, so in the godly marriage is
pure on account of the Word of God and faith.

Again, if purity is properly opposed to concupiscence, it signifies
purity of heart, i.e., mortified concupiscence, because the Law does
not prohibit marriage, but concupiscence, adultery, fornication.
Therefore celibacy is not purity.  For there may be greater purity of
heart in a married man, as in Abraham or Jacob, than in most of those
who are even truly continent [who even, according to bodily purity,
really maintain their chastity].

Lastly, if they understand that celibacy is purity in the sense that
it merits justification more than does marriage, we most emphatically
contradict it.  For we are justified neither on account of virginity
nor on account of marriage, but freely for Christ's sake, when we
believe that for His sake God is propitious to us.  Here perhaps they
will exclaim that, according to the manner of Jovinian, marriage is
made equal to virginity.  But, on account of such clamors we shall
not reject the truth concerning the righteousness of faith, which we
have explained above.  Nevertheless we do not make virginity and
marriage equal.  For just as one gift surpasses another, as prophecy
surpasses eloquence, the science of military affairs surpasses
agriculture, and eloquence surpasses architecture, so virginity is a
more excellent gift than marriage.  And nevertheless, just as an
orator is not more righteous before God because of his eloquence than
an architect because of his skill in architecture, so a virgin does
not merit justification by virginity more than a married person
merits it by conjugal duties but each one ought faithfully to serve
in his own gift, and to believe that for Christ's sake he receives
the remission of sins and by faith is accounted righteous before God.

Neither does Christ or Paul praise virginity because it justifies,
but because it is freer and less distracted with domestic occupations,
in praying, teaching, [writing,] serving.  For this reason Paul says,
1 Cor. 7, 32: He that is unmarried careth for the things which
belong to the Lord.  Virginity, therefore, is praised on account of
meditation and study.  Thus Christ does not simply praise those who
make themselves eunuchs, but adds, for the kingdom of heaven's sake,
i.e., that they may have leisure to learn or teach the Gospel; for He
does not say that virginity merits the remission of sins or salvation.

To the examples of the Levitical priests we have replied that they do
not establish the duty of imposing perpetual celibacy upon the
priests.  Furthermore, the Levitical impurities are not to be
transferred to us.  [The law of Moses, with the ceremonial statutes
concerning what is clean or unclean, do not at all concern us
Christians.] Then intercourse contrary to the Law was an impurity.
Now it is not impurity, because Paul says, Titus 1, 15: Unto the pure
all things are pure.  For the Gospel frees us from these Levitical
impurities [from all the ceremonies of Moses, and not alone from the
laws concerning uncleanness].  And if any one defends the law of
celibacy with the design to burden consciences by these Levitical
observances, we must strive against this, just as the apostles in
Acts 15, 10 sqq. strove against those who required circumcision and
endeavored to impose the Law of Moses upon Christians.

Yet, in the mean while, good men will know how to control the use of
marriage, especially when they are occupied with public offices,
which often, indeed, give good men so much labor as to expel all
domestic thoughts from their minds.  [For to be burdened with great
affairs and transactions, which concern commonwealths and nations,
governments and churches, is a good remedy to keep the old Adam from
lustfulness.] Good men know also this, that Paul, 1 Thess. 4, 4,
commands that every one possess his vessel in sanctification [and
honor, not in the lust of concupiscence].  They know likewise that
they must sometimes retire, in order that there may be leisure for
prayer, but Paul does not wish this to be perpetual, 1 Cor. 7, 5. Now
such continence is easy to those who are good and occupied.  But this
great crowd of unemployed priests which is in the fraternities cannot
afford, in this voluptuousness, even this Levitical continence, as
the facts show.  [On the other hand, what sort of chastity can there
be among so many thousands of monks and priests who live without
worry in all manner of delights, being idle and full, and, moreover,
have not the Word of God, do not learn it, and have no regard for it.
Such conditions bring on all manner of inchastity.  Such people can
observe neither Levitical nor perpetual chastity.] And the lines are
well known: The boy accustomed to pursue a slothful life hates those
who are busy.

Many heretics understanding the Law of Moses incorrectly have treated
marriage with contempt, for whom, nevertheless, celibacy has gained
extraordinary admiration.  And Epiphanius complains that, by this
commendation especially, the Encratites captured the minds of the
unwary.  They abstained from wine even in the Lord's Supper; they
abstained from the flesh of all animals, in which they surpassed the
Dominican brethren who live upon fish.  They abstained also from
marriage; and just this gained the chief admiration.  These works,
these services, they thought, merited grace more than the use of wine
and flesh, and than marriage, which seemed to be a profane and
unclean matter, and which scarcely could please God, even though it
were not altogether condemned.

Paul to the Colossians, 2, 18, greatly disapproves these angelic
forms of worship.  For when men believe that they are pure and
righteous on account of such hypocrisy, they suppress the knowledge
of Christ, and suppress also the knowledge of God's gifts and
commandments.  For God wishes us to use His gifts in a godly way.
And we might mention examples where certain godly consciences were
greatly disturbed on account of the lawful use of marriage.  This
evil was derived from the opinions of monks superstitiously praising
celibacy [and proclaiming the married estate as a life that would be
a great obstacle to salvation, and full of sins].  Nevertheless we do
not find fault with temperance or continence, but we have said above
that exercises and mortifications of the body are necessary.  We
indeed deny that confidence should be placed in certain observances,
as though they made righteous.  And Epiphanies has elegantly said
that these observances ought to be praised dia tehn egkrateian kai
dia tehn politeian, i.e., for restraining the body or on account of
public morals; just as certain rites were instituted for instructing
the ignorant, and not as services that justify.

But it is not through superstition that our adversaries require
celibacy, for they know that chastity is not ordinarily rendered
[that at Rome, also in all their monasteries, there is nothing but
undisguised, unconcealed inchastity.  Nor do they seriously intend to
lead chaste lives, but knowingly practise hypocrisy before the
people].  But they feign superstitious opinions, so as to delude the
ignorant.  They are therefore more worthy of hatred than the
Encratites, who seem to have erred by show of religion; these
Sardanapali [Epicureans] designedly misuse the pretext of religion.

Sixthly.  Although we have so many reasons for disapproving the law
of perpetual celibacy, yet, besides these, dangers to souls and
public scandals also are added, which even, though the law were not
unjust, ought to deter good men from approving such a burden as has
destroyed innumerable souls.

For a long time all good men [their own bishops and canons] have
complained of this burden, either on their own account, or on account
of others whom they saw to be in danger.  But no Popes give ear to
these complaints.  Neither is it doubtful how greatly injurious to
public morals this law is, and what vices and shameful lusts it has
produced.  The Roman satires are extant.  In these Rome still
recognizes and reads its own morals.

Thus God avenges the contempt of His own gift and ordinance in those
who prohibit marriage.  But since the custom in regard to other laws
was that they should be changed if manifest utility would advise it,
why is the same not done with respect to this law, in which so many
weighty reasons concur, especially in these last times, why a change
ought to be made?  Nature is growing old and is gradually becoming
weaker, and vices are increasing; wherefore the remedies divinely
given should have been employed.  We see what vice it was which God
denounced before the Flood, what He denounced before the burning of
the five cities.  Similar vices have preceded the destruction of many
other cities, as of Sybaris and Rome.  And in these there has been
presented an image of the times which will be next to the end of
things.  Accordingly, at this time, marriage ought to have been
especially defended by the most severe laws and warning examples, and
men ought to have been invited to marriage.  This duty pertains to
the magistrates, who ought to maintain public discipline.  [God has
now so blinded the world that adultery and fornication are permitted
almost without punishment, on the contrary, punishment is inflicted
on account of marriage.  Is not this terrible to hear?] Meanwhile the
teachers of the Gospel should do both, they should exhort incontinent
men to marriage, and should exhort others not to despise the gift of
continence.

The Popes daily dispense and daily change other laws which are most
excellent, yet, in regard to this one law of celibacy, they are as
iron and inexorable, although, indeed, it is manifest that this is
simply of human right.  And they are now making this law more
grievous in many ways.  The canon bids them suspend priests, these
rather unfriendly interpreters suspend them not from office, but from
trees.  They cruelly kill many men for nothing but marriage.  [It is
to be feared therefore, that the blood of Abel will cry to heaven so
loudly as not to be endured, and that we shall have to tremble like
Cain.] And these very parricides show that this law is a doctrine of
demons.  For since the devil is a murderer, he defends his law by
these parricides.

We know that there is some offense in regard to schism, because we
seem to have separated from those who are thought to be regular
bishops.  But our consciences are very secure, since we know that,
though we most earnestly desire to establish harmony, we cannot
please the adversaries unless we cast away manifest truth, and then
agree with these very men in being willing to defend this unjust law,
to dissolve marriages that have been contracted, to put to death
priests if they do not obey, to drive poor women and fatherless
children into exile.  But since it is well established that these
conditions are displeasing to God, we can in no way grieve that we
have no alliance with the multitude of murderers among the
adversaries.

We have explained the reasons why we cannot assent with a good
conscience to the adversaries when they defend the pontifical law
concerning perpetual celibacy, because it conflicts with divine and
natural law and is at variance with the canons themselves, and is
superstitious and full of danger, and, lastly, because the whole
affair is insincere.  For the law is enacted not for the sake of
religion [not for holiness' sake, or because they do not know better;
they know very well that everybody is well acquainted with the
condition of the great cloisters, which we are able to name], but for
the sake of dominion, and this is wickedly given the pretext of
religion.  Neither can anything be produced by sane men against these
most firmly established reasons.  The Gospel allows marriage to those
to whom it is necessary.  Nevertheless, it does not compel those to
marry who can be continent, provided they be truly continent.  We
hold that this liberty should also be conceded to the priests, nor do
we wish to compel any one by force to celibacy, nor to dissolve
marriages that have been contracted.

We have also indicated incidentally, while we have recounted our
arguments, how the adversaries cavil at several of these; and we have
explained away these false accusations.  Now we shall relate as
briefly as possible with what important reasons they defend the law.
First, they say that it has been revealed by God.  You see the
extreme impudence of these sorry fellows.  They dare to affirm that
the law of perpetual celibacy has been divinely revealed, although it
is contrary to manifest testimonies of Scripture, which command that
to avoid fornication each one should have his own wife, 1 Cor. 7, 2;
which likewise forbid to dissolve marriages that have been contracted;
cf.  Matt. 6, 32; 19, 6; 1 Cor. 7, 27. [What can the knaves say in
reply?  And how dare they wantonly and shamelessly misapply the great,
most holy name of the divine Majesty?] Paul reminds us what an
author such a law was to have when he calls it a doctrine of demons,
1 Tim. 4, 1. And the fruits show their author, namely, so many
monstrous lusts and so many murders which are now committed under the
pretext of that law [as can be seen at Rome].

The second argument of the adversaries is that the priests ought to
be pure, according to Is. 52, 11: Be ye clean that bear the vessels
of the Lord.  And they cite many things to this effect.  This reason
which they display we have above removed as especially specious.  For
we have said that virginity without faith is not purity before God,
and marriage, on account of faith, is pure, according to Titus 1, 16:
Unto the pure all things are pure.  We have said also this, that
outward purity and the ceremonies of the Law are not to be
transferred hither, because the Gospel requires purity of heart, and
does not require the ceremonies of the Law.  And it may occur that
the heart of a husband, as of Abraham or Jacob, who were polygamists,
is purer and burns less with lusts than that of many virgins who are
even truly continent.  But what Isaiah says: Be ye clean that bear
the vessels of the Lord, ought to be understood as referring to
cleanness of heart and to the entire repentance.  Besides, the saints
will know in the exercise of marriage how far it is profitable to
restrain its use, and as Paul says, 1 Thess. 4, 4, to possess his
vessel in sanctification.  Lastly, since marriage is pure, it is
rightly said to those who are not continent in celibacy that they
should marry wives in order to be pure.  Thus the same law: Be ye
clean that bear the vessels of the Lord, commands that impure
celibates become pure husbands [impure unmarried priests become pure
married priests].

The third argument is horrible, namely, that the marriage of priests
is the heresy of Jovinian.  Fine-sounding words!  [Pity on our poor
souls, dear sirs; proceed gently!] This is a new crime, that marriage
[which God instituted in Paradise] is a heresy!  [In that case all
the world would be children of heretics.] In the time of Jovinian the
world did not as yet know the law concerning perpetual celibacy.
[This our adversaries know very well.] Therefore it is an impudent
falsehood that the marriage of priests is the heresy of Jovinian, or
that such marriage was then condemned by the Church.  In such
passages we can see what design the adversaries had in writing the
_Confutation_.  They judged that the ignorant would be thus most
easily excited, if they would frequently hear the reproach of heresy,
if they pretend that our cause had been dispatched and condemned by
many previous decisions of the Church.  Thus they frequently cite
falsely the judgment of the Church.  Because they are not ignorant of
this, they were unwilling to exhibit to us a copy of their Apology,
lest this falsehood and these reproaches might be exposed.  Our
opinion, however, as regards the case of Jovinian, concerning the
comparison of virginity and marriage, we have expressed above.  For
we do not make marriage and virginity equal, although neither
virginity nor marriage merits justification.

By such false arguments they defend a law that is godless and
destructive to good morals.  By such reasons they set the minds of
princes firmly against God's judgment [the princes and bishops who
believe this teaching will see whether their reasons will endure the
test when the hour of death arrives], in which God will call them to
account as to why they have dissolved marriages, and why they have
tortured [flogged and impaled] and killed priests [regardless of the
cries, wails, and tears of so many widows and orphans].  For do not
doubt but that, as the blood of dead Abel cried out, Gen. 4, 10, so
the blood of many good men against whom they have unjustly raged,
will also cry out.  And God will avenge this cruelty; there you will
discover how empty are these reasons of the adversaries, and you will
perceive that in God's judgment no calumnies against God's Word
remain standing, as Isaiah says, 40, 6: All flesh is grass, and all
the goodliness thereof is as the flower of the field [that their
arguments are straw and hay, and God a consuming fire, before whom
nothing but God's Word can abide, 1 Pet. 1, 24].

Whatever may happen, our princes will be able to console themselves
with the consciousness of right counsels, because even though the
priests would have done wrong in contracting marriages, yet this
disruption of marriages, these proscriptions, and this cruelty are
manifestly contrary to the will and Word of God.  Neither does
novelty or dissent delight our princes, but especially in a matter
that is not doubtful more regard had to be paid to the Word of God
than to all other things.




Part 29


Article XXIV (XII): _Of the Mass._

At the outset we must again make the preliminary statement that we do
not abolish the Mass, but religiously maintain and defend it.  For
among us masses are celebrated every Lord's Day and on the other
festivals, in which the Sacrament is offered to those who wish to use
it, after they have been examined and absolved.  And the usual public
ceremonies are observed, the series of lessons of prayers, vestments,
and other like things.

The adversaries have a long declamation concerning the use of the
Latin language in the Mass, in which they absurdly trifle as to how
it profits [what a great merit is achieved by] an unlearned hearer to
hear in the faith of the Church a Mass which he does not understand.
They evidently imagine that the mere work of hearing is a service,
that it profits without being understood.  We are unwilling to
malignantly pursue these things, but we leave them to the judgment of
the reader.  We mention them only for the purpose of stating in
passing, that also among us the Latin lessons and prayers are
retained.

Since ceremonies, however, ought to be observed both to teach men
Scripture, and that those admonished by the Word may conceive faith
and fear [of God, and obtain comfort] and thus also may pray (for
these are the designs of ceremonies ), we retain the Latin language
on account of those who are learning and understand Latin, and we
mingle with it German hymns, in order that the people also may have
something to learn, and by which faith and fear may be called forth.
This custom has always existed in the churches.  For although some
more frequently, and others more rarely, introduced German hymns,
nevertheless the people almost everywhere sang something in their own
tongue.  [Therefore, this is not such a new departure.] It has,
however, nowhere been written or represented that the act of hearing
lessons not understood profits men, or that ceremonies profit, not
because they teach or admonish, but _ex opere operato_, because they
are thus performed or are looked upon.  Away with such pharisaic
opinions!  [Ye sophists ought to be heartily ashamed of such dreams!]

The fact that we hold only Public or Common Mass [at which the people
also commune, not Private Mass] is no offense against the Church
catholic.  For in the Greek churches even to-day private Masses are
not held, but there is only a public Mass, and that on the Lord's Day
and festivals.  In the monasteries daily Mass is held, but this is
only public.  These are the traces of former customs.  For nowhere do
the ancient writers before Gregory make mention of private Masses.
We now omit noticing the nature of their origin.  It is evident that
after the mendicant monks began to prevail, from most false opinions
and on account of gain they were so increased that all good men for a
long time desired some limit to this thing.  Although St. Francis
wished to provide aright for this matter, as he decided that each
fraternity should be content with a single common Mass daily,
afterwards this was changed, either by superstition or for the sake
of gain.  Thus, where it is of advantage, they themselves change the
institutions of the Fathers; and afterwards they cite against us the
authority of the Fathers.  Epiphanius writes that in Asia the
Communion was celebrated three times a week, and that there were no
daily Masses.  And indeed he says that this custom was handed down
from the apostles.  For he speaks thus: Assemblies for Communion were
appointed by the apostles to be held on the fourth day, on Sabbath
eve, and the Lord's Day.

Moreover, although the adversaries collect many testimonies on this
topic to prove that the Mass is a sacrifice, yet this great tumult of
words will be quieted when the single reply is advanced that this
line of authorities, reasons and testimonies, however long, does not
prove that the Mass confers grace er opere operato, or that, when
applied on behalf of others, it merits for them the remission of
venial and mortal sins, of guilt and punishment.  This one reply
overthrows all objections of the adversaries, not only in this
_Confutation_, but in all writings which they have published
concerning the Mass.

And this is the issue [the principal question] of the case of which
our readers are to be admonished, as Aeschines admonished the judges
that just as boxers contend with one another for their position, so
they should strive with their adversary concerning the controverted
point, and not permit him to wander beyond the case.  In the same
manner our adversaries ought to be here compelled to speak on the
subject presented.  And when the controverted point has been
thoroughly understood, a decision concerning the arguments on both
sides will be very easy.

For in our Confession we have shown that we hold that the Lord's
Supper does not confer _grace ex opere operato_, and that, when
applied on behalf of others, alive or dead, it does not merit for
them _ex opere operato_ the remission of sins, of guilt or of
punishment.  And of this position a clear and firm proof exists in
that it is impossible to obtain the remission of our sins on account
of our own work _ex opere operato_ [even when there is not a good
thought in the heart], but the terrors of sin and death must be
overcome by faith when we comfort our hearts with the knowledge of
Christ, and believe that for Christ's sake we are forgiven, and that
the merits and righteousness of Christ are granted us, Rom. 5, 1:
Being justified by faith, we have peace.  These things are so sure
and so firm that they can stand against all the gates of hell.

If we are to say only as much as is necessary, the case has already
been stated.  For no sane man can approve that pharisaic and heathen
opinion concerning the _opus operatum_.  And nevertheless this
opinion inheres in the people, and has increased infinitely the
number of masses.  For masses are purchased to appease God's wrath,
and by this work they wish to obtain the remission of guilt and of
punishment; they wish to procure whatever is necessary in every kind
of life [health riches, prosperity, and success in business].  They
wish even to liberate the dead.  Monks and sophists have taught this
pharisaic opinion in the Church.

But although our case has already been stated, yet, because the
adversaries foolishly pervert many passages of Scripture to the
defense of their errors, we shall add a few things on this topic.  In
the _Confutation_ they have said many things concerning "sacrifice,"
although in our Confession we purposely avoided this term on account
of its ambiguity.  We have set forth what those persons whose abuses
we condemn now understand as a sacrifice.  Now, in order to explain
the passages of Scripture that have been wickedly perverted, it is
necessary in the beginning to set forth what a sacrifice is.  Already
for an entire period of ten years the adversaries have published
almost infinite volumes concerning sacrifice, and yet not one of them
thus far has given a definition of sacrifice.  They only seize upon
the name "sacrifices" either from the Scriptures or the Fathers [and
where they find it in the Concordances of the Bible apply it here,
whether it fits or not].  Afterward they append their own dreams, as
though indeed a sacrifice signifies whatever pleases them.




Part 30


_What a Sacrifice Is, and What Are the Species of Sacrifice._

[Now, lest we plunge blindly into this business, we must indicate, in
the first place, a distinction as to what is, and what is not, a
sacrifice.  To know this is expedient and good for all Christians.]
Socrates, in the Phaedrus of Plato, says that he is especially fond
of divisions, because without these nothing can either be explained
or understood in speaking, and if he discovers any one skilful in
making divisions, he says that he attends and follows his footsteps
as those of a god.  And he instructs the one dividing to separate the
members in their very joints, lest, like an unskilful cook, he break
to pieces some member.  But the adversaries wonderfully despise these
precepts, and, according to Plato, are truly _kakoi mageiroi_ (poor
butchers), since they break the members of "sacrifice," as can be
understood when we have enumerated the species of sacrifice.
Theologians are rightly accustomed to distinguish between a Sacrament
and a sacrifice.  Therefore let the genus comprehending both of these
be either a ceremony or a sacred work.  A Sacrament is a ceremony or
work in which God presents to us that which the promise annexed to
the ceremony offers; as Baptism is a work, not which we offer to God
but in which God baptizes us, i.e., a minister in the place of God;
and God here offers and presents the remission of sins, etc.,
according to the promise, Mark 16, 16: He that believeth and is
baptized shall be saved.  A sacrifice, on the contrary, is a ceremony
or work which we render God in order to afford Him honor.

Moreover, the proximate species of sacrifice are two, and there are
no more.  One is the propitiatory sacrifice, i.e., a work which makes
satisfaction for guilt and punishment, i.e., one that reconciles God,
or appeases God's wrath, or which merits the remission of sins for
others.  The other species is the eucharistic sacrifice, which does
not merit the remission of sins or reconciliation, but is rendered by
those who have been reconciled, in order that we may give thanks or
return gratitude for the remission of sins that has been received, or
for other benefits received.

These two species of sacrifice we ought especially to have in view
and placed before the eyes in this controversy, as well as in many
other discussions; and especial care must be taken lest they be
confounded.  But if the limits of this book would suffer it, we would
add the reasons for this division.  For it has many testimonies in
the Epistle to the Hebrews and elsewhere.  And all Levitical
sacrifices can be referred to these members as to their own homes
[genera].  For in the Law certain sacrifices were named propitiatory
on account of their signification or similitude; not because they
merited the remission of sins before God, but because they merited
the remission of sins according to the righteousness of the Law, in
order that those for whom they were made might not be excluded from
that commonwealth [from the people of Israel].  Therefore they were
called sin-offerings and burnt offerings for a trespass.  Whereas the
eucharistic sacrifices were the oblation, the drink-offering,
thank-offerings, first-fruits, tithes.

[Thus there have been in the Law emblems of the true sacrifice.] But
in fact there has been only one propitiatory sacrifice in the world,
namely, the death of Christ, as the Epistle to the Hebrews teaches,
which says, 10, 4: It is not possible that the blood of bulls and of
goats should take away sins.  And a little after, of the [obedience
and] will of Christ, v. 10: By the which will we are sanctified by
the offering of the body of Jesus Christ once for all.  And Isaiah
interprets the Law, in order that we may know that the death of
Christ is truly a satisfaction for our sins, or expiation, and that
the ceremonies of the Law are not, wherefore he says, 53, 10: When
Thou shalt make His soul an offering for sin, He will see His seed,
etc. For the word employed here, _'shm_, signifies a victim for
transgression; which signified in the Law that a certain Victim was
to come to make satisfaction for our sins and reconcile God in order
that men might know that God wishes to be reconciled to us, not on
account of our own righteousnesses, but on account of the merits of
another, namely, of Christ.  Paul interprets the same word _'shm_ as
sin, Rom. 8, 3: For sin (God) condemned sin, i.e., He punished sin
for sin, i.e., by a Victim for sin.  The significance of the word can
be the more easily understood from the customs of the heathen, which,
we see, have been received from the misunderstood expressions of the
Fathers.  The Latins called a victim that which in great calamities,
where God seemed to be especially enraged, was offered to appease
God's wrath, a _piaculum_; and they sometimes sacrificed human
victims, perhaps because they had heard that a human victim would
appease God for the entire human race.  The Greeks sometimes called
them _katharmata_ and sometimes _peripsehmata_.  Isaiah and Paul,
therefore, mean that Christ became a victim i.e., an expiation, that
by His merits, and not by our own, God might be reconciled.
Therefore let this remain established in the case namely, that the
death of Christ alone is truly a propitiatory sacrifice.  For the
Levitical propitiatory sacrifices were so called only to signify a
future expiation.  On account of a certain resemblance, therefore,
they were satisfactions redeeming the righteousness of the Law, lest
those persons who sinned should be excluded from the commonwealth.
But after the revelation of the Gospel [and after the true sacrifice
has been accomplished] they had to cease, and because they had to
cease in the revelation of the Gospel, they were not truly
propitiations, since the Gospel was promised for this very reason,
namely, to set forth a propitiation.

Now the rest are eucharistic sacrifices which are called sacrifices
of praise, Lev. 3, 1 f.; 7, 11 f.; Ps. 56, 12 f., namely, the
preaching of the Gospel, faith, prayer, thanksgiving, confession, the
afflictions of saints yea, all good works of saints.  These
sacrifices are not satisfactions for those making them, or applicable
on behalf of others, so as to merit for these, ex opere operato, the
remission of sins or reconciliation.  For they are made by those who
have been reconciled.  And such are the sacrifices of the New
Testament, as Peter teaches, 1. Ep. 2, 5: An holy priesthood, to
offer up spiritual sacrifices.  Spiritual sacrifices, however, are
contrasted not only with those of cattle, but even with human works
offered _ex opere operato_, because spiritual refers to the movements
of the Holy Ghost in us.  Paul teaches the same thing Rom. 12, 1:
Present your bodies a living sacrifice, holy, acceptable, which is
your reasonable service.  Reasonable service signifies, however, a
service in which God is known and apprehended by the mind, as happens
in the movements of fear and trust towards God.  Therefore it is
opposed not only to the Levitical service, in which cattle are slain,
but also to a service in which a work is imagined to be offered _ex
opere operato_.  The Epistle to the Hebrews, 13, 15, teaches the same
thing: By Him, therefore, let us offer the sacrifice of praise to God
continually; and he adds the interpretation, that is, the fruit of
our lips, giving thanks to His name.  He bids us offer praises, i.e.,
prayer, thanksgiving, confession, and the like.  These avail not _ex
opere operato_, but on account of faith.  This is taught by the
clause: By Him let us offer, i.e., by faith in Christ.

In short, the worship of the New Testament is spiritual, i.e., it is
the righteousness of faith in the heart and the fruits of faith.  It
accordingly abolishes the Levitical services.  [In the New Testament
no offering avails _ex opere operato, sine bono motu utentis_, i.e.
on account of the work, without a good thought in the heart.] And
Christ says, John 4, 23. 24: True worshipers shall worship the Father
in spirit and in truth, for the Father seeketh such to worship Him.
God is a Spirit; and they that worship Him must worship Him in spirit
and in truth [that is from the heart, with heartfelt fear and cordial
faith].  This passage clearly condemns [as absolutely devilish,
pharisaical, and antichristian] opinions concerning sacrifices which
they imagine, avail _ex opere operato_, and teaches that men ought to
worship in spirit i.e., with the dispositions of the heart and by
faith.  [The Jews also did not understand their ceremonies aright,
and imagined that they were righteous before God when they had
wrought works _ex opere operato_.  Against this the prophets contend
with the greatest earnestness.] Accordingly, the prophets also in the
Old Testament condemn the opinion of the people concerning the opus
operatum and teach the righteousness and sacrifices of the Spirit.
Jer. 7, 22. 23: For I spake not unto your fathers, nor commanded them,
in the day that I brought them out of the land of Egypt, concerning
burnt offerings or sacrifices; but this thing commanded I them,
saying, Obey My voice, and I will be your God, etc. How do we suppose
that the Jews received this arraignment, which seems to conflict
openly with Moses?  For it was evident that God had given the fathers
commands concerning burnt offerings and victims.  But Jeremiah
condemns the opinion concerning sacrifices which God had not
delivered namely, that these services should please Him _ex opere
operato_.  But he adds concerning faith that God had commanded this:
Hear Me, i.e., believe Me that I am your God; that I wish to become
thus known when I pity and aid; neither have I need of your victims;
believe that I wish to be God the Justifier and Savior, not on
account of works, but on account of My word and promise, truly and
from the heart seek and expect aid from Me.

Ps. 50, 13. 15, which rejects the victims and requires prayer, also
condemns the opinion concerning the opus operatum: Will I eat the
flesh of bulls? etc. (Call upon Me in the day of trouble; I will
deliver thee, and thou shalt glorify Me.  The Psalmist testifies that
this is true service, that this is true honor, if we call upon Him
from the heart.

Likewise Ps. 40, 6: Sacrifice and offering Thou didst not desire;
mine ears hast Thou opened, i.e., Thou hast offered to me Thy Word
that I might hear it, and Thou dost require that I believe Thy Word
and The promises, that Thou truly desirest to pity, to bring aid, etc.
Likewise Ps. 51, 16. 17: Thou delightest not in burnt offering.  The
sacrifices of God are a broken spirit; a broken and a contrite heart,
O God, Thou wilt not despise.  Likewise Ps. 4, 5: Offer the
sacrifices of righteousness, and put your trust [hope, V.] in the
Lord.  He bids us hope, and says that this is a righteous sacrifice,
signifying that other sacrifices are not true and righteous
sacrifices.  And Ps. 116, 17: I will offer to Thee the sacrifices of
thanksgiving, and will call upon the name of the Lord They call
invocation a sacrifice of thanksgiving.

But Scripture is full of such testimonies as teach that sacrifices
_ex opere operato_ do not reconcile God.  Accordingly the New
Testament, since Levitical services have been abrogated, teaches that
new and pure sacrifices will be made, namely, faith, prayer,
thanksgiving, confession, and the preaching of the Gospel,
afflictions on account of the Gospel, and the like.

And of these sacrifices Malachi speaks, 1, 11: From the rising of the
sun even unto the going down of the same My name shall be great among
the Gentiles; and in every place incense shall be offered unto My
name and a pure offering.  The adversaries perversely apply this
passage to the Mass, and quote the authority of the Fathers.  A reply,
however, is easy, for even if it spoke most particularly of the Mass,
it would not follow that the Mass justifies _ex opere operato_, or
that when applied to others, it merits the remission of sins, etc.
The prophet says nothing of those things which the monks and sophists
impudently fabricate.  Besides, the very words of the prophet express
his meaning.  For they first say this, namely, that the name of the
Lord will be great.  This is accomplished by the preaching of the
Gospel.  For through this the name of Christ is made known, and the
mercy of the Father, promised in Christ is recognized.  The preaching
of the Gospel produces faith in those who receive the Gospel.  They
call upon God, they give thanks to God, they bear afflictions for
their confession, they produce good works for the glory of Christ.
Thus the name of the Lord becomes great among the Gentiles.
Therefore incense and a pure offering signify not a ceremony _ex
opere operato_ [not the ceremony of the Mass alone], but all those
sacrifices through which the name of the Lord becomes great, namely,
faith, invocation, the preaching of the Gospel, confession, etc. And
if any one would have this term embrace the ceremony [of the Mass],
we readily concede it, provided he neither understands the ceremony
alone, nor teaches that the ceremony profits _ex opere operato_.  For
just as among the sacrifices of praise, i.e., among the praises of
God, we include the preaching of the Word so the reception itself of
the Lord's Supper can be praise or thanksgiving, but it does not
justify _ex opere operato_; neither is it to be applied to others so
as to merit for them the remission of sins.  But after a while we
shall explain how even a ceremony is a sacrifice.  Yet, as Malachi
speaks of all the services of the New Testament, and not only of the
Lord's Supper; likewise, as he does not favor the pharisaic opinion
of the _opus operatum_, he is not against us, but rather aids us.
For he requires services of the heart, through which the name of the
Lord becomes truly great.

Another passage also is cited from Malachi 3, 3: And He shall purify
the sons of Levi, and purge them as gold and silver, that they may
offer unto the Lord an offering of righteousness.  This passage
clearly requires the sacrifices of the righteous, and hence does not
favor the opinion concerning the _opus operatum_.  But the sacrifices
of the sons of Levi i.e., of those teaching in the New Testament, are
the preaching of the Gospel, and the good fruits of preaching, as
Paul says, Rom. 15, 16: Ministering the Gospel of God, that the
offering up of the Gentiles might be acceptable, being sanctified by
the Holy Ghost, i.e., that the Gentiles might be offerings acceptable
to God by faith, etc. For in the Law the slaying of victims signified
both the death of Christ and the preaching of the Gospel, by which
this oldness of flesh should be mortified, and the new and eternal
life be begun in us.

But the adversaries everywhere perversely apply the name sacrifice to
the ceremony alone.  They omit the preaching of the Gospel, faith,
prayer, and similar things, although the ceremony has been
established on account of these, and the New Testament ought to have
sacrifices of the heart, and not ceremonials for sin that are to be
performed after the manner of the Levitical priesthood.

They cite also the daily sacrifice (cf.  Ex. 29, 38 f.; Dan. 8, ll f.,
12, 11), that, just as in the Law there was a daily sacrifice, so
the Mass ought to be a daily sacrifice of the New Testament.  The
adversaries have managed well if we permit ourselves to be overcome
by allegories.  It is evident, however, that allegories do not
produce firm proofs [that in matters so highly important before God
we must have a sure and clear word of God, and not introduce by force
obscure and foreign passages, such uncertain explanations do not
stand the test of God's judgment].  Although we indeed readily suffer
the Mass to be understood as a daily sacrifice, provided that the
entire Mass be understood, i.e., the ceremony with the preaching of
the Gospel, faith, invocation, and thanksgiving.  For these joined
together are a daily sacrifice of the New Testament, because the
ceremony [of the Mass, or the Lord's Supper] was instituted on
account of these things, neither is it to be separated from these.
Paul says accordingly, 1 Cor. 11, 26: As often as ye eat this bread
and drink this cup, ye do show the Lord's death till He come.  But it
in no way follows from this Levitical type that a ceremony justifying
_ex opere operato_ is necessary, or ought to be applied on behalf of
others, that it may merit for them the remission of sins.

And the type aptly represents not only the ceremony, but also the
preaching of the Gospel.  In Num. 28, 4 f. three parts of that daily
sacrifice are represented, the burning of the lamb, the libation, and
the oblation of wheat flour.  The Law had pictures or shadows of
future things.  Accordingly, in this spectacle Christ and the entire
worship of the New Testament are portrayed.  The burning of the lamb
signifies the death of Christ.  The libation signifies that
everywhere in the entire world, by the preaching of the Gospel,
believers are sprinkled with the blood of that Lamb, i.e., sanctified,
as Peter says, 1. Ep. 1, 2: Through sanctification of the Spirit,
unto obedience and sprinkling of the blood of Jesus Christ.  The
oblation of wheat flour signifies faith, prayer, and thanksgiving in
hearts.  As, therefore, in the Old Testament, the shadow is perceived,
so in the New the thing signified should be sought, and not another
type, as sufficient for a sacrifice.

Therefore, although a ceremony is a memorial of Christ's death,
nevertheless it alone is not the daily sacrifice; but the memory
itself is the daily sacrifice, i.e., preaching and faith, which truly
believes that, by the death of Christ, God has been reconciled.  A
libation is required, i.e., the effect of preaching, in order that,
being sprinkled by the Gospel with the blood of Christ, we may be
sanctified, as those put to death and made alive.  Oblations also are
required, i.e., thanksgiving, confessions, and afflictions.

Thus the pharisaic opinion of the _opus operatum_ being cast aside,
let us understand that spiritual worship and a daily sacrifice of the
heart are signified, because in the New Testament the substance of
good things should be sought for [as Paul says: In the Old Testament
is the shadow of things to come but the body and the truth is in
Christ], i.e., the Holy Ghost, mortification, and quickening.  From
these things it is sufficiently apparent that the type of the daily
sacrifice testifies nothing against us, but rather for us, because we
seek for all the parts signified by the daily sacrifice.  [We have
clearly shown all the parts that belonged to the daily sacrifice in
the law of Moses, that it must mean a true cordial offering, not an
_opus operatum_.] The adversaries falsely imagine that the ceremony
alone is signified, and not also the preaching of the Gospel,
mortification, and quickening of heart, etc. [which is the best part
of the Mass, whether they call it a sacrifice or anything else].

Now, therefore, good men will be able to judge readily that the
complaint against us that we abolish the daily sacrifice is most
false.  Experience shows what sort of Antiochi they are who hold
power in the Church; who under the pretext of religion assume to
themselves the kingdom of the world, and who rule without concern for
religion and the teaching of the Gospel; who wage war like kings of
the world, and have instituted new services in the Church.  For in
the Mass the adversaries retain only the ceremony, and publicly apply
this to sacrilegious gain.  Afterward they feign that this work, as
applied on behalf of others, merits for them grace and all good
things.  In their sermons they do not teach the Gospel, they do not
console consciences they do not show that sins are freely remitted
for Christ's sake, but they set forth the worship of saints, human
satisfactions, human traditions, and by these they affirm that men
are justified before God.  And although some of these traditions are
manifestly godless, nevertheless they defend them by violence.  If
any preachers wish to be regarded more learned, they treat of
philosophical questions, which neither the people nor even those who
propose them understand.  Lastly, those who are more tolerable teach
the Law, and say nothing concerning the righteousness of faith.

The adversaries in the _Confutation_ make a great ado concerning the
desolation of churches, namely, that the altars stand unadorned,
without candles and without images.  These trifles they regard as
ornaments to churches.  [Although it is not true that we abolish all
such outward ornaments; yet, even if it were so, Daniel is not
speaking of such things as are altogether external and do not belong
to the Christian Church.] It is a far different desolation which
Daniel means, 11, 31; 12, 11, namely, ignorance of the Gospel.  For
the people, overwhelmed by the multitude and variety of traditions
and opinions, were in no way able to embrace the sum of Christian
doctrine.  [For the adversaries preach mostly of human ordinances,
whereby consciences are led from Christ to confidence in their own
works.] For who of the people ever understood the doctrine of
repentance of which the adversaries treat?  And yet this is the chief
topic of Christian doctrine.

Consciences were tormented by the enumeration of offenses and by
satisfactions.  Of faith by which we freely receive the remission of
sins, no mention whatever was made by the adversaries.  Concerning
the exercises of faith struggling with despair, and the free
remission of sins for Christ's sake, all the books and all the
sermons of the adversaries were silent [worse than worthless, and,
moreover, caused untold damage].  To these, the horrible profanation
of the masses and many other godless services in the churches were
added.  This is the desolation which Daniel describes.

On the contrary, by the favor of God, the priests among us attend to
the ministry of the Word, teach the Gospel concerning the blessings
of Christ, and show that the remission of sins occurs freely for
Christ's sake.  This doctrine brings sure consolation to consciences.
The doctrine of [the Ten Commandments and] good works which God
commands is also added.  The worth and use of the Sacraments are
declared.

But if the use of the Sacrament would be the daily sacrifice,
nevertheless we would retain it rather than the adversaries, because
with them priests hired for pay use the Sacrament.  With us there is
a more frequent and more conscientious use.  For the people use it,
but after having first been instructed and examined.  For men are
taught concerning the true use of the Sacrament that it was
instituted for the purpose of being a seal and testimony of the free
remission of sins, and that, accordingly, it ought to admonish
alarmed consciences to be truly confident and believe that their sins
are freely remitted.  Since, therefore, we retain both the preaching
of the Gospel and the lawful use of the Sacrament, the daily
sacrifice remains with us.

And if we must speak of the outward appearance, attendance upon
church is better among us than among the adversaries.  For the
audiences are held by useful and clear sermons.  But neither the
people nor the teachers have ever understood the doctrine of the
adversaries.  [There is nothing that so attaches people to the church
as good preaching.  But our adversaries preach their people out of
the churches; for they teach nothing of the necessary parts of
Christian doctrine; they narrate the legends of saints and other
fables.] And the true adornment of the churches is godly, useful, and
clear doctrine, the devout use of the Sacraments, ardent prayer, and
the like.  Candles, golden vessels [tapers, altar-cloths, images],
and similar adornments are becoming, but they are not the adornment
that properly belongs to the Church.  But if the adversaries make
worship consist in such matters, and not in the preaching of the
Gospel, in faith, and the conflicts of faith they are to be numbered
among those whom Daniel describes as worshiping their God with gold
and silver, Dan. 11, 38.

They quote also from the Epistle to the Hebrews, 5, 1: Every high
priest taken from among men is ordained for men in things pertaining
to God that he may offer both gifts and sacrifices for sins.  Hence
they conclude that, since in the New Testament there are high priests
and priests, it follows that there is also a sacrifice for sins.
This passage particularly makes an impression on the unlearned,
especially when the pomp of the priesthood [the garments of Aaron,
since in the Old Testament there were many ornaments of gold, silver,
and purple] and the sacrifices of the Old Testament are spread before
the eyes.  This resemblance deceives the ignorant, so that they judge
that, according to the same manner, a ceremonial sacrifice ought to
exist among us, which should be applied on behalf of the sins of
others, just as in the Old Testament.  Neither is the service of the
masses and the rest of the polity of the Pope anything else than
false zeal in behalf of the misunderstood Levitical polity.  [They
have not understood that the New Testament is occupied with other
matters, and that, if such ceremonies are used for the training of
the young, a limit must be fixed for them.]

And although our belief has its chief testimonies in the Epistle to
the Hebrews, nevertheless the adversaries distort against us
mutilated passages from this Epistle, as in this very passage, where
it is said that every high priest is ordained to offer sacrifices for
sins.  Scripture itself immediately adds that Christ is High Priest,
Heb. 5, 5. 6. 10. The preceding words speak of the Levitical
priesthood, and signify that the Levitical priesthood was an image of
the priesthood of Christ.  For the Levitical sacrifices for sins did
not merit the remission of sins before God; they were only an image
of the sacrifice of Christ, which was to be the one propitiatory
sacrifice, as we have said above.  Therefore the Epistle is occupied
to a great extent with the topic that the ancient priesthood and the
ancient sacrifices were instituted not for the purpose of meriting
the remission of sins before God or reconciliation, but only to
signify the future sacrifice of Christ alone.  For in the Old
Testament it was necessary for saints to be justified by faith
derived from the promise of the remission of sins that was to be
granted for Christ's sake, just as saints are also justified in the
New Testament.  From the beginning of the world it was necessary for
all saints to believe that Christ would be the promised offering and
satisfaction for sins, as Isaiah teaches, 53, 10: When Thou shalt
make His soul an offering for sin.

Since, therefore, in the Old Testament, sacrifices did not merit
reconciliation, unless by a figure (for they merited civil
reconciliation), but signified the coming sacrifice, it follows that
Christ is the only sacrifice applied on behalf of the sins of others.
Therefore, in the New Testament no sacrifice is left to be applied
for the sins of others, except the one sacrifice of Christ upon the
cross.

They altogether err who imagine that Levitical sacrifices merited the
remission of sins before God, and, by this example in addition to the
death of Christ, require in the New Testament sacrifices that are to
be applied on behalf of others.  This imagination absolutely destroys
the merit of Christ's passion and the righteousness of faith, and
corrupts the doctrine of the Old and New Testaments, and instead of
Christ makes for us other mediators and propitiators out of the
priests and sacrificers, who daily sell their work in the churches.

Therefore, if any one would thus infer that in the New Testament a
priest is needed to make offering for sins, this must be conceded
only of Christ.  And the entire Epistle to the Hebrews confirms this
explanation.  And if, in addition to the death of Christ, we were to
seek for any other satisfaction to be applied for the sins of others
and to reconcile God, this would be nothing more than to make other
mediators in addition to Christ.  Again, as the priesthood of the New
Testament is the ministry of the Spirit, as Paul teaches 2 Cor. 3, 6,
it, accordingly, has but the one sacrifice of Christ, which is
satisfactory and applied for the sins of others.  Besides it has no
sacrifices like the Levitical, which could be applied _ex opere
operato_ on behalf of others, but it tenders to others the Gospel and
the Sacraments, that by means of these they may conceive faith and
the Holy Ghost and be mortified and quickened, because the ministry
of the Spirit conflicts with the application of an _opus operatum_.
[For, unless there is personal faith and a life wrought by the Holy
Spirit, the _opus operatum_ of another cannot render me godly nor
save me.] For the ministry of the Spirit is that through which the
Holy Ghost is efficacious in hearts; and therefore this ministry is
profitable to others, when it is efficacious in them, and regenerates
and quickens them.  This does not occur by the application _ex opere
operato_ of the work of another on behalf of others.

We have shown the reason why the Mass does not justify _ex opere
operato_, and why, when applied on behalf of others, it does not
merit remission, because both conflict with the righteousness of
faith.  For it is impossible that remission of sins should occur, and
the terrors of death and sin be overcome by any work or anything,
except by faith in Christ, according to Rom. 5, 1: Being justified by
faith, we have peace.

In addition, we have shown that the Scriptures, which are cited
against us, in no way favor the godless opinion of the adversaries
concerning the opus operatum.  All good men among all nations can
judge this.  Therefore the error of Thomas is to be rejected, who
wrote: That the body of the Lord, once offered on the cross for
original debt, is continually offered for daily offenses on the altar
in order that, in this, the Church might have a service whereby to
reconcile God to herself.  The other common errors are also to be
rejected, as, that the Mass _ex opere operato_ confers grace upon one
employing it; likewise that when applied for others, even for wicked
persons, provided they do not interpose an obstacle, it merits for
them the remission of sins, of guilt and punishment.  All these
things are false and godless, and lately invented by unlearned monks,
and obscure the glory of Christ's passion and the righteousness of
faith.

And from these errors infinite others sprang, as, that the masses
avail when applied for many, just as much as when applied
individually.  The sophists have particular degrees of merit, just as
money-changers have grades of weight for gold or silver.  Besides
they sell the Mass, as a price for obtaining what each one seeks: to
merchants, that business may be prosperous; to hunters, that hunting
may be successful, and infinite other things.  Lastly, they apply it
also to the dead; by the application of the Sacrament they liberate
souls from the pains of purgatory; although without faith the Mass is
of service not even to the living.  Neither are the adversaries able
to produce even one syllable from the Scriptures in defense of these
fables which they teach with great authority in the Church, neither
do they have the testimonies of the ancient Church nor of the Fathers.
[Therefore they are impious and blind people who knowingly despise
and trample under foot the plain truth of God.]




Part 31


_What the Fathers Thought concerning Sacrifice._

And since we have explained the passages of Scripture which are cited
against us, we must reply also concerning the Fathers.  We are not
ignorant that the Mass is called by the Fathers a sacrifice; but they
do not mean that the Mass confers grace _ex opere operato_, and that,
when applied on behalf of others, it merits for them the remission of
sins, of guilt and punishment.  Where are such monstrous stories to
be found in the Fathers?  But they openly testify that they are
speaking of thanksgiving.  Accordingly they call it a eucharist.  We
have said above, however, that a eucharistic sacrifice does not merit
reconciliation, but is made by those who have been reconciled, just
as afflictions do not merit reconciliation, but are eucharistic
sacrifices when those who have been reconciled endure them.

And this reply, in general, to the sayings of the Fathers defends us
sufficiently against the adversaries.  For it is certain that these
figments concerning the merit of the opus operatum are found nowhere
in the Fathers.  But in order that the whole case may be the better
understood, we also shall state those things concerning the use of
the Sacrament which actually harmonize with the Fathers and Scripture.




Part 32


Some clever men imagine that the Lord's Supper was instituted for two
reasons.  First, that it might be a mark and testimony of profession,
just as a particular shape of hood is the sign of a particular
profession.  Then they think that such a mark was especially pleasing
to Christ, namely, a feast to signify mutual union and friendship
among Christians, because banquets are signs of covenant and
friendship.  But this is a secular view; neither does it show the
chief use of the things delivered by God; it speaks only of the
exercise of love, which men, however profane and worldly, understand,
it does not speak of faith, the nature of which few understand.

The Sacraments are signs of God's will toward us, and not merely
signs of men among each other, and they are right in defining that
Sacraments in the New Testament are signs of grace.  And because in a
sacrament there are two things, a sign and the Word, the Word, in the
New Testament, is the promise of grace added.  The promise of the New
Testament is the promise of the remission of sins, as the text, Luke
22, 19, says: This is My body, which is given for you.  This cup is
the New Testament in My blood which is shed for many for the
remission of sins.  Therefore the Word offers the remission of sins.
And a ceremony is, as it were, a picture or seal, as Paul, Rom. 4, 11,
calls it, of the Word, making known the promise.  Therefore, just as
the promise is useless unless it is received by faith, so a ceremony
is useless unless such faith is added as is truly confident that the
remission of sins is here offered.  And this faith encourages
contrite minds.  And just as the Word has been given in order to
excite this faith, so the Sacrament has been instituted in order that
the outward appearance meeting the eyes might move the heart to
believe [and strengthen faith].  For through these, namely, through
Word and Sacrament, the Holy Ghost works.

And such use of the Sacrament, in which faith quickens terrified
hearts, is a service of the New Testament, because the New Testament
requires spiritual dispositions, mortification and quickening.  [For
according to the New Testament the highest service of God is rendered
inwardly in the heart.] And for this use Christ instituted it, since
He commanded them thus to do in remembrance of Him.  For to remember
Christ is not the idle celebration of a show [not something that is
accomplished only by some gestures and actions], or one instituted
for the sake of example, as the memory of Hercules or Ulysses is
celebrated in tragedies, but it is to remember the benefits of Christ
and receive them by faith so as to be quickened by them.  Psalm 111,
4. 5 accordingly says: He hath made His wonderful works to be
remembered: the Lord is gracious and full of compassion.  He hath
given meat unto them that fear Him.  For it signifies that the will
and mercy of God should be discerned in the ceremony.  But that faith
which apprehends mercy quickens.  And this is the principal use of
the Sacrament, in which it is apparent who are fit for the Sacrament,
namely, terrified consciences and how they ought to use it.

The sacrifice [thank-offering or thanksgiving] also is added.  For
there are several ends for one object.  After conscience encouraged
by faith has perceived from what terrors it is freed, then indeed it
fervently gives thanks for the benefit and passion of Christ, and
uses the ceremony itself to the praise of God, in order by this
obedience to show its gratitude; and testifies that it holds in high
esteem the gifts of God.  Thus the ceremony becomes a sacrifice of
praise.

And the Fathers, indeed, speak of a twofold effect, of the comfort of
consciences, and of thanksgiving, or praise.  The former of these
effects pertains to the nature [the right use] of the Sacrament; the
latter pertains to the sacrifice.  Of consolation Ambrose says: Go to
Him and be absolved, because He is the remission of sins.  Do you ask
who He is?  Hear Him when He says, John 6, 35: I am the Bread of life;
he that cometh to Me shall never hunger; and he that believeth on Me
shall never thirst.  This passage testifies that in the Sacrament the
remission of sins is offered; it also testifies that this ought to be
received by faith.  Infinite testimonies to this effect are found in
the Fathers, all of which the adversaries pervert to the _opus
operatum_, and to a work to be applied on behalf of others; although
the Fathers clearly require faith, and speak of the consolation
belonging to every one, and not of the application.

Besides these, expressions are also found concerning thanksgiving,
such as that most beautifully said by Cyprian concerning those
communing in a godly way.  Piety, says he, in thanksgiving the
Bestower of such abundant blessing, makes a distinction between what
has been given and what has been forgiven, i.e., piety regards both
what has been given and what has been forgiven, i.e., it compares the
greatness of God's blessings and the greatness of our evils, sin and
death, with each other, and gives thanks, etc. And hence the term
eucharist arose in the Church.  Nor indeed is the ceremony itself,
the giving of thanks ex opere operato, to be applied on behalf of
others, in order to merit for them the remission of sins, etc., in
order to liberate the souls of the dead.  These things conflict with
the righteousness of faith, as though, without faith, a ceremony can
profit either the one performing it or others.




Part 33


_Of the Term Mass._

The adversaries also refer us to philology.  From the names of the
Mass they derive arguments which do not require a long discussion.
For even though the Mass be called a sacrifice, it does not follow
that it must confer grace _ex opere operato_, or, when applied on
behalf of others, merit for them the remission of sins, etc.
_Leitourgia_, they say, signifies a sacrifice, and the Greeks call
the Mass liturgy.  Why do they here omit the old appellation synaxris,
which shows that the Mass was formerly the communion of many?  But
let us speak of the word liturgy.  This word done not properly
signify a sacrifice, but rather the public ministry, and agrees aptly
with our belief, namely, that one minister who consecrates tenders
the body and blood of the lord to the rest of the people, just as one
minister who preaches tenders the Gospel to the people, as Paul says,
1 Cor. 4, 1: Let a man so account of us as of the ministers of Christ
and stewards of the mysteries of God, i.e., of the Gospel and the
Sacraments.  And 2 Cor. 5, 20: We are ambassadors for Christ as
though God did beseech you by us; we pray you in Christ's stead, Be
ye reconciled to God.  Thus the term _Leitourgia_ agrees aptly with
the ministry.  For it is an old word, ordinarily employed in public
civil administrations, and signified to the Greeks public burdens, as
tribute, the expense of equipping a fleet, or similar things, as the
oration of Demosthenes, _FOR LEPTINES_, testifies, all of which is
occupied with the discussion of public duties and immunities:
_Phehsei de anaxious tinas anthrohpous euromenous ateleian
ekdedukenai tas leitourgias_, i.e.: He will say that some unworthy
men, having found an immunity, have withdrawn from public burdens.
And thus they spoke in the time of the Romana, as the rescript of
Pertinax, _De Iure Immunitatis_, l.  Semper, shows: _Ei kai meh
pasohn leitourgiohn tous pateras ho tohn teknohn arithmos aneitai_,
Even though the number of children does not liberate parents from all
public burdens.  And the Commentary upon Demosthenes states that
_leitourgia_ is a kind of tribute, the expense of the games, the
expense of equipping vessels, of attending to the gymnasia and
similar public offices.  And Paul in 2 Cor. 9, 12 employs it for a
collection.  The taking of the collection not only supplies those
things which are wanting to the saints, but also causes them to give
more thanks abundantly to God, etc. And in Phil. 2, 25 he calls
Epaphroditus a _leitourgos_, one who ministered to my wants, where
assuredly a sacrificer cannot be understood.  But there is no need of
more testimonies, since examples are everywhere obvious to those
reading the Greek writers, in whom _leitourgia_ is employed for
public civil burdens or ministries.  And on account of the diphthong,
grammarians do not derive it from _liteh_, which signifies prayers,
but from public goods, which they call _leita_, so that _leitourgeoh_
means, I attend to, I administer public goods.

Ridiculous is their inference that, since mention is made in the Holy
Scriptures of an altar, therefore the Mass must be a sacrifice; for
the figure of an altar is referred to by Paul only by way of
comparison.  And they fabricate that the Mass has been so called from
_mzbh_, an altar.  What need is there of an etymology so far fetched,
unless it be to show their knowledge of the Hebrew language?  What
need is there to seek the etymology from a distance, when the term
Mass is found in Deut. 16, 10, where it signifies the collections or
gifts of the people, not the offering of the priest?  For individuals
coming to the celebration of the Passover were obliged to bring some
gift as a contribution.  In the beginning the Christians also
retained this custom.  Coming together they brought bread, wine, and
other things, as the Canons of the Apostles testify.  Thence a part
was taken to be consecrated; the rest was distributed to the poor.
With this custom they also retained Mass as the name of the
contributions.  And on account of such contributions it appears also
that the Mass was elsewhere called _agapeh_, unless one would prefer
that it was so called on account of the common feast.  But let us
omit these trifles.  For it is ridiculous that the adversaries should
produce such trifling conjectures concerning a matter of such great
importance.  For although the Mass is called an offering, in what
does the term favor the dreams concerning the _opus operatum_, and
the application which, they imagine, merits for others the remission
of sins?  And it can be called an offering for the reason that
prayers, thanksgivings, and the entire worship are there offered, as
it is also called a eucharist.  But neither ceremonies nor prayers
profit _ex opere operato_, without faith.  Although we are disputing
here not concerning prayers, but particularly concerning the Lord's
Supper.

[Here you can see what rude asses our adversaries are.  They say that
the term _missa_ is derived from the term _misbeach_, which signifies
an altar; hence we are to conclude that the Mass is a sacrifice; for
sacrifices are offered on an altar.  Again, the word _liturgia_, by
which the Greeks call the Mass, is also to denote a sacrifice.  This
claim we shall briefly answer.  All the world sees that from such
reasons this heathenish and antichristian error does not follow
necessarily, that the Mass benefits _ex opere operato sine bono motu
utentis_.  Therefore they are asses, because in such a highly
important matter they bring forward such silly things.  Nor do the
asses know any grammar.  For missa and liturgia do not mean sacrifice.
_Missa_, in Hebrew, denotes a joint contribution.  For this may have
been a custom among Christians, that they brought meat and drink for
the benefit of the poor to their assemblies.  This custom was derived
from the Jews, who had to bring such contributions on their festivals,
these they called _missa_.  Likewise, _liturgia_, in Greek, really
denotes an office in which a person ministers to the congregation.
This is well applied to our teaching, because with us the priest, as
a common servant of those who wish to commune, ministers to them the
holy Sacrament.

Some think that _missa_ is not derived from the Hebrew, but signifies
as much as _remissio_ the forgiveness of sin.  For, the communion
being ended, the announcement used to be made: _Ite, missa est_:
Depart, you have forgiveness of sins.  They cite, as proof that this
is so, the fact that the Greeks used to say: _Lais Aphesis (laois
aphsesis)_, which also means that they had been pardoned.  If this
were so, it would be an excellent meaning, for in connection with
this ceremony forgiveness of sins must always be preached and
proclaimed.  But the case before us is little aided, no matter what
the meaning of the word _missa_ is.]

The Greek canon says also many things concerning the offering, but it
shows plainly that it is not speaking properly of the body and blood
of the Lord, but of the whole service of prayers and thanksgivings.
For it says thus: _Kai poiehson hemas axious genesthai tou
prospserein soi deehseis kai hikesias kai thusias anaimaktous huper
pantos laou._ When this is rightly understood, it gives no offense.
For it prays that we be made worthy to offer prayers and
supplications and bloodless sacrifices for the people.  For he calls
even prayers bloodless sacrifices.  Just as also a little afterward:
_Eti prospheromen soi tehn logikehn tautehn kai anaimakton latreian_,
We offer, he says this reasonable and bloodless service.  For they
explain this inaptly who would rather interpret this of a reasonable
sacrifice, and transfer it to the very body of Christ, although the
canon speaks of the entire worship, and in opposition to the _opus
operatum_ Paul has spoken of _logikeh latreia_ [reasonable service],
namely, of the worship of the mind, of fear, of faith, of prayer, of
thanksgiving, etc.




Part 34


_Of the Mass for the Dead._

Our adversaries have no testimonies and no command from Scripture for
defending the application of the ceremony for liberating the souls of
the dead, although from this they derive infinite revenue.  Nor,
indeed, is it a light sin to establish such services in the Church
without the command of God and without the example of Scripture, and
to apply to the dead the Lord's Supper, which was instituted for
commemoration and preaching among the living [for the purpose of
strengthening the faith of those who use the ceremony].  This is to
violate the Second Commandment, by abusing God's name.

For, in the first place, it is a dishonor to the Gospel to hold that
a ceremony _ex opere operato_, without faith, is a sacrifice
reconciling God, and making satisfaction for sins.  It is a horrible
saying to ascribe as much to the work of a priest as to the death of
Christ.  Again, sin and death cannot be overcome unless by faith in
Christ, as Paul teaches, Rom. 5, 1: Being justified by faith, we have
peace with God, and therefore the punishment of purgatory cannot be
overcome by the application of the work of another.

Now we shall omit the sort of testimonies concerning purgatory that
the adversaries have: what kinds of punishments they think there are
in purgatory, what grounds the doctrine of satisfactions has, which
we have shown above to be most vain.  We shall only present this in
opposition: It is certain that the Lord's Supper was instituted on
account of the remission of guilt.  For it offers the remission of
sins, where it is necessary that guilt be truly understood.  [For
what consolation would we have if forgiveness of sin were here
offered us, and yet there would be no remission of guilt?] And
nevertheless it does not make satisfaction for guilt, otherwise the
Mass would be equal to the death of Christ.  Neither can the
remission of guilt be received in any other way than by faith.
Therefore the Mass is not a satisfaction, but a promise and Sacrament
that require faith.

And, indeed, it is necessary that all godly persons be seized with
the most bitter grief [shed tears of blood, from anguish and sorrow]
if they consider that the Mass has been in great part transferred to
the dead and to satisfactions for punishments.  This is to banish the
daily sacrifice from the Church; this is the kingdom of Antiochus,
who transferred the most salutary promises concerning the remission
of guilt and concerning faith to the most vain opinions concerning
satisfactions; this is to defile the Gospel, to corrupt the use of
the Sacraments.  These are the persons [the real blasphemers] whom
Paul has said, 1 Cor. 11, 27, to be guilty of the body and blood of
the Lord, who have suppressed the doctrine concerning faith and the
remission of sins, and, under the pretext of satisfactions, have
devoted the body and blood of the Lord to sacrilegious gain.  And
they will at some time pay the penalty for this sacrilege.  [God will
one day vindicate the Second Commandment, and pour out a great,
horrible wrath upon them.] Therefore we and all godly consciences
should be on our guard against approving the abuses of the
adversaries.

But let us return to the case.  Since the Mass is not a satisfaction,
either for punishment or for guilt, _ex opere operato_, without faith,
it follows that the application on behalf of the dead is useless.
Nor is there need here of a longer discussion.  For it is evident
that these applications on behalf of the dead have no testimonies
from the Scriptures.  Neither is it safe, without the authority of
Scripture, to institute forms of worship in the Church.  And if it
will at any time be necessary, we shall speak at greater length
concerning this entire subject.  For why should we now contend with
adversaries who understand neither what a sacrifice, nor what a
sacrament, nor what remission of sins, nor what faith is?

Neither does the Greek canon apply the offering as a satisfaction for
the dead, because it applies it equally for all the blessed
patriarchs, prophets, apostles.  It appears therefore that the Greeks
make an offering as thanksgiving, and do not apply it as satisfaction
for punishments.  [For, of course, it is not their intention to
deliver the prophets and apostles from purgatory, but only to offer
up thanks along and together with them for the exalted eternal
blessings that have been given to them and us.] Although they speak,
moreover, not of the offering alone of the body and blood of the Lord,
but of the other parts of the Mass, namely, prayers and thanksgiving.
For after the consecration they pray that it may profit those who
partake of it, they do not speak of others.  Then they add: _Eti
prospheromen soi tehn logikehn tautehn latreian huper tohn en pistei
anapausamenohn propatorohn, paterohn, patriarchohn, prophertohn,
apostolohn_, etc. ["Yet we offer to you this reasonable service for
those having departed in faith, forefathers, fathers, patriarchs
prophets, apostles," etc.] Reasonable service, however, does not
signify the offering itself, but prayers and all things which are
there transacted.  Now, as regards the adversaries' citing the
Fathers concerning the offering for the dead, we know that the
ancients speak of prayer for the dead, which we do not prohibit, but
we disapprove of the application _ex opere operato_ of the Lord's
Supper on behalf of the dead.  Neither do the ancients favor the
adversaries concerning the _opus operatum_.  And even though they
have the testimonies especially of Gregory or the moderns, we oppose
to them the most clear and certain Scriptures.  And there is a great
diversity among the Fathers.  They were men, and could err and be
deceived.  Although if they would now become alive again, and would
see their sayings assigned as pretexts for the notorious falsehoods
which the adversaries teach concerning the opus operatum, they would
interpret themselves far differently.

The adversaries also falsely cite against us the condemnation of
Aerius, who, they say was condemned for the reason that he denied
that in the Mass an offering is made for the living and the dead.
They frequently use this dexterous turn, cite the ancient heresies
and falsely compare our cause with these in order by this comparison
to crush us.  [The asses are not ashamed of any lies.  Nor do they
know who Aerius was and what he taught.] Epiphanius testifies that
Aerius held that prayers for the dead are useless.  With this he
finds fault.  Neither do we favor Aerius, but we on our part are
contending with you who are defending a heresy manifestly conflicting
with the prophets, apostles and holy Fathers, namely, that the Mass
justifies _ex opere operato_, that it merits the remission of guilt
and punishment even for the unjust, to whom it is applied, if they do
not present an obstacle.  Of these pernicious errors, which detract
from the glory of Christ's passion, and entirely overthrow the
doctrine concerning the righteousness of faith, we disapprove.  There
was a similar persuasion of the godless in the Law, namely, that they
merited the remission of sins, not freely by faith, but through
sacrifices _ex opere operato_.  Therefore they increased these
services and sacrifices, instituted the worship of Baal in Israel,
and even sacrificed in the groves in Judah.  Therefore the prophets
condemn this opinion, and wage war not only with the worshipers of
Baal, but also with other priests who, with this godless opinion,
made sacrifices ordained by God.  But this opinion inheres in the
world, and always will inhere namely, that services and sacrifices
are propitiations.  Carnal men cannot endure that alone to the
sacrifice of Christ the honor is ascribed that it is a propitiation,
because they do not understand the righteousness of faith, but
ascribe equal honor to the rest of the services and sacrifices.  Just
as, therefore, in Judah among the godless priests a false opinion
concerning sacrifices inhered, just as in Israel, Baalitic services
continued, and, nevertheless, a Church of God was there which
disapproved of godless services, so Baalitic worship inheres in the
domain of the Pope, namely, the abuse of the Mass, which they apply,
that by it they may merit for the unrighteous the remission of guilt
and punishment.  [And yet, as God still kept His Church, i.e., some
saints, in Israel and Judah, so God still preserved His Church, i.e.,
some saints, under the Papacy, so that the Christian Church has not
entirely perished.] And it seems that this Baalitic worship will
endure as long as the reign of the Pope, until Christ will come to
judge, and by the glory of His advent destroy the reign of Antichrist.
Meanwhile all who truly believe the Gospel [that they may truly
honor God and have a constant comfort against sins; for God has
graciously caused His Gospel to shine, that we might be warned and
saved] ought to condemn these wicked services, devised, contrary to
God's command, in order to obscure the glory of Christ and the
righteousness of faith.

We have briefly said these things of the Mass in order that all good
men in all parts of the world may be able to understand that with the
greatest zeal we maintain the dignity of the Mass and show its true
use, and that we have the most just reasons for dissenting from the
adversaries.  And we would have all good men admonished not to aid
the adversaries in the profanation of the Mass lest they burden
themselves with other men's sin.  It is a great cause and a great
subject not inferior to the transaction of the prophet Elijah, who
condemned the worship of Baal.  We have presented a case of such
importance with the greatest moderation, and now reply without
casting any reproach.  But if the adversaries will compel us to
collect all kinds of abuses of the Mass, the case will not be treated
with such forbearance.




Part 35


Article XXVII (XIII): _Of Monastic Vows._

In the town of Eisenach, in Thuringia, there was, to our knowledge, a
monk, John Hilten, who, thirty years ago, was cast by his fraternity
into prison because he had protested against certain most notorious
abuses.  For we have seen his writings, from which it can be well
understood what the nature of his doctrine was [that he was a
Christian, and preached according to the Scriptures].  And those who
knew him testify that he was a mild old man, and serious indeed, but
without moroseness.  He predicted many things, some of which have
thus far transpired, and others still seem to impend which we do not
wish to recite, lest it may be inferred that they are narrated either
from hatred toward one or from partiality to another.  But finally,
when, either on account of his age or the foulness of the prison, he
fell into disease, he sent for the guardian in order to tell him of
his sickness; and when the guardian, inflamed with pharisaic hatred,
had begun to reprove the man harshly on account of his kind of
doctrine, which seemed to be injurious to the kitchen, then, omitting
all mention of his sickness, he said with a sigh that he was bearing
these injuries patiently for Christ's sake, since he had indeed
neither written nor taught anything which could overthrow the
position of the monks, but had only protested against some well-known
abuses.  But another one he said, will come in A.D. 1516, who will
destroy you, neither will you be able to resist him.  This very
opinion concerning the downward career of the power of the monks, and
this number of years, his friends afterwards found also written by
him in his commentaries, which he had left, concerning certain
passages of Daniel.  But although the outcome will teach how much
weight should be given to this declaration, yet there are other signs
which threaten a change in the power of the monks, that are no less
certain than oracles.  For it is evident how much hypocrisy, ambition,
avarice there is in the monasteries, how much ignorance and cruelty
among all the unlearned, what vanity in their sermons and in devising
continually new means of gaining money.  [The more stupid asses the
monks are, the more stubborn, furious bitter, the more venomous asps
they are in persecuting the truth and the Word of God.] And there are
other faults, which we do not care to mention.  While they once were
[not jails or everlasting prisons, but] schools for Christian
instruction, now they have degenerated, as though from a golden to an
iron age, or as the Platonic cube degenerates into bad harmonies,
which, Plato says brings destruction.  [Now this precious gold is
turned to dross, and the wine to water.] All the most wealthy
monasteries support only an idle crowd, which gluttonizes upon the
public alms of the Church.  Christ, however, teaches concerning the
salt that has lost its savor that it should be cast out and be
trodden under foot, Matt. 5, 13. Therefore the monks by such morals
are singing their own fate [requiem, and it will soon be over with
them].  And now another sign is added, because they are in many
places, the instigators of the death of good men.  [This blood of
Abel cries against them and] These murders God undoubtedly will
shortly avenge.  Nor indeed do we find fault with all, for we are of
the opinion that there are here and there some good men in the
monasteries who judge moderately concerning human and factitious
services, as some writers call them, and who do not approve of the
cruelty which the hypocrites among them exercise.

But we are now discussing the kind of doctrine which the composers of
the _Confutation_ are now defending and not the question whether vows
should be observed.  For we hold that lawful vows ought to be
observed; but whether these services merit the remission of sins and
justification; whether they are satisfactions for sins, whether they
are equal to Baptism, whether they are the observance of precepts and
counsels; whether they are evangelical perfection; whether they have
the merits of supererogation; whether these merits, when applied on
behalf of others save them, whether vows made with these opinions are
lawful; whether vows are lawful that are undertaken under the pretext
of religion, merely for the sake of the belly and idleness, whether
those are truly vows that have been extorted either from the
unwilling or from those who on account of age were not able to judge
concerning the kind of life, whom parents or friends thrust into the
monasteries that they might be supported at the public expense,
without the loss of private patrimony, whether vows are lawful that
openly tend to an evil issue, either because on account of weakness
they are not observed, or because those who are in these fraternities
are compelled to approve and aid the abuses of the Mass, the godless
worship of saints, and the counsels to rage against good men:
concerning these questions we are treating.  And although we have
said very many things in the Confession concerning such vows as even
the canons of the Popes condemn, nevertheless the adversaries command
that all things which we have produced be rejected.  For they have
used these words.

And it is worth while to hear how they pervert our reasons, and what
they adduce to fortify their own cause.  Accordingly, we will briefly
run over a few of our arguments, and in passing, explain away the
sophistry of the adversaries in reference to them.  Since, however,
this entire cause has been carefully and fully treated by Luther in
the book to which he gave the title _De Votis Monasticis_, we wish
here to consider that book as reiterated.

First, it is very certain that a vow is not lawful by which he who
vows thinks that he merits the remission of sins before God, or makes
satisfaction before God for sins.  For this opinion is a manifest
insult to the Gospel, which teaches that the remission of sins is
freely granted us for Christ's sake, as has been said above at some
length.  Therefore we have correctly quoted the declaration of Paul
to the Galatians, Gal. 5, 4: Christ is become of no effect unto you,
whosoever of you are justified by the Law; ye are fallen from grace.
Those who seek the remission of sins not by faith in Christ, but by
monastic works detract from the honor of Christ, and crucify Christ
afresh.  But hear, hear how the composers of the _Confutation_ escape
in this place!  They explain this passage of Paul only concerning the
Law of Moses, and they add that the monks observe all things for
Christ's sake, and endeavor to live the nearer the Gospel in order to
merit eternal life.  And they add a horrible peroration in these
words: Wherefore those things are wicked that are here alleged
against monasticism.  O Christ, how long wilt Thou bear these
reproaches with which our enemies treat Thy Gospel?  We have said in
the Confession that the remission of sins is received freely for
Christ's sake, through faith.  If this is not the very voice of the
Gospel, if it is not the judgment of the eternal Father, which Thou
who art in the bosom of the Father hast revealed to the world, we are
justly blamed.  But Thy death is a witness, Thy resurrection is a
witness, the Holy Ghost is a witness, Thy entire Church is a witness,
that it is truly the judgment of the Gospel that we obtain remission
of sins, not on account of our merits, but on account of Thee,
through faith.

When Paul denies that by the Law of Moses men merit the remission of
sins, he withdraws this praise much more from human traditions, and
this he clearly testifies Col. 2, 16. If the Law of Moses, which was
divinely revealed, did not merit the remission of sins, how much less
do these silly observances [monasticism rosaries, etc.], averse to
the civil custom of life, merit the remission of sins!


The adversaries feign that Paul abolishes the Law of Moses, and that
Christ succeeds in such a way that He does not freely grant the
remission of sins, but on account of the works of other laws, if any
are now devised.  By this godless and fanatical imagination they bury
the benefit of Christ.  Then they feign that among those who observe
this Law of Christ, the monks observe it more closely than others, on
account of their hypocritical poverty, obedience, and chastity, since
indeed all these things are full of sham.  In the greatest abundance
of all things they boast of poverty.  Although no class of men has
greater license than the monks [who have masterfully decreed that
they are exempt from obedience to bishops and princes], they boast of
obedience.  Of celibacy we do not like to speak, how pure this is in
most of those who desire to be continent, Gerson indicates.  And how
many of them desire to be continent [not to mention the thoughts of
their hearts]?

Of course, in this sham life the monks live more closely in
accordance with the Gospel!  Christ does not succeed Moses in such a
way as to remit sins on account of our works, but so as to set His
own merits and His own propitiation on our behalf against God's wrath
that we may be freely forgiven.  Now, he who apart from Christ's
propitiation, opposes his own merits to God's wrath, and on account
of his own merits endeavors to obtain the remission of sins, whether
he present the works of the Mosaic Law, or of the Decalog, or of the
rule of Benedict, or of the rule of Augustine, or of other rules,
annuls the promise of Christ, has cast away Christ, and has fallen
from grace.  This is the verdict of Paul.

But, behold, most clement Emperor Charles behold, ye princes, behold,
all ye ranks, how great is the impudence of the adversaries!
Although we have cited the declaration of Paul to this effect, they
have written: Wicked are those things that are here cited against
monasticism.  But what is more certain than that men obtain the
remission of sins by faith for Christ's sake?  And these wretches
dare to call this a wicked opinion!  We do not at all doubt that if
you had been advised of this passage, you would have taken [will
take] care that such blasphemy be removed from the _Confutation._

But since it has been fully shown above that the opinion is wicked,
that we obtain the remission of sins on account of our works, we
shall be briefer at this place.  For the prudent reader will easily
be able to reason thence that we do not merit the remission of sins
by monastic works.  Therefore this blasphemy also is in no way to be
endured which is read in

Thomas, that the monastic profession is equal to Baptism.  It is
madness to make human tradition, which has neither God's command nor
promise, equal to the ordinance of Christ which has both the command
and promise of God, which contains the covenant of grace and of
eternal life.

Secondly.  Obedience, poverty, and celibacy, provided the latter is
not impure, are, as exercises, adiaphora [in which we are not to look
for either sin or righteousness].  And for this reason the saints can
use these without impiety, just as Bernard, Franciscus, and other
holy men used them.  And they used them on account of bodily
advantage, that they might have more leisure to teach and to perform
other godly offices, and not that the works themselves are, by
themselves, works that justify or merit eternal life.  Finally they
belong to the class of which Paul says, 1 Tim. 4, 8: Bodily exercise
profiteth little.  And it is credible that in some places there are
also at present good men, engaged in the ministry of the Word, who
use these observances without wicked opinions [without hypocrisy and
with the understanding that they do not regard their monasticism as
holiness].  But to hold that these observances are services on
account of which they are accounted just before God, and through
which they merit eternal life, conflicts with the Gospel concerning
the righteousness of faith, which teaches that for Christ's sake
righteousness and eternal life are granted us.  It conflicts also
with the saying of Christ, Matt. 15, 9: In vain do they worship Me,
teaching for doctrines the commandments of men.  It conflicts also
with this statement, Rom. 14, 23: Whatsoever is not of faith is sin.
But how can they affirm that they are services which God approves as
righteousness before Him when they have no testimony of God's Word?

But look at the impudence of the adversaries!  They not only teach
that these observances are justifying services, but they add that
these services are more perfect, i.e. meriting more the remission of
sins and justification, than do other kinds of life [that they are
states of perfection, i.e., holier and higher states than the rest,
such as marriage, rulership].  And here many false and pernicious
opinions concur.  They imagine that they [are the most holy people
who] observe [not only] precepts and [but also] counsels [that is,
the superior counsels, which Scripture issues concerning exalted
gifts, not by way of command but of advice].  Afterwards these
liberal men, since they dream that they have the merits of
supererogation, sell these to others.  All these things are full of
pharisaic vanity.  For it is the height of impiety to hold that they
satisfy the Decalog in such a way that merits remain, while such
precepts as these are accusing all the saints: Thou shalt love the
Lord, thy God, with all shine heart, Deut. 6, 5. Likewise: Thou shalt
not covet, Rom. 7, 7. [For

as the First Commandment of God (Thou shalt love the Lord, thy God,
with all thy heart and with all thy soul and with all thy mind ) is
higher than a man upon earth can comprehend as it is the highest
theology, from which all the prophets and all the apostles have drawn
as from a spring their best and highest doctrines, yea, as it is such
an exalted commandment, according to which alone all divine service,
all honor to God, every offering, all thanksgiving in heaven and upon
earth, must be regulated and judged, so that all divine service high
and precious and holy though it appear if it be not in accordance
with this commandment, is nothing but husks and shells without a
kernel, yea, nothing but filth and abomination before God; which
exalted commandment no saint whatever has perfectly fulfilled, so
that even Noah and Abraham, David, Peter and Paul acknowledged
themselves imperfect and sinners: it is an unheard-of, pharisaic, yea,
an actually diabolical pride for a sordid Barefooted monk or any
similar godless hypocrite to say, yea, preach and teach, that he has
observed and fulfilled the holy high commandment so perfectly, and
according to the demands and will of God has done so many good works,
that merit even superabounds to him.  Yea, dear hypocrites, if the
holy Ten Commandments and the exalted First Commandment of God were
fulfilled as easily as the bread and remnants are put into the sack!
They are shameless hypocrites with whom the world is plagued in this
last time.] The prophet says, Ps. 116, 11: All men are liars, i.e.,
not thinking aright concerning God, not fearing God sufficiently, not
believing Him sufficiently.  Therefore the monks falsely boast that
in the observance of a monastic life the commandments are fulfilled,
and more is done than what is commanded [that their good works and
several hundredweights of superfluous, superabundant holiness remain
in store for them].

Again, this also is false, namely, that monastic observances are
works of the counsels of the Gospel.  For the Gospel does not advise
concerning distinctions of clothing and meats and the renunciation of
property.  These are human traditions, concerning all of which it has
been said, 1 Cor. 8, 8: Meat commendeth us not to God.  Therefore
they are neither justifying services nor perfection; yea, when they
are presented covered with these titles, they are mere doctrines of
demons.

Virginity is recommended, but to those who have the gift, as has been
said above.  It is, however, a most pernicious error to hold that
evangelical perfection lies in human traditions.  For thus the monks
even of the Mohammedans would be able to boast that they have
evangelical perfection.  Neither does it lie in the observance of
other things which are called adiaphora, but because the kingdom of
God is righteousness and life

in hearts, Rom. 14, 17, perfection is growth in the fear of God, and
in confidence in the mercy promised in Christ, and in devotion to
one's calling just as Paul also describes perfection 2 Cor. 3, 18: We
are changed from glory to glory, even as by the Spirit of the Lord.
He does not say: We are continually receiving another hood, or other
sandals, or other girdles.  It is deplorable that in the Church such
pharisaic, yea, Mohammedan expressions should be read and heard as,
that the perfection of the Gospel of the kingdom of Christ, which is
eternal life, should be placed in these foolish observances of
vestments and of similar trifles.

Now hear our Areopagites [excellent teachers] as to what an unworthy
declaration they have recorded in the Confutation.  Thus they say: It
has been expressly declared in the Holy Scriptures that the monastic
life merits eternal life if maintained by a due observance, which by
the grace of God any monk can maintain; and, indeed, Christ has
promised this as much more abundant to those who have left home or
brothers, etc., Matt. 19, 29. These are the words of the adversaries
in which it is first said most impudently that it is expressed in the
Holy Scriptures that a monastic life merits eternal life.  For where
do the Holy Scriptures speak of a monastic life!  Thus the
adversaries plead their case thus men of no account quote the
Scriptures.  Although no one is ignorant that the monastic life has
recently been devised, nevertheless they cite the authority of
Scripture, and say, too, that this their decree has been expressly
declared in the Scriptures.

Besides, they dishonor Christ when they say that by monasticism men
merit eternal life.  God has ascribed not even to His Law the honor
that it should merit eternal life, as He clearly says in Ezek. 20, 25:
I gave them also statutes that were not good, and judgments whereby
they should not live.  In the first place, it is certain that a
monastic life does not merit the remission of sins, but we obtain
this by faith freely, as has been said above.  Secondly, for Christ's
sake, through mercy, eternal life is granted to those who by faith
receive remission, and do not set their own merits against God's
judgment, as Bernard also says with very great force: It is necessary
first of all to believe that you cannot have the remission of sine
unless by God's indulgence.  Secondly, that you can have no good work
whatever, unless He has given also this.  Lastly, that you can merit
eternal life by no works, unless this also is given freely.  The rest
that follows to the same effect we have above recited.  Moreover,
Bernard adds at the end: Let no one deceive himself, because if he
will reflect well, he will undoubtedly find that with ten thousand he
cannot meet Him [namely, God] who cometh against him with twenty
thousand.  Since however, we do not

merit the remission of sins or eternal life by the works of the
divine Law, but it is necessary to seek the mercy promised in Christ,
much less is this honor of meriting the remission of sins or eternal
life to be ascribed to monastic observances since they are mere human
traditions.

Thus those who teach that the monastic life merits the remission of
sins or eternal life, and transfer the confidence due Christ to these
foolish observances, altogether suppress the Gospel concerning the
free remission of sins and the promised mercy in Christ that is to be
apprehended.  Instead of Christ they worship their own hoods and
their own filth.  But since even they need mercy, they act wickedly
in fabricating works of supererogation, and selling them [their
superfluous claim upon heaven] to others.

We speak the more briefly concerning these subjects, because from
those things which we have said above concerning justification,
concerning repentance, concerning human traditions, it is
sufficiently evident that monastic vows are not a price on account of
which the remission of sins and life eternal are granted.  And since
Christ calls traditions useless services, they are in no way
evangelical perfection.

But the adversaries cunningly wish to appear as if they modify the
common opinion concerning perfection.  They say that a monastic life
is not perfection, but that it is a state in which to acquire
perfection.  It is prettily phrased!  We remember that this
correction is found in Gerson.  For it is apparent that prudent men,
offended by these immoderate praises of monastic life, since they did
not venture to remove entirely from it the praise of perfection, have
added the correction that it is a state in which to acquire
perfection.  If we follow this, monasticism will be no more a state
of perfection than the life of a farmer or mechanic.  For these are
also states in which to acquire perfection.  For all men, in every
vocation, ought to seek perfection, that is, to grow in the fear of
God in faith, in love towards one's neighbor, and similar spiritual
virtues.

In the histories of the hermits there are examples of Anthony and of
others which make the various spheres of life equal.  It is written
that when Anthony asked God to show him what progress he was making
in this kind of life, a certain shoemaker in the city of Alexandria
was indicated to him in a dream to whom he should be compared.  The
next day Anthony came into the city, and went to the shoemaker in
order to ascertain his exercises and gifts, and, having conversed
with the man, heard nothing except that early in the morning he
prayed in a few words for the entire state, and then attended to his
trade. Here Anthony learned that justification is not to be ascribed to
the kind of life which he had entered [what God had meant by the
revelation; for we are justified before God not through this or that
life, but alone through faith in Christ].

But although the adversaries now moderate their praises concerning
perfection, yet they actually think otherwise.  For they sell merits,
and apply them on behalf of others under the pretext that they are
observing precepts and counsels, hence they actually hold that they
have superfluous merits.  But what is it to arrogate to one's self
perfection, if this is not?  Again, it has been laid down in the
_Confutation_ that the monks endeavor to live more nearly in
accordance with the Gospel.  Therefore it ascribes perfection to
human traditions if they are living more nearly in accordance with
the Gospel by not having property, being unmarried, and obeying the
rule in clothing, meats, and like trifles.

Again, the _Confutation_ says that the monks merit eternal life the
more abundantly, and quotes Scripture, Matt. 19, 29: Every one that
hath forsaken houses, etc. Accordingly, here, too, it claims
perfection also for factitious religious rites.  But this passage of
Scripture in no way favors monastic life.  For Christ does not mean
that to forsake parents, wife, brethren, is a work that must be done
because it merits the remission of sins and eternal life.  Yea, such
a forsaking is cursed.  For if any one forsakes parents or wife in
order by this very work to merit the remission of sins or eternal
life, this is done with dishonor to Christ.

There is, moreover, a twofold forsaking.  One occurs without a call,
without God's command; this Christ does not approve, Matt. 15, 9. For
the works chosen by us are useless services.  But that Christ does
not approve this flight appears the more clearly from the fact that
He speaks of forsaking wife and children.  We know, however, that
God's commandment forbids the forsaking of wife and children.  The
forsaking which occurs by God's command is of a different kind,
namely, when power or tyranny compels us either to depart or to deny
the Gospel.  Here we have the command that we should rather bear
injury, that we should rather suffer not only wealth, wife, and
children, but even life, to be taken from us.  This forsaking Christ
approves, and accordingly He adds: For the Gospel's sake, Mark 10, 29,
in order to signify that He is speaking not of those who do injury
to wife and children, but who bear injury on account of the
confession of the Gospel.  For the Gospel's sake we ought even to
forsake our body.  Here it would be ridiculous to hold that it would
be a service to God to kill one's self, and without God's command to
leave the body.  So, too, it is ridiculous to hold that it is a service
to God without God's command to forsake possessions, friends, wife,
children.

Therefore it is evident that they wickedly distort Christ's word to a
monastic life.  Unless perhaps the declaration that they "receive a
hundredfold in this life" be in place here.  For very many become
monks not on account of the Gospel but on account of sumptuous living
and idleness, who find the most ample riches instead of slender
patrimonies.  But as the entire subject of monasticism is full of
shams, so, by a false pretext they quote testimonies of Scripture,
and as a consequence they sin doubly, i.e., they deceive men, and
that, too, under the pretext of the divine name.

Another passage is also cited concerning perfection Matt. 19, 21: If
thou wilt be perfect, go and sell that thou hast, and give to the
poor, and come and follow Me.  This passage has exercised many, who
have imagined that it is perfection to cast away possessions and the
control of property.  Let us allow the philosophers to extol
Aristippus, who cast a great weight of gold into the sea.  [Cynics
like Diogenes, who would have no house, but lay in a tub, may commend
such heathenish holiness.] Such examples pertain in no way to
Christian perfection.  [Christian holiness consists in much higher
matters than such hypocrisy.] The division, control and possession of
property are civil ordinances, approved by God's Word in the
commandment, Ex. 20, 15: Thou shalt not steal.  The abandonment of
property has no command or advice in the Scriptures.  For evangelical
poverty does not consist in the abandonment of property, but in not
being avaricious, in not trusting in wealth, just as David was poor
in a most wealthy kingdom.

Therefore, since the abandonment of property is merely a human
tradition, it is a useless service.  Excessive also are the praises
in the Extravagant, which says that the abdication of the ownership
of all things for God's sake is meritorious and holy, and a way of
perfection.  And it is very dangerous to extol with such excessive
praises a matter conflicting with political order.  [When
inexperienced people hear such commendations, they conclude that it
is unchristian to hold property whence many errors and seditions
follow, through such commendations Muentzer was deceived, and thereby
many Anabaptists were led astray.] But [they say] Christ here speaks
of perfection.  Yea, they do violence to the text who quote it
mutilated.  Perfection is in that which Christ adds: Follow Me.  An
example of obedience in one's calling is here presented.  And as
callings are unlike [one is called to rulership, a second to be
father of a family, a third to be a preacher], so this calling does
not belong to all, but pertains properly to

that person with whom Christ there speaks, just as the call of David
to the kingdom, and of Abraham to slay his son, are not to be
imitated by us.  Callings are personal, just as matters of business
themselves vary with times and persons; but the example of obedience
is general.  Perfection would have belonged to that young man if he
had believed and obeyed this vocation.  Thus perfection with us is
that every one with true faith should obey his own calling.  [Not
that I should undertake a strange calling for which I have not the
commission or command of God.]

Thirdly.  In monastic vows chastity is promised.  We have said above,
however, concerning the marriage of priests, that the law of nature
[or of God] in men cannot be removed by vows or enactments.  And as
all do not have the gift of continence, many because of weakness are
unsuccessfully continent.  Neither, indeed, can any vows or any
enactments abolish the command of the Holy Ghost 1 Cor. 7, 2: To
avoid fornication, let every man have his own wife.  Therefore this
vow is not lawful in those who do not have the gift of continence,
but who are polluted on account of weakness.  Concerning this entire
topic enough has been said above, in regard to which indeed it is
strange, since the dangers and scandals are occurring before men's
eyes that the adversaries still defend their traditions contrary to
the manifest command of God.  Neither does the voice of Christ move
them, who chides the Pharisees, Matt. 23, 13 f., who had made
traditions contrary to God's command.

Fourthly.  Those who live in monasteries are released from their vows
by such godless ceremonies as of the Mass applied on behalf of the
dead for the sake of gain, the worship of saints, in which the fault
is twofold, both that the saints are put in Christ's place, and that
they are wickedly worshiped, just as the Dominicasters invented the
rosary of the Blessed Virgin, which is mere babbling not less foolish
than it is wicked, and nourishes the most vain presumption.  Then,
too, these very impieties are applied only for the sake of gain.
Likewise, they neither hear nor teach the Gospel concerning the free
remission of sins for Christ's sake, concerning the righteousness of
faith, concerning true repentance, concerning works which have God's
command.  But they are occupied either in philosophic discussions or
in the handing down of ceremonies that obscure Christ.

We will not here speak of the entire service of ceremonies, of the
lessons, singing, and similar things, which could be tolerated if
they [were regulated as regards number, and if they] would be
regarded as exercises, after the manner of lessons in the schools
[and preaching], whose design is to teach the hearers, and, while
teaching, to move some to fear or faith.

But now they feign that these ceremonies are services of God, which
merit the remission of sins for themselves and for others.  For on
this account they increase these ceremonies.  But if they would
undertake them in order to teach and exhort the hearers, brief and
pointed lessons would be of more profit than these infinite babblings.
Thus the entire monastic life is full of hypocrisy and false
opinions [against the First and Second Commandments, against Christ].
To all these this danger also is added, that those who are in these
fraternities are compelled to assent to those persecuting the truth.
There are, therefore, many important and forcible reasons which free
good men from the obligation to this kind of life.

Lastly, the canons themselves release many who either without
judgment [before they have attained a proper age] have made vows when
enticed by the tricks of the monks, or have made vows under
compulsion by friends.  Such vows not even the canons declare to be
vows.  From all these considerations it is apparent that there are
very many reasons which teach that monastic vows such as have
hitherto been made are not vows; and for this reason a sphere of life
full of hypocrisy and false opinions can be safely abandoned.

Here they present an objection derived from the Law concerning the
Nazarites, Num. 6, 2f.  But the Nazarites did not take upon
themselves their vows with the opinions which, we have hitherto said
we censure in the vows of the monks.  The rite of the Nazarites was
an exercise [a bodily exercise with fasting and certain kinds of
food] or declaration of faith before men, and did not merit the
remission of sins before God, did not justify before God.  [For they
sought this elsewhere, namely, in the promise of the blessed Seed.]
Again, just as circumcision or the slaying of victims would not be a
service of God now, so the rite of the Nazarites ought not to be
presented now as a service, but it ought to be judged simply as an
adiaphoron.  It is not right to compare monasticism, devised without
God's Word, as a service which should merit the remission of sins and
justification, with the rite of the Nazarites, which had God's Word,
and was not taught for the purpose of meriting the remission of sins,
but to be an outward exercise, just as other ceremonies of the Law.
The same can be said concerning other ceremonies prescribed in the
Law.

The Rechabites also are cited, who did not have any possessions, and
did not drink wine, as Jeremiah writes, chap. 35, 6f.  Yea, truly,
the example of the Rechabites accords beautifully with our monks,
whose monasteries excel the palaces of kings, and who live most
sumptuously!  And the Rechabites, in their poverty of all things,
were nevertheless married.  Our monks, although abounding in all
voluptuousness, profess celibacy.

Besides, examples ought to be interpreted according to the rule, i.e.,
according to certain and clear passages of Scripture, not contrary
to the rule, that is, contrary to the Scriptures.  It is very certain,
however, that our observances do not merit the remission of sins or
justification.  Therefore, when the Rechabites are praised, it is
necessary [it is certain] that these have observed their custom, not
because they believed that by this they merited remission of sins, or
that the work was itself a justifying service, or one on account of
which they obtained eternal life, instead of, by God's mercy, for the
sake of the promised Seed.  But because they had the command of their
parents, their obedience is praised, concerning which there is the
commandment of God: Honor thy father and mother.

Then, too, the custom had a particular purpose: Because they were
foreigners, not Israelites, it is apparent that their father wished
to distinguish them by certain marks from their countrymen, so that
they might not relapse into the impiety of their countrymen.  He
wished by these marks to admonish them of the [fear of God, the]
doctrine of faith and immortality.  Such an end is lawful.  But for
monasticism far different ends are taught.  They feign that the works
of monasticism are a service, they feign that they merit the
remission of sins and justification.  The example of the Rechabites
is therefore unlike monasticism; to omit here other evils which
inhere in monasticism at present.

They cite also from 1 Tim. 5, 11ff. concerning widows, who, as they
served the Church, were supported at the public expense, where it is
said: They will marry, having damnation, because they have cast off
their first faith.  First, let us suppose that the Apostle is here
speaking of vows [which, however, he is not doing]; still this
passage will not favor monastic vows, which are made concerning
godless services, and in this opinion that they merit the remission
of sins and justification.  For Paul with ringing voice condemns all
services, all laws, all works, if they are observed in order to merit
the remission of sins, or that, on account of them instead of through
mercy on account of Christ we obtain remission of sins.  On this
account the vows of widows, if there were any, must have been unlike
monastic vows.

Besides, if the adversaries do not cease to misapply the passage to
vows, the prohibition that no widow be selected who is less than
sixty years, 1 Tim. 5, 9, must be misapplied in the same way.  Thus
vows made before this age will be of no account.  But the Church did
not yet know these vows. Therefore Paul condemns widows, not because
they marry, for he commands the younger to marry; but because, when
supported at the public expense, they became wanton, and thus cast
off faith. He calls this first faith, clearly not in a monastic vow,
but in Christianity [of their Baptism, their Christian duty, their
Christianity]. And in this sense he understands faith in the same
chapter, v. 8: If any one provide not for his own, and specially for
those of his own house, he hath denied the faith. For he speaks
otherwise of faith than the sophists. He does not ascribe faith to
those who have mortal sin. He, accordingly, says that those cast off
faith who do not care for their relatives. And in the same way he
says that wanton women cast off faith.

We have recounted some of our reasons and, in passing, have explained
away the objections urged by the adversaries.  And we have collected
these matters, not only on account of the adversaries, but much more
on account of godly minds, that they may have in view the reasons why
they ought to disapprove of hypocrisy and fictitious monastic
services, all of which indeed this one saying of Christ annuls, which
reads, Matt. 15, 9: In vain they do worship Me, teaching for
doctrines the commandments of men.  Therefore the vows themselves and
the observances of meats, lessons, chants, vestments, sandals,
girdles are useless services in God's sight.  And all godly minds
should certainly know that the opinion is simply pharisaic and
condemned that these observances merit the remission of sins; that on
account of them we are accounted righteous, that on account of them,
and not through mercy on account of Christ, we obtain eternal life.
And the holy men who have lived in these kinds of life must
necessarily have learned, confidence in such observance having been
rejected, that they had the remission of sins freely, that for
Christ's sake through mercy they would obtain eternal life, and not
for the sake of these services [therefore godly persons who were
saved and continued to live in monastic life had finally come to this,
namely, that they despaired of their monastic life, despised all
their works as dung, condemned all their hypocritical service of God,
and held fast to the promise of grace in Christ, as in the example of
St. Bernard, saying, _Perdite vixi_, I have lived in a sinful way],
because God only approves services instituted by His Word, which
services avail when used in faith.




Part 36


Article XXVIII (XIV): _Of Ecclesiastical Power._

Here the adversaries cry out violently concerning the privileges and
immunities of the ecclesiastical estate, and they add the peroration:
All things are vain which are presented in the present article
against the immunity of the churches and priests.  This is mere
calumny; for in this article we have disputed concerning other things.
Besides, we have frequently testified that we do not find fault
with political ordinances, and the gifts and privileges granted by
princes.

But would that the adversaries would hear, on the other hand, the
complaints of the churches and of godly minds!  The adversaries
courageously guard their own dignities and wealth; meanwhile, they
neglect the condition of the churches; they do not care that the
churches are rightly taught, and that the Sacraments are duly
administered.  To the priesthood they admit all kinds of persons
indiscriminately.  [They ordain rude asses; thus the Christian
doctrine perished, because the Church was not supplied with efficient
preachers.] Afterwards they impose intolerable burdens, as though
they were delighted with the destruction of their fellowmen, they
demand that their traditions be observed far more accurately than the
Gospel.  Now, in the most important and difficult controversies,
concerning which the people urgently desire to be taught, in order
that they may have something certain which they may follow, they do
not release the minds which are most severely tortured with doubt,
they only call to arms.  Besides, in manifest matters [against
manifest truth] they present decrees written in blood, which threaten
horrible punishments to men unless they act clearly contrary to God's
command.  Here, on the other hand, you ought to see the tears of the
poor, and hear the pitiable complaints of many good men, which God
undoubtedly considers and regards, to whom one day you will render an
account of your stewardship.

But although in the Confession we have in this article embraced
various topics, the adversaries make no reply [act in true popish
fashion], except that the bishops have the power of rule and coercive
correction, in order to direct their subjects to the goal of eternal
blessedness; and that the power of ruling requires the power to judge,
to define, to distinguish and fix those things which are serviceable
or conduce to the aforementioned end.  These are the words of the
_Confutation_, in which the adversaries teach us [but do not prove]
that the bishops have the authority to frame laws [without the
authority of the Gospel] useful for obtaining eternal life.  The
controversy is concerning this article.

[Regarding this matter we submit the following:] But we must retain
in the Church this doctrine, namely, that we receive the remission of
sins freely for Christ's sake, by faith.  We must also retain this
doctrine, namely, that human traditions are useless services, and
therefore neither sin nor righteousness should be placed in meat
drink, clothing and like things, the use of which Christ wished to be
left free, since He says, Matt. 15, 11: Not that which goeth into the
mouth defileth the man; and Paul, Rom. 14, 17: The kingdom of God is
not meat and drink.  Therefore the bishops have no right to frame
traditions in addition to the Gospel, that they may merit the
remission of sins, that they may be services which God is to approve
as righteousness and which burden consciences, as though it were a
sin to omit them.  All this is taught by that one passage in Acts, 15,
9ff., where the apostles say [Peter says] that hearts are purified
by faith.  And then they prohibit the imposing of a yoke, and show
how great a danger this is, and enlarge upon the sin of those who
burden the Church.  Why tempt ye God they say.  By this thunderbolt
our adversaries are in no way terrified, who defend by violence
traditions and godless opinions.

For above they have also condemned Article XV, in which we have
stated that traditions do not merit the remission of sins, and they
here say that traditions conduce to eternal life.  Do they merit the
remission of sins?  Are they services which God approves as
righteousness?  Do they quicken hearts!  Paul to the Colossians, 2,
20ff., says that traditions do not profit with respect to eternal
righteousness and eternal life; for the reason that food, drink,
clothing and the like are things that perish with the using.  But
eternal life [which begins in this life inwardly by faith] is wrought
in the heart by eternal things, i.e., by the Word of God and the Holy
Ghost.  Therefore let the adversaries explain how traditions conduce
to eternal life.

Since, however, the Gospel clearly testifies that traditions ought
not to be imposed upon the Church in order to merit the remission of
sins; in order to be services which God shall approve as
righteousness; in order to burden consciences, so that to omit them
is to be accounted a sin, the adversaries will never be able to show
that the bishops have the power to institute such services.

Besides, we have declared in the Confession what power the Gospel
ascribes to bishops.  Those who are now bishops do not perform the
duties of bishops according to the Gospel although, indeed, they may
be bishops according to canonical polity, which we do not censure.
But we are speaking of a bishop according to the Gospel.  And we are
pleased with the ancient division of power into power of the order
and power of jurisdiction [that is the administration of the
Sacraments and the exercise of spiritual jurisdiction].  Therefore
the bishop has the power of the order, i.e., the ministry of the Word
and Sacraments; he has also the power of jurisdiction, i.e., the
authority to excommunicate those guilty of open crimes, and again to
absolve them if they are converted and seek absolution.  But their
power is not to be tyrannical, i.e., without a fixed law; nor regal,
i.e., above law; but they have a fixed command and a fixed Word of
God, according to which they ought to teach and according to which
they ought to exercise their jurisdiction.  Therefore, even though
they should have some jurisdiction, it does not follow that they are
able to institute new services.  For services pertain in no way to
jurisdiction.  And they have the Word, they have the command, how far
they ought to exercise jurisdiction, namely, if any one would do
anything contrary to that Word which they have received from Christ.
[For the Gospel does not set up a rule independently of the Gospel;
that is quite clear and certain.]

Although in the Confession we also have added how far it is lawful
for them to frame traditions, namely, not as necessary services, but
so that there may be order in the Church, for the sake of
tranquillity.  And these traditions ought not to cast snares upon
consciences, as though to enjoin necessary services; as Paul teaches
when he says, Gal. 5, 1: Stand fast, therefore, in the liberty
wherewith Christ hath made us free, and be not entangled again with
the yoke of bondage.  The use of such ordinances ought therefore to
be left free, provided that offenses be avoided, and that they be not
judged to be necessary services; just as the apostles themselves
ordained [for the sake of good discipline] very many things which
have been changed with time.  Neither did they hand them down in such
a way that it would not be permitted to change them.  For they did
not dissent from their own writings, in which they greatly labor lest
the Church be burdened with the opinion that human rites are
necessary services.

This is the simple mode of interpreting traditions, namely, that we
understand them not as necessary services, and nevertheless, for the
sake of avoiding offenses, we should observe them in the proper place.
And thus many learned and great men in the Church have held.  Nor
do we see what can be said against this.  For it is certain that the
expression Luke 10, 16: He that heareth you heareth Me, does not
speak of traditions, but is chiefly directed against traditions.  For
it is not a _mandatum cum libera_ ( a bestowal of unlimited
authority), as they call it, but it is a _cautio de rato_ (a caution
concerning something prescribed), namely, concerning the special
command [not a free, unlimited order and power, but a limited order,
namely, not to preach their own word, but God's Word and the Gospel],
i.e., the testimony given to the apostles that we believe them with
respect to the word of another, not their own.  For Christ wishes to
assure us, as was necessary, that we should know that the Word
delivered by men is efficacious, and that no other word from heaven
ought to be sought.  He that heareth you heareth Me, cannot be
understood of traditions.  For Christ requires that they teach in
such a way that [by their mouth] He Himself be heard, because He says:
He heareth Me.  Therefore He wishes His own voice, His own Word, to
be heard, not human traditions.  Thus a saying which is most
especially in our favor, and contains the most important consolation
and doctrine, these stupid men pervert to the most trifling matters,
the distinctions of food, vestments, and the like.

They quote also Heb. 13, 17: Obey them that have the rule over you.
This passage requires obedience to the Gospel.  For it does not
establish a dominion for the bishops apart from the Gospel.  Neither
should the bishops frame traditions contrary to the Gospel, or
interpret their traditions contrary to the Gospel.  And when they do
this, obedience is prohibited, according to Gal. 1, 9: If any man
preach any other gospel, let him be accursed.

We make the same reply to Matt. 23, 3: Whatsoever they bid you
observe, that observe, because evidently a universal command is not
given that we should receive all things [even contrary to God's
command and Word], since Scripture elsewhere, Acts 5, 29, bids us
obey God rather than men.  When, therefore they teach wicked things,
they are not to be heard.  But these are wicked things, namely, that
human traditions are services of God that they are necessary services,
that they merit the remission of sins and eternal life.

They present, as an objection, the public offenses and commotions
which have arisen under pretext of our doctrine.  To these we briefly
reply.  If all the scandals be brought together, still the one
article concerning the remission of sins, that for Christ's sake
through faith we freely obtain the remission of sins, brings so much
good as to hide all evils.  And this, in the beginning, gained for
Luther not only our favor, but also, that of many who are now
contending against us.  "For former favor ceases, and mortals are
forgetful," says Pindar.  Nevertheless, we neither desire to desert
truth that is necessary to the Church, nor can we assent to the
adversaries in condemning it.  For we ought to obey God rather than
men.  Those who in the beginning condemned manifest truth, and are
now persecuting it with the greatest cruelty, will give an account
for the schism that has been occasioned.  Then, too, are there no
scandals among the adversaries?  How much evil is there in the
sacrilegious profanation of the Mass applied to gain!  How great
disgrace in celibacy!  But let us omit a comparison.  This is what we
hare replied to the _Confutation_ for the time being.  Now we leave
it to the judgment of all the godly whether the adversaries are right
in boasting that they have actually refuted our Concession from the
Scriptures.




Part 37


_THE END._

[As regards the slander and complaint of the adversaries at the end
of the _Confutation_, namely, that this doctrine is causing
disobedience and other scandals, this is unjustly imputed to our
doctrine.  For it is evident that by this doctrine the authority of
magistrates is most highly praised.  Moreover, it is well known that
in those localities where this doctrine is preached, the magistrates
have hitherto by the grace of God, been treated with all respect by
the subjects.

But as to the want of unity and dissension in the Church, it is well
known how these matters first happened, and who have caused the
division, namely, the sellers of indulgences, who shamelessly
preached intolerable lies, and afterwards condemned Luther for not
approving of those lies, and besides, they again and again excited
more controversies, so that Luther was induced to attack many other
errors.  But since our opponents would not tolerate the truth, and
dared to promote manifest errors by force, it is easy to judge who is
guilty of the schism.  Surely, all the world, all wisdom, all power
ought to yield to Christ and His holy Word.  But the devil is the
enemy of God, and therefore rouses all his might against Christ, to
extinguish and suppress the Word of God.  Therefore the devil with
his members, setting himself against the Word of God, is the cause of
the schism and want of unity.  For we have most zealously sought
peace, and still most eagerly desire it, provided only we are not
forced to blaspheme and deny Christ.  For God, the discerner of all
men's hearts, is our witness that we do not delight and have no joy
in this awful disunion.  On the other hand, our adversaries have so
far not been willing to conclude peace without stipulating that we
must abandon the saving doctrine of the forgiveness of sin by Christ
without our merit; though Christ would be most foully blasphemed
thereby.

And although, as is the custom of the world it cannot be but that
offenses have occurred in this schism through malice and by imprudent
people; for the devil causes such offenses, to disgrace the Gospel,
yet all this is of no account in view of the great comfort which this
teaching has brought men, that for Christ's sake, without our merit,
we have forgiveness of sins and a gracious God.  Again, that men have
been instructed that forsaking secular estates and magistracies is
not a divine worship, but that such estates and magistracies are
pleasing to God and to be engaged in them is a real holy work and
divine service.

If we also were to narrate the offenses of the adversaries, which,
indeed, we have no desire to do, it would be a terrible list: what an
abominable, blasphemous fair the adversaries have made of the Mass;
what unchaste living has been instituted by their celibacy; how the
Popes have for more than 400 years been engaged in wars against the
emperors, have forgotten the Gospel, and only sought to be emperors
themselves, and to bring all Italy into their power how they have
juggled the possessions of the Church; how through their neglect many
false teachings and forms of worship have been set up by the monks.
Is not their worship of the saints manifest pagan idolatry?  All
their writers do not say one word concerning faith in Christ, by
which forgiveness of sin is obtained; the highest degree of holiness
they ascribe to human traditions, it is chiefly of these that they
write and preach.  Moreover this, too, ought to be numbered with
their offenses, that they clearly reveal what sort of a spirit is in
them, because they are now putting to death so many innocent, pious
people on account of Christian doctrine.  But we do not now wish to
say more concerning this; for these matters should be decided in
accordance with God's Word, regardless of the offenses on either aide.

We hope that all God-fearing men will sufficiently see from this
writing of ours that ours is the Christian doctrine and comforting
and salutary to all godly men.  Accordingly, we pray God to extend
His grace to the end that His holy Gospel may be known and honored by
all, for His glory, and for the peace, unity, and salvation of all of
us.  Regarding all these articles we offer to make further statements
if required.]
