Epitome of the Formula of Concord

Comprehensive Summary, Rule and Norm According to which all dogmas should be judged, and the erroneous teachings [controversies]that have occurred should be decided and explained in a Christian way.

1. We believe, teach, and confess that the sole rule and standard according to which all dogmas together with [all] teachers should be estimated and judged are the prophetic and apostolic Scriptures of the Old and of the New Testament alone, as it is written Ps. 119:105: Thy Word is a lamp unto my feet and a light unto my path. And St. Paul: Though an angel from heaven preach any other gospel unto you, let him be accursed, Gal. 1:8.

Other writings, however, of ancient or modern teachers, whatever name they bear, must not be regarded as equal to the Holy Scriptures, but all of them together be subjected to them, and should not be received otherwise or further than as witnesses, [which are to show] in what manner after the time of the apostles, and at what places, this [pure] doctrine of the prophets and apostles was preserved.

2. And because directly after the times of the apostles, and even while they were still living, false teachers and heretics arose, and symbols, i. e., brief, succinct [categorical] confessions, were composed against them in the early Church, which were regarded as the unanimous, universal Christian faith and confession of the orthodox and true Church, namely, the Apostles' Creed, the Nicene Creed, and the Athanasian Creed, we pledge ourselves to them, and hereby reject all heresies and dogmas which, contrary to them, have been introduced into the Church of God.

3. As to the schisms in matters of faith, however, which have occurred in our time, we regard as the unanimous consensus and declaration of our Christian faith and confession, especially against the Papacy and its false worship, idolatry, superstition, and against other sects, as the symbol of our time, the First, Unaltered Augsburg Confession, delivered to the Emperor Charles V at Augsburg in the year 1530, in the great Diet, together with its Apology, and the Articles composed at Smalcald in the year 1537, and subscribed at that time by the chief theologians.

And because such matters concern also the laity and the salvation of their souls, we also confess the Small and Large Catechisms of Dr. Luther, as they are included in Luther's works, as the Bible of the laity, wherein everything is comprised which is treated at greater length in Holy Scripture, and is necessary for a Christian man to know for his salvation.

To this direction, as above announced, all doctrines are to be conformed, and what is, contrary thereto is to be rejected and condemned, as opposed to the unanimous declaration of our faith.

In this way the distinction between the Holy Scriptures of the Old and of the New Testament and all other writings is preserved, and the Holy Scriptures alone remain the only judge, rule, and standard, according to which, as the only test-stone, all dogmas shall and must be discerned and judged, as to whether they are good or evil, right or wrong.

But the other symbols and writings cited are not judges, as are the Holy Scriptures, but only a testimony and declaration of the faith, as to how at any time the Holy Scriptures have been understood and explained in the articles in controversy in the Church of God by those then living, and how the opposite dogma was rejected and condemned [by what arguments the dogmas conflicting with the Holy Scripture were rejected and condemned].
I. Original Sin.

STATUS CONTROVERSIAE.
The Principal Question in This Controversy.

Whether original sin is properly and without any distinction man's corrupt nature, substance, and essence, or at any rate the principal and best part of his essence [substance], namely, the rational soul itself in its highest state and powers; or whether, even after the Fall, there is a distinction between man's substance, nature, essence, body, soul, and original sin, so that the nature [itself] is one thing, and original sin, which inheres in the corrupt nature and corrupts the nature, another.

Affirmative Theses.
The Pure Doctrine, Faith, and Confession according to the Aforesaid Standard and Summary Declaration.

1. We believe, teach, and confess that there is a distinction between man's nature, not only as he was originally created by God pure and holy and without sin, but also as we have it [that nature] now after the Fall, namely, between the nature [itself], which even after the Fall is and remains a creature of God, and original sin, and that this distinction is as great as the distinction between a work of God and a work of the devil.

2. We believe, teach, and confess also that this distinction should be maintained with the greatest care, because this doctrine, that no distinction is to be made between our corrupt human nature and original sin, conflicts with the chief articles of our Christian faith concerning creation, redemption, sanctification, and the resurrection of our body, and cannot coexist therewith.

For God created not only the body and soul of Adam and Eve before the Fall, but also our bodies and souls after the Fall, notwithstanding that they are corrupt, which God also still acknowledges as His work, as it is written Job 10:8: Thine hands have made me and fashioned me together round about. Deut. 32:18; Is. 45:9ff; 54:5; 64:8; Acts 17:28; Job 10:8; Ps. 100:3; 139:14; Eccl. 12:1.

Moreover, the Son of God has assumed this human nature, however, without sin, and therefore not a foreign, but our own flesh, into the unity of His person, and according to it is become our true Brother. Heb. 2:14: Forasmuch, then, as the children were partakers of flesh and blood, He also Himself likewise took part of the same. Again, 16; 4:15: He took not on Him the nature of angels, but He took on Him the seed of Abraham. Wherefore in all things it behooved Him to be made like unto His brethren, yet without sin. 6] In like manner Christ has also redeemed it as His work, sanctifies it as His work, raises it from the dead, and gloriously adorns it as His work. But original sin He has not created, assumed, redeemed, sanctified; nor will He raise it, will neither adorn nor save it in the elect, but in the [blessed] resurrection it will be entirely destroyed.

Hence the distinction between the corrupt nature and the corruption which infects the nature and by which the nature became corrupt, can easily be discerned.

3. But, on the other hand, we believe, teach, and confess that original sin is not a slight, but so deep a corruption of human nature that nothing healthy or uncorrupt has remained in man's body or soul, in his inner or outward powers, but, as the Church sings:

Through Adam's fall is all corrupt,
Nature and essence human.

This damage is unspeakable, and cannot be discerned by reason, but only from God's Word. 10] And [we affirm] that no one but God alone can separate from one another the nature and this corruption of the nature, which will fully come to pass through death, in the [blessed] resurrection, where our nature which we now bear will rise and live eternally without original sin and separated and sundered from it, as it is written Job 19:26: I shall be compassed again with this my skin, and in my flesh shall I see God, whom I shall see for myself, and mine eyes shall behold.

Negative Theses.
Rejection of the False Opposite Dogmas.

 1. Therefore we reject and condemn the teaching that original sin is only a reatus or debt on account of what has been committed by another [diverted to us] without any corruption of our nature.

 2. Also, that evil lusts are not sin, but con-created, essential properties of the nature, or, as though the above-mentioned defect and damage were not truly sin, because of which man without Christ [not ingrafted into Christ] would be a child of wrath.

 3. We likewise reject the Pelagian error, by which it is alleged that man's nature even after the Fall is incorrupt, and especially with respect to spiritual things has remained entirely good and pure in naturalibus, i. e., in its natural powers.

 4. Also, that original sin is only a slight, insignificant spot on the outside, dashed upon the nature, or a blemish that has been blown upon it, beneath which [nevertheless] the nature has retained its good powers even in spiritual things.

 5. Also, that original sin is only an external impediment to the good spiritual powers, and not a despoliation or want of the same, as when a magnet is smeared with garlic-juice, its natural power is not thereby removed, but only impeded; or that this stain can be easily wiped away like a spot from the face or pigment from the wall.

 6. Also, that in man the human nature and essence are not entirely corrupt, but that man still has something good in him, even in spiritual things, namely, capacity, skill, aptness, or ability in spiritual things to begin, to work, or to help working for something [good].

 7. On the other hand, we also reject the false dogma of the Manicheans, when it is taught that original sin, as something essential and self-subsisting, has been infused by Satan into the nature, and intermingled with it, as poison and wine are mixed.

 8. Also, that not the natural man, but something else and extraneous to man, sins, on account of which not the nature, but only original sin in the nature, is accused.

 9. We reject and condemn also as a Manichean error the doctrine that original sin is properly and without any distinction the substance, nature, and essence itself of the corrupt man, so that a distinction between the corrupt nature, as such, after the Fall and original sin should not even be conceived of, nor that they could be distinguished from one another [even] in thought.

 10. Now, this original sin is called by Dr. Luther nature-sin, person-sin, essential sin, not because the nature, person, or essence of man is, without any distinction, itself original sin, but in order to indicate by such words the distinction between original sin, which inheres in human nature, and other sins, which are called actual sins.

 11. For original sin is not a sin which is committed, but it inheres in the nature, substance, and essence of man, so that, though no wicked thought ever should arise in the heart of corrupt man, no idle word were spoken, no wicked deed were done, yet the nature is nevertheless corrupted through original sin, which is born in us by reason of the sinful seed, and is a fountainhead of all other actual sins, as wicked thoughts, words, and works, as it is written Matt. 15:19: Out of the heart proceed evil thoughts. Also Gen. 6:5; 8:21: The imagination of man's heart is evil from his youth.

 12. Thus there is also to be noted well the diverse signification of the word nature, whereby the Manicheans cover their error and lead astray many simple men. For sometimes it means the essence [the very substance] of man, as when it is said: God created human nature. But at other times it means the disposition and the vicious quality [disposition, condition, defect, or vice] of a thing, which inheres in the nature or essence, as when it is said: The nature of the serpent is to bite, and the nature and disposition of man is to sin, and is sin; here the word nature does not mean the substance of man, but something that inheres in the nature or substance.

 13. But as to the Latin terms substantia and accidens, because they are not words of Holy Scripture, and besides unknown to the ordinary man, they should not be used in sermons before ordinary, uninstructed people, but simple people should be spared them.

 But in the schools, among the learned, these words are rightly retained in disputations concerning original sin, because they are well known and used without any misunderstanding, to distinguish exactly between the essence of a thing and what attaches to it in an accidental way.

 For the distinction between God's work and that of the devil is thereby designated in the clearest way, because the devil can create no substance, but can only, in an accidental way, by the providence of God [God permitting], corrupt the substance created by God.
II. Free Will.

STATUS CONTROVERSIAE.
The Principal Question in This Controversy.

Since the will of man is found in four unlike states, namely: 1. before the Fall; 2. since the Fall; 3. after regeneration; 4. after the resurrection of the body, the chief question is only concerning the will and ability of man in the second state, namely, what powers in spiritual things he has of himself after the fall of our first parents and before regeneration, and whether he is able by his own powers, prior to and before his regeneration by God's Spirit, to dispose and prepare himself for God's grace, and to accept [and apprehend], or not, the grace offered through the Holy Ghost in the Word and holy [divinely instituted] Sacraments.

Affirmative Theses.
The Pure Doctrine concerning This Article, according to God's Word.

1. Concerning this subject, our doctrine, faith, and confession is, that in spiritual things the understanding and reason of man are [altogether] blind, and by their own powers understand nothing, as it is written 1 Cor. 2:14: The natural man receiveth not the things of the Spirit of God, for they are foolishness to him; neither can he know them when he is examined concerning spiritual things.

2. Likewise we believe, teach, and confess that the unregenerate will of man is not only turned away from God, but also has become an enemy of God, so that it only has an inclination and desire for that which is evil and contrary to God, as it is written Gen. 8:21: The imagination of man's heart is evil from his youth. Also Rom. 8:7: The carnal mind is enmity against God; for it is not subject to the Law of God, neither, indeed, can be. Yea, as little as a dead body can quicken itself to bodily, earthly life, so little can man, who by sin is spiritually dead, raise himself to spiritual life, as it is written Eph. 2:5: Even when we were dead in sins, He hath quickened us together with Christ; 2 Cor. 3:5: Not that we are sufficient of ourselves to think anything good as of ourselves, but that we are sufficient is of God.

3. God the Holy Ghost, however, does not effect conversion without means, but uses for this purpose the preaching and hearing of God's Word, as it is written Rom. 1:16: The Gospel is the power of God 5] unto salvation to every one that believeth. Also Rom. 10:17: Faith cometh by hearing of the Word of God. And it is God's will that His Word should be heard, and that man's ears should not be closed. Ps. 95:8. With this Word the Holy Ghost is present, and opens hearts, so that they, as Lydia in Acts 16:14, are attentive to it, and are thus converted alone through the grace and power of the Holy Ghost, whose 6] work alone the conversion of man is. For without His grace, and if He do not grant the increase, our willing and running, our planting, sowing, and watering, all are nothing, as Christ says John 15:5: Without Me ye can do nothing. With these brief words He denies to the free will its powers, and ascribes everything to God's grace, in order that no one may boast before God. 1 Cor. 1:29; 2 Cor. 12:5; Jer. 9:23.

Negative Theses.
Contrary False Doctrine.

Accordingly, we reject and condemn all the following errors as contrary to the standard of God's Word:

1. The delirium [insane dogma] of philosophers who are called Stoics, as also of the Manicheans, who taught that everything that happens must so happen, and cannot happen otherwise, and that everything that man does, even in outward things, he does by compulsion, and that he is coerced to evil works and deeds, as inchastity, robbery, murder, theft, and the like.

2. We reject also the error of the gross Pelagians, who taught that man by his own powers, without the grace of the Holy Ghost, can turn himself to God, believe the Gospel, be obedient from the heart to God's Law, and thus merit the forgiveness of sins and eternal life.

3. We reject also the error of the Semi-Pelagians, who teach that man by his own powers can make a beginning of his conversion, but without the grace of the Holy Ghost cannot complete it.

4. Also, when it is taught that, although man by his free will before regeneration is too weak to make a beginning, and by his own powers to turn himself to God, and from the heart to be obedient to God, yet, if the Holy Ghost by the preaching of the Word has made a beginning, and therein offered His grace, then the will of man from its own natural powers can add something, though little and feebly, to this end, can help and cooperate, qualify and prepare itself for grace, and embrace and accept it, and believe the Gospel.

5. Also, that man, after he has been born again, can perfectly observe and completely fulfil God's Law, and that this fulfilling is our righteousness before God, by which we merit eternal life.

6. Also, we reject and condemn the error of the Enthusiasts, who imagine that God without means, without the hearing of God's Word, also without the use of the holy Sacraments, draws men to Himself, and enlightens, justifies, and saves them. (Enthusiasts we call those who expect the heavenly illumination of the Spirit [celestial revelations] without the preaching of God's Word.)

7. Also, that in conversion and regeneration God entirely exterminates the substance and essence of the old Adam, and especially the rational soul, and in conversion and regeneration creates a new essence of the soul out of nothing.

8. Also, when the following expressions are employed without explanation, namely, that the will of man before, in, and after conversion resists the Holy Ghost, and that the Holy Ghost is given to those who resist Him intentionally and persistently; for, as Augustine says, in conversion God makes willing persons out of the unwilling and dwells in the willing.

As to the expressions of ancient and modern teachers of the Church, when it is said: Deus trahit, sed volentem trahit, i. e., God draws, but He draws the willing; likewise, Hominis voluntas in conversione non est otiosa, sed agit aliquid, i. e., In conversion the will of man is not idle, but also effects something, we maintain that, inasmuch as these expressions have been introduced for confirming [the false opinion concerning] the powers of the natural free will in man's conversion, against the doctrine of God's grace, they do not conform to the form of sound doctrine, and therefore, when we speak of conversion to God, justly ought to be avoided.

But, on the other hand, it is correctly said that in conversion God, through the drawing of the Holy Ghost, makes out of stubborn and unwilling men willing ones, and that after such conversion in the daily exercise of repentance the regenerate will of man is not idle, but also cooperates in all the works of the Holy Ghost, which He performs through us.

9. Also what Dr. Luther has written, namely, that man's will in his conversion is pure passive, that is, that it does nothing whatever, is to be understood respectu divinae gratiae in accendendis novis motibus, that is, when God's Spirit, through the Word heard or the use of the holy Sacraments, lays hold upon man's will, and works [in man] the new birth and conversion. For when [after] the Holy Ghost has wrought and accomplished this, and man's will has been changed and renewed by His divine power and working alone, then the new will of man is an instrument and organ of God the Holy Ghost, so that he not only accepts grace, but also cooperates with the Holy Ghost in the works which follow.

Therefore, before the conversion of man there are only two efficient causes, namely, the Holy Ghost and the Word of God, as the instrument of the Holy Ghost, by which He works conversion. This Word man is [indeed] to hear; however, it is not by his own powers, but only through the grace and working of the Holy Ghost that he can yield faith to it and accept it.
III. The Righteousness of Faith Before God.

STATUS CONTROVERSIAE.
The Principal Question In This Controversy.

Since it is unanimously confessed in our churches, in accordance with God's Word and the sense of the Augsburg Confession, that we poor sinners are justified before God and saved alone by faith in Christ, and thus Christ alone is our Righteousness, who is true God and man, because in Him the divine and human natures are personally united with one another, Jer. 23:6; 1 Cor. 1:30; 2 Cor. 5:21, the question has arisen: According to which nature is Christ our Righteousness? and thus two contrary errors have arisen in some churches.

For the one side has held that Christ according to His divinity alone is our Righteousness, if He dwell in us by faith; contrasted with this divinity, dwelling in us by faith, the sins of all men must be regarded as a drop of water compared to the great ocean. Others, on the contrary, have held that Christ is our Righteousness before God according to the human nature alone.

Affirmative Theses.
Pure Doctrine of the Christian Churches against Both Errors Just Mentioned.

1. Against both the errors just recounted, we unanimously believe, teach, and confess that Christ is our Righteousness neither according to the divine nature alone nor according to the human nature alone, but that it is the entire Christ according to both natures, in His obedience alone, which as God and man He rendered to the Father even unto death, and thereby merited for us the forgiveness of sins and eternal life, as it is written: As by one man's disobedience many were made sinners, so by the obedience of One shall many be made righteous, Rom. 5:19.

2. Accordingly, we believe, teach, and confess that our righteousness before God is (this very thing], that God forgives us our sins out of pure grace, without any work, merit, or worthiness of ours preceding, present, or following, that He presents and imputes to us the righteousness of Christ's obedience, on account of which righteousness we are received into grace by God, and regarded as righteous.

3. We believe, teach, and confess that faith alone is the means and instrument whereby we lay hold of Christ, and thus in Christ of that righteousness which avails before God, for whose sake this faith is imputed to us for righteousness, Rom. 4:5.

4. We believe, teach, and confess that this faith is not a bare knowledge of the history of Christ, but such a gift of God by which we come to the right knowledge of Christ as our Redeemer in the Word of the Gospel, and trust in Him that for the sake of His obedience alone we have, by grace, the forgiveness of sins, are regarded as holy and righteous before God the Father, and eternally saved.

5. We believe, teach, and confess that according to the usage of Holy Scripture the word justify means in this article, to absolve, that is, to declare free from sins. Prov. 17:15: He that justifieth the wicked, and he that condemneth the righteous, even they both are abomination to the Lord. Also Rom. 8:33: Who shall lay anything to the charge of God's elect? It is God that justifieth.

And when, in place of this, the words regeneratio and vivificatio, that is, regeneration and vivification, are employed, as in the Apology, this is done in the same sense. By these terms, in other places, the renewal of man is understood, and distinguished from justification by faith.

6. We believe, teach, and confess also that notwithstanding the fact that many weaknesses and defects cling to the true believers and truly regenerate, even to the grave, still they must not on that account doubt either their righteousness which has been imputed to them by faith, or the salvation of their souls, but must regard it as certain that for Christ's sake, according to the promise and [immovable] Word of the holy Gospel, they have a gracious God.

7. We believe, teach, and confess that for the preservation of the pure doctrine concerning the righteousness of faith before God it is necessary to urge with special diligence the particulae exclusivae, that is, the exclusive particles, i. e., the following words of the holy Apostle Paul, by which the merit of Christ is entirely separated from our works, and the honor given to Christ alone, when the holy Apostle Paul writes: Of grace, without merit, without Law, without works, not of works. All these words together mean as much as that we are justified and saved alone by faith in Christ. Eph. 2:8; Rom. 1:17; 3:24; 4:3ff.; Gal. 3:11; Heb. 11.

8. We believe, teach, and confess that, although the contrition that precedes, and the good works that follow, do not belong to the article of justification before God, yet one is not to imagine a faith of such a kind as can exist and abide with, and alongside of, a wicked intention to sin and to act against the conscience. But after man has been justified by faith, then a true living faith worketh by love, Gal. 5:6, so that thus good works always follow justifying faith, and are surely found with it, if it be true and living; for it never is alone, but always has with it love and hope.

Antitheses: Contrary Doctrines Rejected.

Therefore we reject and condemn all the following errors:

1. That Christ is our Righteousness according to His divine nature alone.

2. That Christ is our Righteousness according to His human nature alone.

3. That in the sayings of the prophets and apostles where the righteousness of faith is spoken of the words justify and to be justified are not to signify declaring or being declared free from sins, and obtaining the forgiveness of sins, but actually being made righteous before God, because of love infused by the Holy Ghost, virtues, and the works following them.

4. That faith looks not only to the obedience of Christ, but to His divine nature, as it dwells and works in us, and that by this indwelling our sins are covered.

5. That faith is such a trust in the obedience of Christ as can exist and remain in a man even when he has no genuine repentance, in whom also no love follows, but who persists in sins against his conscience.

6. That not God Himself, but only the gifts of God, dwell in believers.

7. That faith saves on this account, because by faith the renewal, which consists in love to God and one's neighbor, is begun in us.

8. That faith has the first place in justification, nevertheless also renewal and love belong to our righteousness before God in such a manner that they [renewal and love] are indeed not the chief cause of our righteousness, but that nevertheless our righteousness before God is not entire or perfect without this love and renewal.

9. That believers are justified before God and saved jointly by the imputed righteousness of Christ and by the new obedience begun in them, or in part by the imputation of Christ's righteousness, but in part also by the new obedience begun in them.

10. That the promise of grace is made our own by faith in the heart, and by the confession which is made with the mouth, and by other virtues.

11. That faith does not justify without good works; so that good works are necessarily required for righteousness, and without their presence man cannot be justified.
IV. Good Works.

STATUS CONTROVERSIAE.
The Principal Question In the Controversy concerning Good Works.

Concerning the doctrine of good works two divisions have arisen in some churches:

1. First, some theologians have become divided because of the following expressions, where the one side wrote: Good works are necessary for salvation. It is impossible to be saved without good works. Also: No one has ever been saved without good works. But the other side, on the contrary, wrote: Good works are injurious to salvation.

2. Afterwards a schism arose also between some theologians with respect to the two words necessary and free, since the one side contended that the word necessary should not be employed concerning the new obedience, which, they say, does not flow from necessity and coercion, but from a voluntary spirit. The other side insisted on the word necessary, because, they say, this obedience is not at our option, but regenerate men are obliged to render this obedience.

From this disputation concerning the terms a controversy afterwards occurred concerning the subject itself; for the one side contended that among Christians the Law should not be urged at all, but men should be exhorted to good works from the Holy Gospel alone; the other side contradicted this.

Affirmitive Theses.

Pure Doctrine of the Christian Churches concerning This Controversy.

For the thorough statement and decision of this controversy our doctrine, faith, and confession is:

1. That good works certainly and without doubt follow true faith, if it is not a dead, but a living faith, as fruits of a good tree.

2. We believe, teach, and confess also that good works should be entirely excluded, just as well in the question concerning salvation as in the article of justification before God, as the apostle testifies with clear words, when he writes as follows: Even as David also describeth the blessedness of the man unto whom God imputeth righteousness without works, saying, Blessed is the man to whom the Lord will not impute sin, Rom. 4:6ff And again: By grace are ye saved through faith; and that not of yourselves, it is the gift of God; not of works, lest any man should boast, Eph. 2:8-9.

3. We believe, teach, and confess also that all men, but those especially who are born again and renewed by the Holy Ghost, are bound to do good works.

4. In this sense the words necessary, shall, and must are employed correctly and in a Christian manner also with respect to the regenerate, and in no way are contrary to the form of sound words and speech.

5. Nevertheless, by the words mentioned, necessitas, necessarium, necessity and necessary, if they be employed concerning the regenerate, not coercion, but only due obedience is to be understood, which the truly believing, so far as they are regenerate, render not from coercion or the driving of the Law, but from a voluntary spirit; because they are no more under the Law, but under grace, Rom. 6:14; 7:6; 8:14.

6. Accordingly, we also believe, teach, and confess that when it is said: The regenerate do good works from a free spirit, this is not to be understood as though it is at the option of the regenerate man to do or to forbear doing good when he wishes, and that he can nevertheless retain faith if he intentionally perseveres in sins.

7. Yet this is not to be understood otherwise than as the Lord Christ and His apostles themselves declare, namely, regarding the liberated spirit, that it does not do this from fear of punishment, like a servant, but from love of righteousness, like children, Rom. 8:15.

8. Although this voluntariness [liberty of spirit] in the elect children of God is not perfect, but burdened with great weakness, as St. Paul complains concerning himself, Rom. 7:14-25; Gal. 5:17;

9. Nevertheless, for the sake of the Lord Christ, the Lord does not impute this weakness to His elect, as it is written: There is therefore now no condemnation to them which are in Christ Jesus, Rom. 8:1.

10. We believe, teach, and confess also that not works maintain faith and salvation in us, but the Spirit of God alone, through faith, of whose presence and indwelling good works are evidences.

Negative Theses.
False Contrary Doctrine.

1. Accordingly, we reject and condemn the following modes of speaking: when it is taught and written that good works are necessary to salvation; also, that no one ever has been saved without good works; also, that it is impossible to be saved without good works.

2. We reject and condemn as offensive and detrimental to Christian discipline the bare expression, when it is said: Good works are injurious to salvation.

For especially in these last times it is no less needful to admonish men to Christian discipline [to the way of living aright and godly] and good works, and remind them how necessary it is that they exercise themselves in good works as a declaration of their faith and gratitude to God, than that the works be not mingled in the article of justification; because men may be damned by an Epicurean delusion concerning faith, as well as by papistic and Pharisaic confidence in their own works and merits.

3. We also reject and condemn the dogma that faith and the indwelling of the Holy Ghost are not lost by wilful sin, but that the saints and elect retain the Holy Ghost even though they fall into adultery and other sins and persist therein.
V. Law and Gospel

STATUS CONTROVERSIAE.
The Principal Question In This Controversy.

Whether the preaching of the Holy Gospel is properly not only a preaching of grace, which announces the forgiveness of sins, but also a preaching of repentance and reproof, rebuking unbelief, which, they say, is rebuked not in the Law, but alone through the Gospel.

Affirmative Theses.
Pure Doctrine of God's Word.

1. We believe, teach, and confess that the distinction between the Law and the Gospel is to be maintained in the Church with great diligence as an especially brilliant light, by which, according to the admonition of St. Paul, the Word of God is rightly divided.

2. We believe, teach, and confess that the Law is properly a divine doctrine, which teaches what is right and pleasing to God, and reproves everything that is sin and contrary to God's will.

3. For this reason, then, everything that reproves sin is, and belongs to, the preaching of the Law.

4. But the Gospel is properly such a doctrine as teaches what man who has not observed the Law, and therefore is condemned by it, is to believe, namely, that Christ has expiated and made satisfaction for all sins, and has obtained and acquired for him, without any merit of his [no merit of the sinner intervening], forgiveness of sins, righteousness that avails before God, and eternal life.

5. But since the term Gospel is not used in one and the same sense in the Holy Scriptures, on account of which this dissension originally arose, we believe, teach, and confess that if by the term Gospel is understood the entire doctrine of Christ which He proposed in His ministry, as also did His apostles (in which sense it is employed, Mark 1:15; Acts 20:21), it is correctly said and written that the Gospel is a preaching of repentance and of the forgiveness of sins.

6. But if the Law and the Gospel, likewise also Moses himself [as] a teacher of the Law and Christ as a preacher of the Gospel are contrasted with one another, we believe, teach, and confess that the Gospel is not a preaching of repentance or reproof, but properly nothing else than a preaching of consolation, and a joyful message which does not reprove or terrify, but comforts consciences against the terrors of the Law, points alone to the merit of Christ, and raises them up again by the lovely preaching of the grace and favor of God, obtained through Christ's merit.

7. As to the revelation of sin, because the veil of Moses hangs before the eyes of all men as long as they hear the bare preaching of the Law, and nothing concerning Christ, and therefore do not learn from the Law to perceive their sins aright, but either become presumptuous hypocrites [who swell with the opinion of their own righteousness] as the Pharisees, or despair like Judas, Christ takes the Law into His hands, and explains it spiritually, Matt. 5:21ff ; Rom. 7:14. And thus the wrath of God is revealed from heaven against all sinners [ Rom. 1:18 ], how great it is; by this means they are directed [sent back] to the Law, and then first learn from it to know aright their sins-a knowledge which Moses never could have forced out of them.

Accordingly, although the preaching of the suffering and death of Christ, the Son of God, is an earnest and terrible proclamation and declaration of God's wrath, whereby men are first led into the Law aright, after the veil of Moses has been removed from them, so that they first know aright how great things God in His Law requires of us, none of which we can observe, and therefore are to seek all our righteousness in Christ:

8. Yet as long as all this (namely, Christ's suffering and death) proclaims God's wrath and terrifies man, it is still not properly the preaching of the Gospel, but the preaching of Moses and the Law, and therefore a foreign work of Christ, by which He arrives at His proper office, that is, to preach grace, console, and quicken, which is properly the preaching of the Gospel.

Negative Theses.
Contrary Doctrine which is Rejected.

Accordingly we reject and regard as incorrect and injurious the dogma that the Gospel is properly a preaching of repentance or reproof, and not alone a preaching of grace; for thereby the Gospel is again converted into a doctrine of the Law, the merit of Christ and Holy Scripture are obscured, Christians robbed of true consolation, and the door is opened again to [the errors and superstitions of] the Papacy.
VI. The Third Use of the Law.

STATUS CONTROVERSIAE.
The Principal Question In This Controversy.

Since the Law was given to men for three reasons: first, that thereby outward discipline might be maintained against wild, disobedient men [and that wild and intractable men might be restrained, as though by certain bars]; secondly, that men thereby may be led to the knowledge of their sins; thirdly, that after they are regenerate and [much of] the flesh notwithstanding cleaves to them, they might on this account have a fixed rule according to which they are to regulate and direct their whole life, a dissension has occurred between some few theologians concerning the third use of the Law, namely, whether it is to be urged or not upon regenerate Christians. The one side has said, Yea; the other, Nay.

Affirmative Theses.
The True Christian Doctrine concerning This Controversy.

1. We believe, teach, and confess that, although men truly believing [in Christ] and truly converted to God have been freed and exempted from the curse and coercion of the Law, they nevertheless are not on this account without Law, but have been redeemed by the Son of God in order that they should exercise themselves in it day and night [that they should meditate upon God's Law day and night, and constantly exercise themselves in its observance, Ps. 1:2 ], Ps. 119. For even our first parents before the Fall did not live without Law, who had the Law of God written also into their hearts, because they were created in the image of God, Gen. 1:26f.; 2:16ff; 3:3.

2. We believe, teach, and confess that the preaching of the Law is to be urged with diligence, not only upon the unbelieving and impenitent, but also upon true believers, who are truly converted, regenerate, and justified by faith.

3. For although they are regenerate and renewed in the spirit of their mind, yet in the present life this regeneration and renewal is not complete, but only begun, and believers are, by the spirit of their mind, in a constant struggle against the flesh, that is, against the corrupt nature and disposition which cleaves to us unto death. On account of this old Adam, which still inheres in the understanding, the will, and all the powers of man, it is needful that the Law of the Lord always shine before them, in order that they may not from human devotion institute wanton and self-elected cults [that they may frame nothing in a matter of religion from the desire of private devotion, and may not choose divine services not instituted by God's Word]; likewise, that the old Adam also may not employ his own will, but may be subdued against his will, not only by the admonition and threatening of the Law, but also by punishments and blows, so that he may follow and surrender himself captive to the Spirit, 1 Cor. 9:27; Rom. 6:12, Gal. 6:14; Ps. 119:1ff ; Heb. 13:21 (Heb. 12:1).

4. Now, as regards the distinction between the works of the Law and the fruits of the Spirit, we believe, teach, and confess that the works which are done according to the Law are and are called works of the Law as long as they are only extorted from man by urging the punishment and threatening of God's wrath.

5. Fruits of the Spirit, however, are the works which the Spirit of God who dwells in believers works through the regenerate, and which are done by believers so far as they are regenerate [spontaneously and freely], as though they knew of no command, threat, or reward; for in this manner the children of God live in the Law and walk according to the Law of God, which [mode of living] St. Paul in his epistles calls the Law of Christ and the Law of the mind, Rom. 7:25; 8:7; Rom. 8:2; Gal. 6:2.

6. Thus the Law is and remains both to the penitent and impenitent, both to regenerate and unregenerate men, one [and the same] Law, namely, the immutable will of God; and the difference, so far as concerns obedience, is alone in man, inasmuch as one who is not yet regenerate does for the Law out of constraint and unwillingly what it requires of him (as also the regenerate do according to the flesh); but the believer, so far as he is regenerate, does without constraint and with a willing spirit that which no threatenings [however severe] of the Law could ever extort from him.

Negative Theses.
False Contrary Doctrine.

8] Accordingly, we reject as a dogma and error injurious to, and conflicting with, Christian discipline and true godliness the teaching that the Law in the above-mentioned way and degree is not to be urged upon Christians and true believers, but only upon unbelievers, non-Christians, and the impenitent.
VII. The Lord's Supper.

Although the Zwinglian teachers are not to be reckoned among the theologians who affiliate with [acknowledge and profess] the Augsburg Confession, as they separated from them at the very time when this Confession was presented, nevertheless, since they are intruding themselves (into their assembly], and are attempting, under the name of this Christian Confession, to spread their error, we intend also to make a needful statement [we have judged that the Church of Christ should be instructed also] concerning this controversy.

STATUS CONTROVERSIAE.
Chief Controversy between Our Doctrine and That of the Sacramentarians regarding This Article.

Whether in the Holy Supper the true body and blood of our Lord Jesus Christ are truly and essentially present, are distributed with the bread and wine, and received with the mouth by all those who use this Sacrament, whether they be worthy or unworthy, godly or ungodly, believing or unbelieving; by the believing for consolation and life, by the unbelieving for judgment? The Sacramentarians say, No; we say, Yes.

For the explanation of this controversy it is to be noted in the beginning that there are two kinds of Sacramentarians. Some are gross Sacramentarians, who declare in plain (deutschen), clear words as they believe in their hearts, that in the Holy Supper nothing but bread and wine is present, and distributed and received with the mouth. 4] Others, however, are subtle Sacramentarians, and the most injurious of all, who partly speak very speciously in our own words, and pretend that they also believe a true presence of the true, essential, living body and blood of Christ in the Holy Supper, however, that 5] this occurs spiritually through faith. Nevertheless they retain under these specious words precisely the former gross opinion, namely, that in the Holy Supper nothing is present and received with the mouth except bread and wine. For with them the word spiritually means nothing else than the Spirit of Christ or the power of the absent body of Christ and His merit, which is present; but the body of Christ is in no mode or way present, except only above in the highest heaven, to which we should elevate ourselves into heaven by the thoughts of our faith, and there, not at all, however, in the bread and wine of the Holy Supper, should seek this body and blood [of Christ].

Affirmative Theses.
Confession of the Pure Doctrine concerning the Holy Supper against the Sacramentarians.

1. We believe, teach, and confess that in the Holy Supper the body and blood of Christ are truly and essentially present, and are truly distributed and received with the bread and wine.

2. We believe, teach, and confess that the words of the testament of Christ are not to be understood otherwise than as they read, according to the letter, so that the bread does not signify the absent body and the wine the absent blood of Christ, but that, on account of the sacramental union, they [the bread and wine] are truly the body and blood of Christ.

3. Now, as to the consecration, we believe, teach, and confess that no work of man or recitation of the minister [of the church] produces this presence of the body and blood of Christ in the Holy Supper, but that this is to be ascribed only and alone to the almighty power of our Lord Jesus Christ.

4. But at the same time we also believe, teach, and confess unanimously that in the use of the Holy Supper the words of the institution of Christ should in no way be omitted, but should be publicly recited, as it is written 1 Cor. 10:16: The cup of blessing which we bless, etc. This blessing occurs through the recitation of the words of Christ.

5. The grounds, however, on which we stand against the Sacramentarians in this matter are those which Dr. Luther has laid down in his Large Confession concerning the Lord's Supper.

The first is this article 11] of our Christian faith: Jesus Christ is true, essential, natural, perfect God and man in one person, undivided and inseparable.

The second: That God's right hand is everywhere; at which Christ is placed in deed and in truth according to His human nature, [and therefore] being present, rules, and has in His hands and beneath His feet everything that is in heaven and on earth [as Scripture says, Eph. 1:22 ], where no man else, nor angel, but only the Son of Mary is placed; hence He can do this [those things which we have said].

The third: That God's Word is not false, and does not deceive.

The fourth: That God has and knows of various modes of being in any place, and not only the one [is not bound to the one] which philosophers call localis (local) for circumscribed].

6. We believe, teach, and confess that the body and blood of Christ are received with the bread and wine, not only spiritually by faith, but also orally; yet not in a Capernaitic, but in a supernatural, heavenly mode, because of the sacramental union; as the words of Christ clearly show, when Christ gives direction to take, eat, and drink, as was also done by the apostles; for it is written Mark 14:23: And they all drank of it. St. Paul likewise says, 1 Cor. 10:16: The bread which we break, is it not the communion of the body of Christ? that is: He who eats this bread eats the body of Christ, which also the chief ancient teachers of the Church, Chrysostom, Cyprian, Leo I, Gregory, Ambrose, Augustine, unanimously testify.

7. We believe, teach, and confess that not only the true believers [in Christ] and the worthy, but also the unworthy and unbelievers, receive the true body and blood of Christ; however, not for life and consolation, but for judgment and condemnation, if they are not converted and do not repent, 1 Cor. 11:27-29.

For although they thrust Christ from themselves as a Savior, yet they must admit Him even against their will as a strict Judge, who is just as present also to exercise and render judgment upon impenitent guests as He is present to work life and consolation in the hearts of the true believers and worthy guests.

8. We believe, teach, and confess also that there is only one kind of unworthy guests, namely, those who do not believe, concerning whom it is written John 3:18: He that believeth not is condemned already. And this judgment becomes greater and more grievous, being aggravated, by the unworthy use of the Holy Supper, 1 Cor. 11:29.

9. We believe, teach, and confess that no true believer, as long as he retains living faith, however weak he may be, receives the Holy Supper to his judgment, which was instituted especially for Christians weak in faith, yet penitent, for the consolation and strengthening of their weak faith [Matt. 9:12; 11:5. 28].

10. We believe, teach, and confess that all the worthiness of the guests of this heavenly feast is and consists in the most holy obedience and perfect merit of Christ alone, which we appropriate to ourselves by true faith, and whereof [of the application of this merit] we are assured by the Sacrament, and not at all in [but in nowise does this worthiness depend upon] our virtues or inward and outward preparations.

Negative Theses.
Contrary, Condemned Doctrines of the Sacramentarians.

On the other hand, we unanimously reject and condemn all the following erroneous articles, which are opposed and contrary to the doctrine presented above, the simple faith, and the [pure] confession concerning the Lord's Supper;

1. The papistic transubstantiation, when it is taught in the Papacy that in the Holy Supper the bread and wine lose their substance and natural essence, and are thus annihilated; that they are changed into the body of Christ, and the outward form alone remains.

2. The papistic sacrifice of the Mass for the sins of the living and the dead.

3. That [the sacrilege whereby] to laymen one form only of the Sacrament is given, and, contrary to the plain words of the testament of Christ, the cup is withheld from them, and they are [thus] deprived of His blood.

4. When it is taught that the words of the testament of Christ must not be understood or believed simply as they read, but that they are obscure expressions, whose meaning must be sought first in other passages of Scripture.

5. That in the Holy Supper the body of Christ is not received orally with the bread; but that with the mouth only bread and wine are received, the body of Christ, however, only spiritually by faith.

6. That the bread and wine in the Holy Supper are nothing more than [symbols or] tokens by which Christians recognize one another.

7. That the bread and wine are only figures, similitudes, and representations of the far absent body and blood of Christ.

8. That the bread and wine are no more than a memorial, seal, and pledge, through which we are assured that when faith elevates itself to heaven, it there becomes partaker of the body and blood of Christ as truly as we eat bread and drink wine in the Supper.

9. That the assurance and confirmation of our faith [concerning salvation] in the Holy Supper occur through the external signs of bread and wine alone, and not through the true, [verily] present body and blood of Christ.

10. That in the Holy Supper only the power, efficacy, and merit of the absent body and blood of Christ are distributed.

11. That the body of Christ is so enclosed in heaven that it can in no way be at once and at one time in many or all places upon earth where His Holy Supper is celebrated.

12. That Christ has not promised, neither could have effected, the essential presence of His body and blood in the Holy Supper, because the nature and property of His assumed human nature cannot suffer nor permit it.

13. That God, according to [even by] all His omnipotence (which is dreadful to hear), is not able to cause His body to be essentially present in more than one place at one time.

14. That not the omnipotent words of Christ's testament, but faith, produces and makes [is the cause of] the presence of the body and blood of Christ in the Holy Supper.

15. That believers must not seek the body [and blood] of Christ in the bread and wine of the Holy Supper, but raise their eyes from the bread to heaven and there seek the body of Christ.

16. That unbelieving, impenitent Christians do not receive the true body and blood of Christ in the Holy Supper, but only bread and wine.

17. That the worthiness of the guests at this heavenly meal consists not alone in true faith in Christ, but also in the external preparation of men.

18. That even the true believers, who have and retain a true, living, pure faith in Christ, can receive this Sacrament to their judgment, because they are still imperfect in their outward life.

19. That the external visible elements of the bread and wine should be adored in the Holy Sacrament.

20. Likewise, we consign also to the just judgment of God all presumptuous, frivolous, blasphemous questions (which decency forbids to mention) and [other] expressions, which most blasphemously and with great offense [to the Church] are proposed by the Sacramentarians in a gross, carnal, Capernaitic way concerning the supernatural, heavenly mysteries of this Sacrament.

21. Hence we hereby utterly [reject and] condemn the Capernaitic eating of the body of Christ, as though [we taught that] His flesh were rent with the teeth, and digested like other food, which the Sacramentarians, against the testimony of their conscience, after all our frequent protests, wilfully force upon us, and in this way make our doctrine odious to their hearers; and on the other hand, we maintain and believe, according to the simple words of the testament of Christ, the true, yet supernatural eating of the body of Christ, as also the drinking of His blood, which human senses and reason do not comprehend, but as in all other articles of faith our reason is brought into captivity to the obedience of Christ, and this mystery is not apprehended otherwise than by faith alone, and revealed in the Word alone.
VIII. The Person of Christ.

From the controversy concerning the Holy Supper a disagreement has arisen between the pure theologians of the Augsburg Confession and the Calvinists, who also have confused some other theologians, concerning the person of Christ and the two natures in Christ and their properties.

STATUS CONTROVERSIAE.
Chief Controversy In This Dissension.

The chief question, however, has been whether, because of the personal union, the divine and human natures, as also their properties, have realiter, that is, in deed and truth, a communion with one another in the person of Christ, and how far this communion extends.

The Sacramentarians have asserted that the divine and human natures in Christ are united personally in such a way that neither has realiter, that is, in deed and truth, in common with the other that which is peculiar to either nature, but that they have in common nothing more than the name alone. For unio, they plainly say, facit communia nomina, i. e., the personal union makes nothing more than the names common, namely, that God is called man, and man God, yet in such a way that God has nothing realiter, that is, in deed and truth, in common with humanity, and humanity nothing in common with divinity, its majesty and properties. Dr. Luther, and those who held with him, have contended for the contrary against the Sacramentarians.

Affirmative Theses.
Pure Doctrine of the Christian Church concerning the Person of Christ.

To explain this controversy, and settle it according to the guidance [analogy] of our Christian faith, our doctrine, faith, and confession is as follows:

1. That the divine and human natures in Christ are personally united, so that there are not two Christs, one the Son of God, the other the Son of man, but that one and the same is the Son of God and Son of man, Luke 1:35; Rom. 9:5.

2. We believe, teach, and confess that the divine and human natures are not mingled into one substance, nor the one changed into the other, but that each retains its own essential properties, which [can] never become the properties of the other nature.

3. The properties of the divine nature are: to be almighty, eternal, infinite, and to be, according to the property of its nature and its natural essence, of itself, everywhere present, to know everything, etc.; which never become properties of the human nature.

4. The properties of the human nature are: to be a corporeal creature, to be flesh and blood, to be finite and circumscribed, to suffer, to die, to ascend and descend, to move from one place to another, to suffer hunger, thirst, cold, heat, and the like; which never become properties of the divine nature.

5. As the two natures are united personally, i. e., in one person, we believe, teach, and confess that this union is not such a copulation and connection that neither nature has anything in common with the other personally, i.e . because of the personal union, as when two boards are glued together, where neither gives anything to the other or takes anything from the other. But here is the highest communion, which God truly has with the [assumed] man, from which personal union, and the highest and ineffable communion resulting therefrom, there flows everything human that is said and believed concerning God, and everything divine that is said and believed concerning the man Christ; as the ancient teachers of the Church explained this union and communion of the natures by the illustration of iron glowing with fire, and also by the union of body and soul in man.

6. Hence we believe, teach, and confess that God is man and man is God, which could not be if the divine and human natures had in deed and truth absolutely no communion with one another.

For how could the man, the son of Mary, in truth be called or be God, or the Son of God the Most High, if His humanity were not personally united with the Son of God, and He thus had realiter, that is, in deed and truth, nothing in common with Him except only the name of God?

7. Hence we believe, teach, and confess that Mary conceived and bore not a mere man and no more, but the true Son of God; therefore she also is rightly called and truly is the mother of God.

8. Hence we also believe, teach, and confess that it was not a mere man who suffered, died, was buried, descended to hell, arose from the dead, ascended into heaven, and was raised to the majesty and almighty power of God for us, but a man whose human nature has such a profound [close], ineffable union and communion with the Son of God that it is [has become] one person with Him.

9. Therefore the Son of God truly suffered for us, however, according to the property of the human nature which He assumed into the unity of His divine person and made His own, so that He might be able to suffer and be our High Priest for our reconciliation with God, as it is written 1 Cor. 2:8: They have crucfied the Lord of glory. And Acts 20:28: We are purchased with God's blood.

10. Hence we believe, teach, and confess that the Son of Man is realiter, that is, in deed and truth, exalted according to His human nature to the right hand of the almighty majesty and power of God, because He [that man] was assumed into God when He was conceived of the Holy Ghost in His mother's womb, and His human nature was personally united with the Son of the Highest.

11. This majesty He [Christ] always had according to the personal union, and yet He abstained from it in the state of His humiliation, and on this account truly increased in all wisdom and favor with God and men; therefore He exercised this majesty, not always, but when [as often as] it pleased Him, until after His resurrection He entirely laid aside the form of a servant, but not the [human] nature, and was established in the full use, manifestation, and declaration of the divine majesty, and thus entered into His glory, Phil. 2:6ff , so that now not only as God, but also as man He knows all things, can do all things, is present with all creatures, and has under His feet and in His hands everything that is in heaven and on earth and under the earth, as He Himself testifies Matt. 28:18; John 13:3: All power is given unto Me in heaven and in earth. And St. Paul says Eph. 4:10: He ascended up far above all heavens, that He might fill all things. And this His power, He, being present, can exercise everywhere, and to Him everything is possible and everything is known.

12. Hence He also is able and it is very easy for Him to impart, as one who is present, His true body and blood in the Holy Supper, not according to the mode or property of the human nature, but according to the mode and property of the right hand of God, as Dr. Luther says in accordance with our Christian faith for children, which presence (of Christ in the Holy Supper] is not [physical or] earthly, nor Capernaitic; nevertheless it is true and substantial, as the words of His testament read: This is, is, is My body, etc.

By this our doctrine, faith, and confession the person of Christ is not divided, as it was by Nestorius, who denied the communicatio idiomatum, that is, the true communion of the properties of both natures in Christ, and thus divided the person, as Luther has explained in his book Concerning Councils. Neither are the natures together with their properties confounded with one another [or mingled] into one essence (as Eutyches erred); nor is the human nature in the person of Christ denied or annihilated; nor is either nature changed into the other; but Christ is and remains to all eternity God and man in one undivided person, which, next to the Holy Trinity, is, as the Apostle testifies, 1 Tim. 3:16, the highest mystery, upon which our only consolation, life, and salvation depends.

Negative Theses.
Contrary False Doctrine concerning the Person of Christ.

Accordingly, we reject and condemn as contrary to God's Word and our simple [pure] Christian faith all the following erroneous articles, when it is taught:

1. That God and man in Christ are not one person, but that the Son of God is one, and the Son of Man another, as Nestorius raved.

2. That the divine and human natures have been mingled with one another into one essence, and the human nature has been changed into the Deity, as Eutyches fanatically asserted.

3. That Christ is not true, natural, and eternal God, as Arius held [blasphemed].

4. That Christ did not have a true human nature [consisting] of body and soul, as Marcion imagined.

5. Quod unio personalis faciat tantum communia nomina, that is, that the personal union renders only the names and titles common.

6. That it is only phrasis et modus loquendi, that is, a phrase and mode of speaking, when it is said: God is man, man is God; since Divinity, as they say, has realiter, that is, in deed [and truth], nothing in common with the humanity, nor the humanity with the Deity.

7. That there is merely communicatio [idiomatum] verbalis [without reality], that is, that it is nothing but words when it is said the Son of God died for the sins of the world; the Son of Man has become almighty.

8. That the human nature in Christ has become an infinite essence in the same manner as the Divinity, and that it is everywhere present in the same manner as the divine nature because of this essential power and property, communicated to, and poured out into, the human nature and separated from God.

9. That the human nature has become equal to and like the divine nature in its substance and essence, or in its essential properties.

10. That the human nature of Christ is locally extended to all places of heaven and earth, which should not be ascribed even to the divine nature.

11. That because of the property of the human nature it is impossible for Christ to be able to be at the same time in more than one place, much less everywhere, with His body.

12. That only the mere humanity has suffered for us and redeemed us, and that the Son of God in the suffering had actually no communion with the humanity, as though it did not concern Him.

13. That Christ is present with us on earth in the Word, the Sacraments, and in all our troubles, only according to His divinity, and that this presence does not at all pertain to His human nature, according to which also, as they say, He, after having redeemed us by His suffering and death, has nothing to do with us any longer upon earth.

14. That the Son of God who assumed the human nature, after He has laid aside the form of a servant, does not perform all the works of His omnipotence in, through, and with His human nature, but only some, and only in the place where His human nature is locally.

15. That according to His human nature He is not at all capable of omnipotence and other attributes of the divine nature, against the express declaration of Christ, Matt. 28:18: All power is given unto He in heaven and in earth, and of St. Paul, Col. 2:9: In Him dwelleth all the fulness of the Godhead bodily.

16. That to Him [to Christ according to His humanity] greater power is given in heaven and upon earth, namely, greater and more than to all angels and other creatures, but that He has no communion with the omnipotence of God, nor that this has been given Him. Hence they devise mediam potentiam, that is, a power between the almighty power of God and the power of other creatures given to Christ according to His humanity by the exaltation, such as would be less than God's almighty power and greater than that of other creatures.

17. That Christ according to His human mind has a certain limit as to how much He is to know, and that He knows no more than is becoming and needful for Him to know for [the execution of] His office as Judge.

18. That Christ does not yet have a perfect knowledge of God and all His works; of whom nevertheless it is written Col. 2:3: In whom are hid all the treasures of wisdom and knowledge.

19. That it is impossible for Christ according to His human mind to know what has been from eternity, what at present is occurring everywhere, and what will be in eternity.

20. When it is taught, and the passage Matt. 28:18: All power is given unto Me, etc., is thus interpreted and blasphemously perverted, namely, that all power in heaven and on earth was restored, that is, delivered again to Christ according to the divine nature, at the resurrection and His ascension to heaven, as though He had also according to His divinity laid this aside and abandoned it in His state of humiliation. By this doctrine not only the words of the testament of Christ are perverted, but also the way is prepared for the accursed Arian heresy, so that finally the eternal deity of Christ is denied, and thus Christ, and with Him our salvation, are entirely lost if this false doctrine were not firmly contradicted from the immovable foundation of the divine Word and our simple Christian [catholic] faith.
IX. The Descent of Christ Into Hell.

STATUS CONTROVERSIAE.
Chief Controversy concerning This Article.

It has also been disputed among some theologians who have subscribed to the Augsburg Confession concerning this article: When and in what manner the Lord Christ, according to our simple Christian faith, descended to hell: whether this was done before or after His death; also, whether it occurred according to the soul alone, or according to the divinity alone, or with body and soul, spiritually or bodily; also, whether this article belongs to the passion or to the glorious victory and triumph of Christ.

But since this article, as also the preceding, cannot be comprehended by the senses or by our reason, but must be grasped by faith alone, it is our unanimous opinion that there should be no disputation concerning it, but that it should be believed 3] and taught only in the simplest manner; according as Dr. Luther, of blessed memory, in his sermon at Torgau in the year 1533 has explained this article in an altogether Christian manner, separated from it all useless, unnecessary questions, and admonished all godly Christians to Christian simplicity of faith.

For it is sufficient that we know that Christ descended into hell, destroyed hell for all believers, and delivered them from the power of death and of the devil, from eternal condemnation and the jaws of hell. But how this occurred we should [not curiously investigate, but] reserve until the other world, where not only this point [mystery], but also still others will be revealed, which we here simply believe, and cannot comprehend with our blind reason.
X. Church Rites

Which are [Commonly] Called Adiaphora or Matters of Indifference.

1] Concerning ceremonies or church rites which are neither commanded nor forbidden in God's Word, but have been introduced into the Church for the sake of good order and propriety, a dissension has also occurred among the theologians of the Augsburg Confession.

STATUS CONTROVERSIAE.
Chief Controversy concerning This Article.

The chief question, however, has been, whether, in time of persecution and in case of confession, even if the enemies of the Gospel have not reached an agreement with us in doctrine, some abrogated ceremonies, which in themselves are matters of indifference and are neither commanded nor forbidden by God, may nevertheless, upon the pressure and demand of the adversaries, be reestablished without violence to conscience, and we may thus [rightly] have conformity with them in such ceremonies and adiaphora. To this the one side has said Yea, the other, Nay.

Affirmative Theses.
The Correct and True Doctrine and Confession concerning This Article.

1. For settling also this controversy we unanimously believe, teach, and confess that the ceremonies or church rites which are neither commanded nor forbidden in God's Word, but have been instituted alone for the sake of propriety and good order, are in and of themselves no divine worship, nor even a part of it. Matt. 15:9:In vain they do worship Me, teaching for doctrines the commandments of men.

2. We believe, teach, and confess that the congregation of God of every place and every time has the power, according to its circumstances, to change such ceremonies in such manner as may be most useful and edifying to the congregation of God.

3. Nevertheless, that herein all frivolity and offense should be avoided, and special care should be taken to exercise forbearance towards the weak in faith. 1 Cor. 8:9; Rom. 14:13.

4. We believe, teach, and confess that in time of persecution, when a plain [and steadfast] confession is required of us, we should not yield to the enemies in regard to such adiaphora, as the apostle has written Gal. 5:1: Stand fast, therefore, in the liberty wherewith Christ hath made us free, and be not entangled again in the yoke of bondage. Also 2 Cor. 6:14: Be ye not unequally yoked together with unbelievers, etc. For what concord hath light with darkness? Also Gal. 2:5: To whom we gave place, no, not for an hour, that the truth of the Gospel might remain with you. For in such a case it is no longer a question concerning adiaphora, but concerning the truth of the Gospel, concerning [preserving] Christian liberty, and concerning sanctioning open idolatry, as also concerning the prevention of offense to the weak in the faith [how care should be taken lest idolatry be openly sanctioned and the weak in faith be offended]; in which we have nothing to concede, but should plainly confess and suffer on that account what God sends, and what He allows the enemies of His Word to inflict upon us.

5. We believe, teach, and confess also that no Church should condemn another because one has less or more external ceremonies not commanded by God than the other, if otherwise there is agreement among them in doctrine and all its articles, as also in the right use of the holy Sacraments, according to the well-known saying: Dissonantia ieiunii non dissolvit consonantiam fidei, Disagreement in fasting does not destroy agreement in faith.

Negative Theses.
False Doctrine concerning This Article.

Accordingly, we reject and condemn as wrong and contrary to God's Word when it is taught:

1. That human ordinances and institutions in the church should be regarded as in themselves a divine worship or part of it.

2. When such ceremonies, ordinances, and institutions are violently forced upon the congregation of God as necessary, contrary to its Christian liberty which it has in external things.

3. Also, that in time of persecution and public confession [when a clear confession is required] we may yield to the enemies of the Gospel in such adiaphora and ceremonies, or may come to an agreement with them (which causes injury to the truth).

4. Also, when these external ceremonies and adiaphora are abrogated in such a manner as though it were not free to the congregation of God to employ one or more [this or that] in Christian liberty, according to its circumstances, as may be most useful at any time to the Church [for edification].
XI. Election.

1] Concerning this article no public dissension has occurred among the theologians of the Augsburg Confession. But since it is a consolatory article, if treated properly, and lest offensive disputations concerning the same be instituted in the future, it is also explained in this writing.

Affirmative Theses.
The Pure and True Doctrine concerning This Article.

1. To begin with [First of all], the distinction between praescientia et praedestinatio, that is, between God's foreknowledge and His eternal election, ought to be accurately observed.

2. For the foreknowledge of God is nothing else than that God knows all things before they happen, as it is written Dan. 2:28: There is a God in heaven that revealeth secrets and maketh known to the king Nebuchadnezzar what shall be in the latter days.

3. This foreknowledge extends alike over the godly and the wicked, but it is not the cause of evil, neither of sin, namely, of doing what is wrong (which originally arises from the devil and the wicked, perverse will of man), nor of their ruin [that men perish], for which they themselves are responsible [which they must ascribe to themselves]; but it only regulates it, and fixes a limit to it [how far it should progress and] how long it should last, and all this to the end that it should serve His elect for their salvation, notwithstanding that it is evil in itself.

4. The predestination or eternal election of God, however, extends only over the godly, beloved children of God, being a cause of their salvation, which He also provides, as well as disposes what belongs thereto. Upon this [predestination of God] our salvation is founded so firmly that the gates of hell cannot overcome it. John 10:28; Matt. 16:18.

5. This [predestination of God] is not to be investigated in the secret counsel of God, but to be sought in the Word of God, where it is also revealed.

6. But the Word of God leads us to Christ, who is the Book of Life, in whom all are written and elected that are to be saved in eternity, as it is written Eph. 1:4: He hath chosen us in Him [Christ] before the foundation of the world.

7. This Christ calls to Himself all sinners and promises them rest, and He is in earnest [seriously wills] that all men should come to Him and suffer themselves to be helped, to whom He offers Himself in His Word, and wishes them to hear it and not to stop their ears or [neglect and] despise the Word. Moreover, He promises the power and working of the Holy Ghost, and divine assistance for perseverance and eternal salvation [that we may remain steadfast in the faith and attain eternal salvation].

8. Therefore we should judge concerning this our election to eternal life neither from reason nor from the Law of God, which lead us either into a reckless, dissolute, Epicurean life or into despair, and excite pernicious thoughts in the hearts of men, for they cannot, as long as they follow their reason, successfully refrain from thinking: If God has elected me to salvation, I cannot be condemned, no matter what I do; and again: If I am not elected to eternal life, it is of no avail what good I do; it is all [all my efforts are] in vain anyway.

9. But it [the true judgment concerning predestination] must be learned alone from the holy Gospel concerning Christ, in which it is clearly testified that God hath concluded them all in unbelief, that He might have mercy upon all, and that He is not willing that any should perish, but that all should come to repentance and believe in the Lord Christ. Rom. 11:32; Ezek. 18:23; 33:11; 2 Pet. 3:9; 1 John 2:2.

10. Whoever, now, is thus concerned about the revealed will of God, and proceeds according to the order which St. Paul has observed in the Epistle to the Romans, who first directs men to repentance, to knowledge of sins, to faith in Christ, to divine obedience, before he speaks of the mystery of the eternal election of God, to him this doctrine [concerning God's predestination] is useful and consolatory.

11. However, that many are called and few chosen, Matt. 22:14, does not mean that God is not willing to save everybody; but the reason is that they either do not at all hear God's Word, but wilfully despise it, stop their ears and harden their hearts, and in this manner foreclose the ordinary way to the Holy Ghost, so that He cannot perform His work in them, or, when they have heard it, make light of it again and do not heed it, for which [that they perish] not God or His election, but their wickedness, is responsible. [2 Pet. 2:1ff ; Luke 11:49. 52; Heb. 12:25f.]

12. Thus far a Christian should occupy himself [in meditation] with the article concerning the eternal election of God, as it has been revealed in God's Word, which presents to us Christ as the Book of Life, which He opens and reveals to us by the preaching of the holy Gospel, as it is written Rom. 8:30: Whom He did predestinate, them He also called. In Him we are to seek the eternal election of the Father, who has determined in His eternal divine counsel that He would save no one except those who know His Son Christ and truly believe on Him. Other thoughts are to be [entirely] banished [from the minds of the godly], as they proceed not from God, but from the suggestion of the Evil Foe, whereby he attempts to weaken or entirely to remove from us the glorious consolation which we have in this salutary doctrine, namely, that we know [assuredly] that out of pure grace, without any merit of our own, we have been elected in Christ to eternal life, and that no one can pluck us out of His hand; as He has not only promised this gracious election with mere words, but has also certified it with an oath and sealed it with the holy Sacraments, which we can [ought to] call to mind in our most severe temptations, and take comfort in them, and therewith quench the fiery darts of the devil.

13. Besides, we should use the greatest diligence to live according to the will of God, and, as St. Peter admonishes, 2 Pet. 1:10, make our calling sure, and especially adhere to [not recede a finger's breadth from] the revealed Word: that can and will not fail us.

14. By this brief explanation of the eternal election of God His glory is entirely and fully given to God, that out of pure mercy alone, without all merit of ours, He saves us according to the purpose of His will; besides, also, no cause is given any one for despondency or a vulgar, wild life [no opportunity is afforded either for those more severe agitations of mind and faintheartedness or for Epicureanism].

Negative Theses
False Doctrine concerning This Article.

Accordingly, we believe and hold: When any teach the doctrine concerning the gracious election of God to eternal life in such a manner that troubled Christians cannot comfort themselves therewith, but are thereby led to despondency or despair, or the impenitent are strengthened in their wantonness, that such doctrine is treated [wickedly and erroneously] not according to the Word and will of God, but according to reason and the instigation of the cursed Satan. For, as the apostle testifies, Rom. 15:4, whatsoever things were written aforetime were written for our learning, that we, through patience and comfort of the Scriptures, might have hope. Therefore we reject the following errors:

1. As when it is taught that God is unwilling that all men repent and believe the Gospel.

2. Also, that when God calls us to Himself, He is not in earnest that all men should come to Him.

3. Also, that God is unwilling that every one should be saved, but that some, without regard to their sins, from the mere counsel, purpose, and will of God, are ordained to condemnation so that they cannot be saved.

4. Also, that not only the mercy of God and the most holy merit of Christ, but also in us there is a cause of God's election, on account of which God has elected us to everlasting life.

All these are blasphemous and dreadful erroneous doctrines, whereby all the comfort which they have in the holy Gospel and the use of the holy Sacraments is taken from Christians, and therefore should not be tolerated in the Church of God.

----------

This is the brief and simple explanation of the controverted articles, which for a time have been debated and taught controversially among the theologians of the Augsburg Confession. Hence every simple Christian, according to the guidance of God's Word and his simple Catechism, can perceive what is right or wrong, since not only the pure doctrine has been stated, but also the erroneous contrary doctrine has been repudiated and rejected, and thus the offensive divisions that have occurred are thoroughly settled [and decided].

May Almighty God and the Father of our Lord Jesus grant the grace of His Holy Ghost that we all may be one in Him, and constantly abide in this Christian unity, which is well pleasing to Him! Amen.
(XII.) Other Heresies and Sects Which Never Embraced the Augsburg Confession.

In order that such [heresies and sects] may not silently be ascribed to us, because, in the preceding explanation, we have made no mention of them, we intend at the end [of this writing] simply to enumerate the mere articles wherein they [the heretics of our time] err and teach contrary to our Christian faith and confession to which we have often referred.
Erroneous Articles of the Anabaptists.

The Anabaptists are divided among themselves into many factions, as one contends for more, another for less errors; however, they all in common propound [profess] such doctrine as is to be tolerated or allowed neither in the Church, nor in the commonwealth and secular government, nor in domestic life.

Articles that Cannot be Tolerated in the Church.

1. That Christ did not assume His body and blood from the Virgin Mary, but brought them with Him from heaven.

2. That Christ is not true God, but only [is superior to other saints, because He] has more gifts of the Holy Ghost than any other holy man.

3. That our righteousness before God consists not in the sole merit of Christ alone, but in renewal, and hence in our own godliness [uprightness] in which we walk. This is based in great part upon one's own special, self-chosen [and humanly devised] spirituality [holiness], and in fact is nothing else than a new sort of monkery.

4. That children who are not baptized are not sinners before God, but righteous and innocent, who in their innocency, because they have not yet attained their reason [the use of reason], are saved without Baptism (which, according to their assertion, they do not need). Therefore they reject the entire doctrine concerning original sin and what belongs to it.

5. That children are not to be baptized until they have attained their reason [the use of reason], and can themselves confess their faith.

6. That the children of Christians, because they have been born of Christian and believing parents, are holy and children of God even without and before Baptism; and for this reason they neither attach much importance to the baptism of children nor encourage it, contrary to the express words of God's promise which pertains only to those who keep His covenant and do not despise it. Gen. 17:7ff

7. That that is no true Christian congregation [church] in which sinners are still found.

8. That no sermon is to be heard nor attended in those churches in which formerly papal masses have been celebrated and said.

9. That one [a godly man] must not have anything to do with the ministers of the Church who preach the Gospel according to the Augsburg Confession, and rebuke the sermons and errors of the Anabaptists; also that he is neither to serve nor in any way to labor for them, but to flee from and shun them as perverters of God's Word.

Articles that Cannot be Tolerated in the Government.

1. That under the New Testament the magistracy is not an estate pleasing to God.

2. That a Christian cannot with a good, inviolate conscience hold or discharge the office of magistrate.

3. That a Christian cannot without injury to conscience use the office of the magistracy against the wicked in matters as they occur [matters so requiring], nor that subjects may invoke for their protection and defense the power which the magistrates possess and have received from God.

4. That a Christian cannot with a good conscience take an oath, nor with an oath do homage [promise fidelity] to the hereditary prince of his country or sovereign.

5. That under the New Testament magistrates cannot, without injury to conscience, inflict capital punishment upon malefactors.

Articles that Cannot be Tolerated in Domestic Life.

1. That a Christian cannot with a good conscience hold or possess property, but is in duty bound to devote it to the common treasury.

2. That a Christian cannot with a good conscience be an innkeeper, merchant, or cutler [maker of arms].

3. That the married may be divorced on account of [diverse] faith, and the one may abandon the other and be married to another person who is of his faith.
Erroneous Articles of the Schwenkfeldians.

1. That all those have no true knowledge of Christ as reigning King of heaven who regard Christ according to the flesh as a creature.

2. That the flesh of Christ by His exaltation has assumed all divine properties in such a manner that Christ as man is in might, power, majesty, and glory altogether, as regards degree and position of essence equal to the Father and to the Word, so that now there is only one essence, property, will, and glory of both natures in Christ, and that the flesh of Christ belongs to the essence of the Holy Trinity.

3. That the ministry of the Church [ministry of the Word], the Word preached and heard, is not a means whereby God the Holy Ghost teaches men, and works in them the saving knowledge of Christ, conversion, repentance, faith, and new obedience.

4. That the water of Baptism is not a means whereby God the Lord seals the adoption of sons and works regeneration.

5. That bread and wine in the Holy Supper are not means through and by which Christ distributes His body and blood.

6. That a Christian who is truly regenerated by God's Spirit can perfectly observe and fulfil the Law of God in this life.

7. That it is not a true Christian congregation [church] in which no public excommunication [some formal mode of excommunication] or no regular process of the ban [as it is commonly called] is observed.

8. That the minister of the church who is not on his part truly renewed, regenerate, righteous, and godly cannot teach other men with profit or distribute genuine, true Sacraments.
Error of the New Arians.

That Christ is not true, essential, natural God, of one eternal, divine essence with God the Father and the Holy Ghost, but is only adorned with divine majesty inferior to and alongside of God the Father [is so adorned with divine majesty, with the Father, that He is inferior to the Father].
Error of the Anti-Trinitarians.

This is an entirely new sect, not heard of before in Christendom, [composed of those] who believe, teach, and confess that there is not one only, eternal, divine essence of the Father Son, and Holy Ghost, but as God the Father, Son, and Holy Ghost are three distinct persons, so each person has its essence distinct and separate from the other persons of the Godhead; and that nevertheless they are either [some think] all three of equal power, wisdom, majesty, and glory, just as otherwise three men are distinct and separate from one another in their essence, or [others think that these three persons and essences are] unequal with one another in essence and properties, so that the Father alone is properly and truly God.

These and similar articles, one and all, with whatever other errors depend upon and follow from them, we reject and condemn as wrong, false, heretical, contrary to the Word of God, the three Creeds, the Augsburg Confession and Apology, the Smalcald Articles, and Luther's Catechisms, against which all godly Christians of both high and low station are to be on their guard as they love the welfare and salvation of their souls.

That this is the doctrine, faith, and confession of us all, for which we will answer at the last day before the just Judge, our Lord Jesus Christ, and will neither secretly nor publicly speak or write anything against it, but that we intend by the grace of God to persevere therein, we have after mature deliberation testified, in the true fear of God and invocation of His name, by signing with our own hands [this Epitome]. 
