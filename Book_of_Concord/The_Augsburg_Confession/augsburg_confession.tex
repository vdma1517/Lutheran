THE AUGSBURG CONFESSION

Delivered to the Emperor, Charles V, at the

Diet of Augsburg, A.D. 1530

“I will speak of Your testimonies also before kings, and will not be ashamed.”

(Psalm 119:46)

 

 
PREFACE

Most Invincible Emperor, Caesar Augustus, most Clement Lord: Your Imperial Majesty summoned a Diet of the Empire here at Augsburg to resolve what measures should be taken against the Turk, that most atrocious, hereditary and ancient enemy of the Christian name and religion; namely, that it might be decided how to effectually withstand his furor and assaults with a strong and lasting military arrangement. This Diet was also summoned to resolve dissensions in the matter of our holy religion and Christian Faith, that in this matter of religion the opinions and judgments of parties might be heard in each other’s presence, and considered and weighed among ourselves in charity, leniency and mutual kindness. Then the things in the Scriptures which on either side have been differently interpreted or misunderstood, being corrected and laid aside, may be settled and brought back to one perfect truth and Christian concord. Thus in the future one pure and true religion may be embraced and maintained by us, that as we all serve and do battle under Christ, so we may be able also to live in unity and concord in one Christian Church. And since we, the undersigned Electors and Princes, with others joined with us, have been called to the aforesaid Diet, the same as the other Electors, Princes and Representatives, in obedient compliance with the Imperial mandate we have come to Augsburg, and without boasting we can say that we were among the first to be here.

Since then Your Imperial Majesty caused to be proposed to the Electors, Princes, and other Representatives of the Empire, also here at Augsburg, at the very beginning of this Diet, that among other things, by order of the Imperial Edict the several Representatives of the Empire should present their opinions and judgments in the German and Latin languages, after due deliberation, answer was given to Your Imperial Majesty last Wednesday, that on the next Friday the Articles of our Confession would be presented.

Therefore, in obedience to Your Imperial Majesty’s wishes, we offer in this matter of religion the Confession of our preachers and of ourselves, showing what sort of doctrine from the Holy Scriptures and the pure Word of God has been up to this time presented in our lands, territories, dominions and cities, and what has been taught in our churches. And if the other Electors, Princes and Representatives of the Empire will present similar writings in Latin and German, according to the Imperial proposition and give their opinions in this matter of religion here before Your Imperial Majesty, our most clement Lord, we are prepared to confer amicably with them concerning all possible ways and means, as far as may be honorably done, that we may come together. And when the matter between us has been peacefully discussed on both sides without offensive strife, by God’s help the dissension may be done away and we of one accord returned to the true religion. For as we all serve and do battle under one Christ, we ought to confess the one Christ and conduct ourselves according to the truth of God; and this we entreat of God with our most fervent prayers.

But, should the other Electors, Princes and Representatives decide that this mutual presentation of writings and calm conferring among ourselves, so wisely proposed by Your Imperial Majesty, should not proceed, or be unfruitful in results, so be it. We, at least, leave behind the clear testimony that we decline or refuse nothing at all allowed of God and a good conscience which might bring about Christian concord. Your Imperial Majesty, the other Electors and Representatives of the Empire, and all who are moved by sincere love and zeal for religion and will give an impartial hearing to this matter, will graciously perceive and understand this more and more from our Confession.

Also in dealing with this matter of religion Your Imperial Majesty, not only once but often, graciously signified to the Electors, Princes and Representatives of the Empire, and publicly proclaimed at the Diet of Spires, held in 1526 according to the prescribed form of Your Imperial instruction and commission, that while for certain reasons that were alleged in Your Majesty’s name Your Majesty was not willing to decide and could not determine anything, yet Your Majesty would diligently use Your Majesty’s office with the Roman Pontiff for the convening of a General Council. All of this was publicly set forth at greater length over a year ago at the last Diet which met in Spires. There Your Imperial Majesty, through His Highness Ferdinand, King of Bohemia and Hungary, our friend and most clement Lord, as well as through the Orator and Imperial Commissioners, caused this, among other things, to be proclaimed: that Your Imperial Majesty had known of and pondered the resolution of Your Majesty’s Ambassador in the Empire, and of the President and Imperial Counselors, and the Legates from other Estates convened at Ratisbon, concerning the calling of a Council, and that this also was judged by Your Imperial Majesty to be of advantage to all; and because the matters to be negotiated between Your Imperial Majesty and the Roman Pontiff were nearing agreement and Christian reconciliation, Your Imperial Majesty did not doubt the Roman Pontiff could be induced to hold a General Council; therefore Your Imperial Majesty himself signified that he would endeavor to secure the Chief Pontiff’s consent together with Your Imperial Majesty to convene such General Council, and that letters to that effect would be publicly issued as quickly as possible.

In the event, therefore, that the differences between us and the other parties in the matter of religion cannot be amicably and charitably settled here before Your Imperial Majesty, in all obedience we make this offer: in addition to what we have already done we are prepared to defend the cause in such a general, free, Christian Council. In all the Imperial Diets held during Your Majesty’s reign there has always been accordant action and agreement of votes for the convening of such a council on the part of the Electors, Princes, and other Representatives of the Empire. To this General Council, and at the same time to Your Imperial Majesty, we have made appeal in this greatest and gravest of matters even before this, in due manner and form of law. To this appeal, both to Your Imperial Majesty and to a Council, we still adhere, neither do we intend, nor would it be possible for us to relinquish it by this or any other document, unless the matter between us and the other side, according to the tone of the latest Imperial citation, can be amicably and charitably settled and brought to Christian accord, for this is our solemn and public testimony.

 

 

CHIEF ARTICLES OF FAITH

Article I

God

Our churches with common consent teach that the decree of the Council of Nicea concerning the Unity of the Divine Essence and concerning the Three Persons, is true and to be believed without any doubting; that is to say, there is one Divine Essence which is called and which is God: eternal, without body, without parts, of infinite power, wisdom and goodness, the Maker and Preserver of all things, visible and invisible; and yet that there are Three Persons, of the same essence and power, who are co-eternal, the Father, the Son and the Holy Spirit. And we use the term “person” as the Fathers have used it, to signify not a part or a quality in another, but that which subsists of itself.

We condemn all heresies which have sprung up against this article, such as that of the Manichaeans, who assert two gods, one Good and the other Evil; also that of the Valentinians, Arians, Eunomians, Mohammedans, and all like them. We also condemn the Samosatenes, old and new, who contend that there is but one Person, arguing with sophistry and impiety that the Word and the Holy Spirit are not distinct Persons, but that “Word” signifies a spoken word, and “Spirit” signifies the motion created in things.

 

Article II

Original Sin

We teach that since the fall of Adam all who are propagated according to nature are born in sin, that is, without the fear of God, without trust in God, and with concupiscence; and that this disease, or vice of origin is truly sin, even now condemning and bringing eternal death upon those not born again through Baptism and the Holy Spirit.

We condemn the Pelagians and others who deny that the vice of origin is sin, and who, to obscure the glory of Christ’s merit and benefits, argue that a person can be justified before God by his own strength and reason.

 

Article III

The Son of God

We teach that the Word, that is, the Son of God, took on man’s nature in the womb of the blessed Virgin Mary, so that there are two natures, the divine and the human, inseparably conjoined in one person, one Christ, true God and true Man, who was born of the Virgin Mary, truly suffered, was crucified, died and was buried, that He might reconcile the Father unto us and be a sacrifice, not only for original guilt, but for all actual sins of men.

He also descended into hell, and truly rose again the third day; afterward He ascended into heaven, that He might sit at the right hand of the Father, and forever reign, and have dominion over all creatures and sanctify those who believe in Him by sending the Holy Spirit into their hearts to rule, comfort and enliven them, and to defend them against the devil and the power of sin. The same Christ shall openly come again to judge the living and the dead, etc., according to the Apostles’ Creed.

 

Article IV

Justification

We teach that men cannot be justified before God by their own strength, merits or works, but are freely justified for Christ’s sake through faith, when they believe that they are received into favor and that their sins are forgiven for Christ’s sake, who by His death has made satisfaction for our sins. This faith God accounts as righteousness in His sight, Rom. 3 and 4.

 

Article V

The Ministry of the Church

That we may obtain this faith, the ministry of teaching the Gospel and administering the sacraments was instituted. For through the Word and sacraments, as through instruments, the Holy Spirit is given, who works faith where and when it pleases God in those who hear the Gospel. That is, God, not because of our own merits, but for Christ’s sake, justifies those who believe that they are received into favor for Christ’s sake.

We condemn the Anabaptists and others who think that the Holy Spirit comes without the external Word but through their own preparations and works.

 

Article VI

New Obedience

We teach that this faith is bound to bring forth good fruits, and that it is necessary to do good works commanded by God because it is God’s will. Yet we should not rely on those works to merit justification before God. For the forgiveness of sins and justification are apprehended by faith, as the words of Christ attest, “When you have done all those things which you are commanded, say, ‘We are unprofitable servants’” [Luke 17:10]. The same is also taught by the Fathers. For Ambrose says, “It is ordained of God that he who believes in Christ is saved, freely receiving remission of sins, without works, by faith alone.”

 

Article VII

The Church

We teach that one holy Church is to continue forever. The Church is the congregation of saints, in which the Gospel is rightly taught and the sacraments rightly administered. And concerning the true unity of the Church, it is enough to agree concerning the doctrine of the Gospel and the administration of the sacraments. Nor is it necessary that human traditions, rites, or ceremonies, instituted by men, should be the same everywhere. As St. Paul says, “One faith, one baptism, one God and Father of all,” etc. [Eph. 4:5,6].

 

Article VIII

What the Church Is

Although the Church is properly the congregation of saints and true believers, nevertheless, since in this life many hypocrites and evil persons are mingled among the believers, it is allowable to use the sacraments which are administered by evil men, according to the saying of Christ, “The Scribes and the Pharisees sit in Moses’ seat,” etc. [Matt. 23:2]. Both the sacraments and the Word are effective because of the institution and commandment of Christ, even when administered by evil men.

We condemn the Donatists, and all like them, who denied it to be allowable to use the ministry of evil men in the Church, and who thought the ministry of evil men to be unprofitable and of no effect.

 

Article IX

Baptism

Of Baptism we teach that it is necessary to salvation, and that through Baptism is offered the grace of God; and that children are to be baptized, who, being offered to God through Baptism, are received into His grace.

We condemn the Anabaptists, who do not allow the baptism of children and say that children are saved without Baptism.

 

Article X

The Lord’s Supper

Of the Lord’s Supper we teach that the Body and Blood of Christ are truly present and are distributed to those who eat in the Lord’s Supper. We reject those who teach otherwise.

 

Article XI

Confession

Of Confession we teach that Private Absolution ought to be retained in the churches, although an enumeration of all sins is not necessary in confession. Indeed, it is impossible according to the Psalm, “Who can understand his errors?” [Psalm 19:12]

 

Article XII

Repentance

Of repentance we teach that for those that have fallen after Baptism there is remission of sins whenever they are converted; and that the Church ought to impart Absolution to those thus returning to repentance.

Now repentance consists properly of two parts: One is contrition, that is, terrors smiting the conscience through the knowledge of sin; the other is faith, which, born of the Gospel, or of Absolution, believes that for Christ’s sake sins are forgiven, comforts the conscience and delivers it from terrors. Then good works are bound to follow, being the fruits of repentance.

We condemn the Anabaptists, who deny that those once justified can lose the Holy Spirit. We also condemn those who contend that some may attain to such perfection in this life that they cannot sin. The Novatians are also condemned, who would not absolve those who had fallen after Baptism even though they repented. They are also rejected who do not teach that forgiveness of sins comes through faith, but command us to merit grace by satisfactions of our own.

 

Article XIII

The Use of the Sacraments

Of the use of the sacraments we teach that the sacraments were ordained not simply to be marks of profession among us but rather to be signs and testimonies of the will of God toward us, instituted to awaken and confirm faith in those who use them. Therefore we must so use the sacraments that faith be added to believe the promises which are offered and set forth through the sacraments.

We therefore condemn those who teach that the sacraments justify by the outward act and do not teach that, in the use of the sacraments, faith, which believes that sins are forgiven, is required.

 

Article XIV

Ecclesiastical Order

Of ecclesiastical order we teach that no one should publicly teach in the Church or administer the sacraments unless he is rightfully called.

 

Article XV

Ecclesiastical Rites

Of rites in the Church we teach that those ought to be observed which may be observed without sin, and which are profitable for tranquility and good order in the Church, such as particular holidays, festivals, and the like.

Nevertheless, concerning such things, let all be admonished that consciences are not to be burdened, as though such an observance were necessary to salvation. We also admonish that human traditions instituted to appease God, to merit grace and to make satisfaction for sins, are opposed to the Gospel and the doctrine of faith. Therefore vows and traditions concerning foods and days, etc., instituted to merit grace and to make satisfaction for sins, are useless and contrary to the Gospel.

 

 

Article XVI

Civil Affairs

Of civil affairs we teach that lawful civil ordinances are good works of God, and that it is right for Christians to hold civil office, to sit as judges, to determine matters by the Imperial and other existing laws, to award just punishments, to engage in just wars, to serve as soldiers, to make legal contracts, to own property, to make an oath when required by a judge, to marry and to be given in marriage.

We condemn the Anabaptists who forbid these civil offices to Christians. Also we condemn those who place the perfection of the Gospel not in the fear of God and in faith, but in forsaking civil offices; for the Gospel teaches an eternal righteousness in the heart. Meanwhile, it does not destroy the State or the family, but very much requires their preservation as the ordinances of God, and that charity be practiced in them. Therefore, Christians are necessarily bound to obey their own governments and laws, unless they command them to sin, for then they ought to obey God rather than men [Acts 5:29].

 

Article XVII

Christ’s Return to Judgment

We teach that at the consummation of the world Christ shall appear for its judgment, and shall raise up all the dead; He shall give to the godly and elect eternal life and everlasting joys, but the ungodly and the devils He shall condemn to be tormented without end.

We condemn the Anabaptists who think that there will be an end to the punishments of the damned and the devils. We also condemn others who are now spreading certain Jewish opinions that, before the resurrection of the dead, the godly shall take possession of the kingdom of the world, the ungodly being everywhere suppressed.

 

Article XVIII

The Freedom of the Will

Of the freedom of the will we teach that a person’s will has some liberty to choose civil righteousness, and to choose in things subject to reason. Nevertheless it has no power without the Holy Spirit to work the righteousness of God, that is, spiritual righteousness, since “the natural man does not receive the things of the Spirit of God” [1 Cor. 2:14]. Rather, this righteousness is worked in the heart when the Holy Spirit is received through the Word. These things are said in as many words by Augustine in his Hypognosticon, book 3: “We grant that all have a certain freedom of the will, inasmuch as we have the ability to reason. Yet the human will does not have a freedom which without God is capable to begin, much less complete, any one of the things pertaining to God. Only in the deeds of this life are we able to choose good or evil. ‘Good’ I call those works which spring from the good in nature; that is, to be willing to labor in the field, to eat and drink, to have a friend, to clothe oneself, to build a house, to marry, to keep cattle, to learn various and useful arts, or whatever good pertains to this life, none of which things are without dependence on the providence of God; yea, of Him and through Him they are and have their beginning. ‘Evil’ I call such works as to want to worship an idol, to commit murder,” etc.

We condemn the Pelagians and others who teach that without the Holy Spirit, by the power of human nature alone, we are able to love God above all things; also the notion that we can keep the commandments of God as touching “the substance of the act.” For, although human nature is able in some sort to do the outward work (for it is able to keep the hands from theft and murder), yet it cannot work the inward motions, such as the fear of God, trust in God, chastity, patience, etc.

 

Article XIX

The Cause of Sin

Of the cause of sin we teach that although God does create and preserve nature, yet the cause of sin is the will of the wicked, that is, of the devil and the ungodly. For unaided by God, the human will turns itself away from God, as Christ says, “When he speaks a lie, he speaks from his own resources” [John 8:44].

 

Article XX

Faith and Good Works

Our teachers are falsely accused of forbidding good works. Indeed their published writings on the Ten Commandments and similar things bear witness that we have taught with good intentions concerning all stations and duties of life, as to what stations of life and what duties in every calling are pleasing to God. Before this preachers taught very little about these things, and urged only childish and needless works, such as particular Church festivals, particular fasts, brotherhoods, pilgrimages, services in honor of saints, the use of rosaries, monasticism, and the like. And now that our adversaries have been admonished about these things they are unlearning them, and do not preach these unprofitable works as they did before. They are beginning to mention faith, about which before there was extraordinary silence. They now teach that we are not justified by works only, but they join faith and works, and say that we are justified by faith and works. This doctrine is more tolerable than the former one, and at least gives more consolation than their old doctrine.

Seeing that the doctrine concerning faith, which ought to be the foremost in the Church, has so long lain unknown (as all must concede that there was the deepest silence in their sermons concerning the righteousness of faith, while only the doctrine of works was treated in the churches), our teachers have instructed the churches concerning faith as follows:

First, that our works cannot reconcile us to God or merit forgiveness of sins, grace and justification; but that we obtain this only by faith when we believe that we are received into grace for Christ’s sake, who alone has been set forth the Mediator and Propitiation, in order that the Father may be reconciled through Him. Whoever, therefore, trusts that by works he merits grace, despises the merit and grace of Christ and seeks a way to God by human strength without Christ, even though Christ has said of Himself, “I am the Way, the Truth, and the Life” [John 14:6].

This doctrine concerning faith is everywhere treated by Paul in this way: “By grace you have been saved through faith, and that not of yourselves; it is the gift of God, not of works” [Eph. 2:8], etc.

And lest anyone should sneer and say that we have devised a new interpretation of Paul, this entire matter is supported by the testimonies of the Fathers. For in many volumes Augustine defends grace and the righteousness of faith over against the merits of works. And Ambrose, in his De Vocatione Gentium and elsewhere, teaches to like effect. For in his De Vocatione Gentium he writes as follows: “Redemption by the Blood of Christ would become of little value, neither would the preeminence of man’s works be superseded by the mercy of God, if justification, which is worked through grace, were due to previous merits, making it not the gift of a donor but the reward due a laborer.”

But although this doctrine is despised by the inexperienced, nevertheless God-fearing and distressed consciences have through experience found that it brings the greatest consolation. Consciences can never be pacified through any works, but only by faith, being assured that for Christ’s sake they have a gracious God, just as Paul teaches: “Having been justified by faith, we have peace with God” [Rom. 5:1]. This whole doctrine is to be linked to that struggle in the conscience. Therefore, inexperienced and profane men judge poorly in this matter, imagining that Christian righteousness is nothing but the civil righteousness of natural reason.

Before this consciences were plagued with the doctrine of works and never heard any consolation from the Gospel. Some were driven by conscience into the desert, some into monasteries, hoping to merit grace there by a monastic life. Some devised other works to merit grace and make satisfaction for sins. There was a very great need to treat of and renew this doctrine of faith in Christ, so that distressed consciences are not without consolation, but know that grace, forgiveness of sins, and justification are apprehended by faith in Christ.

We also caution that here the term “faith” does not mean a mere knowledge of history, for even the ungodly and the devil possess that. Rather it means a faith which believes not only the history, but also the effect of the history—namely, the article of the forgiveness of sins; that through Christ we have grace, righteousness and forgiveness of sins.

Now he that knows he has a Father reconciled to him through Christ, since he truly knows God, also knows that God cares for him and calls upon God. In short, he is not without God as the heathen are. For devils and the ungodly are not able to believe this article of the forgiveness of sins. Hence, they hate God as an enemy, do not call upon Him, and expect no good from Him. Augustine also admonishes his readers concerning the word “faith” and teaches that the term “faith” is used in the Scriptures not for knowledge such as you find in the ungodly, but for confidence which consoles and encourages the terrified mind.

Furthermore, it is taught by us that it is necessary to do good works, not so that we can believe we merit grace by them, but because it is the will of God. It is only by faith that forgiveness of sins and grace are apprehended. We also teach this because it is through faith that the Holy Spirit is received, hearts are renewed and endowed with new affections, so as to be able to bring forth good works. For Ambrose says, “Faith is the mother of a good will and right doing.” For without the Holy Spirit human powers are full of ungodly affections and are too weak to do works that are good in God’s sight. Besides, they are in the power of the devil, who impels men to various sins, to ungodly opinions and to open crimes. We see this in the philosophers, who although they endeavored to live an honest life, could not succeed but were defiled with many open crimes. Such is the impotence of humanity when it is without faith and without the Holy Spirit and governs itself only by human strength.

Hence it may be readily seen that this doctrine is not to be charged with prohibiting good works, but rather it is to be further commended because it shows how we are able to do good works. For without faith, human nature can never do the works of the First or of the Second Commandment. Without faith, it does not call upon God, nor expect any help from Him, nor bear the cross; but it seeks and trusts in man’s help. And thus, when there is no faith and trust in God, all manner of lusts and human vices rule in the heart. For this reason Christ said [John 15:5], “Without Me you can do nothing,” and the Church sings:

We know no dawn but Thine:

Send forth Thy beams divine

On our dark souls to shine

And make us blest.

 

Article XXI

The Worship of Saints

Of the worship of the saints we teach that the memory of saints may be set before us that we may follow their faith and good works according to our calling, as the Emperor may follow the example of David in making war to drive away the Turk from his country, for both are kings. But the Scriptures do not teach the invocation of saints, or to ask help of saints, since it sets before us Christ as the only Mediator, Propitiation, High Priest, and Intercessor. He is to be prayed to and has promised that He will hear our prayer. This worship He approves above all: that in afflictions He be called upon: “If any man sins, we have an Advocate with the Father” [1 John 2:1], etc.

 

This is about the sum of our doctrine, in which it can be seen there is nothing that varies from the Scriptures or from the Church Catholic or from the Church of Rome as known from its writers. This being the case, they judge harshly who insist that our teachers be regarded as heretics. Rather the disagreement is on certain abuses which have crept into the Church without rightful authority. And even in these, if there were some difference, there should be proper leniency on the part of bishops to bear with us by reason of the Confession which we have now drawn up. For even the Canons are not so severe as to demand the same rites everywhere, neither at any time have the rites of all churches been the same, although the majority among us diligently observe the ancient rites. For it is a false and malicious charge that all the ceremonies, all the things instituted of old, are abolished in our churches. But it has been a common complaint that some abuses were connected with ordinary rites. These, insofar as they could not be approved with good conscience, have been to some extent corrected.

 

Articles in which are Recounted the

Abuses which have been Corrected

Inasmuch as our churches dissent in no article of the faith from the Church Catholic, but omit some abuses which are new and, contrary to the intent of the Canons, have been erroneously accepted by fault of the times, we pray that Your Imperial Majesty would graciously hear both what has been changed, and also what were the reasons, in order that the people be not compelled to observe those abuses against their conscience. Nor should Your Imperial Majesty believe those who, in order to excite the hatred of men against our side, disseminate outlandish slanders among our people. Having thus excited the minds of good people, they have first given rise to this controversy, and now endeavor by those same methods to increase the discord. For Your Imperial Majesty will undoubtedly find that the form of doctrine and of ceremonies with us is not so intolerable as these ungodly and malicious men represent. Furthermore, the truth cannot be gathered from common rumors or the revilings of our enemies. But it can be readily judged that nothing would serve better to maintain the dignity of worship and to nourish reverence and pious devotion among the people than proper observance of the ceremonies in the churches.

 

Article XXII

Both Kinds in the Lord’s Supper

[Note: It was the practice of the Roman Church to distribute only in one kind, to give only the bread to the laity during the celebration of the Lord’s Supper. The reformers show this to be contrary to the command of Christ and the practice of the ancient Church.]

To the laity are given both kinds in the sacrament of the Lord’s Supper, because this practice has the commandment of the Lord: “Drink of it, all of you,” where Christ has expressly commanded concerning the cup that all should drink. And lest anyone should craftily say that this refers only to priests, Paul [1 Cor. 11:27] recites an example from which it appears that the whole congregation did use both kinds. And this practice has long remained in the Church, nor is it known when, or by whose authority, it was changed, although Cardinal Cusanus mentions the time when it was approved. Cyprian in some places testifies that the Blood was given to the people. We have the same testimony from Jerome, who says, “The priests administer the Eucharist and distribute the Blood of Christ to the people.” Indeed, Pope Gelasius commands that the sacrament not be divided in De Consecratione, dist. 2, chapter “Comperimus.” Only custom, and then not so ancient, has it otherwise. But it is evident that any custom introduced against the commandments of God is not to be allowed, as the Canons witness (Dist. 3, chapter “Veritate,” and the following chapters). But this custom has been received not only against the Scriptures, but also against the old Canons and the example of the Church. Therefore if any preferred to use both kinds of the sacrament, they ought not to have been compelled with offense to their consciences to do otherwise.

And because the division of the sacrament does not agree with the ordinance of Christ we are accustomed to omit the procession, which up until now has been in use.

 

Article XXIII

The Marriage of Priests

[Note: This article shows how the forbidding of the marriage of priests was also a departure from Scripture and the practice of the ancient Church.]

There has been common complaint concerning the examples of priests who were not chaste. For that reason also, Pope Pius is reported to have said that there were certain reasons why marriage was taken away from priests, but that there were far weightier ones why it ought to be given back, as reported by Platina. Since therefore, our priests wished to avoid these open scandals, they married wives and taught that it was lawful for them to contract matrimony. First, because Paul says [1 Cor. 7:2]: “Because of sexual immorality, let each man have his own wife,” and [v. 9], “It is better to marry than to burn.” Secondly, Christ says [Matt. 19:11], “All cannot accept this saying”, where He teaches that not all men are fit to lead a single life, for God created man for procreation [Gen. 1:28]. Nor is it in man’s power, without a singular gift and work from God, to alter this creation. Therefore those who are not fit to lead a single life ought to contract matrimony. For no human law, no vow, can annul the commandment and ordinance of God. For these reasons the priests teach that it is lawful for them to marry wives. It is also evident that in the ancient Church priests were married men, for Paul says [1 Tim. 3:2] that a bishop must be the husband of one wife. And four hundred years ago in Germany priests were for the first time violently compelled to lead a single life. They offered such resistance that the Archbishop of Mainz was almost killed in the insurrection raised by angry priests as he was about to proclaim the Pope’s decree in this matter. And so harsh was the dealing in the matter that not only were future marriages forbidden, but existing marriages were annulled, contrary to all laws, divine and human, contrary even to the Canons themselves, which were made not only by the Popes but by the finest councils.

Seeing also that, as the world is aging, human nature is gradually growing weaker, it is well to guard that no more vices steal into Germany. Furthermore, God ordained marriage to be a help against human infirmity. The Canons themselves say that the old rigors ought now and then be relaxed in the later times because of human weakness, and it is devoutly desired this be done regarding the marriage of priests. And it is to be expected that churches shall soon lack pastors if marriage should be forbidden much longer.

But while the commandment of God is in force, while the custom of the Church is well known, while impure celibacy causes many scandals, adulteries and other crimes deserving the punishments of good judges, yet it is astonishing that nowhere do we see more cruelty than what is exercised against the marriage of priests. God has given the commandment to honor marriage. By the laws of all well-ordered nations, even among the heathen, marriage is most highly honored. But now, against the intent of the Canons, men, even priests, are cruelly put to death for no other cause than marriage. In 1 Timothy 4:3 Paul calls a teaching that forbids marriage a “doctrine of demons.” This is now easily understood when the law against marriage is enforced by such penalties.

And as no human law can annul the commandment of God, so neither can it be done by any vow. Accordingly, Cyprian also advises that women who do not keep the chastity they have promised should marry. He writes [Book I, Epistle XI], “But if they are unwilling or unable to persevere, it is better for them to marry than to fall into the fire by their lusts, for at least then they will give no offense to their brothers and sisters.” And even the Canons show some leniency toward those who have taken vows before the proper age, as up until now has generally been the case.

 

Article XXIV

The Mass

[Note: The Roman Church accused the Lutherans of abolishing the observance of the Lord’s Supper. In this article the Lutherans show their high regard for the Sacrament of the Altar, and that what they abolished were practices which degraded the Lord’s Supper, obscured its benefits or misinterpreted its purpose.]

Falsely are our churches accused of abolishing the Mass, for the Mass is retained by us and celebrated with the highest reverence. All the usual ceremonies are also preserved, except that the parts sung in Latin are interspersed here and there with German hymns, which have been added to teach the people. For ceremonies are needed for this reason alone: that the unlearned be taught. And not only has Paul, in 1 Corinthians 14, commanded that the Church use a language understood by the people, but it has also been so ordained by human law.

The people are accustomed to receive the sacrament together, if any are fit to do so, and this increases the reverence and devotion of public worship. For none are admitted unless they are first examined. The people are also advised concerning the dignity and use of the sacrament, what great consolation it brings troubled consciences, that they might learn to believe God and to expect and ask of Him all that is good. This worship pleases God; for such use of the sacrament nourishes true devotion toward God. It does not, therefore, appear that the Mass is more devoutly celebrated among our adversaries than among us.

But it is evident that for a long time it has been the public and most serious complaint of all good people that Masses have been basely profaned and even used for the purpose of collecting money. For it is unknown how far this abuse extends in all the churches, what kind of men say Masses only for fees and stipends, and how many celebrate them contrary to the Canons. But Paul severely threatens those who deal unworthily with the Eucharist when he says [1 Cor. 11:27], “Whoever eats this bread or drinks this cup of the Lord in an unworthy manner will be guilty of the body and blood of the Lord.” So when our priests were admonished concerning this sin, private Masses were discontinued among us, as hardly any private Masses were celebrated except for collecting money.

Neither were the bishops ignorant of these abuses, and if they had corrected them in time there would now be less dissension. Up until the present time, by their own negligence they allowed many corruptions to creep into the Church. Now, when it is too late, they begin to complain about the troubles of the Church, seeing that this disturbance has been occasioned simply by those abuses, which are so open that they could be tolerated no longer. Great dissensions have arisen concerning the Mass, concerning the Sacrament. Perhaps the world is being punished for such long and continued profanations of the Mass as have been tolerated in the churches for so many centuries by the very men who were both able and duty bound to correct them. For in the Ten Commandments [Exodus 20:7] it is written, “The LORD will not hold him guiltless who takes His name in vain.” But since the world began, nothing that God ever ordained seems to have been so abused for dishonest gain as the Mass.

There was also added the opinion, which infinitely increased Private Masses, that Christ by His Passion had made satisfaction for original sin and instituted the Mass so that offerings might be made for daily sins, venial and mortal. From this has arisen the common opinion that the Mass takes away the sins of the living and the dead, simply by the outward act. Then they began to argue whether one Mass said for many was worth as much as special Masses for individuals, and this brought forth that infinite multitude of Masses. Our teachers have warned that these opinions depart from the Holy Scriptures and diminish the glory of the Passion of Christ. For Christ’s Passion was both sacrifice and satisfaction, not only for original guilt, but also for all sins, just as it is written [Hebrews 10:10]: “We have been sanctified through the offering of the body of Jesus Christ once for all.” Also [v. 14], “By one offering He has perfected forever those who are being sanctified.” Scripture also teaches that we are justified before God through faith in Christ when we believe that our sins are forgiven for Christ’s sake. Now if the Mass takes away the sins of the living and the dead by the outward act, then justification comes through the work of the Masses and not of faith. This the Scriptures do not allow.

Rather, Christ commanded us, “This do in remembrance of Me” [Luke 22:19]; therefore the Mass was instituted that the faith of those who use the Sacrament should remember what benefits are received through Christ and should cheer and comfort the troubled conscience. For to remember Christ is to remember His benefits and to realize that they are truly offered to us. Nor is it enough only to remember the history, for this the Jews and the ungodly also can remember. Therefore the Mass is to be used for this purpose, that there the sacrament may be administered to them that have need of consolation, as Ambrose says, “Because I always sin, I am always bound to take the medicine.”

Now, inasmuch as the Mass is such a giving of the sacrament, we hold one Communion every holy day, and on other days it is given to those who ask for it, should any desire the sacrament. Nor is this custom new in the Church; for the Fathers before Gregory make no mention of any private Mass, but of the common Mass they speak very much. Chrysostom says, “The priest stands daily at the altar, inviting some to Communion and keeping back others.” And it appears from the ancient Canons that one man celebrated the Mass, from whom all the other presbyters and deacons received the Body of the Lord, for the words of the Nicene Canon say: “Let the deacons, according to their order, receive the Holy Communion after the presbyters, from the bishop or from a presbyter.” And Paul in 1 Cor. 11:33 commands concerning the Communion, “Wait for one another,” so that there may be a common participation.

Since we celebrate the Mass according to the example of the Church, taken from Scripture and the Fathers, we are confident that it cannot be disapproved, especially since the public ceremonies are retained for the most part, like those currently in use. Only the number of Masses differs, a number which, in consideration of the manifest abuses, might without doubt be profitably reduced. For in the past, even in churches most frequented, the Mass was not celebrated every day, as the Tripartite History, book 9, chapter 33, testifies: “Again in Alexandria, every Wednesday and Friday, the Scriptures are read and the doctors expound them; all things are done except the solemn rite of Communion.”

 

Article XXV

Confession

[Note: When accused of abolishing the practice of private confession, the Lutherans respond by showing they retained this worthy practice, as stated in Article XI. However, while the Roman Church had focused on the enumeration, or listing of sins, the Lutherans retained private confession for the sake of private Absolution, that Christians might individually hear the blessed Gospel of forgiveness.]

Confession in our churches is not abolished, for it is not usual to give the Body of the Lord except to those who have been previously examined and absolved. And the people are most carefully taught concerning the faith and assurance of absolution, about which before there was profound silence. Our people are taught that they should highly esteem absolution, for it is the voice of God proclaimed by God’s command. The power of the Keys is acclaimed as an ornament of the Church, and we show what great consolation it brings to troubled consciences; furthermore that God requires faith to believe the absolution as a voice sounding from heaven and that such faith in Christ truly obtains and receives the forgiveness of sins.

Formerly, satisfactions were excessively advocated, and no mention was made of faith or the merit of Christ or the righteousness of faith. Because of this there is no reason to blame our churches on this point. For even our adversaries must concede this to us, that the doctrine concerning repentance has been most diligently treated and laid open by our teachers.

But of confession we teach that enumeration of sins is not necessary, and that consciences not be burdened with the anxiety to enumerate all sins, for it is impossible to remember all sins, as Psalm 19:12 testifies, “Who can understand his errors?” Also Jeremiah 17:9, “The heart is deceitful…who can know it?” If no sins were forgiven except those recounted, consciences could never find peace, for there are many sins they neither see nor remember.

The ancient writers also testify that an enumeration is not necessary. For in the Decrees Chrysostom is quoted as saying, “I do not say to you that you should disclose yourself in public or that you should accuse yourself before others, but I would have you obey the prophet who says, ‘Disclose your way before God.’ Therefore, with prayer confess your sins before God, the true Judge. Tell your errors not with your tongue but with the memory of your conscience.” And the gloss in De Poenitentia, dist. 5, chapter “Consideret,” admits that confession comes to us by human right only. Nevertheless, on account of the great benefit of absolution and because it is otherwise useful to the conscience, confession is retained among us.

 

Article XXVI

The Distinction of Foods

[Note: The Roman Church taught that observing special times of fasting (such as not eating meat during Lent) and other traditions were works that merited grace, indeed that these were better works than those of secular duties at work or home. The Reformers show that while such traditions are often beneficial, their observance should never be allowed to burden consciences or obscure Christ.]

It has been the general opinion, and not of the people alone, but also of some who teach in the churches, that making distinction of foods and similar human traditions are good works; that they merit grace and are able to make satisfaction for sins. And that the world so thought is obvious from this, that new ceremonies, new orders, new Church festivals and new fastings were daily instituted, and the teachers in the Church demanded these works as a service necessary to merit grace, and greatly terrified consciences if a person should omit any of these things. From this opinion concerning traditions much detriment has resulted in the Church.

First, the doctrine of grace and of the righteousness of faith has been obscured by it. This is the chief part of the Gospel and ought to stand out as the most prominent teaching in the Church, so that the merits of Christ may be well known and that faith, which believes that sins are forgiven for Christ’s sake, may be exalted far above works. For this reason Paul also lays the greatest stress on this article, putting aside the law and human traditions in order to show that the righteousness of the Christian is apart from such works, indeed, it is the faith that believes that sins are freely forgiven for Christ’s sake. But this doctrine of Paul has been almost wholly smothered by traditions, which has produced the opinion that by making distinctions in food and like services we must merit grace and righteousness. In treating repentance there was no mention made of faith; all that was done was to set forth those works of satisfaction, and repentance seemed to consist solely in these.

Secondly, these traditions have obscured the commandments of God, because traditions were placed far above the commandments of God. Christianity was thought to consist wholly in the observance of certain festivals, rites, fasts and vestures. These observances have won for themselves the exalted title of being the spiritual life and the perfect life. Meanwhile, the commandments of God, according to each one’s calling, were without honor; namely, that the father brought up his family, that the mother bore children, that the prince governed the state—these works were labeled worldly and imperfect, and far below those glittering observances. And this error greatly tormented devout consciences, which grieved that they were trapped in an imperfect state in life, such as marriage, the office of magistrate or other civil administrations. On the other hand, they admired the monks and those like them, and falsely imagined that the observances of these were more acceptable to God.

Thirdly, traditions brought great danger to consciences, for it was impossible to keep all traditions, and yet men judged these observances to be necessary acts of worship. Gerson writes that many fell into despair and that some even took their own lives, because they felt that they were not able to satisfy the traditions. Meanwhile, they had not heard the consolation of the righteousness of faith and grace.

We see that the summists and theologians gather the traditions together and seek mitigations to ease consciences. Yet they do not succeed in releasing them but sometimes entangle consciences even more. And the schools and sermons have been so much occupied with the assembling of these traditions that they do not have the leisure to touch upon Scripture, and to seek the more beneficial doctrines of faith, of the cross, of hope, of the dignity of civil affairs, and of the consolation of sorely tried consciences. Hence Gerson and some other theologians have with sorrow complained that by these strivings concerning traditions they were prevented from giving attention to a better kind of doctrine. Augustine also forbids that a person’s conscience should be burdened with such observances, and prudently advises Januarius that he must know that they are to be observed as indifferent things; these are his words.

Therefore our teachers must not be looked upon as having taken up this matter rashly or out of hatred of the bishops, as some falsely suspect. There was great need to warn the churches of these errors which had arisen from misunderstanding the traditions. For the Gospel compels us to insist in the churches upon the doctrine of grace and of the righteousness of faith. These however cannot be understood if people think they merit grace by observances of their own choice.

Thus they have taught that by the observance of human traditions we cannot merit grace or be justified; and hence we must not think such observances to be necessary acts of worship.

They add to this the testimonies of Scripture. Christ [Matt. 15:3] defends the Apostles who had not observed the usual tradition, which seemed to pertain to a matter which was not unlawful, but indifferent, and was related to the purifications of the law. He says, “In vain they worship Me, teaching as doctrines the commandments of men.” Christ, therefore, does not demand an unnecessary service. Shortly thereafter He adds, “Not what goes into a mouth defiles a man.” So also St. Paul [Rom. 14:17]: “The kingdom of God is not food and drink.” And in Colossians [2:16]: “Let no one judge you in food or in drink, or regarding a festival or a new moon or sabbaths”; also, “If you died with Christ from the basic principles of this world, why, as though living in the world, do you subject yourself to regulations—‘Do not touch, do not taste, do not handle’?” And St. Peter says [Acts 15:10–11]: “Why do you test God by putting a yoke on the neck of the disciples which neither our fathers nor we were able to bear? But we believe that through the grace of the Lord Jesus Christ we shall be saved in the same manner as they.” Here Peter forbids the burdening of consciences with many rites, either of Moses or of others.

And in 1 Timothy [4:1, 3] St. Paul calls the prohibition of foods a “doctrine of demons,” for it is against the Gospel to institute or to do such works so that by them we may merit grace, or as though Christianity could not exist without such service to God.

Here our adversaries object that our teachers are opposed to discipline and mortification of the flesh, calling them Jovinian. But the contrary may be learned from the writings of our teachers. For they have always taught concerning the cross that Christians are under obligation to bear afflictions. This is the true, earnest and unfeigned mortification: being harassed with various afflictions and to be crucified with Christ.

Moreover, they teach that every Christian ought to discipline and subdue himself with exercise and labor, so that neither plenty nor laziness tempt him to sin; but not that we may merit grace or make satisfaction for sins with such activities. And such external discipline ought to be urged at all times, not only on a few and set days. So Christ commands [Luke 21:34]: “Take heed to yourselves, lest your hearts be weighed down with carousing;” also [Matt. 17:21]: “This kind does not go out except by prayer and fasting.” St. Paul also says [1 Cor. 9:27]: “I discipline my body and bring it under subjection.” Here he clearly shows he was disciplining his body not to merit forgiveness of sins by that discipline, but to have his body in subjection and fitted for spiritual things, and for the discharge of duty according to his calling. Therefore we do not condemn fasting, but the traditions that prescribe certain days and certain foods with peril of conscience, as though works of such kind were a necessary service.

Nevertheless, very many traditions are kept on our part, for they lead to good order in the Church, such as the Order of Lessons in the Mass and the chief festivals. But at the same time people are warned that such observances do not justify before God, and that in such things it should not be called a sin if they are omitted without scandal. Such liberty in human rites was not unknown to the Fathers. For in the East they kept Easter at a different time than in Rome, and when on account of this diversity the Romans accused the Eastern Church of schism, they were admonished by others that such rites need not be alike everywhere. And Irenaeus says: “Diversity concerning fasting does not destroy the harmony of faith,” as also Pope Gregory intimates, in Dist. xii, that such diversity does not violate any unity of the Church. And in the Tripartite History, book 9, many examples of dissimilar rites are gathered, and the following statement is made: “It was not the mind of the Apostles to enact rules concerning Church festivals, but to preach godliness and a holy life.”

 

Article XXVII

Monastic Vows

[Note: The Lutherans had three criticisms concerning monastic life. First, monastic life used to be voluntary; one could freely join and freely leave at any time. However by Luther’s time many in the monasteries and convents had been forced to join, and once they had taken their vows it was almost impossible for them to leave. Secondly, the Roman Church touted monastic life as the Christian life, all other occupations being something less. Finally, the Roman Church taught that one could earn grace by becoming a monk or nun, and thus obscured the work of Christ. The Lutherans did not necessarily want to abolish the monasteries and convents, but wished to eliminate the vows of perpetual celibacy, poverty and obedience, whose introduction had given rise to the abuses listed.]

What is taught among us concerning monastic vows will be better understood if it is remembered what the state of the monasteries has been, and how many things were daily done in those very monasteries contrary to the Canons. In Augustine’s time they were voluntary associations. Afterward, when discipline declined, vows were everywhere added for the purpose of restoring discipline, as in a carefully planned prison. Gradually many other observances were added besides vows. And these fetters were laid upon many before a lawful age, contrary to the Canons. Many also entered into this kind of life through ignorance, being unable to judge their own strength, though they were of sufficient age. Being thus ensnared, they were compelled to remain even though some could have been freed by the provision of the Canons. And this was more the case in the convents of women than of monks, even though more consideration should have been shown the weaker sex. This rigor displeased many good men before now, who saw that many good young men and women were thrown into convents for a living, and what unfortunate results came from this procedure, what scandals were created, what snares were cast upon consciences! They were grieved that the authority of the Canons in so important a matter was utterly despised and set aside.

To these evils was added an opinion concerning vows, which, it is well known, in former times displeased even those monks who were more thoughtful. They taught that vows were equal to Baptism! They taught that by this kind of life they merited forgiveness of sins and justification before God. They even added that monastic life not only merited righteousness before God, but even greater things, because it kept not only the precepts, but also the so-called “evangelical counsels.”

Thus they made people believe that the profession of monasticism was far better than Baptism, and that the monastic life was more meritorious than that of magistrates, than the life of pastors and such who serve their calling in accordance with God’s commands, without man-made services. None of these things can be denied, for they appear in their own books.

What then happened in the monasteries? They once were schools of theology and of other branches beneficial to the Church, and from there pastors and bishops were obtained. Now it is another thing. It is needless to repeat what is known to all. They used to come together to learn; now they pretend that it is a kind of life instituted to merit grace and righteousness; indeed they even preach that it is a state of perfection and put it far above all other kinds of life ordained of God.

These things we have repeated without offensive exaggeration, in order that the doctrine of our teachers on this point might be better understood. First, concerning those who contract matrimony, they teach on our part that it is lawful for all who are not suited for the single life to marry, because vows cannot annul the ordinance and commandment of God. But the commandment of God is [1 Cor. 7:2]: “Because of sexual immorality, let each man have his own wife, and let every woman have her own husband.” Nor is it the commandment only, but also the creation and ordinance of God which compels those to marry who are not excepted by a singular work of God, according to the text [Gen. 2:18]: “It is not good that man should be alone.” Therefore they do not sin who obey this commandment and ordinance of God.

What objection can be raised to this? Let men extol the obligation of a vow as much as they wish, yet they shall not make it so that the vow annuls the commandment of God. The Canons teach that the right of the superior is excepted in every vow. Much less, therefore, are the vows in force that are against the commandments of God.

Now if the obligation of vows could not be changed for any cause whatever, the Roman Pontiffs could never have given dispensation; for it is not lawful for man to annul an obligation which is altogether divine. But the Roman Pontiffs have wisely judged that leniency is to be observed in this obligation, and therefore we read that many times they have released from vows. The case of the King of Aragon who was called back from the monastery is well known, and there are also examples in our own times.

In the second place, why do our adversaries exaggerate the obligation or effect of a vow, when at the same time they have not a word to say on the nature of the vow itself, that it ought to be a thing possible, voluntary, and chosen of one’s own accord and deliberately? But it is not known to what extent perpetual chastity is within human powers. And how few there are who have taken the vow of their own accord and deliberately! Young men and women, before they are able to judge, are persuaded and sometimes even compelled, to take the vow. For this reason it is not fair to insist so rigorously on the obligation, since it is granted by all that it is against the nature of a vow when it is not taken of one’s own accord or deliberately.

Many canonical laws rescind vows made before the age of fifteen, for before that age there does not seem sufficient judgment in a person to make a decision regarding the rest of his life. Another Canon, granting even more liberty to human weakness, adds a few years and forbids a vow to be made before the age of eighteen. But whether we followed the one or the other, most of them have an excuse for leaving the monasteries because most of them took their vows before they reached these ages.

But finally, even though the violation of a vow might be rebuked, yet it does not follow that the marriages of such persons ought to be dissolved. For Augustine denies that they ought to be dissolved in Nuptiarum, Question 27, chapter 1; and his authority is not to be regarded lightly, although men afterwards thought otherwise.

But although it appears that God’s command concerning marriage delivers many from their vows, yet our teachers introduce yet another argument concerning vows, to show that they are void. For every service of God that is ordained and chosen of men without the commandment of God to merit justification and grace is wicked, as Christ says [Matt. 15:9]: “In vain they worship Me, teaching as doctrines the commandments of men.” And Paul teaches everywhere that righteousness is not to be sought by our own observances and acts of worship, devised by us, but that it comes by faith to those who believe that they are received by God into grace for Christ’s sake.

But it is evident that monks have taught that human services of our making satisfy for sins and merit grace and justification. What else is this but to detract from the glory of Christ and to obscure and deny the righteousness of faith? It follows, therefore, that the vows thus commonly taken have been wicked services, and consequently are void. For a wicked vow, taken against the commandment of God, is not valid. Just as the Canon says, no vow should bind a person to wickedness.

Paul says [Gal. 5:4]: “You have become estranged from Christ, you who attempt to be justified by the law; you have fallen from grace.” They, therefore, who want to be justified by their vows are severed from Christ and fall from grace. For those who ascribe justification to vows, ascribe to their own works that which properly belongs to the glory of Christ. But it is undeniable that the monks have taught us that by their vows and observances they were justified and merited forgiveness of sin. Indeed, they invented still greater absurdities, saying that they could give others a share of their works. If anyone should be inclined to expand on those things with evil intent, how many things could he bring together, of which things even the monks are now ashamed! Over and above this, they persuaded people that services of human origin were a state of Christian perfection. Is not this assigning justification to works? It is no light offense in the Church to set forth to the people a service of human origin, without the commandment of God, and to teach that such service justifies. For the righteousness of faith in Christ, which ought to be first in the Church, is obscured when these wonderful angelic forms of worship, with their show of poverty, humility and chastity are cast before the people’s eyes.

Furthermore, the precepts of God and the true service of God are obscured when people hear that only monks are in a state of perfection. For Christian perfection is to fear God from the heart, again to have great faith, to trust that for Christ’s sake we have a gracious God, and to ask of God and assuredly to expect His aid in all things that are to be borne according to our calling; and meanwhile to be diligent in outward good works and to serve our calling. In these things consist the true perfection and the true service of God. It does not consist in the unmarried life, or in begging, or in shabby apparel. But the people conceive many harmful opinions from the erroneous commendations of monastic life. They hear unmarried life praised above measure, therefore they lead their married life with offense to their consciences. They hear that only beggars are perfect, therefore they keep their possessions and do business with offense to their consciences. They hear that it is an evangelical counsel not to avenge, therefore some in private life are not afraid to take revenge, for they hear that it is only a counsel and not a commandment; while others believe that the Christian cannot rightfully hold a civil office or be a magistrate.

There are on record examples of men who, forsaking marriage and the administration of the State, have hid themselves in monasteries. They call this “fleeing from the world” and “seeking a life which should be more pleasing to God.” They did not see that God ought to be served in those commandments which He Himself has given, and not by commandments of human origin. A good and perfect kind of life is that which has the commandment of God in its favor. It is necessary to admonish people in these things. Before this, Gerson denounced this error concerning perfection, and testified that in his day it was a new thing to say that the monastic life is a state of perfection.

So many wicked opinions are inherent in the vows; such as that they justify, that they constitute Christian perfection, that they keep the counsels and commandments, and that they have works of supererogation. All these things, since they are false and empty, make vows null and void.

 

Article XXVIII

Ecclesiastical Power

[Note: By the time of the Reformation, the Roman Church had become a civil power. It was not unusual for the Pope or bishops to intervene in civil affairs to protect their interests. They had also allowed the traditions of the Church to become regulations, and they told the people they sinned when they did not observe them. The Lutherans insist on the distinction between the power of the Church and the power of the State, and that traditions not be allowed to obscure Christ or burden consciences. They then repeat the theme of Article XV, that we observe those traditions which may be observed without sin.]

There has been a great controversy concerning the power of bishops, in which some have unfortunately confused the power of the Church and the power of the sword. And from this confusion very great wars and tumults have resulted, while the Pontiffs, emboldened by the power of the Keys, have not only instituted new services and burdened consciences with reservation of cases[11] and harsh excommunications, but have undertaken to transfer kingdoms of this world and to take the Empire from the Emperor. These wrongs have long since been rebuked in the Church by learned and godly men. Therefore, our teachers, for the comforting of consciences, were constrained to show the difference between the power of the Church and the power of the sword, and taught that both of them, because of God’s commandment, are to be held in reverence and honor, as among the chief blessings of God on earth.

This is their opinion: that the power of the Keys, or the power of the bishops, according to the Gospel, is a power or commandment of God to preach the Gospel, to forgive and retain sins, and to administer the sacraments. For with that commandment Christ sends forth His Apostles [John 20:21ff]: “As the Father has sent Me, I also send you…. Receive the Holy Spirit. If you forgive the sins of any, they are forgiven them; if you retain the sins of any, they are retained.” [Mark 16:15]: “Go, preach the Gospel to every creature.”

According to the calling this power is exercised only by teaching or preaching the Gospel and administering the sacraments, either to many or to individuals. For thereby are granted not physical but eternal things, as eternal righteousness, the Holy Spirit, and eternal life. These things cannot come except by the ministry of the Word and sacraments. As Paul says [Rom. 1:16]: “The Gospel is the power of God to salvation for everyone who believes.” Therefore, since the power of the Church grants eternal things and is exercised only by the ministry of the Word, it does not interfere with civil government; no more than the art of singing interferes with civil government. For civil government deals with other things than does the Gospel. The civil rulers do not defend souls, but bodies and physical things against various injuries. It restrains with the sword and physical punishments in order to preserve civil justice and peace.

Therefore the power of the Church and the civil power must not be confused. The power of the Church has its own commission: to teach the Gospel and administer the sacraments. Let it not interfere with the office of another; let it not transfer the kingdoms of this world; let it not abrogate the laws of civil rulers; let it not abolish lawful obedience; let it not interfere with judgments concerning civil ordinances and contracts; let it not prescribe laws to civil rulers concerning the form of the State. As Christ says [John 18:36]: “My kingdom is not of this world”; also [Luke 12:14]: “Who made Me a judge or an arbitrator over you?” Paul also says [Phil. 3:20]: “Our citizenship is in heaven”; [2 Cor. 10:4]: “The weapons of our warfare are not carnal but mighty in God for casting down arguments.” In this way our teachers distinguish between the duties of both these powers, and command that both be honored and acknowledged as gifts and blessings of God.

If bishops have any power of the sword, they have that power not as bishops by the commission of the Gospel, but by human law, having received it of Kings and Emperors, for the civil administration is theirs. Yet this is a different office than the ministry of the Gospel.

So when a question arises concerning the jurisdiction of bishops, civil authority must be distinguished from ecclesiastical jurisdiction. Again, according to the Gospel (or as they say, according to Divine Law), “to the bishops as bishops,” that is, to those to whom has been committed the ministry of Word and sacraments, no jurisdiction belongs except to forgive sins, to discern doctrine, to reject doctrines contrary to the Gospel, and to exclude from the communion of the Church wicked people whose wickedness is known, and this without human force but simply by the Word. Herein the congregations are bound by Divine Law to obey them, according to Luke 10:16: “He who hears you hears Me.”

But when they teach or ordain anything against the Gospel, then the congregations have a commandment of God prohibiting obedience [Matt. 7:15]: “Beware of false prophets”; [Gal. 1:8]: “Even if an angel from heaven preach any other gospel, let him be accursed”; [2 Cor. 13:8]: “We can do nothing against the truth, but for the truth”; also [v. 10]: “The authority of the Lord has [been] given me for edification and not for destruction.” So also the Canonical Laws command (II, question 7, in chapters “Sacerdotes” and “Oves”). And Augustine writes in his Letter Against Petilian: “We must not even submit to the Catholic bishops if they happen to err or hold anything contrary to the Canonical Scriptures of God.”

If they have any other power of jurisdiction in hearing and judging cases, as of matrimony or tithes, they have it by human law. But where the bishops fail, princes are bound, even against their will, to dispense justice to their subjects for the maintenance of peace.

Moreover, it is disputed whether bishops or pastors have the right to introduce ceremonies in the Church and to make laws concerning foods, festivals and degrees, that is, orders of ministers, etc. They who claim this right for bishops refer to this passage [John 16:12–13]: “I still have many things to say to you, but you cannot bear them now. However, when He, the Spirit of truth, has come, He will guide you into all truth.” They also refer to the example of the Apostles who commanded to abstain from blood and things strangled in Acts 15:29. They refer to the Sabbath Day as having been changed to Sunday; seemingly contrary to the Ten Commandments. Indeed, there is no other example they make more of than the one concerning the Sabbath Day. Great, they say, is the power of the Church, since it has dispensed with one of the Ten Commandments!

But concerning this question it is taught by us (as has been shown above) that bishops have no power to decree anything against the Gospel. The Canonical Laws teach the same thing in Dist. 9. Now it is against Scripture to establish or require the observance of any traditions, so that by such observance we may make satisfaction for sins or merit grace and righteousness. For the glory of Christ’s merit is dishonored when by such observations we undertake to merit justification. Yet it is obvious that by such belief traditions have almost infinitely multiplied in the Church, the doctrine concerning faith and the righteousness of faith being meanwhile suppressed. For gradually more festivals were created, fasts appointed, new ceremonies and services in honor of saints were instituted; because the authors of these things thought that by these works they were meriting grace. Thus, in times past, the Penitential Canons increased, and we still see traces of them in the satisfactions.

Again, the authors of traditions do contrary to the command of God when they find matters of sin in food, in days and similar things, and burden the Church with the bondage of the law; as if we ought to find among Christians a Levitical-like service to merit justification that God committed to the Apostles and bishops. For thus some of them write, and the Pontiffs in some measure seem to be misled by the example of the Law of Moses. From this come burdens such as these: that it is a mortal sin to do work on festival days, even without offense to others, to omit the Canonical Hours, that certain foods corrupt the conscience, that fasting is a work which appeases God, that sin in a reserved case cannot be forgiven except by the authority of him who reserved it; whereas the Canons speak only of reserving the ecclesiastical penalty, not of reserving the guilt.

Where do the bishops get the right to lay these traditions on the Church for the ensnaring of consciences, when Peter [Acts 15:10] forbids that a yoke be put on the neck of the disciples, and Paul says [2 Cor. 13:10] that the authority given to him was for edification and not for destruction? Why do they increase sins by these traditions?

But there are clear testimonies that prohibit the making of such traditions as though they merited grace or were necessary to salvation. Paul says [Col. 2:16]: “Let no one judge you in food or in drink, or regarding a festival or a new moon or sabbaths;” also [v. 20–23] “If you died with Christ from the basic principles of this world, why, as though living in the world, do you subject yourself to regulations—‘Do not touch, do not taste, do not handle,’ which all concern things which perish with the using—according to the commandments and doctrines of men? These things indeed have an appearance of wisdom.” Also in Titus [1:14] he openly forbids traditions: “Not giving heed to Jewish fables and commandments of men who turn from the truth.” And Christ [Matt. 15:14] says of those who require traditions: “Let them alone. They are blind leaders of the blind”; and He condemns such works [v. 13]: “Every plant which My heavenly Father has not planted will be uprooted.”

If bishops have the right to burden churches with infinite traditions and to ensnare consciences, why does Scripture so often prohibit making and listening to traditions? Why does it call them a “doctrine of demons” [1 Tim. 4:1]? Did the Holy Spirit forewarn against these things in vain?

Therefore since ordinances instituted as being necessary or with the idea that they merit grace are contrary to the Gospel, it follows that it is not lawful for any bishop to institute or require such services. For it is necessary that the doctrine of Christian liberty be preserved in the churches; namely that bondage to law is not necessary to justification, as it is written in the Epistle to the Galatians [5:1]: “Do not be entangled again with a yoke of bondage.” It is necessary that the chief article of the Gospel be preserved: that we obtain grace freely by faith in Christ, and not because of certain observances or acts of worship devised by men.

What then, are we to think of Sunday and similar rites in the house of God? To this we answer that it is lawful for bishops and pastors to make ordinances so that things are done orderly in the Church, not so that we should merit grace or make satisfaction for sins by them, or that consciences be bound to think them necessary services and that it is a sin to break them without offense to others. Thus Paul ordains [1 Cor. 11:5] that women should cover their heads in the congregation; [1 Cor. 14:30] and that interpreters of Scripture be heard in an orderly way in the Church, etc.

It is proper that the Church keep such ordinances for the sake of charity and tranquility, so far that one does not offend another, and that all things be done in the churches in an orderly way, without confusion. Yet consciences are not to be burdened, thinking they are necessary to salvation, or that they sin when they break them without offense to others, as no one will say that a woman sins who goes out in public with her head uncovered, provided that no offense is given.

Likewise the observance of Sunday, Easter, Pentecost, and like festivals and rites. For they greatly err who judge that by the authority of the Church the observance of Sunday instead of the Sabbath Day was ordained as being necessary. Scripture has abrogated the Sabbath Day; for it teaches that since the Gospel has been revealed, all the ceremonies of Moses can be omitted. And yet because it was necessary to appoint a certain day so that the people would know when they might come together, it appears that the Church designated Sunday for this purpose. This day seems to have been chosen all the more for this additional reason: that people might have an example of Christian liberty and might know that neither the keeping of the Sabbath or any other day is necessary.

There are monstrous arguments concerning the changing of the law, the ceremonies of the new law, the changing of the Sabbath Day, all of which have sprung from the false belief that the Church must have a Levitical-like service, that Christ had commissioned the Apostles and bishops to devise new ceremonies as necessary to salvation. These errors crept into the Church when the righteousness of faith was not taught clearly enough. Some argue that while the keeping of Sunday is not actually a divine right, it is almost a divine right. They prescribe how much a person may work on a festival day. What else are such arguments but snares of conscience? For although they endeavor to modify the traditions, the moderation can never be achieved so long as the opinion remains that they are necessary, which must remain where the righteousness of faith and Christian liberty are cast aside.

The Apostles commanded to abstain from blood [Acts 15:20]. Who now observes this? And yet those who do not observe it do not sin, for not even the Apostles themselves wished to burden consciences with such bondage. They only forbade it for a time, to avoid offense. For in any decree we must consider the perpetual aim of the Gospel. Hardly any Canons are kept with exactness; from day to day many fall into disuse even with those who are the most zealous advocates of traditions. Neither can any counsel be given to consciences unless this moderation be observed: that we know the Canons are kept without holding them to be necessary, and that no harm is done to consciences even though traditions fall into disuse.

The bishops might easily retain the lawful obedience of the people if they would not insist upon the observance of those traditions which cannot be kept with a good conscience. But now they demand celibacy and will admit none unless they swear that they will not teach the pure doctrine of the Gospel. The churches do not ask that the bishops restore concord at the expense of their honor (which, nevertheless, it would be proper for good pastors to do). They only ask them to relieve unjust burdens which are new and have been allowed contrary to the custom of the Church Catholic. It may be that there were plausible reasons for some of these ordinances, and yet they are not adapted to later times. It is also evident that some were adopted because of erroneous conceptions. Therefore it would be befitting the clemency of the Pontiffs to mitigate them now, because such a modification does not shake the unity of the Church. For many human traditions have been changed with the passing of time, as the Canons themselves show. But if it be impossible to obtain a mitigation of those observances which cannot be kept without sin, we are bound to follow the Apostolic rule [Acts 5:29] which commands us to obey God rather than men. Peter [1 Pet. 5:3] forbids bishops to be lords and to rule over the churches. Nor is it our intent to wrest the government from the bishops, but rather to ask this one thing: that they allow the Gospel to be purely taught and that they relax a few observances which cannot be kept without sin. But if they make no concession, they will have to give account to God for having created a schism by their obstinacy.

 

CONCLUSION

These are the chief articles which seem to be in controversy. For although we might have spoken of more abuses, yet to avoid undue length we have set forth the chief points, from which the rest may be readily judged. There have been a great many complaints concerning indulgences, pilgrimages, and the abuses of excommunication. The parishes have been vexed in many ways by the dealers of indulgences. There were endless contentions between the pastors and the monks concerning the parochial rites, confessions, burials, sermons on special occasions and innumerable other things. Things of this sort we have passed over, so that the chief points in this matter, having been briefly set forth, might be the most readily understood. Nor has anything been said or presented here to show contempt of anyone. Only those things have been recounted which we thought necessary to say, so that it might be understood that in doctrine and in ceremonies, nothing has been allowed by us that is against Scripture or the Church Catholic, since it is obvious that we have been very careful that no new and ungodly doctrine should creep into our churches.

The above articles we desire to present in accordance with the edict of Your Imperial Majesty, so that in them our Confession should be shown forth and a summary of the doctrine of our teachers be discerned. If anything further be desired, we are ready, God willing, to present ampler information according to the Scriptures.

John, Duke of Saxony, Elector

George, Margrave of Brandenburg

Ernest, Duke of Lüneburg

Philip, Landgrave of Hesse

John Frederick, Duke of Saxony

Francis, Duke of Lüneburg

Wolfgang, Prince of Anhalt

Senate and Magistracy of Nuremburg

Senate of Reutlingen

 
