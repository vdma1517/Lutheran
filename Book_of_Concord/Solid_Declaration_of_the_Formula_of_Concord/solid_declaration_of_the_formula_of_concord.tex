Preface

1 When, by the special grace and mercy of the Almighty, the doctrine concerning the chief articles of our Christian religion (which under the Papacy had been horribly obscured by human teachings and ordinances) had been explained and purified again from [in accordance with the direction and analogy of] God’s Word by Dr. Luther, of blessed and holy memory, and the papistic errors, abuses, and idolatries had been rebuked;

2 and this pure reformation was nevertheless regarded by its opponents as [introducing] a new doctrine and was violently (though without foundation) charged with being entirely contrary to God’s Word and the Christian ordinances, and, in addition, was loaded with [almost endless] unsupportable calumnies and accusations,

3 the Christian [the most illustrious and in religious piety most prominent] Electors and Princes, and the Estates [of the Empire] which at that time had embraced the pure doctrine of the Holy Gospel and had their churches reformed in a Christian manner according to God’s Word, had a Christian Confession prepared from God’s Word at the great Diet of Augsburg in the year 1530 and delivered it to the Emperor Charles V. In this they clearly and plainly made their Christian confession as to what was being held and taught in the Christian evangelical churches concerning the chief articles, especially those in controversy between them and the Papists; and although this Confession was received with disfavor by their opponents, still, thank God, it remains to this day unrefuted and unoverthrown.

4 To this Christian [pious] Augsburg Confession, so thoroughly grounded in God’s Word, we herewith pledge ourselves again [publicly and solemnly] from our inmost hearts; we abide by its simple, clear, and unadulterated meaning as the words convey it, and regard the said Confession as a pure Christian symbol, with which at the present time true Christians ought to be found next to [which pious hearts ought to receive next to the matchless authority of] God’s Word; just as in former times concerning certain great controversies that had arisen in the Church of God, symbols and confessions were proposed, to which the pure teachers and hearers at that time pledged themselves with heart and mouth.

5 We intend also, by the grace of the Almighty, faithfully to abide until our end by [the doctrine of] this Christian Confession, mentioned several times, as it was delivered in the year 1530 to the Emperor Charles V; and it is our purpose, neither in this nor in any other writing, to recede in the least from that oft-cited Confession, nor to propose another or new confession.

6 Now, although the Christian doctrine of this Confession has in great part remained unchallenged (save what has been done by the Papists), yet it cannot be denied that some theologians have departed from some great [principal] and important articles of the said Confession, and either have not attained to their true meaning, or at any rate have not continued steadfastly therein, and occasionally [some] have even undertaken to attach to it a foreign meaning, while at the same time they wished to be regarded as adherents of [they professed to embrace] the Augsburg Confession, and to avail themselves and make their boast of it [for a pretext].

7 From this, grievous and injurious dissensions have arisen in the pure evangelical churches; just as even during the lives of the holy apostles among those who wished to be called Christians, and boasted of Christ’s doctrine, horrible errors arose likewise. For some sought to be justified and saved by the works of the Law, Acts 15, 1-029, others denied the resurrection of the dead, 1 Cor. 15, 12, and still others did not believe that Christ was true and eternal God. Against these the holy apostles had to inveigh strenuously in their sermons and writings, although [they were well aware that] also at that time such fundamental errors and severe controversies could not occur without offense both to unbelievers and to those weak in the faith.

8 In a similar manner at present our opponents, the Papists, rejoice at the dissensions that have arisen among us, in the unchristian and vain hope that these discords might finally cause the suppression of the pure doctrine, while those who are weak in faith are [greatly] offended [and disturbed], and some of them doubt whether, amid such dissensions, the pure doctrine is with us, and others do not know with whom to side with respect to the articles in controversy.

9 For the controversies which have occurred are not, as some would regard them, mere misunderstandings or disputes concerning words [as are apt to occur], one side not having sufficiently grasped the meaning of the other, and the difficulty lying thus in a few words which are not of great moment; but here the subjects of controversy are important and great, and of such a nature that the opinion of the party in error cannot be tolerated in the Church of God, much less be excused or defended.
10 Necessity, therefore, requires us to explain these controverted articles according to God’s Word and approved writings, so that every one who has Christian understanding can notice which opinion concerning the matters in controversy accords with God’s Word and the Christian Augsburg Confession, and which does not. And sincere Christians who have the truth at heart may guard and protect themselves against [flee and avoid] the errors and corruptions that have arisen.

Comprehension Summary, Foundation, Rule, and Norm

Whereby All Dogmas should be Judged according to God’s Word, and the Controversies that have Occurred should be Explained and Decided in a Christian Manner.

1 Since for thorough, permanent unity in the Church it is, above all things, necessary that we have a comprehensive, unanimously approved summary and form wherein is brought together from God’s Word the common doctrine, reduced to a brief compass, which the churches that are of the true Christian religion confess, just as the ancient Church always had for this use its fixed symbols;

2 moreover, since this [comprehensive form of doctrine] should not be based on private writings, but on such books as have been composed, approved, and received in the name of the churches which pledge themselves to one doctrine and religion, we have declared to one another with heart and mouth that we will not make or receive a separate or new confession of our faith, but confess the public common writings which always and everywhere were held and used as such symbols or common confessions in all the churches of the Augsburg Confession before the dissensions arose among those who accept the Augsburg Confession, and as long as in all articles there was on all sides a unanimous adherence to [and maintenance and use of] the pure doctrine of the divine Word, as the sainted Dr. Luther explained it.

3 1. First [, then, we receive and embrace with our whole heart] the Prophetic and Apostolic Scriptures of the Old and New Testaments as the pure, clear fountain of Israel, which is the only true standard by which all teachers and doctrines are to be judged.

4 2. And since of old the true Christian doctrine, in a pure, sound sense, was collected from God’s Word into brief articles or chapters against the corruption of heretics, we confess, in the second place, the three Ecumenical Creeds, namely, the Apostles’, the Nicene, and the Athanasian, as glorious confessions of the faith, brief, devout, and founded upon God’s Word, in which all the heresies which at that time had arisen in the Christian Church are clearly and unanswerably refuted.

5 3. In the third place, since in these last times God, out of especial grace, has brought the truth of His Word to light again from the darkness of the Papacy through the faithful service of the precious man of God, Dr. Luther, and since this doctrine has been collected from, and according to, God’s Word into the articles and chapters of the Augsburg Confession against the corruptions of the Papacy and also of other sects, we confess also the First, Unaltered Augsburg Confession as our symbol for this time, not because it was composed by our theologians, but because it has been taken from God’s Word and is founded firmly and well therein, precisely in the form in which it was committed to writing, in the year 1530, and presented to the Emperor Charles V at Augsburg by some Christian Electors, Princes, and Estates of the Roman Empire as a common confession of the reformed churches, whereby our reformed churches are distinguished from the Papists and other repudiated and condemned sects and heresies, after the custom and usage of the early Church, whereby succeeding councils, Christian bishops and teachers appealed to the Nicene Creed, and confessed it [publicly declared that they embraced it].

6 4. In the fourth place, as regards the proper and true sense of the oft-quoted Augsburg Confession, an extensive Apology was composed and published in print in 1531, after the presentation of the Confession, in order that we might explain ourselves at greater length and guard against the [slanders of the] Papists, and that condemned errors might not steal into the Church of God under the name of the Augsburg Confession, or dare to seek cover under the same. We unanimously confess this also, because not only is the said Augsburg Confession explained as much as is necessary and guarded [against the slanders of the adversaries], but also proven [confirmed] by clear, irrefutable testimonies of Holy Scripture.

7 5. In the fifth place, we also confess the Articles composed, approved, and received at Smalcald in the large assembly of theologians, in the year 1537, as they were first framed and printed in order to be delivered in the council at Mantua, or wherever it would be held, in the name of the Estates, Electors, and Princes, as an explanation of the above-mentioned Augsburg Confession, wherein by God’s grace they were resolved to abide. In them the doctrine of the Augsburg Confession is repeated, and some articles are explained at greater length from God’s Word, and, besides, the cause and grounds are indicated, as far as necessary, why we have abandoned the papistical errors and idolatries, and can have no fellowship with them, and also why we know, and can think of, no way for coming to any agreement with the Pope concerning them.

8 6. And now, in the sixth place, because these highly important matters [the business of religion] concern also the common people and laymen [as they are called], who, inasmuch as they are Christians, must for their salvation distinguish between pure and false doctrine, we confess also the Small and the Large Catechisms of Dr. Luther, as they were written by him and incorporated in his works, because they have been unanimously approved and received by all churches adhering to the Augsburg Confession, and have been publicly used in churches, schools, and in [private] houses, and, moreover, because the Christian doctrine from God’s Word is comprised in them in the most correct and simple way, and, in like manner, is explained, as far as necessary [for simple laymen].

9 In the pure churches and schools these public common writings have been always regarded as the sum and model of the doctrine which Dr. Luther, of blessed memory, has admirably deduced from God’s Word, and firmly established against the Papacy and other sects; and to his full explanations in his doctrinal and polemical writings we wish to appeal, in the manner and as far as Dr. Luther himself in the Latin preface to his published works has given necessary and Christian admonition concerning his writings, and has expressly drawn this distinction namely, that the Word of God alone should be and remain the only standard and rule of doctrine, to which the writings of no man should be regarded as equal, but to which everything should be subjected.

10 But [this is not to be understood as if] hereby other good, useful, pure books, expositions of the Holy Scriptures, refutations of errors, explanations of doctrinal articles, are not rejected; for as far as they are consistent with the above-mentioned type of doctrine, these are regarded as useful expositions and explanations, and can be used with advantage. But what has thus far been said concerning the summary of our Christian doctrine is intended to mean only this, that we should have a unanimously accepted, definite, common form of doctrine, which all our evangelical churches together and in common confess, from and according to which, because it has been derived from God’s Word, all other writings should be judged and adjusted as to how far they are to be approved and accepted.

11 For that we embodied the above-mentioned writing, namely, the Augsburg Confession, Apology, Smalcald Articles, Luther’s Large and Small Catechisms, in the oft-mentioned Sum of our Christian doctrine, was done for the reason that these have always and everywhere been regarded as the common, unanimously accepted meaning of our churches, and, moreover, have been subscribed at that time by the chief and most enlightened theologians, and have held sway in all evangelical churches and schools.

12 So also, as before mentioned, they were all written and sent forth before the divisions among the theologians of the Augsburg Confession arose; therefore, since they are held to be impartial, and neither can nor should be rejected by either part of those who have entered into controversy, and no one who without guile is an adherent of the Augsburg Confession will complain of these writings, but will cheerfully accept and tolerate them as witnesses [of the truth], no one can think ill of [blame] us that we derive from them an explanation and decision of the articles in controversy,

13 and that, as we lay down God’s Word, the eternal truth, as the foundation, so we introduce and quote also these writings as a witness of the truth and as the unanimously received correct understanding of our predecessors who have steadfastly held to the pure doctrine.
Articles in Controversy with Respect to the Antithesis, or Opposite Doctrine.

14 Moreover, since for the preservation of pure doctrine and for thorough, permanent, godly unity in the Church it is necessary, not only that the pure, wholesome doctrine be rightly presented, but also that the opponents who teach otherwise be reproved, 1 Tim. 3 (2 Tim. 3:16); Titus 1:9, — for faithful shepherds, as Luther says, should do both, namely, feed or nourish the lambs and resist the wolves, so that the sheep may flee from strange voices, John 10:12, and may separate the precious from the vile, Jer. 15:19, —

15 Therefore we have thoroughly and clearly declared ourselves to one another, also regarding these matters, as follows: that a distinction should and must by all means be observed between unnecessary and useless wrangling, on the one hand, whereby the Church ought not to be disturbed, since it destroys more than it builds up, and necessary controversy, on the other hand, as, when such a controversy occurs as involves the articles of faith or the chief heads of the Christian doctrine, where for the defense of the truth the false opposite doctrine must be reproved.

16 Now, although the aforesaid writings afford the Christian reader, who delights in and has a love for the divine truth, clear and correct information concerning each and every controverted article of our Christian religion, as to what he should regard and receive as right and true according to God’s Word of the Prophetic and Apostolic Scriptures, and what he should reject, shun, and avoid as false and wrong; yet, in order that the truth may be preserved the more distinctly and clearly, and be distinguished from all errors, and that nothing be hidden and concealed under ordinary terms [rather general words and phrases], we have clearly and expressly declared ourselves to one another concerning the chief and most important articles taken one by one, which at the present time have come into controversy, so that there might be a public, definite testimony, not only for those now living, but also for our posterity, what is and should remain the unanimous understanding and judgment [decision] of our churches in reference to the articles in controversy, namely:

17 1. First, that we reject and condemn all heresies and errors which were rejected and condemned in the primitive, ancient, orthodox Church, upon the true, firm ground of the holy divine Scriptures.

18 2. Secondly, we reject and condemn all sects and heresies which are rejected in the writings, just mentioned, of the comprehensive summary of the Confession of our churches.

19 3. Thirdly, since within thirty years some divisions arose among some theologians of the Augsburg Confession on account of the Interim and otherwise, it has been our purpose to state and declare plainly [categorically], purely, and clearly our faith and confession concerning each and every one of these in thesis and antithesis, i. e., the true doctrine and its opposite, in order that the foundation of divine truth might be manifest in all articles, and that all unlawful, doubtful, suspicious, and condemned doctrines, wherever and in whatever books they may be found, and whoever may have written them, or even now may be disposed to defend them, might be exposed [distinctly repudiated],
20 so that every one may be faithfully warned against the errors, which are spread here and there in the writings of some theologians, and no one be misled in this matter by the reputation [authority] of any man. From this declaration the Christian reader will inform himself in every emergency, and compare it with the writings enumerated above, and he will find out exactly that what was confessed in the beginning concerning each article in the comprehensive summary of our religion and faith, and what was afterward restated at various times, and is repeated by us in this document, is in no way contradictory, but the simple, immutable, permanent truth, and that we, therefore, do not change from one doctrine to another, as our adversaries falsely assert, but earnestly desire to be found loyal to the once-delivered Augsburg Confession and its unanimously accepted Christian sense, and through God’s grace to abide thereby firmly and constantly in opposition to all corruptions which have entered.

I. Original Sin

1 And, to begin with, a controversy has occurred among some theologians of the Augsburg Confession concerning Original Sin, what it properly [and really] is. For one side contended that, since through the fall of Adam man’s nature, and essence are entirely corrupt, the nature, substance, and essence of the corrupt, man, now, since the Fall, or, at any rate, the principal, highest part of his essence, namely, the rational soul in its highest state or principal powers is original sin itself, which has been called nature-sin or person-sin, for the reason that it is not a thought, word, or work, but the nature itself whence, as from a root, spring all other sins, and that on this account there is now, since the Fall, because the nature is corrupt through sin, no difference whatever between the nature and essence of man and original sin.

2 But the other side taught, in opposition, that original sin is not properly the nature, substance, or essence of man, that is, man’s body or soul, which even now, since the Fall, are and remain the creation and creatures of God in us, but that it is something in the nature, body, and soul of man, and in all his powers, namely, a horrible, deep, inexpressible corruption of the same, so that man is destitute of the righteousness wherein he was originally created, and in spiritual things is dead to good and perverted to all evil; and that, because of this corruption and inborn sin, which inheres in the nature, all actual sins flow forth from the heart; and that hence a distinction must be maintained between the nature and essence of the corrupt man, or his body and soul, which are the creation and creatures of God in us even since the Fall, and original sin, which is a work of the devil, by which the nature has become corrupt.

3 Now this controversy concerning original sin is not unnecessary wrangling, but if this doctrine is rightly presented from, and according to, God’s Word, and separated from all Pelagian and Manichean errors, then (as the Apology says) the benefits of the Lord Christ and His precious merit, also the gracious operation of the Holy Ghost, are the better known and the more extolled; moreover, due honor is rendered to God, if His work and creation in man is rightly distinguished from the work of the devil, by which the nature has been corrupted.

4 In order, therefore, to explain this controversy in the Christian way and according to God’s Word, and to maintain the correct, pure doctrine of original sin, we shall collect from the above-mentioned writings the thesis and antithesis, that is, the correct doctrine and its opposite, into brief chapters.

5 1. And first, it is true that Christians should regard and recognize as sin not only the actual transgression of God’s commandments; but also that the horrible, dreadful hereditary malady by which the entire nature is corrupted should above all things be regarded and recognized as sin indeed, yea, as the chief sin, which is a root and fountain-head of all actual sins.

6 And by Dr. Luther it is called a nature-sin or person-sin, thereby to indicate that, even though a person would think, speak, or do nothing evil (which, however, is impossible in this life, since the fall of our first parents), his nature and person are nevertheless sinful, that is, thoroughly and utterly infected and corrupted before God by original sin, as by a spiritual leprosy; and on account of this corruption and because of the fall of the first man the nature or person is accused or condemned by God’s Law, so that we are by nature the children of wrath, death, and damnation, unless we are delivered therefrom by the merit of Christ.

7 2. In the second place, this, too, is clear and true, as the Nineteenth Article of the Augsburg Confession teaches, that God is not a creator, author, or cause of sin, but by the instigation of the devil through one man sin (which is a work of the devil) has entered the world, Rom. 5, 12; 1 John 3, 7. And even at the present day, in this corruption [in this corruption of nature], God does not create and make sin in us, but with the nature which God at the present day still creates and makes in men original sin is propagated from sinful seed, through carnal conception and birth from father and mother.

8 3. In the third place, what [and how great] this hereditary evil is no reason knows and understands, but, as the Smalcald Articles say, it must be learned and believed from the revelation of Scripture. And in the Apology this is briefly comprehended under the following main heads:

9 1. That this hereditary evil is the guilt [by which it comes to pass] that, by reason of the disobedience of Adam and Eve, we are all in God’s displeasure, and by nature children of wrath, as the apostle shows Rom. 5:12ff ; Eph. 2:3.

10 2. Secondly, that it is an entire want or lack of the concreated hereditary righteousness in Paradise, or of God’s image, according to which man was originally created in truth, holiness, and righteousness; and at the same time an inability and unfitness for all the things of God, or, as the Latin words read: Desciptio peccati originalis detrahit naturae non renovatae et dona et vim seu facultatem et actus inchoandi et efficiendi spiritualia; that is: The definition of original sin takes away from the unrenewed nature the gifts, the power, and all activity for beginning and effecting anything in spiritual things.

11 3. That original sin (in human nature) is not only this entire absence of all good in spiritual, divine things, but that, instead of the lost image of God in man, it is at the same time also a deep, wicked, horrible, fathomless, inscrutable, and unspeakable corruption of the entire nature and all its powers, especially of the highest, principal powers of the soul in the understanding, heart, and will, so that now, since the Fall, man inherits an inborn wicked disposition and inward impurity of heart, evil lust and propensity;

12 that we all by disposition and nature inherit from Adam such a heart, feeling, and thought as are, according to their highest powers and the light of reason, naturally inclined and disposed directly contrary to God and His chief commandments, yea, that they are enmity against God, especially as regards divine and spiritual things. For in other respects, as regards natural, external things which are subject to reason, man still has to a certain degree understanding, power, and ability, although very much weakened, all of which, however, has been so infected and contaminated by original sin that before God it is of no use.

13 4. The punishment and penalty of original sin, which God has imposed upon the children of Adam and upon original sin, are death, eternal damnation, and also other bodily and spiritual, temporal and eternal miseries, and the tyranny and dominion of the devil, so that human nature is subject to the kingdom of the devil and has been surrendered to the power of the devil, and is held captive under his sway, who stupefies [fascinates] and leads astray many a great, learned man in the world by means of dreadful error, heresy, and other blindness, and otherwise rushes men into all sorts of crime.

14 5. Fifthly, this hereditary evil is so great and horrible that only for the sake of the Lord Christ it can be covered and forgiven before God in the baptized and believing. Moreover, human nature, which is perverted and corrupted thereby, must and can be healed only by the regeneration and renewal of the Holy Ghost, which, however, is only begun in this life, but will not be perfect until in the life to come.

15 These points, which have been quoted here only in a summary way, are set forth more fully in the above-mentioned writings of the common confession of our Christian doctrine.

16 Now this doctrine must be so maintained and guarded that it may not deflect either to the Pelagian or the Manichean side. For this reason the contrary doctrine concerning this article, which is censured and rejected in our churches, should also be briefly stated.

17 1. And first, in opposition to the old and the new Pelagians, the following false opinions and dogmas are censured and rejected, namely, that original sin is only a reatus or guilt, on account of what has been committed by another, without any corruption of our nature.

18 2. Also, that sinful, evil lusts are not sins, but conditiones, or concreated and essential properties of the nature.

19 3. Or as though the above-mentioned defect and evil were not properly and truly sin before God, on account of which man without Christ [unless he be grafted into Christ and be delivered through Him] must be a child of wrath and damnation, also in the dominion and beneath the power of Satan.

20 4. The following and similar Pelagian errors are also censured and rejected, namely: that nature, even since the Fall, is said to be incorrupt, and that especially with respect to spiritual things entirely good and pure, and in naturalibus, that is, in its natural powers, it is said to be perfect.

21 5. Or that original sin is only external, a slight, insignificant spot sprinkled or a stain dashed upon the nature of man, or corruptio tantum accidentium aut qualitatum, i. e., a corruption only in some accidental things, along with and beneath which the nature nevertheless possesses and retains its integrity and power even in spiritual things.

22 6. Or that original sin is not a despoliation or deficiency, but only an external impediment to these spiritual good powers, as when a magnet is smeared with garlic-juice, whereby its natural power is not removed, but only hindered; or that this stain can be easily washed away, as a spot from the face or pigment from the wall.

23 7. They are rebuked and rejected likewise who teach that the nature has indeed been greatly weakened and corrupted through the Fall, but that nevertheless it has not entirely lost all good with respect to divine, spiritual things, and that what is sung in our churches, Through Adam’s fall is all corrupt, Nature and essence human, is not true, but from natural birth it still has something good, small, little and inconsiderable though it be, namely, capacity, skill, aptness or ability to begin, to effect, or to help effect something in spiritual things.

24 For concerning external, temporal, worldly things and transactions, which are subject to reason, there will be an explanation in the succeeding article.

25 These and contrary doctrines of like kind are censured and rejected for the reason that God’s Word teaches that the corrupt nature, of and by itself, has no power for anything good in spiritual, divine things, not even for the least, as good thoughts; and not only this, but that of and by itself it can do nothing in the sight of God but sin, Gen. 6:5; 8:21.

26 In the same manner this doctrine must also be guarded on the other side against Manichean errors. Accordingly, the following and similar erroneous doctrines are rejected, namely: that now, since the Fall, human nature is in the beginning created pure and good, and that afterwards original sin from without is infused and mingled with the nature by Satan (as something essential), as poison is mingled with wine [that in the beginning human nature was created by God pure and good, but that now, since the Fall, original sin, etc. ].

27 For although in Adam and Eve the nature was originally created pure, good, and holy, nevertheless sin did not enter their nature through the Fall in the way fanatically taught by the Manicheans, as though Satan had created or made some evil substance, and mingled it with their nature. But since man, by the seduction of Satan through the Fall, has lost his concreated hereditary righteousness according to God’s judgment and sentence, as a punishment, human nature, as has been said above, is so perverted and corrupted by this deprivation or deficiency, want, and injury, which has been caused by Satan, that at present the nature is transmitted, together with this defect and corruption [propagated in a hereditary way], to all men, who are conceived and born in a natural way from father and mother.

28 For since the Fall human nature is not at first created pure and good, and only afterward corrupted by original sin, but in the first moment of our conception the seed from which man is formed is sinful and corrupt. Moreover, original sin is not something by itself, existing independently in, or apart from, the nature of the corrupt man, as it neither is the real essence, body, or soul of the corrupt man, or the man himself.

29 Nor can and should original sin and the nature of man corrupted thereby be so distinguished as though the nature were pure, good, holy, and uncorrupted before God, while original sin alone which dwells therein were evil.

30 Also, as Augustine writes concerning the Manicheans, as though it were not the corrupt man himself that sins by reason of inborn original sin, but something different and foreign in man, and that God, accordingly, accuses and condemns by the Law, not the nature as corrupt by sin, but only the original sin therein. For, as stated above in thesi, that is, in the explanation of the pure doctrine concerning original sin, the entire nature of man, which is born in the natural way of father and mother, is entirely and to the farthest extent corrupted and perverted by original sin, in body and soul, in all its powers, as regards and concerns the goodness, truth, holiness, and righteousness concreated with it in Paradise. Non tamen in aliam substantiam genere aut specie diversam, priori abolita, transmutata est, that is: Nevertheless the nature is not entirely exterminated or changed into another substance, which, according to its essence, could not be said to be like our nature [but is diverse in genus or species], and therefore cannot be of one essence with us.

31 Because of this corruption, too, the entire corrupt nature of man is accused and condemned by the Law, unless the sin is forgiven for Christ’s sake.

32 But the Law accuses and condemns our nature, not because we have been created men by God, but because we are sinful and wicked; not because and so far as nature and its essence, even since the Fall, is a work and creature of God in us, but because and so far as it has been poisoned and corrupted by sin.

33 But although original sin, like a spiritual poison and leprosy (as Luther says), has poisoned and corrupted the whole human nature, so that we cannot show and point out to the eye the nature apart by itself, and original sin apart by itself, nevertheless the corrupt nature, or essence of the corrupt man, body and soul, or the man himself whom God has created (and in whom dwells original sin, which also corrupts the nature, essence, or the entire man), and original sin, which dwells in man’s nature or essence, and corrupts it, are not one thing; as also in external leprosy the body which is leprous, and the leprosy on or in the body, are not, properly speaking, one thing. But a distinction must be maintained also between our nature as created and preserved by God, in which sin is indwelling, and original sin, which dwells in the nature. These two must and also can be considered, taught, and believed separately according to Holy Scripture.

34 Moreover, the chief articles of our Christian faith urge and compel us to preserve this distinction. For instance, in the first place, in the article of Creation, Scripture testifies that God has created human nature not only before the Fall, but that it is a creature and work of God also since the Fall, Deut. 32:6; Is. 45:11, 54:5, 64:8; Acts 17:25; Rev. 4:11.

35 Thine hands, says Job, have made me and fashioned me together round about; yet Thou dost destroy me. Remember, I beseech Thee, that Thou hast made me as the clay; and wilt Thou bring me into dust again? Hast Thou not poured me out as milk and curdled me as cheese? Thou hast clothed me with skin and flesh, and fenced me with bones and sinews. Thou hast granted me life and favor, and Thy visitation hath preserved my spirit. Job 10:8-012.

36 I will praise Thee, says David, for I am fearfully and wonderfully made. Marvelous are Thy works, and that my soul knoweth right well. My substance was not hid from Thee when I was made in secret and curiously wrought in the lowest parts of the earth. Thine eyes did see my substance yet being unperfect, and in Thy book all my members were written which in continuance were fashioned, when as yet there was none of them, Ps. 139:14-016.

37 In the Ecclesiastes of Solomon it is written: Then shall the dust return to the earth as it was, and the spirit to God, who gave it, Eccl. 12:7.

38 These passages clearly testify that God even since the Fall is the Creator of man, and creates his body and soul. Therefore corrupt man cannot, without any distinction, be sin itself, otherwise God would be a creator of sin; as also our Small Catechism confesses in the explanation of the First Article, where it is written: I believe that God has made me and all creatures, that He has given me my body and soul, eyes, ears, and all my members, my reason and all my senses, and still preserves them. Likewise in the Large Catechism it is thus written: This is what I believe and mean, that is, that I am a creature of God; that He has given and constantly preserves to me my body, soul, and life, members great and small, and all my senses, mind, and reason. Nevertheless, this same creature and work of God is lamentably corrupted by sin; for the mass (massa) from which God now forms and makes man was corrupted and perverted in Adam, and is thus transmitted by inheritance to us.

39 And here pious Christian hearts justly ought to consider the unspeakable goodness of God, that God does not immediately cast from Himself into hell-fire this corrupt, perverted, sinful mass, but forms and makes from it the present human nature, which is lamentably corrupted by sin, in order that He may cleanse it from all sin, sanctify and save it by His dear Son.

40 From this article, now, the distinction is found indisputably and clearly. For original sin does not come from God. God is not a creator or author of sin. Nor is original sin a creature or work of God, but it is a work of the devil.

41 Now, if there were to be no difference whatever between the nature or essence of our body and soul, which is corrupted by original sin, and original sin, by which the nature is corrupted, it would follow either that God, because He is the Creator of this our nature, also created and made original sin, which, accordingly would also be His work and creature; or, because sin is a work of the devil, that Satan would be the creator of this our nature, of our body and soul, which would also have to be a work or creation of Satan if, without any distinction, our corrupt nature should have to be regarded as sin itself; both of which teachings are contrary to the article of our Christian faith.

42 Therefore, in order that God’s creation and work in man may be distinguished from the work of the devil, we say that it is God’s creation that man has body and soul; also, that it is God’s work that man can think, speak, do, and work anything; for in Him we live, and move, and have our being, Acts 17:28. But that the nature is corrupt, that its thoughts, words, and works are wicked, is originally a work of Satan, who has thus corrupted God’s work in Adam through sin, which from him is transmitted by inheritance to us.

43 Secondly, in the article of Redemption the Scriptures testify forcibly that God’s Son assumed our human nature without sin, so that He was in all things, sin excepted, made like unto us, His brethren, Heb. 2:14. Unde veteres dixerunt: Christum nobis, fratribus suis, consubstantialem esse secundum assumptam naturam, quia naturam, quae, excepto peccato, eiusdem generis, speciei et substantiae cum nostra est, assumpsit; et contrariam sententiam manifeste haereseos damnarunt. That is: Hence all the old orthodox teachers have maintained that Christ, according to His assumed humanity, is of one essence with us, His brethren; for He has assumed His human nature, which in all respects (sin alone excepted) is like our human nature in its essence and all essential attributes; and they have condemned the contrary doctrine as manifest heresy.

44 Now, if there were no distinction between the nature or essence of corrupt man and original sin, it must follow that Christ either did not assume our nature, because He did not assume sin, or that, because He assumed our nature, He also assumed sin; both of which ideas are contrary to the Scriptures. But inasmuch as the Son of God assumed our nature, and not original sin, it is clear from this fact that human nature, even since the Fall, and original sin, are not one [and the same] thing, but must be distinguished.

45 Thirdly, in the article of Sanctification Scripture testifies that God cleanses, washes, and sanctifies man from sin, 1 John 1:7, and that Christ saves His people from their sins, Matt. 1:21. Sin, therefore, cannot be man himself; for God receives man into grace for Christ’s sake, but to sin He remains hostile to eternity. Therefore it is unchristian and horrible to hear that original sin is baptized in the name of the Holy Trinity, sanctified and saved, and other similar expressions found in the writings of the recent Manicheans, with which we will not offend simple-minded people.

46 Fourthly, in the article of the Resurrection Scripture testifies that precisely the substance of this our flesh, but without sin, will rise again, and that in eternal life we shall have and retain precisely this soul, but without sin.

47 Now, if there were no difference whatever between our corrupt body and soul and original sin, it would follow, contrary to this article of the Christian faith, either that this our flesh will not rise again at the last day, and that in eternal life we shall not have the present essence of our body and soul, but another substance (or another soul), because then we shall be without sin; or that [at the last day] sin also will rise again, and will be and remain in the elect in eternal life.

48 Hence it is clear that this doctrine [of the Manicheans] (with all that depends upon it and follows from it) must be rejected, when it is asserted and taught that original sin is the nature, substance, essence, body, or soul itself of corrupt man, so that between our corrupt nature, substance, and essence and original sin there is no distinction whatever. For the chief articles of our Christian faith forcibly and emphatically testify why a distinction should and must be maintained between man’s nature or substance, which is corrupted by sin, and the sin, with which and by which man is corrupted.

49 For a simple statement of the doctrine and the contrary teaching (in thesi et antithesi) in this controversy, as regards the principal matter itself, is sufficient in this place, where the subject is not argued at length, but only the principal points are treated, article by article.

50 But as regards terms and expressions, it is best and safest to use and retain the form of sound words employed concerning this article in the Holy Scriptures and the above-mentioned books.

51 Also, to avoid strife about words, aequivocationes vocabulorum, that is, words and expressions which are applied and used in various meanings, should be carefully and distinctly explained; as when it is said: God creates the nature of men, there by the term nature the essence, body, and soul of men are understood. But often the disposition or vicious quality of a thing is called its nature, as when it is said: It is the nature of the serpent to bite and poison. Thus Luther says that sin and sinning are the disposition and nature of corrupt man.

52 Therefore original sin properly signifies the deep corruption of our nature, as it is described in the Smalcald Articles. But sometimes the concrete person or the subject, that is, man himself with body and soul, in which sin is and inheres, is also comprised under this term, for the reason that man is corrupted by sin, poisoned and sinful, as when Luther says: “Thy birth, thy nature, and thy entire essence is sin,” that is, sinful and unclean.

53 Luther himself explains that by nature-sin, person-sin, essential sin he means that not only the words, thoughts, and works are sin, but that the entire nature, person, and essence of man are altogether corrupted from the root by original sin.

54 However, as to the Latin words substantia and accidens, a church of plain people ought to be spared these terms in public sermons, because they are unknown to ordinary men. But when learned men among themselves, or with others to whom these words are not unknown, employ such terms in treating this subject, as Eusebius, Ambrose, and especially Augustine, and also still other eminent church-teachers have done, because they were necessary to explain this doctrine in opposition to the heretics, they assume immediatam divisionem, that is, a division between which there is no mean, so that everything that is must be either substantia, that is, a self-existent essence, or accidens, that is, an accidental matter which does not exist by itself essentially, but is in another self-existent essence and can be distinguished from it; which division Cyril and Basil also use.

55 And since, among others, this, too, is an indubitable, indisputable axiom in theology, that every substantia or self-existing essence, so far as it is a substance, is either God Himself or a work and creation of God, Augustine, in many writings against the Manicheans, in common with all true teachers, has, after due consideration and with earnestness, condemned and rejected the statement: Peccatum originis est substantia vel natura, that is, original sin is man’s nature or substance. After him all the learned and intelligent also have always maintained that whatever does not exist by itself, nor is a part of another self-existing essence, but exists, subject to change, in another thing, is not a substantia, that is, something self-existing, but an accidens, that is, something accidental. Accordingly, Augustine is accustomed constantly to speak in this way: Original sin is not the nature itself, but an accidens vitium in natura, that is, an accidental defect and damage in the nature.

56 In this way, previous to this controversy, [learned] men spoke, also in our schools and churches, according to the rules of logic, freely and without being suspected [of heresy], and were never censured on this account either by Dr. Luther or any orthodox teacher of our pure, evangelical churches.

57 Now, then, since it is the indisputable truth that everything that is, is either a substance or an accidens, that is, either a self-existing essence or something accidental in it, as has just been shown and proved by testimonies of the church-teachers, and no truly intelligent man has ever had any doubts concerning this, necessity here constrains, and no one can evade it, if the question be asked whether original sin is a substance, that is, such a thing as exists by itself, and is not in another or whether it is an accidens, that is, such a thing as does not exist by itself, but is in another, and cannot exist or be by itself, he must confess straight and pat that original sin is no substance, but an accidens.

58 For this reason, too, the Church of God will never be helped to permanent peace in this controversy, but the dissension will rather be strengthened and kept up, if the ministers of the Church remain in doubt as to whether original sin is a substance or an accidens, and whether it is rightly and properly named thus.

59 Hence, if the churches and schools are to be thoroughly relieved of this scandalous and very mischievous controversy, it is necessary that each and every one be properly instructed concerning this matter.

60 But if it be further asked what kind of an accidens original sin is, that is another question, of which no philosopher, no papist, no sophist, yea, no human reason, however acute it may be, can give the right explanation, but all understanding and every explanation of it must be derived solely from the Holy Scriptures, which testify that original sin is an unspeakable evil and such an entire corruption of human nature that in it and all its internal and external powers nothing pure or good remains, but everything is entirely corrupt, so that on account of original sin man is in God’s sight truly spiritually dead, with all his powers dead to that which is good.

61 In this way, then, original sin is not extenuated by the word accidens, [namely] when it is explained according to [the analogy of] God’s Word, after the manner in which Dr. Luther, in his Latin exposition of the third chapter of Genesis, has written with great earnestness against the extenuation of original sin; but this word serves only to indicate the distinction between the work of God (which our nature is, notwithstanding that it is corrupt) and the work of the devil (which the sin is that inheres in God’s work, and is the most profound and indescribable corruption of it).
62 Therefore Luther also in his treatment of this subject has employed the term accidens, as also the term qualitas [quality], and has not rejected them; but at the same time he has, with special earnestness and great zeal, taken the greatest pains to explain and to inculcate upon each and every one what a horrible quality and accidens it is, by which human nature is not merely polluted, but so deeply corrupted that nothing pure or incorrupt has remained in it, as his words on Ps. 90 run: Sive igitur peccatum originis qualitatem sive morbum vocaverimus, profecto extremum malum est non solum pati aeternam iram et mortem, sed ne agnoscere quidem, quae pateris. That is: Whether we call original sin a quality or a disease, it is indeed the utmost evil, that we are not only to suffer the eternal wrath of God and eternal death, but that we do not even understand what we suffer. And again, on Gen. 3: Qui isto veneno peccati originis a planta pedis usque ad verticem infecti sumus, siquidem in natura adhuc integra accidere. That is: We are infected with the poison of original sin from the sole of the foot to the crown of the head, inasmuch as this happened to us in a nature still perfect.

III. The Righteousness of Faith

1 The third controversy which has arisen among some theologians of the Augsburg Confession is concerning the righteousness of Christ or of faith, which God imputes by grace, through faith, to poor sinners for righteousness.

2 For one side has contended that the righteousness of faith, which the apostle calls the righteousness of God, is God’s essential righteousness, which is Christ Himself as the true, natural, and essential Son of God, who dwells in the elect by faith and impels them to do right, and thus is their righteousness, compared with which righteousness the sins of all men are as a drop of water compared with the great ocean.

3 Over against this, others have held and taught that Christ is our righteousness according to His human nature alone.

4 In opposition to both these parties it has been unanimously taught by the other teachers of the Augsburg Confession that Christ is our righteousness not according to His divine nature alone, nor according to His human nature alone, but according to both natures; for He has redeemed, justified, and saved us from our sins as God and man, through His complete obedience; that therefore the righteousness of faith is the forgiveness of sins, reconciliation with God, and our adoption as God’s children only on account of the obedience of Christ, which through faith alone, out of pure grace, is imputed for righteousness to all true believers, and on account of it they are absolved from all their unrighteousness.

5 Besides this [controversy] there have been still other disputes caused and excited on account of the Interim [on occasion of the formula of the Interim or of Interreligion], and otherwise, concerning the article of justification, which will hereafter be explained in antithesi, that is, in the enumeration of those errors which are contrary to the pure doctrine in this article.

6 This article concerning justification by faith (as the Apology says) is the chief article in the entire Christian doctrine, without which no poor conscience can have any firm consolation, or can truly know the riches of the grace of Christ, as Dr. Luther also has written: If this only article remains pure on the battlefield, the Christian Church also remains pure, and in goodly harmony and without any sects; but if it does not remain pure, it is not possible that any error or fanatical spirit can be resisted. (Tom. 5, Jena, p. 159.)

7 And concerning this article especially Paul says that a little leaven leaveneth the whole lump. Therefore, in this article he urges with so much zeal and earnestness the particulas exclusivas, that is, the words whereby the works of men are excluded (namely, without Law, without works, by grace [freely], Rom. 3:28; 4:5; Eph. 2:8-9), in order to indicate how highly necessary it is that in this article, aside from [the presentation of] the pure doctrine, the antithesis, that is, all contrary dogmas, be stated separately, exposed, and rejected by this means.

8 Therefore, in order to explain this controversy in a Christian way by means of God’s Word, and, by His grace, to settle it, our doctrine, faith, and confession are as follows:

9 Concerning the righteousness of faith before God we believe, teach, and confess unanimously, in accordance with the comprehensive summary of our faith and confession presented above, that poor sinful man is justified before God, that is, absolved and declared free and exempt from all his sins, and from the sentence of well-deserved condemnation, and adopted into sonship and heirship of eternal life, without any merit or worth of our own, also without any preceding, present, or any subsequent works, out of pure grace, because of the sole merit, complete obedience, bitter suffering, death, and resurrection of our Lord Christ alone, whose obedience is reckoned to us for righteousness.

10 These treasures are offered us by the Holy Ghost in the promise of the holy Gospel; and faith alone is the only means by which we lay hold upon, accept, and apply, and appropriate them to ourselves.

11 This faith is a gift of God, by which we truly learn to know Christ, our Redeemer, in the Word of the Gospel, and trust in Him, that for the sake of His obedience alone we have the forgiveness of sins by grace, are regarded as godly and righteous by God the father, and are eternally saved.

12 Therefore it is considered and understood to be the same thing when Paul says that we are justified by faith, Rom. 3:28, or that faith is counted to us for righteousness, Rom. 4:5, and when he says that we are made righteous by the obedience of One, Rom. 5:19, or that by the righteousness of One justification of faith came to all men, Rom. 5:18.

13 For faith justifies, not for this cause and reason that it is so good a work and so fair a virtue, but because it lays hold of and accepts the merit of Christ in the promise of the holy Gospel; for this must be applied and appropriated to us by faith, if we are to be justified thereby.

14 Therefore the righteousness which is imputed to faith or to the believer out of pure grace is the obedience, suffering, and resurrection of Christ, since He has made satisfaction for us to the Law, and paid for [expiated] our sins.

15 For since Christ is not man alone, but God and man in one undivided person, He was as little subject to the Law, because He is the Lord of the Law, as He had to suffer and die as far as His person is concerned. For this reason, then, His obedience, not only in suffering and dying, but also in this, that He in our stead was voluntarily made under the Law, and fulfilled it by this obedience, is imputed to us for righteousness, so that, on account of this complete obedience, which He rendered His heavenly Father for us, by doing and suffering, in living and dying, God forgives our sins, regards us as godly and righteous, and eternally saves us.

16 This righteousness is offered us by the Holy Ghost through the Gospel and in the Sacraments, and is applied, appropriated, and received through faith, whence believers have reconciliation with God, forgiveness of sins, the grace of God, sonship, and heirship of eternal life.

17 Accordingly, the word justify here means to declare righteous and free from sins, and to absolve one from eternal punishment for the sake of Christ’s righteousness, which is imputed by God to faith, Phil. 3:9. For this use and understanding of this word is common in the Holy Scriptures of the Old and the New Testament. Prov. 17:15: He that justifieth the wicked, and he that condemneth the just, even they both are abomination to the Lord. Is. 5:23: Woe unto them which justify the wicked for reward, and take away the righteousness of the righteous from him! Rom. 8:33: Who shall lay anything to the charge of God’s elect? It is God that justifieth, that is, absolves from sins and acquits.

18 However, since the word regeneratio, regeneration, is sometimes employed for the word iustificatio, justification, it is necessary that this word be properly explained, in order that the renewal which follows justification of faith may not be confounded with the justification of faith, but that they may be properly distinguished from one another.

19 For, in the first place, the word regeneratio, that is, regeneration, is used so as to comprise at the same time the forgiveness of sins for Christ’s sake alone, and the succeeding renewal which the Holy Ghost works in those who are justified by faith. Then, again, it is [sometimes] used pro remissione peccatorum et adoptione in filios Dei, that is, so as to mean only the remission of sins, and that we are adopted as sons of God. And in this latter sense the word is much and often used in the Apology, where it is written: Iustificatio est regeneratio, that is, Justification before God is regeneration. St. Paul, too, has employed these words as distinct from one another, Titus 3:5: He saved us by the washing of regeneration and renewal of the Holy Ghost.

20 As also the word vivificatio, that is, making alive, has sometimes been used in a like sense. For when man is justified through faith (which the Holy Ghost alone works), this is truly a regeneration, because from a child of wrath he becomes a child of God, and thus is transferred from death to life, as it is written: When we were dead in sins, He hath quickened us together with Christ, Eph. 2:5. Likewise: The just shall live by faith, Rom. 1:17; Hab. 2:4. In this sense the word is much and often used in the Apology.

21 But again, it is often taken also for sanctification and renewal, which succeeds the righteousness of faith, as Dr. Luther has thus used it in his book concerning the Church and the Councils, and elsewhere.

22 But when we teach that through the operation of the Holy Ghost we are born anew and justified, the sense is not that after regeneration no unrighteousness clings any more to the justified and regenerate in their being and life, but that Christ covers all their sins which nevertheless in this life still inhere in nature with His complete obedience. But irrespective of this they are declared and regarded godly and righteous by faith and for the sake of Christ’s obedience (which Christ rendered the Father for us from His birth to His most ignominious death upon the cross), although, on account of their corrupt nature, they still are and remain sinners to the grave [while they bear about this mortal body]. Nor, on the other hand, is this the meaning, that without repentance, conversion, and renewal we might or should yield to sins, and remain and continue in them.

23 For true [and not feigned] contrition must precede; and to those who, in the manner stated, out of pure grace, for the sake of the only Mediator, Christ, without any works and merit, are righteous before God, that is, are received into grace, the Holy Ghost is also given, who renews and sanctifies them, and works in them love to God and to their neighbor. But since the incipient renewal is imperfect in this life, and sin still dwells in the flesh, even in the regenerate, the righteousness of faith before God consists in the gracious imputation of the righteousness of Christ, without the addition of our works, so that our sins are forgiven us and covered, and are not imputed, Rom. 4:6ff

24 But here very good attention must be given with especial diligence, if the article of justification is to remain pure, lest that which precedes faith, and that which follows after it, be mingled together or inserted into the article of justification as necessary and belonging to it, because it is not one or the same thing to speak of conversion and of justification.

25 For not everything that belongs to conversion belongs likewise to the article of justification, in and to which belong and are necessary only the grace of God, the merit of Christ, and faith, which receives this in the promise of the Gospel, whereby the righteousness of Christ is imputed to us, whence we receive and have forgiveness of sins, reconciliation with God, sonship, and heirship of eternal life.

26 Therefore true, saving faith is not in those who are without contrition and sorrow, and have a wicked purpose to remain and persevere in sins; but true contrition precedes, and genuine faith is in or with true repentance [justifying faith is in those who repent truly, not feignedly].

27 Love is also a fruit which surely and necessarily follows true faith. For the fact that one does not love is a sure indication that he is not justified, but is still in death, or has lost the righteousness of faith again, as John says, 1 John 3:14. But when Paul says, Rom. 3:28: We are justified by faith without works, he indicates thereby that neither the contrition that precedes, nor the works that follow, belong in the article or transaction of justification by faith. For good works do not precede justification, but follow it, and the person must first be justified before he can do good works.

28 In like manner also renewal and sanctification, although it is also a benefit of the Mediator, Christ, and a work of the Holy Ghost, does not belong in the article or affair of justification before God, but follows the same since, on account of our corrupt flesh, it is not entirely perfect and complete in this life, as Dr. Luther writes well concerning this in his beautiful and large exposition of the Epistle to the Galatians, in which he says as follows:

29 We concede indeed that instruction should be given also concerning love and good works, yet in such a way that this be done when and where it is necessary, namely, when otherwise and outside of this matter of justification we have to do with works. But here the chief matter dealt with is the question, not whether we should also do good works and exercise love, but by what means we can be justified before God, and saved. And here we answer thus with St. Paul: that we are justified by faith in Christ alone, and not by the deeds of the Law or by love. Not that we hereby entirely reject works and love, as the adversaries falsely slander and accuse us, but that we do not allow ourselves to be led away, as Satan desires, from the chief matter with which we have to do here to another and foreign affair which does not at all belong to this matter. Therefore, whereas, and as long as we are occupied with this article of justification, we reject and condemn works, since this article is so constituted that it can admit of no disputation or treatment whatever regarding works; therefore in this matter we cut short all Law and works of the Law. So far Luther.

30 In order, therefore, that troubled hearts may have a firm, sure consolation, also, that due honor be given to the merit of Christ and the grace of God, the Scriptures teach that the righteousness of faith before God consists alone in the gracious [gratuitous] reconciliation or the forgiveness of sins, which is presented to us out of pure grace, for the sake of the only merit of the Mediator, Christ, and is received through faith alone in the promise of the Gospel. In like manner, too, in justification before God faith relies neither upon contrition nor upon love or other virtues, but upon Christ alone, and in Him upon His complete obedience by which He has fulfilled the Law for us, which [obedience] is imputed to believers for righteousness.

31 Moreover, neither contrition nor love or any other virtue, but faith alone is the sole means and instrument by which and through which we can receive and accept the grace of God, the merit of Christ, and the forgiveness of sins, which are offered us in the promise of the Gospel.

32 It is also correctly said that believers who in Christ through faith have been justified, have in this life first the imputed righteousness of faith, and then also the incipient righteousness of the new obedience or of good works. But these two must not be mingled with one another or be both injected at the same time into the article of justification by faith before God. For since this incipient righteousness or renewal in us is incomplete and impure in this life because of the flesh, the person cannot stand with and by it [on the ground of this righteousness] before God’s tribunal, but before God’s tribunal only the righteousness of the obedience, suffering, and death of Christ, which is imputed to faith, can stand, so that only for the sake of this obedience is the person (even after his renewal, when he has already many good works and lives the best [upright and blameless] life) pleasing and acceptable to God, and is received into adoption and heirship of eternal life.

33 Here belongs also what St. Paul writes Rom. 4:3, that Abraham was justified before God by faith alone, for the sake of the Mediator, without the cooperation of his works, not only when he was first converted from idolatry and had no good works, but also afterwards, when he had been renewed by the Holy Ghost, and adorned with many excellent good works, Gen. 15:6; Heb. 11:8. And Paul puts the following question, Rom. 4:1ff : On what did Abraham’s righteousness before God for everlasting life, by which he had a gracious God, and was pleasing and acceptable to Him, rest at that time?

34 This he answers: To him who worketh not, but believeth on Him that justifieth the ungodly, his faith is counted for righteousness; as David also, Ps. 32:1, speaks of the blessedness of the man to whom God imputes righteousness without works.

35 Hence, even though the converted and believing [in Christ] have incipient renewal, sanctification, love, virtue, and good works, yet these neither can nor should be drawn into, or mingled with, the article of justification before God, in order that the honor due Him may remain with Christ the Redeemer, and tempted consciences may have a sure consolation, since our new obedience is incomplete and impure.

36 And this is the meaning of the Apostle Paul when in this article he urges so diligently and zealously the particulas exclusivas, that is, the words by which works are excluded from the article of justification: absque operibus, sine lege, gratis, non ex operibus, that is, by grace, without merit, without works, not of works. These exclusivae are all comprised in the expression: By faith alone in Christ we are justified before God and saved. For thereby works are excluded, not in the sense that a true faith can exist without contrition, or that good works should, must, and dare not follow true faith as sure and indubitable fruits, or that believers dare not nor must do anything good; but good works are excluded from the article of justification before God, so that they must not be drawn into, woven into, or mingled with the transaction of the justification of the poor sinner before God as necessary or belonging thereto. And the true sense of the particulae exclusivae in articulo iustificationis, that is, of the aforementioned terms, in the article of justification, consists in the following, and they should also be urged in this article with all diligence and earnestness [on account of these reasons]:

37 1. That thereby [through these particles] all our own works, merit, worthiness, glory, and confidence in all our works are entirely excluded in the article of justification so that our works shall not be constituted or regarded as either the cause or the merit of justification, neither entirely, nor half, nor in the least part, upon which God could or ought to look, or we to rely in this article and action.

38 2. That this remain the office and property of faith alone, that it alone, and nothing else whatever, is the means or instrument by and through which God’s grace and the merit of Christ in the promise of the Gospel are received, apprehended, accepted, applied to us, and appropriated; and that from this office and property of such application or appropriation love and all other virtues or works are excluded.

39 3. That neither renewal, sanctification, virtues nor good works are tamquam forma aut pars aut causa iustificationis, that is, our righteousness before God, nor are they to be constituted and set up as a part or cause of our righteousness, or otherwise under any pretext, title, or name whatever to be mingled in the article of justification as necessary and belonging thereto; but that the righteousness of faith consists alone in the forgiveness of sins out of pure grace, for the sake of Christ’s merit alone; which blessings are offered us in the promise of the Gospel, and are received, accepted, applied, and appropriated by faith alone.

40 In the same manner the order also between faith and good works must abide and be maintained, and likewise between justification and renewal, or sanctification.

41 For good works do not precede faith, neither does sanctification precede justification. But first faith is kindled in us in conversion by the Holy Ghost from the hearing of the Gospel. This lays hold of God’s grace in Christ, by which the person is justified. Then, when the person is justified, he is also renewed and sanctified by the Holy Ghost, from which renewal and sanctification the fruits of good works then follow. Et haec non ita divelluntur, quasi vera fides aliquando et aliquamdiu stare possit cum malo proposito, sed ordine causarum et effectuum, antecedentium et consequentium, ita distribuuntur. Manet enim, quod Lutherus recte dicit: Bene conveniunt et sunt connexa inseparabiliter fides et opera; sed sola fides est, quae apprehendit benedictionem sine operibus, et tamen nunquam est sola. That is: This should not be understood as though justification and renewal were sundered from one another in such a manner that a genuine faith sometimes could exist and continue for a time together with a wicked intention, but hereby only the order [of causes and effects, of antecedents and consequents] is indicated, how one precedes or succeeds the other. For what Luther has correctly said remains true nevertheless: Faith and good works well agree and fit together [are inseparably connected]; but it is faith alone, without works, which lays hold of the blessing; and yet it is never and at no time alone. This has been set forth above.

42 Many disputations also are usefully and well explained by means of this true distinction, of which the Apology treats in reference to the passage James 2:20. For when we speak of faith, how it justifies, the doctrine of St. Paul is that faith alone, without works, justifies, Rom. 3:28, inasmuch as it applies and appropriates to us the merit of Christ, as has been said. But if the question is, wherein and whereby a Christian can perceive and distinguish, either in himself or in others, a true living faith from a feigned and dead faith, (since many idle, secure Christians imagine for themselves a delusion in place of faith, while they nevertheless have no true faith,) the Apology gives this answer: James calls that dead faith where good works and fruits of the Spirit of every kind do not follow. And to this effect the Latin edition of the Apology says: Iacobus recte negat, nos tali fide iustificari, quae est sine operibus, hoc est, quae mortua est. That is: St. James teaches correctly when he denies that we are justified by such a faith as is without works, which is dead faith.

43 But James speaks, as the Apology says, concerning the works of those who have already been justified through Christ, reconciled with God, and obtained forgiveness of sins through Christ. But if the question is, whereby and whence faith has this, and what appertains to this that it justifies and saves, it is false and incorrect to say: Fidem non posse iustificare sine operibus; vel fidem, quatenus caritatem, qua formatur, coniunctam habet, iustificare; vel fidei, ut iustificet, necessariam esse praesentiam bonorum operum; vel bona opera esse causam sine qua non, quae per particulas exclusivas ex articulo iustificationis non excludantur. That is: That faith cannot justify without works; or that faith justifies or makes righteous, inasmuch as it has love with it, for the sake of which love this is ascribed to faith [it has love with it, by which it is formed]; or that the presence of works with faith is necessary if otherwise man is to be justified thereby before God; or that the presence of good works in the article of justification, or for justification, is needful, so that good works are a cause without which man cannot be justified, and that they are not excluded from the article of justification by the particulae exclusivae: absque operibus etc., that is, when St. Paul says: without works. For faith makes righteous only inasmuch as and because, as a means and instrument, it lays hold of, and accepts, the grace of God and the merit of Christ in the promise of the Gospel.

44 Let this suffice, according to the plan of this document, as a summary explanation of the doctrine of justification by faith, which is treated more at length in the above-mentioned writings. From these, the antithesis also, that is, the false contrary dogmas, are manifest, namely, that in addition to the errors recounted above also the following and similar ones, which conflict with the explanation now published, must be censured, exposed, and rejected, as when it is taught:

45 1. That our love or good works are a merit or cause of justification before God, either entirely or at least in part.

46 2. Or that by good works man must render himself worthy and fit that the merit of Christ may be imparted to him.

47 3. Vel formalem nostram iustitiam coram Deo esse inhaerentem nostram novitatem seu caritatem; that is, that our real righteousness before God is the love or renewal which the Holy Ghost works in us, and which is in us.

48 4. Or that two things or parts belong to the righteousness of faith before God in which it consists, namely, the gracious forgiveness of sins, and then, secondly, also renewal or sanctification.

49 5. Item, fidem iustificare tantum initialiter, vel partialiter, vel principaliter; et novitatem vel caritatem nostram iustificare etiam coram Deo vel completive, vel minus principaliter (that is, that faith justifies only initially, either in part or primarily, and that our newness or love justifies even before God, either completively or secondarily).

50 6. Item, credentes coram Deo iustificari vel coram Deo iustos esse simul et imputatione et inchoatione, vel partim imputatione, partim inchoatione novae obedientiae (that is, also that believers are justified before God, or are righteous before God, both by imputation and by inchoation at the same time, or partly by the imputation of Christ’s righteousness and partly by the beginning of new obedience).

51 7. Item, applicationem promissionis gratiae fieri et fide cordis et confessione oris ac reliquis virtutibus (that is, also that the application of the promise of grace occurs both by faith of the heart and confession of the mouth, and by other virtues). That is: Faith makes righteous for this reason alone, that righteousness is begun in us by faith, or in this way, that faith takes the precedence in justification; nevertheless, renewal and love also belong to our righteousness before God, however, in such a way that it is not the chief cause of our righteousness, but that our righteousness before God is not entire and complete without such love and renewal. Likewise, that believers are justified and righteous before God at the same time by the imputed righteousness of Christ and the incipient new obedience, or in part by the imputation of Christ’s righteousness and in part by the incipient new obedience. Likewise, that the promise of grace is appropriated to us by faith in the heart, and confession which is made with the mouth, and by other virtues.

52 Also this is incorrect, when it is taught that man must be saved in some other way or through something else than as he is justified before God, so that we are indeed justified before God by faith alone, without works, but that it is impossible to be saved without works or obtain salvation without works.

53 This is false, for the reason that it is directly opposed to the declaration of Paul, Rom. 4:6: The blessedness is of the man unto whom God imputeth righteousness without works. And Paul’s reason [the basis of Paul’s argument] is that we obtain both, salvation as well as righteousness, in one and the same way; yea, that by: this very means, when we are justified by faith, we receive at the same time adoption and heirship of eternal life and salvation; and on this account Paul employs and emphasizes the particulas exclusivas, that is, those words by which works and our own merits are entirely excluded, namely, by grace, without works, as forcibly in the article concerning salvation as in the article concerning righteousness.

54 Likewise also the disputation concerning the indwelling in us of the essential righteousness of God must be correctly explained. For although in the elect, who are justified by Christ and reconciled with God, God the Father, Son, and Holy Ghost, who is the eternal and essential righteousness, dwells by faith (for all Christians are temples of God the Father, Son, and Holy Ghost, who also impels them to do right), yet this indwelling of God is not the righteousness of faith of which St. Paul treats and which he calls iustitiam Dei, that is, the righteousness of God, for the sake of which we are declared righteous before God; but it follows the preceding righteousness of faith, which is nothing else than the forgiveness of sins and the gracious adoption of the poor sinner, for the sake of Christ’s obedience and merit alone.

55 Accordingly, since in our churches it is acknowledged [established beyond controversy] among the theologians of the Augsburg Confession that all our righteousness is to be sought outside the merits, works, virtues, and worthiness of ourselves and of all men, and rests alone upon Christ the Lord, it must be carefully considered in what respect Christ is called our Righteousness in this affair of justification, namely, that our righteousness rests not upon one or the other nature, but upon the entire person of Christ, who as God and man is our Righteousness in His only, entire, and complete obedience.

56 For even though Christ had been conceived and born without sin by the Holy Ghost, and had fulfilled all righteousness in His human nature alone, and yet had not been true and eternal God, this obedience and suffering of His human nature could not be imputed to us for righteousness. As also, if the Son of God had not become man, the divine nature alone could not be our righteousness. Therefore we believe, teach, and confess that the entire obedience of the entire person of Christ, which He has rendered the Father for us even to His most ignominious death upon the cross, is imputed to us for righteousness. For the human nature alone, without the divine, could neither by obedience nor suffering render satisfaction to eternal almighty God for the sins of all the world; however, the divinity alone, without the humanity, could not mediate between God and us.

57 But, since it is the obedience as above mentioned [not only of one nature, but] of the entire person, it is a complete satisfaction and expiation for the human race, by which the eternal, immutable righteousness of God, revealed in the Law, has been satisfied, and is thus our righteousness, which avails before God and is revealed in the Gospel, and upon which faith relies before God, which God imputes to faith, as it is written, Rom. 5:19: For as by one man’s disobedience many were made sinners, so by the obedience of One shall many be made righteous; and 1 John 1:7: The blood of Jesus Christ, the Son of God, cleanseth us from all sin. Likewise: The just shall live by his faith, Hab. 2:4; Rom. 1:17.

58 Thus neither the divine nor the human nature of Christ by itself is imputed to us for righteousness, but only the obedience of the person who is at the same time God and man. And faith thus regards the person of Christ as it was made under the Law for us, bore our sins, and in His going to the Father offered to His heavenly Father for us poor sinners His entire, complete obedience, from His holy birth even unto death, and has thereby covered all our disobedience which inheres in our nature, and its thoughts, words, and works, so that it is not imputed to us for condemnation, but is pardoned and forgiven out of pure grace, alone for Christ’s sake.

59 Therefore we unanimously reject and condemn, besides the above-mentioned, also the following and all similar errors, as contrary to God’s Word, the doctrine of the prophets and apostles, and our Christian faith:

60 1. When it is taught that Christ is our righteousness before God according to His divine nature alone.

61 2. That Christ is our righteousness according to His human nature alone.

62 3. That in the passages from the prophets and apostles, when the righteousness of faith is spoken of, the words justify and to be justified are not to signify to declare free from sins and to obtain the forgiveness of sins, but to be made actually and really righteous because of love infused by the Holy Ghost, virtues, and the works following from it.

63 4. That faith looks not only to the obedience of Christ, but to His divine nature as it dwells and works in us, and that by this indwelling our sins are covered before God.

64 5. That faith is such a trust in the obedience of Christ as can be and remain in a person notwithstanding he has no genuine repentance, in whom also no love follows, but who persists in sins against his conscience.

65 6. That not God dwells in the believers, but only the gifts of God.

66 These and like errors, one and all, we unanimously reject as contrary to the clear Word of God, and by God’s grace abide firmly and constantly in the doctrine of the righteousness of faith before God, as it is embodied, expounded, and proved from God’s Word in the Augsburg Confession, and the Apology issued after it.
67 Concerning what is needful furthermore for the proper explanation of this profound and chief article of justification before God, upon which depends the salvation of our souls, we direct, and for the sake of brevity herewith refer, every one to Dr. Luther’s beautiful and glorious exposition of the Epistle of St. Paul to the Galatians.

IV. Good Works

1 A disagreement has also occurred among the theologians of the Augsburg Confession concerning good works, one part employing the following words and manner in speaking of them: Good works are necessary for salvation; it is impossible to be saved without good works; likewise, no one has been saved without good works; because, they say, good works are required of true believers as fruits of faith, and faith without love is dead, although such love is no cause of salvation.

2 The other part, however, contended, on the contrary, that good works are indeed necessary; however, not for salvation, but for other reasons; and that on this account the aforecited propositiones, or expressions, which have been used (as they are not in accord with the form of sound doctrine and with the Word, and have been always and are still set by the Papists in opposition to the doctrine of our Christian faith, in which we confess that faith alone justifies and saves) are not to be tolerated in the Church, in order that the merit of Christ, our Savior, be not diminished, and the promise of salvation may be and remain firm and certain to believers.

3 In this controversy also the following controverted proposition, or expression, was employed by some few, that good works are injurious to salvation. It has also been argued by some that good works are not necessary, but are voluntary [free and spontaneous], because they are not extorted by fear and the penalty of the Law, but are to be done from a voluntary spirit and a joyful heart. Over against this the other side contended that good works are necessary.

4 This [latter] controversy was originally occasioned by the words necessitas and libertas, that is, necessary and free, because especially the word necessitas, necessary, signifies not only the eternal, immutable order according to which all men are obliged and in duty bound to obey God, but sometimes also a coercion, by which the Law forces men to good works.

5 But afterwards there was a disputation not only concerning the words, but the doctrine itself was attacked in the most violent manner, and it was contended that the new obedience in the regenerate is not necessary because of the above-mentioned divine order.

6 In order to explain this disagreement in a Christian way and according to the guidance of God’s Word, and by His grace to settle it completely, our doctrine, faith, and confession are as follows:

7 First, there is no controversy among our theologians concerning the following points in this article, namely: that it is God’s will, order, and command that believers should walk in good works; and that truly good works are not those which every one contrives himself from a good intention, or which are done according to traditions of men, but those which God Himself has prescribed and commanded in His Word; also, that truly good works are done, not from our own natural powers, but in this way: when the person by faith is reconciled with God and renewed by the Holy Ghost, or, as Paul says, is created anew in Christ Jesus to good works, Eph. 2:10.

8 Nor is there a controversy as to how and why the good works of believers, although in this flesh they are impure and incomplete, are pleasing and acceptable to God, namely, for the sake of the Lord Christ, by faith, because the person is acceptable to God. For the works which pertain to the maintenance of external discipline, which are also done by, and required of, the unbelieving and unconverted, although commendable before the world, and besides rewarded by God in this world with temporal blessings, are nevertheless, because they do not proceed from true faith, in God’s sight sins, that is, stained with sin, and are regarded by God as sins and impure on account of the corrupt nature and because the person is not reconciled with God. For a corrupt tree cannot bring forth good fruit, Matt. 7:18, as it is also written Rom. 14:23: Whatsoever is not of faith is sin. For the person must first be accepted of God, and that for the sake of Christ alone, if also the works of that person are to please Him.

9 Therefore, of works that are truly good and well-pleasing to God, which God will reward in this world and in the world to come, faith must be the mother and source; and on this account they are called by St. Paul true fruits of faith, as also of the Spirit.

10 For, as Dr. Luther writes in the Preface to St. Paul’s Epistle to the Romans: Thus faith is a divine work in us, that changes us and regenerates us of God, and puts to death the old Adam, makes us entirely different men in heart, spirit, mind, and all powers, and brings with it [confers] the Holy Ghost. Oh, it is a living, busy, active, powerful thing that we have in faith, so that it is impossible for it not to do good without ceasing.

11 Nor does it ask whether good works are to be done; but before the question is asked, it has wrought them, and is always engaged in doing them. But he who does not do such works is void of faith, and gropes and looks about after faith and good works, and knows neither what faith nor what good works are, yet babbles and prates with many words concerning faith and good works.

12 [Justifying] faith is a living, bold [firm] trust in God’s grace, so certain that a man would die a thousand times for it [rather than suffer this trust to be wrested from him]. And this trust and knowledge of divine grace renders joyful, fearless, and cheerful towards God and all creatures, which [joy and cheerfulness] the Holy Ghost works through faith; and on account of this, man becomes ready and cheerful, without coercion, to do good to every one, to serve every one, and to suffer everything for love and praise to God, who has conferred this grace on him, so that it is impossible to separate works from faith, yea, just as impossible as it is for heat and light to be separated from fire.

13 But since there is no controversy on these points among our theologians, we will not treat them here at length, but only explain ourselves, part against part, in a simple and plain manner regarding the controverted points.

14 And first, as regards the necessity or voluntariness of good works, it is manifest that in the Augsburg Confession and its Apology these expressions are often used and repeated that good works are necessary. Likewise, that it is necessary to do good works, which also are necessarily to follow faith and reconciliation. Likewise, that we necessarily are to do and must do such good works as God has commanded. Thus also in the Holy Scriptures themselves the words necessity, needful, and necessary, likewise, ought and must, are used concerning what we are bound to do because of God’s ordinance, command, and will, as Rom. 13:5; 1 Cor. 9:9; Acts 5:29; John 15:12; 1 John 4:21.

15 Therefore the expressions or propositions mentioned [that good works are necessary, and that it is necessary to do good] are unjustly censured and rejected in this Christian and proper sense, as has been done by some; for they are employed and used with propriety to rebuke and reject the secure, Epicurean delusion, by which many fabricate for themselves a dead faith or delusion which is without repentance and without good works, as though there could be in a heart true faith and at the same time the wicked intention to persevere and continue in sins, which is impossible; or, as though one could, indeed, have and retain true faith, righteousness, and salvation even though he be and remain a corrupt and unfruitful tree, whence no good fruits whatever come, yea, even though he persist in sins against conscience, or purposely engages again in these sins, all of which is incorrect and false.

16 But in this connection the following distinction must also be noted, namely, that the meaning must be: necessitas ordinis, mandati et voluntatis Christi ac debiti nostri, non autem necessitas coactionis (a necessity of Christ’s ordinance, command, and will, and of our obligation, but not a necessity of coercion). That is: When this word necessary is employed, it should be understood not of coercion, but only of the ordinance of the immutable will of God, whose debtors we are; thither also

17 His commandment points that the creature should be obedient to its Creator. For in other places, as 2 Cor. 9:7, and in the Epistle of St. Paul to Philemon 14, also 1 Pet. 5:2, that is termed of necessity which is wrung from one against his will, by force or otherwise, so that he acts externally for appearance, but nevertheless without and against his will. For such specious [hypocritical] works God does not want [does not approve], but the people of the New Testament are to be a willing people, Ps. 110:3, and sacrifice freely, Ps. 54:6, not grudgingly or of necessity, but are to be obedient from the heart, 2 Cor. 9:7; Rom. 6:17.

18 For God loveth a cheerful giver, 2 Cor. 9:7. In this understanding and in such sense it is correctly said and taught that truly good works should be done willingly or from a voluntary spirit by those whom the Son of God has made free, even as it was especially for [confirming] this opinion that the disputation concerning the voluntariness of good works was engaged in by some.

19 But here, again, it is well to note also the distinction of which St. Paul says, Rom. 7:22f.: [I am willing] and delight in the Law of God after the inward man. But I see another law in my members, that is not only unwilling or disinclined, but also warring against the law of my mind. And as regards the unwilling and rebellious flesh, Paul says, 1 Cor. 9:27: I keep under my body, and bring it into subjection, and Gal. 5:24; Rom. 8:13: They that are Christ’s have crucified, yea, slain, the flesh with its affections and lusts.

20 But it is false, and must be censured, when it is asserted and taught as though good works were free to believers in the sense that it were optional with them to do or to omit them, or that they might or could act contrary thereto [to the Law of God], and none the less could retain faith and God’s favor and grace.

21 Secondly, when it is taught that good works are necessary, it must also be explained why and for what reasons they are necessary, which reasons are enumerated in the Augsburg Confession and Apology.

22 But here we must be well on our guard lest works are drawn and mingled into the article of justification and salvation. Therefore the propositions are justly rejected, that to believers good works are necessary for salvation, so that it is impossible to be saved without good works. For they are directly contrary to the doctrine de particulis exclusivis in articulo iustificationis et salvationis (concerning the exclusive particles in the article of justification and salvation), that is, they conflict with the words by which St. Paul has entirely excluded our works and merits from the article of justification and salvation, and ascribed everything to the grace of God and the merit of Christ alone, as explained in the preceding article.

23 Again, they [these propositions concerning the necessity of good works for salvation] take from afflicted, troubled consciences the comfort of the Gospel, give occasion for doubt, are in many ways dangerous, strengthen presumption in one’s own righteousness and confidence in one’s own works; besides, they are accepted by the Papists, and in their interest adduced against the pure doctrine of the alone-saving faith.

24 Moreover, they are contrary to the form of sound words, as it is written that blessedness is only of the man unto whom God imputeth righteousness without works, Rom. 4:6. Likewise, in the Sixth Article of the Augsburg Confession it is written that we are saved without works, by faith alone. Thus Dr. Luther, too, has rejected and condemned these propositions:

25 1. In the false prophets among the Galatians [who led the Galatians into error].

26 2. In the Papists, in very many places.

27 3. In the Anabaptists, when they present this interpretation: We should not indeed rest faith upon the merit of works, but we must nevertheless have them as things necessary to salvation.

28 4. Also in some others among his own followers, who wished to interpret this proposition thus: Although we require works as necessary to salvation, yet we do not teach to place trust in works. On Gen. 22.

29 Accordingly, and for the reasons now enumerated, it is justly to remain settled in our churches, namely, that the aforesaid modes of speech should not be taught, defended, or excused, but be thrown out of our churches and repudiated as false and incorrect, and as expressions which were renewed in consequence of the Interim, originated from it, and were [again] drawn into discussion in times of persecution, when there was especial need of a clear, correct confession against all sorts of corruptions and adulterations of the article of justification.

30 Thirdly, since it is also disputed whether good works preserve salvation, or whether they are necessary for preserving faith, righteousness, and salvation, and this again is of high and great importance,–for he that shall endure unto the end, the same shall be saved, Matt. 24:13; also Heb. 3:6-014: We are made partakers of Christ, if we hold the beginning of our confidence steadfast unto the end,–we must also explain well and precisely how righteousness and salvation are preserved in us, lest it be lost again.

31 Above all, therefore, the false Epicurean delusion is to be earnestly censured and rejected, namely, that some imagine that faith and the righteousness and salvation which they have received can be lost through no sins or wicked deeds, not even through wilful and intentional ones, but that a Christian although he indulges his wicked lusts without fear and shame, resists the Holy Ghost, and purposely engages in sins against conscience, yet none the less retains faith, God’s grace, righteousness, and salvation.

32 Against this pernicious delusion the following true, immutable, divine threats and severe punishments and admonitions should be often repeated and impressed upon Christians who are justified by faith: 1 Cor. 6:9: Be not deceived: neither fornicators, nor idolaters, nor adulterers, etc., shall inherit the kingdom of God. Gal. 5:21; Eph. 5:5: They which do such things shall not inherit the kingdom of God. Rom. 8:13: If ye live after the flesh, ye shall die. Col. 3:6: For which thing’s sake the wrath of God cometh upon the children of disobedience.

33 But when and in what way the exhortations to good works can be earnestly urged from this basis without darkening the doctrine of faith and of the article of justification, the Apology shows by an excellent model, when in Article XX, on the passage 2 Pet. 1:10: Give diligence to make your calling and election sure, it says as follows: Peter teaches why good works should be done, namely, that we may make our calling sure, that is, that we may not fall from our calling if we again sin. “Do good works,” he says, “that you may persevere in your heavenly calling, that you may not fall away again, and lose the Spirit and the gifts, which come to you, not on account of works that follow, but of grace, through Christ, and are now retained by faith. But faith does not remain in those who lead a sinful life, lose the Holy Ghost, and reject repentance.” Thus far the Apology.

34 But, on the other hand, the sense is not that faith only in the beginning lays hold of righteousness and salvation, and then resigns its office to the works as though thereafter they had to sustain faith, the righteousness received, and salvation; but in order that the promise, not only of receiving, but also of retaining righteousness and salvation, may be firm and sure to us, St. Paul, Rom. 5:2, ascribes to faith not only the entrance to grace, but also that we stand in grace and boast of the future glory, that is, the beginning, middle, and end he ascribes all to faith alone. Likewise, Rom. 11:20: Because of unbelief they were broken off, and thou standest by faith. Col. 1:22: He will present you holy and unblamable and unreprovable in His sight, if ye continue in the faith. 1 Pet. 1:5. 9: By the power of God we are kept through faith unto salvation. Likewise: Receiving the end of your faith, even the salvation of your souls.

35 Since, then, it is manifest from God’s Word that faith is the proper and only means by which righteousness and salvation are not only received, but also preserved by God, the decree of the Council of Trent, and whatever elsewhere is set forth in the same sense, is justly to be rejected, namely, that our good works preserve salvation, or that the righteousness of faith which has been received, or even faith itself, is either entirely or in part kept and preserved by our works.

36 For although before this controversy quite a few pure teachers employed such and similar expressions in the exposition of the Holy Scriptures, in no way, however, intending thereby to confirm the above-mentioned errors of the Papists, still, since afterwards a controversy arose concerning such expressions, from which all sorts of offensive distractions [debates, offenses, and dissensions] followed, it is safest of all, according to the admonition of St. Paul, 2 Tim. 1:13, to hold fast as well to the form of sound words as to the pure doctrine itself, whereby much unnecessary wrangling may be cut off and the Church preserved from many scandals.

37 Fourthly, as regards the proposition that good works are said to be injurious to salvation, we explain ourselves clearly as follows: If any one should wish to drag good works into the article of justification, or rest his righteousness or trust for salvation upon them, to merit God’s grace and be saved by them, to this not we say, but St. Paul himself says, and repeats it three times, Phil. 3:7ff , that to such a man his works are not only useless and a hindrance, but also injurious. But this is not the fault of the good works themselves, but of the false confidence placed in the works, contrary to the express Word of God.

38 However, it by no means follows thence that we are to say simpliciter and flatly: Good works are injurious to believers for or as regards their salvation; for in believers good works are indications of salvation when they are done propter veras causas et ad veros fines (from true causes and for true ends), that is, in the sense in which God requires them of the regenerate, Phil. 1:20; for it is God’s will and express command that believers should do good works, which the Holy Ghost works in believers, and with which God is pleased for Christ’s sake, and to which He promises a glorious reward in this life and the life to come.

39 For this reason, too, this proposition is censured and rejected in our churches, because as a flat statement it is false and offensive, by which discipline and decency might be impaired, and a barbarous, dissolute, secure, Epicurean life be introduced and strengthened. For what is injurious to his salvation a person should avoid with the greatest diligence.
40 However, since Christians should not be deterred from good works, but should be admonished and urged thereto most diligently, this bare proposition cannot and must not be tolerated, employed, nor defended in the Church [of Christ].

V. Law and Gospel

1 As the distinction between the Law and the Gospel is a special brilliant light, which serves to the end that God’s Word may be rightly divided, and the Scriptures of the holy prophets and apostles may be properly explained and understood, we must guard it with especial care, in order that these two doctrines may not be mingled with one another, or a law be made out of the Gospel, whereby the merit of Christ is obscured and troubled consciences are robbed of their comfort, which they otherwise have in the holy Gospel when it is preached genuinely and in its purity, and by which they can support themselves in their most grievous trials against the terrors of the Law.

2 Now, here likewise there has occurred a dissent among some theologians of the Augsburg Confession; for the one side asserted that the Gospel is properly not only a preaching of grace, but at the same time also a preaching of repentance, which rebukes the greatest sin, namely, unbelief. But the other side held and contended that the Gospel is not properly a preaching of repentance or of reproof [preaching of repentance, convicting sin], as that properly belongs to God’s Law, which reproves all sins, and therefore unbelief also; but that the Gospel is properly a preaching of the grace and favor of God for Christ’s sake, through which the unbelief of the converted, which previously inhered in them, and which the Law of God reproved, is pardoned and forgiven.

3 Now, when we consider this dissent aright, it has been caused chiefly by this, that the term Gospel is not always employed and understood in one and the same sense, but in two ways, in the Holy Scriptures, as also by ancient and modern church teachers.

4 For sometimes it is employed so that there is understood by it the entire doctrine of Christ, our Lord, which He proclaimed in His ministry upon earth, and commanded to be proclaimed in the New Testament, and hence comprised in it the explanation of the Law and the proclamation of the favor and grace of God, His heavenly Father, as it is written, Mark 1:1: The beginning of the Gospel of Jesus Christ, the Son of God. And shortly afterwards the chief heads are stated: Repentance and forgiveness of sins. Thus, when Christ after His resurrection commanded the apostles to preach the Gospel in all the world, Mark 16:15, He compressed the sum of this doctrine into a few words, when He said, Luke 24:46,47: Thus it is written, and thus it behooved Christ to suffer, and to rise from the dead the third day; and that repentance and remission of sins should be preached in His name among all nations. So Paul, too, calls his entire doctrine the Gospel, Acts 20:21; but he embraces the sum of this doctrine under the two heads: Repentance toward God and faith toward our Lord Jesus Christ.

5 And in this sense the generalis definitio, that is, the description of the word Gospel, when employed in a wide sense and without the proper distinction between the Law and the Gospel is correct, when it is said that the Gospel is a preaching of repentance and the remission of sins. For John, Christ, and the apostles began their preaching with repentance and explained and urged not only the gracious promise of the forgiveness of sins, but also the Law of God.

6 Furthermore the term Gospel is employed in another, namely, in its proper sense, by which it comprises not the preaching of repentance, but only the preaching of the grace of God, as follows directly afterwards, Mark 1:15, where Christ says: Repent, and believe the Gospel.

7 Likewise the term repentance also is not employed in the Holy Scriptures in one and the same sense. For in some passages of Holy Scripture it is employed and taken for the entire conversion of man, as Luke 13:5: Except ye repent, ye shall all likewise perish. And in 15:7: Likewise joy shalt be in heaven over one sinner that repenteth.

8 But in this passage, Mark 1:15, as also elsewhere, where repentance and faith in Christ, Acts 20:21, or repentance and remission of sins, Luke 24:46-047, are mentioned as distinct, to repent means nothing else than truly to acknowledge sins, to be heartily sorry for them, and to desist from them.

9 This knowledge comes from the Law, but is not sufficient for saving conversion to God, if faith in Christ be not added, whose merits the comforting preaching of the holy Gospel offers to all penitent sinners who are terrified by the preaching of the Law. For the Gospel proclaims the forgiveness of sins, not to coarse and secure hearts, but to the bruised or penitent, Luke 4:18. And lest repentance or the terrors of the Law turn into despair, the preaching of the Gospel must be added, that it may be a repentance unto salvation, 2 Cor. 7:10.

10 For since the mere preaching of the Law, without Christ, either makes presumptuous men, who imagine that they can fulfill the Law by outward works, or forces them utterly to despair, Christ takes the Law into His hands, and explains it spiritually, Matt. 5:21ff ; Rom. 7:14 and Rom 1:18, and thus reveals His wrath from heaven upon all sinners, and shows how great it is; whereby they are directed to the Law, and from it first learn to know their sins aright-a knowledge which Moses never could extort from them. For as the apostle testifies, 2 Cor. 3:14f, even though Moses is read, yet the veil which he put over his face is never lifted, so that they cannot understand the Law spiritually, and how great things it requires of us, and how severely it curses and condemns us because we cannot observe or fulfil it. Nevertheless, when it shalt turn to the Lord, the veil shalt be taken away, 2 Cor. 3:16.

11 Therefore the Spirit of Christ must not only comfort, but also through the office of the Law reprove the world of sin, John 16:8, and thus must do in the New Testament, as the prophet says, Is. 28:21, opus alienum, ut faciat opus proprium, that is, He must do the work of another (reprove), in order that He may [afterwards] do His own work, which is to comfort and preach of grace. For to this end He was earned [from the Father] and sent to us by Christ, and for this reason, too, He is called the Comforter, as Dr. Luther has explained in his exposition of the Gospel for the Fifth Sunday after Trinity, in the following words:

12 Anything that preaches concerning our sins and God’s wrath, let it be done how or when it will, that is all a preaching of the Law. Again, the Gospel is such a preaching as shows and gives nothing else than grace and forgiveness in Christ, although it is true and right that the apostles and preachers of the Gospel (as Christ Himself also did) confirm the preaching of the Law, and begin it with those who do not yet acknowledge their sins nor are terrified at [by the sense of] God’s wrath; as He says, John 16:8:

13 “The Holy Ghost will reprove the world of sin because they believe not on Me.” Yea, what more forcible, more terrible declaration and preaching of God’s wrath against sin is there than just the suffering and death of Christ, His Son? But as long as all this preaches God’s wrath and terrifies men, it is not yet the preaching of the Gospel nor Christ’s own preaching, but that of Moses and the Law against the impenitent. For the Gospel and Christ were never ordained and given for the purpose of terrifying and condemning, but of comforting and cheering those who are terrified and timid. And again: Christ says, John 16:8: “The Holy Ghost will reprove the world of sin”; which cannot be done except through the explanation of the Law. Jena, Tom. 2, fol. 455.

14 So, too, the Smalcald Articles say: The New Testament retains and urges the office of the Law, which reveals sins and God’s wrath; but to this office it immediately adds the promise of grace through the Gospel.

15 And the Apology says: To a true and salutary repentance the preaching of the Law alone is not sufficient, but the Gospel should be added thereto. Therefore the two doctrines belong together, and should also be urged by the side of each other, but in a definite order and with a proper distinction; and the Antinomians or assailants of the Law are justly condemned, who abolish the preaching of the Law from the Church, and wish sins to be reproved, and repentance and sorrow to be taught, not from the Law, but from the Gospel.

16 But in order that every one may see that in the dissent of which we are treating we conceal nothing, but present the matter to the eyes of the Christian reader plainly and clearly:

17 Therefore [we shall set forth our meaning:] we unanimously believe, teach, and confess that the Law is properly a divine doctrine, in which the righteous, immutable will of God is revealed, what is to be the quality of man in his nature, thoughts, words, and works, in order that he may be pleasing and acceptable to God; and it threatens its transgressors with God’s wrath and temporal and eternal punishments. For as Luther writes against the law-stormers [Antinomians]: Everything that reproves sin is and belongs to the Law, whose peculiar office it is to reprove sin and to lead to the knowledge of sins, Rom. 3:20,7:7; and as unbelief is the root and well-spring of all reprehensible sins [all sins that must be censured and reproved], the Law reproves unbelief also.

18 However, this is true likewise that the Law with its doctrine is illustrated and explained by the Gospel; and nevertheless it remains the peculiar office of the Law to reprove sins and teach concerning good works.

19 Thus, the Law reproves unbelief, [namely,] when men do not believe the Word of God. Now, since the Gospel, which alone properly teaches and commands to believe in Christ, is God’s Word, the Holy Ghost, through the office of the Law, also reproves unbelief, that men do not believe in Christ, although it is properly the Gospel alone which teaches concerning saving faith in Christ.

20 However, now that man has not kept the Law of God, but transgressed it, his corrupt nature, thoughts, words, and works fighting against it, for which reason he is under God’s wrath, death, all temporal calamities, and the punishment of hell-fire, the Gospel is properly a doctrine which teaches what man should believe, that he may obtain forgiveness of sins with God, namely, that the Son of God, our Lord Christ, has taken upon Himself and borne the curse of the Law, has expiated and paid for all our sins, through whom alone we again enter into favor with God, obtain forgiveness of sins by faith, are delivered from death and all the punishments of sins, and eternally saved.

21 For everything that comforts, that offers the favor and grace of God to transgressors of the Law, is, and is properly called, the Gospel, a good and joyful message that God will not punish sins, but forgive them for Christ’s sake.

22 Therefore every penitent sinner ought to believe, that is, place his confidence in the Lord Christ alone, that He was delivered for our offenses, and was raised again for our justification, Rom. 4:25, that He was made sin for us who knew no sin, that we might be made the righteousness of God in Him, 2 Cor. 5:21, who of God is made unto us Wisdom, and Righteousness, and Sanctification, and Redemption, 1 Cor. 1:30, whose obedience is counted to us for righteousness before God’s strict tribunal, so that the Law, as above set forth, is a ministration that kills through the letter and preaches condemnation, 2 Cor. 3:7, but the Gospel is the power of God unto salvation to every one that believeth, Rom. 1:16, that preaches righteousness and gives the Spirit, 1 Cor. 1:18; Gal. 3:2. As Dr. Luther has urged this distinction with especial diligence in nearly all his writings, and has properly shown that the knowledge of God derived from the Gospel is far different from that which is taught and learned from the Law, because even the heathen to a certain extent had a knowledge of God from the natural law, although they neither knew Him aright nor glorified Him aright, Rom. 1:20f.

23 From the beginning of the world these two proclamations [kinds of doctrines] have been ever and ever inculcated alongside of each other in the Church of God, with a proper distinction. For the descendants of the venerated patriarchs, as also the patriarchs themselves, not only called to mind constantly how in the beginning man had been created righteous and holy by God, and through the fraud of the Serpent had transgressed God’s command, had become a sinner, and had corrupted and precipitated himself with all his posterity into death and eternal condemnation, but also encouraged and comforted themselves again by the preaching concerning the Seed of the Woman, who would bruise the Serpent’s head, Gen. 3:15; likewise, concerning the Seed of Abraham, in whom all the nations of the earth shall be blessed, Gen. 22:18; likewise, concerning David’s Son, who should restore again the kingdom of Israel and be a light to the heathen, Ps. 110:1; Is. 49:6; Luke 2:32, who was wounded for our transgressions, and bruised for our iniquities, by whose stripes we are healed, Is. 53:5.

24 These two doctrines, we believe and confess, should ever and ever be diligently inculcated in the Church of God even to the end of the world, although with the proper distinction of which we have heard, in order that, through the preaching of the Law and its threats in the ministry of the New Testament the hearts of impenitent men may be terrified, and brought to a knowledge of their sins and to repentance; but not in such a way that they lose heart and despair in this process, but that (since the Law is a schoolmaster unto Christ that we might be justified by faith, Gal. 3:24, and thus points and leads us not from Christ, but to Christ, who is the end of the Law, Rom. 10:4)

25 they be comforted and strengthened again by the preaching of the holy Gospel concerning Christ, our Lord, namely, that to those who believe the Gospel, God forgives all their sins through Christ, adopts them as children for His sake, and out of pure grace, without any merit on their part, justifies and saves them, however, not in such a way that they may abuse the grace of God,

26 and sin hoping for grace, as Paul, 2 Cor. 3:7ff , thoroughly and forcibly shows the distinction between the Law and the Gospel.
27 Now, in order that both doctrines, that of the Law and that of the Gospel, be not mingled and confounded with one another, and what belongs to the one may not be ascribed to the other, whereby the merit and benefits of Christ are easily obscured and the Gospel is again turned into a doctrine of the Law, as has occurred in the Papacy, and thus Christians are deprived of the true comfort which they have in the Gospel against the terrors of the Law, and the door is again opened in the Church of God to the Papacy, therefore the true and proper distinction between the Law and the Gospel must with all diligence be inculcated and preserved, and whatever gives occasion for confusion inter legem et evangelium (between the Law and the Gospel), that is, whereby the two doctrines, Law and Gospel, may be confounded and mingled into one doctrine, should be diligently prevented. It is, therefore, dangerous and wrong to convert the Gospel, properly so called, as distinguished from the Law, into a preaching of repentance or reproof [a preaching of repentance, reproving sin]. For otherwise, if understood in a general sense of the entire doctrine, also the Apology says several times that the Gospel is a preaching of repentance and the forgiveness of sins. Meanwhile, however, the Apology also shows that the Gospel is properly the promise of the forgiveness of sins and of justification through Christ, but that the Law is a doctrine which reproves sins and condemns.

VI. The Third Use of the Law

1 Since the Law of God is useful, 1. not only to the end that external discipline and decency are maintained by it against wild, disobedient men; 2. likewise, that through it men are brought to a knowledge of their sins; 3. but also that, when they have been born anew by the Spirit of God, converted to the Lord, and thus the veil of Moses has been lifted from them, they live and walk in the law, a dissension has occurred between some few theologians concerning this third and last use of the Law.

2 For the one side taught and maintained that the regenerate do not learn the new obedience, or in what good works they ought to walk, from the Law, and that this teaching [concerning good works] is not to be urged thence [from the law], because they have been made free by the Son of God, have become the temples of His Spirit, and therefore do freely of themselves what God requires of them, by the prompting and impulse of the Holy Ghost, just as the sun of itself, without any [foreign] impulse, completes its ordinary course.

3 Over against this the other side taught: Although the truly believing are verily moved by God’s Spirit, and thus, according to the inner man, do God’s will from a free spirit, yet it is just the Holy Ghost who uses the written law for instruction with them, by which the truly believing also learn to serve God, not according to their own thoughts, but according to His written Law and Word, which is a sure rule and standard of a godly life and walk, how to order it in accordance with the eternal and immutable will of God.

4 For the explanation and final settlement of this dissent we unanimously believe, teach, and confess that although the truly believing and truly converted to God and justified Christians are liberated and made free from the curse of the Law, yet they should daily exercise themselves in the Law of the Lord, as it is written, Ps. 1:2;119:1: Blessed is the man whose delight is in the Law of the Lord, and in His Law doth he meditate day and night. For the Law is a mirror in which the will of God, and what pleases Him, are exactly portrayed, and which should [therefore] be constantly held up to the believers and be diligently urged upon them without ceasing.

5 For although the Law is not made for a righteous man, as the apostle testifies 1 Tim. 1:9, but for the unrighteous, yet this is not to be understood in the bare meaning, that the justified are to live without law. For the Law of God has been written in their heart, and also to the first man immediately after his creation a law was given according to which he was to conduct himself. But the meaning of St. Paul is that the Law cannot burden with its curse those who have been reconciled to God through Christ; nor must it vex the regenerate with its coercion, because they have pleasure in God’s Law after the inner man.

6 And, indeed, if the believing and elect children of God were completely renewed in this life by the indwelling Spirit, so that in their nature and all its powers they were entirely free from sin, they would need no law, and hence no one to drive them either, but they would do of themselves, and altogether voluntarily, without any instruction, admonition, urging or driving of the Law, what they are in duty bound to do according to God’s will; just as the sun, the moon, and all the constellations of heaven have their regular course of themselves, unobstructed, without admonition, urging, driving, force, or compulsion, according to the order of God which God once appointed for them, yea, just as the holy angels render an entirely voluntary obedience.

7 However, believers are not renewed in this life perfectly or completely, completive vel consummative [as the ancients say]; for although their sin is covered by the perfect obedience of Christ, so that it is not imputed to believers for condemnation, and also the mortification of the old Adam and the renewal in the spirit of their mind is begun through the Holy Ghost, nevertheless the old Adam clings to them still in their nature and all its internal and external powers.

8 Of this the apostle has written Rom. 7:18ff.: I know that in me [that is, in my flesh] dwelleth no good thing. And again: For that which I do I allow not; for what I would, that do I not; but what I hate, that I do; Likewise: I see another law in my members, warring against the law of my mind, and bringing me into captivity to the law of sin. Likewise, Gal. 5:17: The flesh lusteth against the spirit and the spirit against the flesh; and these are contrary the one to the other, so that ye cannot do the things that ye would.

9 Therefore, because of these lusts of the flesh the truly believing, elect, and regenerate children of God need in this life not only the daily instruction and admonition, warning, and threatening of the Law, but also frequently punishments, that they may be roused [the old man is driven out of them] and follow the Spirit of God, as it is written Ps. 119:71: It is good for me that I have been afflicted, that I might learn Thy statutes. And again, 1 Cor. 9:27: I keep under my body and bring it into subjection, lest that, by any means, when I have preached to others, I myself should be a castaway. And again, Heb. 12:8: But if ye be without chastisement, whereof all are partakers, then are ye bastards and not sons; as Dr. Luther has fully explained this at greater length in the Summer Part of the Church Postil, on the Epistle for the Nineteenth Sunday after Trinity.

10 But we must also explain distinctively what the Gospel does, produces, and works towards the new obedience of believers, and what is the office of the Law in this matter, as regards the good works of believers.

11 For the Law says indeed that it is God’s will and command that we should walk in a new life, but it does not give the power and ability to begin and do it; but the Holy Ghost, who is given and received, not through the Law, but through the preaching of the Gospel, Gal. 3:14, renews the heart.

12 Thereafter the Holy Ghost employs the Law so as to teach the regenerate from it, and to point out and show them in the Ten Commandments what is the [good and] acceptable will of God, Rom. 12:2, in what good works God hath before ordained that they should walk, Eph. 2:10. He exhorts them thereto, and when they are idle, negligent, and rebellious in this matter because of the flesh, He reproves them on that account through the Law, so that He carries on both offices together: He slays and makes alive; He leads into hell and brings up again. For His office is not only to comfort, but also to reprove, as it is written: When the Holy Ghost is come, He will reprove the world (which includes also the old Adam) of sin, and of righteousness, and of judgment.

13 But sin is everything that is contrary to God’s Law.

14 And St. Paul says: All Scripture given by inspiration of God is profitable for doctrine, for reproof, etc., and to reprove is the peculiar office of the Law. Therefore, as often as believers stumble, they are reproved by the Holy Spirit from the Law, and by the same Spirit are raised up and comforted again with the preaching of the Holy Gospel.

15 But in order that, as far as possible, all misunderstanding may be prevented, and the distinction between the works of the Law and those of the Spirit be properly taught and preserved it is to be noted with especial diligence that when we speak of good works which are in accordance with God’s Law (for otherwise they are not good works), then the word Law has only one sense, namely, the immutable will of God, according to which men are to conduct themselves in their lives.

16 The difference, however, is in the works, because of the difference in the men who strive to live according to this Law and will of God. For as long as man is not regenerate, and [therefore] conducts himself according to the Law and does the works because they are commanded thus, from fear of punishment or desire for reward, he is still under the Law, and his works are called by St. Paul properly works of the Law, for they are extorted by the Law, as those of slaves; and these are saints after the order of Cain [that is, hypocrites].

17 But when man is born anew by the Spirit of God, and liberated from the Law, that is, freed from this driver, and is led by the Spirit of Christ, he lives according to the immutable will of God comprised in the Law, and so far as he is born anew, does everything from a free, cheerful spirit; and these are called not properly works of the Law, but works and fruits of the Spirit, or as St. Paul names it, the law of the mind and the Law of Christ. For such men are no more under the Law, but under grace, as St. Paul says, Rom. 8:2 [Rom. 7:23; 1 Cor. 9:21 ].

18 But since believers are not completely renewed in this world, but the old Adam clings to them even to the grave, there also remains in them the struggle between the spirit and the flesh. Therefore they delight indeed in God’s Law according to the inner man, but the law in their members struggles against the law in their mind; hence they are never without the Law, and nevertheless are not under, but in the Law, and live and walk in the Law of the Lord, and yet do nothing from constraint of the Law.

19 But as far as the old Adam is concerned, which still clings to them, he must be driven not only with the Law, but also with punishments; nevertheless he does everything against his will and under coercion, no less than the godless are driven and held in obedience by the threats of the Law, 1 Cor. 9:27; Rom. 7:18. 19.

20 So, too, this doctrine of the Law is needful for believers, in order that they may not hit upon a holiness and devotion of their own, and under the pretext of the Spirit of God set up a self-chosen worship, without God’s Word and command, as it is written Deut. 12:8,28,32: Ye shall not do … every man whatsoever is right in his own eyes, etc., but observe and hear all these words which I command thee. Thou shalt not add thereto, nor diminish therefrom.

21 So, too, the doctrine of the Law, in and with [the exercise of] the good works of believers, is necessary for the reason that otherwise man can easily imagine that his work and life are entirely pure and perfect. But the Law of God prescribes to believers good works in this way, that it shows and indicates at the same time, as in a mirror, that in this life they are still imperfect and impure in us, so that we must say with the beloved Paul, 1 Cor. 4:4: I know nothing by myself; yet am I not hereby justified. Thus Paul, when exhorting the regenerate to good works, presents to them expressly the Ten Commandments, Rom. 13:9; and that his good works are imperfect and impure he recognizes from the Law, Rom. 7:7ff ; and David declares Ps. 119:32: Viam mandatorum tuorum cucurri, I will run the way of Thy commandments; but enter not into judgment with Thy servant, for in Thy sight shall no man living be justified, Ps. 143:2.

22 But how and why the good works of believers, although in this life they are imperfect and impure because of sin in the flesh, are nevertheless acceptable and well-pleasing to God, is not taught by the Law, which requires an altogether perfect, pure obedience if it is to please God. But the Gospel teaches that our spiritual offerings are acceptable to God through faith for Christ’s sake, 1 Pet. 2:5; Heb. 11:4ff.

23 In this way Christians are not under the Law, but under grace, because by faith in Christ the persons are freed from the curse and condemnation of the Law; and because their good works, although they are still imperfect and impure, are acceptable to God through Christ; moreover, because so far as they have been born anew according to the inner man, they do what is pleasing to God, not by coercion of the Law, but by the renewing of the Holy Ghost, voluntarily and spontaneously from their hearts; however, they maintain nevertheless a constant struggle against the old Adam.

24 For the old Adam, as an intractable, refractory ass, is still a part of them, which must be coerced to the obedience of Christ, not only by the teaching, admonition, force and threatening of the Law, but also oftentimes by the club of punishments and troubles, until the body of sin is entirely put off, and man is perfectly renewed in the resurrection, when he will need neither the preaching of the Law nor its threatenings and punishments, as also the Gospel any longer; for these belong to this [mortal and] imperfect life.

25 But as they will behold God face to face, so they will, through the power of the indwelling Spirit of God, do the will of God [the heavenly Father] with unmingled joy, voluntarily, unconstrained, without any hindrance, with entire purity and perfection, and will rejoice in it eternally.
26 Accordingly, we reject and condemn as an error pernicious and detrimental to Christian discipline, as also to true godliness, the teaching that the Law, in the above-mentioned way and degree, should not be urged upon Christians and the true believers, but only upon the unbelieving, unchristians, and impenitent.

VII. The Holy Supper

1 Although, in the opinion of some, the exposition of this article perhaps should not be inserted into this document, in which we intend to explain the articles which have been drawn into controversy among the theologians of the Augsburg Confession (from which the Sacramentarians soon in the beginning, when the Confession was first composed and presented to the Emperor at Augsburg in 1530, entirely withdrew and separated, and presented their own Confession), still, since some theologians, and others who boast [their adherence to] the Augsburg Confession, have, alas! during the last years, given their assent in this article to the Sacramentarians no longer secretly, but partly publicly and against their own conscience have endeavored to wrest forcibly and to pervert the Augsburg Confession as being in this article in entire harmony with the doctrine of the Sacramentarians, we neither can nor should omit our testimony by our confession of the divine truth also in this document, and must repeat the true sense and proper understanding of the words of Christ and of the Augsburg Confession with reference to this article, and [for we recognize it to be our duty], so far as in us lies, by God’s help, preserve it [this pure doctrine] also for posterity, and faithfully warn our hearers, together with other godly Christians, against this pernicious error, which is entirely contrary to the divine Word and the Augsburg Confession, and has been frequently condemned.
STATUS CONTROVERSIAE.
The Chief Controversy between Our Doctrine and that of the Sacramentarians In This Article.

2 Although some Sacramentarians strive to employ words that come as close as possible to the Augsburg Confession and the form and mode of speech in its [our] churches, and confess that in the Holy Supper the body of Christ is truly received by believers, still, when we insist that they state their meaning properly, sincerely, and clearly, they all declare themselves unanimously thus: that the true essential body and blood of Christ is absent from the consecrated bread and wine in the Holy Supper as far as the highest heaven is from the earth. For thus their own words run: Abesse Christi corpus et sanguinem a signis tanto intervallo dicimus, quanto abest terra ab altissimis coelis. That is: “We say that the body and blood of Christ are as far from the signs as the earth is distant from the highest heaven.”

3 Therefore they understand this presence of the body of Christ not as a presence here upon earth, but only respectu fidei (with respect to faith) [when they speak of the presence of the body and blood of Christ in the Supper, they do not mean that they are present upon earth, except with respect to faith], that is, that our faith, reminded and excited by the visible signs, just as by the Word preached, elevates itself and ascends above all heavens, and receives and enjoys the body of Christ, which is there in heaven present, yea, Christ Himself, together with all His benefits, in a manner true and essential, but nevertheless spiritual only. For [they hold that] as the bread and wine are here upon earth and not in heaven, so the body of Christ is now in heaven and not upon earth, and consequently nothing else is received by the mouth in the Holy Supper than bread and wine.

4 Now, originally, they alleged that the Lord’s Supper is only an external sign, by which Christians are known, and that nothing else is offered in it than mere bread and wine (which are bare signs [symbols] of the absent body of Christ). When this [figment] would not stand the test, they confessed that the Lord Christ is truly present in His Supper, namely per communicationem idiomatum (by the communication of attributes), that is, according to His divine nature alone, but not with His body and blood.

5 Afterwards, when they were forced by Christ’s words to confess that the body of Christ is present in the Supper, they still understood and declared it in no other way than spiritually [only of a spiritual presence], that is, of partaking through faith of His power, efficacy, and benefits, because [they say] through the Spirit of Christ, who is everywhere, our bodies, in which the Spirit of Christ dwells here upon earth, are united with the body of Christ, which is in heaven.

6 The consequence was that many great men were deceived by these fine, plausible words, when they alleged and boasted that they were of no other opinion than that the Lord Christ is present in His [Holy] Supper truly, essentially, and as one alive; but they understand this according to His divine nature alone, and not of His body and blood, which, they say, are now in heaven, and nowhere else, and that He gives us with the bread and wine His true body and blood to eat, to partake of them spiritually through faith, but not bodily with the mouth.

7 For they understand the words of the Supper: Eat, this is My body, not properly, as they read, according to the letter, but figurate, as figurative expressions, so that eating the body of Christ means nothing else than believing, and body is equivalent to symbol, that is, a sign or figure of the body of Christ, which is not in the Supper on earth, but only in heaven. The word is they interpret sacramentaliter seu modo significativo (sacramentally, or in a significative manner), nequis rem cum signis ita putet copulari, ut Christi quoque caro nunc in terris adsit modo quodam invisibili et incomprehensibili (in order that no one may regard the thing so joined with the signs that the flesh also of Christ is now present on earth in an invisible and incomprehensible manner);

8 that is, that the body of Christ is united with the bread sacramentally, or significatively, so that believing, godly Christians as surely partake spiritually of the body of Christ, which is above, in heaven, as they eat the bread with the mouth. But that the body of Christ is present here upon earth in the Supper essentially, although invisibly and incomprehensibly, and is received orally, with the consecrated bread, even by hypocrites or those who are Christians only in appearance [by name] this they are accustomed to execrate and condemn as a horrible blasphemy.

9 Over against this it is taught in the Augsburg Confession from God’s Word concerning the Lord’s Supper: That the true body and blood of Christ are truly present in the Holy Supper under the form of bread and wine, and are there dispensed and received; and the contrary doctrine is rejected (namely, that of the Sacramentarians, who presented their own Confession at the same time at Augsburg, that the body of Christ, because He has ascended to heaven, is not truly and essentially present here upon earth in the Sacrament [which denied the true and substantial presence of the body and blood of Christ in the Sacrament of the Supper administered on earth, namely, for the reason that Christ had ascended into heaven]);

10 even as this opinion is clearly expressed in Luther’s Small Catechism in the following words: The Sacrament of the Altar is the true body and blood of our Lord Jesus Christ under the bread and wine, for us Christians to eat and to drink, instituted by Christ Himself;

11 and in the Apology this is not only explained still more clearly, but also established by the passage from Paul, 1 Cor. 10:16, and by the testimony of Cyril, in the following words: The Tenth Article has been approved, in which we confess that in the Lord’s Supper the body and blood of Christ are truly and substantially present, and are truly tendered with the visible elements, bread and wine, to those who receive the Sacrament. For since Paul says: “The bread which we break is the communion of the body of Christ,” etc., it would follow, if the body of Christ were not, but only the Holy Ghost were truly present, that the bread is not a communion of the body, but of the Spirit of Christ. Besides, we know that not only the Romish, but also the Greek Church has taught the bodily presence of Christ in the Holy Supper. And testimony is produced from Cyril that Christ dwells also bodily in us in the Holy Supper by the communication of His flesh.

12 Afterwards, when those who at Augsburg delivered their own Confession concerning this article had allied themselves with the Confession of our churches [seemed to be willing to approve the Confession of our churches], the following Formula Concordiae, that is, articles of Christian agreement, between the Saxon theologians and those of Upper Germany was composed and signed at Wittenberg, in the year 1536, by Dr. Martin Luther and other theologians on both sides:

13 We have heard how Mr. Martin Bucer explained his own opinion, and that of the other preachers who came with him from the cities, concerning the holy Sacrament of the body and blood of Christ, namely, as follows:

14 They confess, according to the words of Irenaeus, that in this Sacrament there are two things, a heavenly and an earthly. Accordingly, they hold and teach that with the bread and wine the body and blood of Christ are truly and essentially present, offered, and received. And although they believe in no transubstantiation, that is, an essential transformation of the bread and wine into the body and blood of Christ, nor hold that the body and blood of Christ are included in the bread localiter, that is, locally, or are otherwise permanently united therewith apart from the use of the Sacrament, yet they concede that through the sacramental union the bread is the body of Christ, etc. [that when the bread is offered, the body of Christ is at the same time present, and is truly tendered].

15 For apart from the use, when the bread is laid aside and preserved in the sacramental vessel [the pyx], or is carried about in the procession and exhibited, as is done in popery, they do not hold that the body of Christ is present.

16 Secondly, they hold that the institution of this Sacrament made by Christ is efficacious in Christendom [the Church], and that it does not depend upon the worthiness or unworthiness of the minister who offers the Sacrament, or of the one who receives it. Therefore, as St. Paul says, that even the unworthy partake of the Sacrament, they hold that also to the unworthy the body and blood of Christ are truly offered, and the unworthy truly receive them, if [where] the institution and command of the Lord Christ are observed. But such persons receive them to condemnation, as St. Paul says; for they misuse the holy Sacrament, because they receive it without true repentance and without faith. For it was instituted for this purpose, that it might testify that to those who truly repent and comfort themselves by faith in Christ the grace and benefits of Christ are here applied, and that they are incorporated into Christ and are washed by His blood.

17 In the following year, when the chief theologians of the Augsburg Confession assembled from all Germany at Smalcald, and deliberated as to what to present in the Council concerning this doctrine of the Church, by common consent the Smalcald Articles were composed by Dr. Luther and signed by all the theologians, jointly and severally, in which the proper and true meaning is clearly expressed in short, plain words, which agree most accurately with the words of Christ, and every subterfuge and loophole is barred to

18 the Sacramentarians (who had interpreted [perverted] the Formula of Concord, that is, the above-mentioned articles of union, framed the preceding year, to their advantage, as saying that the body of Christ is offered with the bread in no other way than as it is offered, together with all His benefits, by the Word of the Gospel, and that by the sacramental union nothing else than the spiritual presence of the Lord Christ by faith is meant);

19 for they [the Smalcald Articles] declare: The bread and wine in the Holy Supper are the true body and blood of Jesus Christ, which are offered and received, not only by the godly, but also by godless Christians [those who have nothing Christian except the name].

20 Dr. Luther has also more amply expounded and confirmed this opinion from God’s Word in the Large Catechism, where it is written: What, then, is the Sacrament of the Altar? Answer: It is the true body and blood of our Lord Jesus Christ, in and under the bread and wine, which we Christians are commanded by the Word of Christ to eat and to drink.

21 And shortly after: It is the ‘Word,’ I say, which makes and distinguishes this Sacrament, so that it is not mere bread and wine, but is, and is called. the body and blood of Christ.

22 Again: With this Word you can strengthen your conscience and say: If a hundred thousand devils, together with all fanatics, should rush forward, crying, How can bread and wine be the body and blood of Christ? I know that all spirits and scholars together are not as wise as is the Divine Majesty in His little finger. Now, here stands the Word of Christ: “Take, eat; this is My body. Drink ye all of this; this is the new testament in My blood,” etc. Here we abide, and would like to see those who will constitute themselves His masters, and make it different from what He has spoken.

23 It is true, indeed, that if you take away the Word, or regard it without the Word, you have nothing but mere bread and wine. But if the words remain with them, as they shall and must, then, in virtue of the same, it is truly the body and blood of Christ. For as the lips of Christ say and speak, so it is, as He can never lie or deceive.

24 Hence it is easy to reply to all manner of questions about which at the present time men are disturbed, as, for instance, whether a wicked priest can administer and distribute the Sacrament, and such like other points. For here conclude and reply: Even though a knave take or distribute the Sacrament, he receives the true Sacrament, that is, the true body and blood of Christ, just as truly as he who receives or administers it in the most worthy manner. For it is not founded upon the holiness of men, but upon the Word of God. And as no saint upon earth, yea, no angel in heaven, can change bread and wine into the body and blood of Christ, so also can no one change or alter it, even though it be abused.

25 For the Word, by which it became a sacrament and was instituted, does not become false because of the person or his unbelief. For He does not say: If you believe or are worthy, you will receive My body and blood, but: “Take, eat and drink; this is My body and blood”;

26 likewise: “Do this” (namely, what I now do, institute, give, and bid you take). That is as much as to say, No matter whether you be worthy or unworthy, you have here His body and blood, by virtue of these words which are added to the bread and wine. This mark and observe well; for upon these words rest all our foundation, protection, and defense against all error and temptation that have ever come or may yet come.

27 Thus far the Large Catechism, in which the true presence of the body and blood of Christ in the Holy Supper is established from God’s Word; and this [presence] is understood not only of the believing and worthy, but also of the unbelieving and unworthy.

28 But inasmuch as this highly illumined man [Dr. Luther, the hero illumined with unparalleled and most excellent gifts of the Holy Ghost] foresaw in the Spirit that after his death some would endeavor to make him suspected of having receded from the above-mentioned doctrine and other Christian articles, he has appended the following protestation to his large Confession:

29 Since I see that as time wears on, sects and errors increase, and that there is no end to the rage and fury of Satan, in order that henceforth during my life or after my death some of them may not, in future, support themselves by me, and falsely quote my writings to strengthen their error as the Sacramentarians and Anabaptists begin to do, I mean by this writing to confess my faith, point by point [concerning all the articles of our religion], before God and all the world, in which I intend to abide until my death, and therein (so help me God!) to depart from this world and to appear before the judgment-seat of Jesus Christ.

30 And if after my death any one should say: If Dr. Luther were living now, he would teach and hold this or that article differently, for he did not sufficiently consider it, against this I say now as then, and then as now, that, by God’s grace, I have most diligently, compared all these articles with the Scriptures time and again [have examined them, not once, but very often, according to the standard of Holy Scripture], and often have gone over them, and would defend them as confidently as I have now defended the Sacrament of the Altar.

31 I am not drunk nor thoughtless; I know what I say; I also am sensible of what it means for me at the coming of the Lord Christ at the final judgment. Therefore I want no one to regard this as a jest or mere idle talk; it is a serious matter to me; for by God’s grace I know Satan a good deal; if he can pervert or confuse God’s Word, what will he not do with my words or those of another? Tom. 2, Wittenb., German, fol. 243.

32 After this protestation, Doctor Luther, of blessed memory, presents, among other articles, this also: In the same manner I also speak and confess (he says) concerning the Sacrament of the Altar, that there the body and blood of Christ are in truth orally eaten and drunk in the bread and wine, even though the priests [ministers] who administer it [the Lord’s Supper], or those who receive it, should not believe or otherwise misuse it. For it does not depend upon the faith or unbelief of men, but upon God’s Word and ordinance, unless they first change God’s Word and ordinance and interpret it otherwise, as the enemies of the Sacrament do at the present day, who, of course, have nothing but bread and wine; for they also do not have the words and appointed ordinance of God, but have perverted and changed them according to their own [false] notion. Fol. 245.

33 Dr. Luther, who, above others, certainly understood the true and proper meaning of the Augsburg Confession, and who constantly remained steadfast thereto till his end, and defended it, shortly before his death repeated his faith concerning this article with great zeal in his last Confession, where he writes thus: I rate as one concoction, namely, as Sacramentarians and fanatics, which they also are, all who will not believe that the Lord’s bread in the Supper is His true natural body, which the godless or Judas received with the mouth, as well as did St. Peter and all [other] saints; he who will not believe this (I say) should let me alone, and hope for no fellowship with me; this is not going to be altered [thus my opinion stands, which I am not going to change]. Tom. 2, Wittenb., German, fol. 252.

34 From these explanations, and especially from that of Dr. Luther as the leading teacher of the Augsburg Confession, every intelligent man who loves truth and peace, can undoubtedly perceive what has always been the proper meaning and understanding of the Augsburg Confession in regard to this article.

35 For the reason why, in addition to the expressions of Christ and St. Paul (the bread in the Supper is the body of Christ or the communion of the body of Christ), also the forms: under the bread, with the bread, in the bread [the body of Christ is present and offered], are employed, is that by means of them the papistical transubstantiation may be rejected and the sacramental union of the unchanged essence of the bread and of the body of Christ indicated;

36 just as the expression, Verbum caro factum est, The Word was made flesh [ John 1:14 ], is repeated and explained by the equivalent expressions: The Word dwelt among us; likewise [ Col 2:9 ]: In Him dwelleth all the fulness of the Godhead bodily; likewise [ Acts 10:38 ]: God was with Him; likewise [ 2 Cor. 5:19 ]: God was in Christ, and the like; namely, that the divine essence is not changed into the human nature, but the two natures, unchanged, are personally united. [These phrases repeat and declare the expression of John, above mentioned, namely, that by the incarnation the divine essence is not changed into the human nature, but that the two natures without confusion are personally united.]

37 Even as many eminent ancient teachers, Justin, Cyprian, Augustine, Leo, Gelasius, Chrysostom and others, use this simile concerning the words of Christ’s testament: This is My body, that just as in Christ two distinct, unchanged natures are inseparably united, so in the Holy Supper the two substances, the natural bread and the true natural body of Christ, are present together here upon earth in the appointed administration of the Sacrament.

38 Although this union of the body and blood of Christ with the bread and wine is not a personal union, as that of the two natures in Christ, but as Dr. Luther and our theologians, in the frequently mentioned Articles of Agreement [Formula of Concord] in the year 1536 and in other places call it sacramentatem unionem, that is, a sacramental union, by which they wish to indicate that, although they also employ the formas: in pane, sub pane, cum pane, that is, these distinctive modes of speech: in the bread, under the bread, with the bread, yet they have received the words of Christ properly and as they read, and have understood the proposition, that is, the words of Christ’s testament: Hoc est corpus meum, This is My body, not as a figuratam propositionem, but inusitatam (that is, not as a figurative, allegorical expression or comment, but as an unusual expression).

39 For thus Justin says: This we receive not as common bread and common drink; but as Jesus Christ, our Savior, through the Word of God became flesh, and on account of our salvation also had flesh and blood, so we believe that the food blessed by Him through the Word and prayer is the body and blood of our Lord Jesus Christ.

40 Likewise Dr. Luther also in his Large and especially in his last Confession concerning the Lord’s Supper with great earnestness and zeal defends the very form of expression which Christ used at the first Supper.

41 Now, since Dr. Luther is to be regarded as the most distinguished teacher of the churches which confess the Augsburg Confession, whose entire doctrine as to sum and substance is comprised in the articles of the frequently mentioned Augsburg Confession, and was presented to the Emperor Charles V, the proper meaning and sense of the oft-mentioned Augsburg Confession can and should be derived from no other source more properly and correctly than from the doctrinal and polemical writings of Dr. Luther.

42 And, indeed, this very opinion, just cited, is founded upon the only firm, immovable, and indubitable rock of truth, from the words of institution, in the holy, divine Word, and was thus understood, taught, and propagated by the holy evangelists and apostles, and their disciples and hearers.

43 For since our Lord and Savior Jesus Christ, concerning whom, as our only Teacher, this solemn command has been given from heaven to all men: Hunc audite, Hear ye Him, who is not a mere man or angel, neither true, wise, and mighty only, but the eternal Truth and Wisdom itself and Almighty God, who knows very well what and how He is to speak, and who also can powerfully effect and execute everything that He speaks and promises, as He says Luke 21:33: Heaven and earth shalt pass away, but My words shall not pass away; also Matt. 28:18: All power is given unto Me in heaven and in earth,-

44 Since, now, this true, almighty Lord, our Creator and Redeemer, Jesus Christ, after the Last Supper, when He is just beginning His bitter suffering and death for our sins, in those sad last moments, with great consideration and solemnity, in the institution of this most venerable Sacrament, which was to be used until the end of the world with great reverence and obedience [and humility], and was to be an abiding memorial of His bitter suffering and death and all His benefits, a sealing [and confirmation] of the New Testament, a consolation of all distressed hearts, and a firm bond of union of Christians with Christ, their Head, and with one another, in the ordaining and institution of the Holy Supper spake these words concerning the bread which He blessed and gave [to His disciples]: Take, eat; this is My body, which is given for you, and concerning the cup, or wine: This is My blood of the new testament, which is shed for many for the remission of sins;-

45 [Now, since this is so,] We are certainly in duty bound not to interpret and explain these words of the eternal, true, and almighty Son of God, our Lord, Creator, and Redeemer, Jesus Christ, differently, as allegorical, figurative, tropical expressions, according as it seems agreeable to our reason, but with simple faith and due obedience to receive the words as they read, in their proper and plain sense, and allow ourselves to be diverted therefrom [from this express testament of Christ] by no objections or human contradictions spun from human reason, however charming they may appear to reason.

46 Even as Abraham, when he hears God’s Word concerning offering his son, although, indeed, he had cause enough for disputing as to whether the words should be understood according to the letter or with a tolerable or mild interpretation, since they conflicted openly not only with all reason and with the divine and natural law, but also with the chief article of faith concerning the promised Seed, Christ, who was to be born of Isaac, nevertheless, just as previously, when the promise of the blessed Seed from Isaac was given him, he gave God the honor of truth, and most confidently concluded and believed that what God promised He could also do, although it appeared impossible to his reason; so also here he understands and believes God’s Word and command plainly and simply, as they read according to the letter, and commits the matter to God’s omnipotence and wisdom, which, he knows, has many more modes and ways to fulfil the promise of the Seed from Isaac than he can comprehend with his blind reason;-

47 Thus we, too, are simply to believe with all humility and obedience the plain, firm, clear, and solemn words and command of our Creator and Redeemer, without any doubt and disputation as to how it agrees with our reason or is possible. For these words were spoken by that Lord who is infinite Wisdom and Truth itself, and also can execute and accomplish everything which He promises.

48 Now, all the circumstances of the institution of the Holy Supper testify that these words of our Lord and Savior Jesus Christ, which in themselves are simple, plain, clear, firm, and indubitable, cannot and must not be understood otherwise than in their usual, proper, and common signification. For since Christ gives this command [concerning eating His body, etc.] at the table and at supper, there is indeed no doubt that He speaks of real, natural bread and of natural wine, also of oral eating and drinking, so that there can be no metaphor, that is, a change of meaning, in the word bread, as though the body of Christ were a spiritual bread or a spiritual food of souls.

49 Likewise, also Christ Himself takes care that there be no metonymy either, that is, that in the same manner there be no change of meaning in the word body, and that He does not speak concerning a sign of His body, or concerning an emblem [a symbol] or figurative body, or concerning the virtue of His body and the benefits which He has earned by the sacrifice of His body [for us], but of His true, essential body, which He delivered into death for us, and of His true, essential blood, which He shed for us on the tree [altar] of the cross for the remission of sins.

50 Now, surely there is no interpreter of the words of Jesus Christ as faithful and sure as the Lord Christ Himself, who understands best His words and His heart and opinion, and who is the wisest and most knowing for expounding them; and here, as in the making of His last will and testament and of His everabiding covenant and union, as elsewhere in [presenting and confirming] all articles of faith, and in the institution of all other signs of the covenant and of grace or sacraments, as [for example] circumcision, the various offerings in the Old Testament and Holy Baptism, He uses not allegorical, but entirely proper, simple, indubitable, and clear words; and in order that no misunderstanding can occur, He explains them more clearly with the words: Given for you, shed for you.

51 He also allows His disciples to rest in the simple, proper sense, and commands them that they should thus teach all nations to observe what He had commanded them, the apostles.

52 For this reason, too, all three evangelists, Matt. 26:26; Mark 14:22; Luke 22:19, and St. Paul, who received the same [the institution of the Lord’s Supper] after the ascension of Christ [from Christ Himself], 1 Cor. 11:24, unanimously and with the same words and syllables repeat concerning the consecrated and distributed bread these distinct, clear, firm, and true words of Christ: This is My body, altogether in one way, without any interpretation [trope, figure] and change. Therefore there is no doubt that also concerning

53 the other part of the Sacrament these words of Luke and Paul: This cup is the new testament in My blood, can have no other meaning than that which St. Matthew and St. Mark give: This (namely, that which you orally drink out of the cup) is My blood of the new testament, whereby I establish, seal, and confirm with you men this My testament and new covenant, namely, the forgiveness of sins.

54 So also that repetition, confirmation, and explanation of the words of Christ which St. Paul makes 1 Cor. 10:16, where he writes as follows: The cup of blessing which we bless, is it not the communion of the blood of Christ? The bread which we break, is it not the communion of the body of Christ? is to be considered with all diligence and seriousness [accurately], as an especially clear testimony of the true, essential presence and distribution of the body and blood of Christ in the Supper. From this we clearly learn that not only the cup which Christ blessed at the first Supper, and not only the bread which Christ broke and distributed, but also that which we break and bless, is the communion of the body and blood of Christ, so that all who eat this bread and drink of this cup truly receive, and are partakers of, the true body and blood of Christ.

55 For if the body of Christ were present and partaken of, not truly and essentially, but only according to its power and efficacy, the bread would have to be called, not a communion of the body, but of the Spirit, power, and benefits of Christ, as the Apology argues and concludes.

56 And if Paul were speaking only of the spiritual communion of the body of Christ through faith, as the Sacramentarians pervert this passage, he would not say that the bread, but that the spirit or faith, was the communion of the body of Christ. But as he says that the bread is the communion of the body of Christ, that all who partake of the consecrated bread also become partakers of the body of Christ, he must indeed be speaking, not of a spiritual, but of a sacramental or oral participation of the body of Christ, which is common to godly and godless Christians [Christians only in name].

57 This is shown also by the causes and circumstances of this entire exposition of St. Paul, in which he deters and warns those who ate of offerings to idols and had fellowship with heathen devil-worship, and nevertheless went also to the table of the Lord and became partakers of the body and blood of Christ, lest they receive the body and blood of Christ for judgment and condemnation to themselves. For since all those who become partakers of the consecrated and broken bread in the Supper have communion also with the body of Christ, St. Paul indeed cannot be speaking of spiritual communion with Christ, which no man can abuse, and against which also no one is to be warned.

58 Therefore also our dear fathers and predecessors, as Luther and other pure teachers of the Augsburg Confession, explain this statement of Paul with such words that it accords most fully with the words of Christ when they write thus: The bread which we break is the distributed body of Christ, or the common [communicated] body of Christ, distributed to those who receive the broken bread.

59 By this simple, well-founded exposition of this glorious testimony, 1 Cor. 10, we unanimously abide, and we are justly astonished that some are so bold as to venture now to cite this passage, which they themselves previously opposed to the Sacramentarians, as a foundation for their error, that in the Supper the body of Christ is partaken of spiritually only. [For thus they speak]: Panis est communicatio corporis Christi, hoc est, id, quo fit societas cum corpore Christi (quod est ecclesia), seu est medium, per quod fideles unimur Christo, sicut verbum evangelii fide apprehensum est medium, per quod Christo spiritualiter unimur et corpori Christi, quod est ecclesia, inserimur. Translated, this reads as follows: “The bread is the communion of the body of Christ, that is, it is that by which we have fellowship with the body of Christ, which is the Church, or it is the means by which we believers are united with Christ, just as the Word of the Gospel, apprehended by faith, is a means through which we are spiritually united to Christ and incorporated into the body of Christ, which is the Church.”

60 For that not only the godly, pious, and believing Christians, but also unworthy, godless hypocrites, as Judas and his ilk, who have no spiritual communion with Christ, and go to the Table of the Lord without true repentance and conversion to God, also receive orally in the Sacrament the true body and [true] blood of Christ, and by their unworthy eating and drinking grievously sin against the body and blood of Christ, St. Paul teaches expressly. For he says, 1 Cor. 11:27: Whosoever shall eat this bread, and drink this cup of the Lord, unworthily, sins not merely against the bread and wine, not merely against the signs or symbols and emblems of the body and blood, but shall be guilty of the body and blood of the Lord Jesus Christ, which, as there [in the Holy Supper] present, he dishonors, abuses, and disgraces, as the Jews, who in very deed violated the body of Christ and killed Him; just as the ancient Christian Fathers and church-teachers unanimously have understood and explained this passage.

61 There is, therefore, a two-fold eating of the flesh of Christ, one spiritual, of which Christ treats especially John 6:54, which occurs in no other way than with the Spirit and faith, in the preaching and meditation of the Gospel, as well as in the Lord’s Supper, and by itself is useful and salutary, and necessary at all times for salvation to all Christians; without which spiritual participation also the sacramental or oral eating in the Supper is not only not salutary, but even injurious and damning [a cause of condemnation].

62 But this spiritual eating is nothing else than faith, namely, to hear God’s Word (wherein Christ, true God and man, is presented to us, together with all benefits which He has purchased for us by His flesh given into death for us, and by His blood shed for us, namely, God’s grace, the forgiveness of sins, righteousness, and eternal life), to receive it with faith and appropriate it to ourselves, and in all troubles and temptations firmly to rely, with sure confidence and trust, and to abide in the consolation that we have a gracious God, and eternal salvation on account of the Lord Jesus Christ. [He who hears these things related from the Word of God, and in faith receives and applies; them to himself, and relies entirely upon this consolation (that we have God reconciled and life eternal on account of the Mediator, Jesus Christ),-he, I say, who with true confidence rests in the Word of the Gospel in all troubles and temptations, spiritually eats the body of Christ and drinks His blood.]

63 The other eating of the body of Christ is oral or sacramental, when the true, essential body and blood of Christ are also orally received and partaken of in the Holy Supper, by all who eat and drink the consecrated bread and wine in the Supper-by the believing as a certain pledge and assurance that their sins are surely forgiven them, and Christ dwells and is efficacious in them, but by the unbelieving for their judgment and condemnation,

64 as the words of the institution by Christ expressly declare, when at the table and during the Supper He offers His disciples natural bread and natural wine, which He calls His true body and true blood, at the same time saying: Eat and drink. For in view of the circumstances this command evidently cannot be understood otherwise than of oral eating and drinking, however, not in a gross, carnal, Capernaitic, but in a supernatural, incomprehensible way;

65 to which afterwards the other command adds still another and spiritual eating, when the Lord Christ says further: This do in remembrance of Me, where He requires faith [which is the spiritual partaking of Christ’s body).

66 Therefore all the ancient Christian teachers expressly, and in full accord with the entire holy Christian Church, teach, according to these words of the institution of Christ and the explanation of St. Paul, that the body of Christ is not only received spiritually by faith, which occurs also outside of [the use of] the Sacrament, but also orally, not only by believing and godly, but also by unworthy, unbelieving, false, and wicked Christians. As this is too long to be narrated here, we would, for the sake of brevity, have the Christian reader referred to the exhaustive writings of our theologians.

67 Hence it is manifest how unjustly and maliciously the Sacramentarian fanatics (Theodore Beza) deride the Lord Christ, St. Paul, and the entire Church in calling this oral partaking, and that of the unworthy, duos pilos caudae equinae et commentum, cuius vel ipsum Satanam pudeat, as also the doctrine concerning the majesty of Christ, excrementum Satanae, quo diabolus sibi ipsi et hominibus illudat, that is, they speak so horribly of it that a godly Christian man should be ashamed to translate it.

68 But it must [also] be carefully explained who are the unworthy guests of this Supper, namely, those who go to this Sacrament without true repentance and sorrow for their sins, and without true faith and the good intention of amending their lives, and by their unworthy oral eating of the body of Christ load themselves with damnation, that is, with temporal and eternal punishments, and become guilty of the body and blood of Christ.

69 For Christians who are of weak faith, diffident, troubled, and heartily terrified because of the greatness and number of their sins, and think that in this their great impurity they are not worthy of this precious treasure and the benefits of Christ, and who feel and lament their weakness of faith, and from their hearts desire that they may serve God with stronger, more joyful faith and pure obedience, they are the truly worthy guests for whom this highly venerable Sacrament [and sacred feast] has been especially instituted and appointed;

70 as Christ says, Matt. 11:28: Come unto Me, all ye that labor and are heavy laden, and I will give you rest. Also Matt. 9:12: They that be whole need not a physician, but they that be sick. Also [ 2 Cor. 12:9 ]: God’s strength is made perfect in weakness. Also [ Rom. 14:1 ]: Him that is weak in the faith receive ye [ Rom 14:3 ], for God hath received him. For whosoever believeth in the Son of God, be it with a strong or with a weak faith, has eternal life [ John 3:15f. ].

71 And worthiness does not depend upon great or small weakness or strength of faith, but upon the merit of Christ, which the distressed father of little faith [ Mark 9:24 ] enjoyed as well as Abraham, Paul, and others who have a joyful and strong faith.

72 Let the foregoing be said of the true presence and two-fold participation of the body and blood of Christ, which occurs either by faith, spiritually, or also orally, both by worthy and unworthy [which latter is common to worthy and unworthy].

73 Since a misunderstanding and dissension among some teachers of the Augsburg Confession also has occurred concerning consecration and the common rule, that nothing is a sacrament without the appointed use [or divinely instituted act], we have made a fraternal and unanimous declaration to one another also concerning this matter to the following purport,

74 namely, that not the word or work of any man produces the true presence of the body and blood of Christ in the Supper, whether it be the merit or recitation of the minister, or the eating and drinking or faith of the communicants; but all this should be ascribed alone to the power of Almighty God and the word, institution, and ordination of our Lord Jesus Christ.

75 For the true and almighty words of Jesus Christ which He spake at the first institution were efficacious not only at the first Supper, but they endure, are valid, operate, and are still efficacious [their force, power, and efficacy endure and avail even to the present], so that in all places where the Supper is celebrated according to the institution of Christ, and His words are used, the body and blood of Christ are truly present, distributed, and received, because of the power and efficacy of the words which Christ spake at the first Supper. For where His institution is observed and His words are spoken over the bread and cup [wine], and the consecrated bread and cup [wine] are distributed, Christ Himself, through the spoken words, is still efficacious by virtue of the first institution, through His word, which He wishes to be there repeated.

76 As Chrysostom says (in Serm. de Pass.) in his Sermon concerning the Passion: Christ Himself prepared this table and blesses it; for no man makes the bread and wine set before us the body and blood of Christ, but Christ Himself who was crucified for us. The words are spoken by the mouth of the priest, but by God’s power and grace, by the word, where He speaks: “This is My body,” the elements presented are consecrated in the Supper. And just as the declaration, Gen. 1:28: “Be fruitful, and multiply, and replenish the earth,” was spoken only once, but is ever efficacious in nature, so that it is fruitful and multiplies, so also this declaration [“This is My body; this is My blood”] was spoken once, but even to this day and to His advent it is efficacious, and works so that in the Supper of the Church His true body and blood are present.

77 Luther also [writes concerning this very subject in the same manner], Tom. VI, Jena, Fol. 99: This His command and institution have this power and effect that we administer and receive not mere bread and wine, but His body and blood, as His words declare: “This is My body,” etc.; “This is My blood,” etc., so that it is not our work or speaking, but the command and ordination of Christ that makes the bread the body, and the wine the blood, from the beginning of the first Supper even to the end of the world, and that through our service and office they are daily distributed.

78 Also, Tom. III, Jena, Fol. 446: Thus here also, even though I should pronounce over all bread the words: This is Christ’s body, nothing, of course, would result therefrom; but when in the Supper we say, according to His institution and command: “This is My body,” it is His body, not on account of our speaking or word uttered [because these words, when uttered, have this efficacy], but because of His command-that He has commanded us thus to speak and to do, and has united His command and act with our speaking.

79 Now, in the administration of the Holy Supper the words of institution are to be publicly spoken or sung before the congregation distinctly and clearly, and should in no way be omitted [and this for very many and the most important reasons.

80 First,] in order that obedience may be rendered to the command of Christ: This do [that therefore should not be omitted which Christ Himself did in the Holy Supper],

81 and [secondly] that the faith of the hearers concerning the nature and fruit of this Sacrament (concerning the presence of the body and blood of Christ, concerning the forgiveness of sins, and all benefits which have been purchased by the death and shedding of the blood of Christ, and are bestowed upon us in Christ’s testament) may be excited, strengthened, and confirmed by Christ’s Word,

82 and [besides] that the elements of bread and wine may be consecrated or blessed for this holy use, in order that the body and blood of Christ may therewith be administered to us to be eaten and to be drunk, as Paul declares [ 1 Cor. 10:16 ]: The cup of blessing which we bless, which indeed occurs in no other way than through the repetition and recitation of the words of institution.

83 However, this blessing, or the recitation of the words of institution of Christ alone does not make a sacrament if the entire action of the Supper, as it was instituted by Christ, is not observed (as when the consecrated bread is not distributed, received, and partaken of, but is enclosed, sacrificed, or carried about), but the command of Christ, This do (which embraces the entire action or administration in this Sacrament,

84 that in an assembly of Christians bread and wine are taken, consecrated, distributed, received, eaten, drunk, and the Lord’s death is shown forth at the same time) must be observed unseparated and inviolate, as also St. Paul places before our eyes the entire action of the breaking of bread or of distribution and reception, 1 Cor. 10:16.

85 [Let us now come also to the second point, of which mention was made a little before.] To preserve this true Christian doctrine concerning the Holy Supper, and to avoid and abolish manifold idolatrous abuses and perversions of this testament, the following useful rule and standard has been derived from the words of institution: Nihil habet rationem sacramenti extra usum a Christo institutum (“Nothing has the nature of a sacrament apart from the use instituted by Christ”) or extra actionem divinitus institutam (“apart from the action divinely instituted”). That is: If the institution of Christ be not observed as He appointed it, there is no sacrament. This is by no means to be rejected, but can and should be urged and maintained with profit in the Church of God.

86 And the use or action here does not mean chiefly faith, neither the oral participation only, but the entire external, visible action of the Lord’s Supper instituted by Christ, [to this indeed is required] the consecration, or words of institution, the distribution and reception, or oral partaking [manducation] of the consecrated bread and wine, [likewise the partaking] of the body and blood of Christ.

87 And apart from this use, when in the papistic mass the bread is not distributed, but offered up or enclosed, borne about, and exhibited for adoration, it is to be regarded as no sacrament; just as the water of baptism, when used to consecrate bells or to cure leprosy, or otherwise exhibited for worship, is no sacrament or baptism. For against such papistic abuses this rule has been set up at the beginning [of the reviving Gospel], and has been explained by Dr. Luther himself, Tom. IV, Jena.

88 Meanwhile, however, we must call attention also to this, that the Sacramentarians artfully and wickedly pervert this useful and necessary rule, in order to deny the true, essential presence and oral partaking of the body of Christ, which occurs here upon earth alike by the worthy and the unworthy, and interpret it as referring to the usus fidei, that is, to the spiritual and inner use of faith, as though it were no sacrament to the unworthy, and the partaking of the body occurred only spiritually, through faith, or as though faith made the body of Christ present in the Holy Supper, and therefore unworthy, unbelieving hypocrites did not receive the body of Christ as being present.

89 Now, it is not our faith that makes the sacrament, but only the true word and institution of our almighty God and Savior Jesus Christ, which always is and remains efficacious in the Christian Church, and is not invalidated or rendered inefficacious by the worthiness or unworthiness of the minister, nor by the unbelief of the one who receives it. Just as the Gospel, even though godless hearers do not believe it, yet is and remains none the less the true Gospel, only it does not work for salvation in the unbelieving; so, whether those who receive the Sacrament believe or do not believe, Christ remains none the less true in His words when He says: Take, eat: this is My body, and effects this [His presence] not by our faith, but by His omnipotence.

90 Accordingly, it is a pernicious, shameless error that some from a cunning perversion of this familiar rule ascribe more to our faith, which [in their opinion] alone renders present and partakes of the body of Christ, than to the omnipotence of our Lord and Savior, Jesus Christ.

91 Now, as regards the various imaginary reasons and futile counter-arguments of the Sacramentarians concerning the essential and natural attributes of a human body, concerning the ascension of Christ, concerning His departure from this world, and such like, inasmuch as these have one and all been refuted thoroughly and in detail, from God’s Word, by Dr. Luther in his controversial writings: Against the Heavenly Prophets, That These Words, “This Is My Body,” Still Stand Firm; likewise in his Large and his Small Confession concerning the Holy Supper [published some years afterwards], and in other of his writings, and inasmuch as since his death nothing new has been advanced by the factious spirits, we would for the sake of brevity have the Christian reader directed to them and have referred to them.

92 For that we neither will, nor can, nor should allow ourselves to be led away by thoughts of human wisdom, whatever outward appearance or authority they may have, from the simple, distinct, and clear sense of the Word and testament of Christ to a strange opinion, other than the words read, but that, in accordance with what is above stated, we understand and believe them simply, our reasons upon which we have rested in this matter ever since the controversy concerning

93 this article arose, are those which Dr. Luther himself, in the very beginning, presented against the Sacramentarians in the following words (Dr. Luther in his Large Confession concerning the Holy Supper): My reasons upon which I rest in this matter are the following:

94 1. The first is this article of our faith: Jesus Christ is essential, natural, true, perfect God and man in one person, inseparable and undivided.

95 2. The second, that God’s right hand is everywhere.

96 3. The third, that God’s Word is not false, nor does it lie.

97 4. The fourth, that God has and knows of many modes of being in any place, and not only the single one concerning which the fanatics talk flippantly, and which philosophers call localem, or local.

98 Also: The one body of Christ [says Luther] has a threefold mode or all three modes of being anywhere.

99 First, the comprehensible, bodily mode, as He went about bodily upon earth, when, according to His size, He vacated and occupied space [was circumscribed by a fixed place]. This mode He can still use whenever He will, as He did after the resurrection, and will use at the last day, as Paul says, 1 Tim. 6:15: “Which in His times He shall show, who is the blessed God [and only Potentate, the King of kings and Lord of lords].” And to the Colossians, 3:4: “When Christ, who is our Life, shall appear.” In this manner He is not in God or with the Father, neither in heaven, as the mad spirits dream; for God is not a bodily space or place. And this is what the passages how Christ leaves the world and goes to the Father refer to which the false spirits cite.

100 Secondly, the incomprehensible, spiritual mode, according to which He neither occupies nor vacates space, but penetrates all creatures wherever He pleases [according to His most free will]; as, to make an imperfect comparison, my sight penetrates and is in air, light, or water, and does not occupy or vacate space; as a sound or tone penetrates and is in air or water or board and wall, and also does not occupy or vacate space; likewise, as light and heat penetrate and are in air, water, glass, crystal, and the like, and also do not vacate or occupy space; and much more of the like [many comparisons of this matter could be adduced]. This mode He used when He rose from the closed [and sealed] sepulcher, and passed through the closed door [to His disciples], and in the bread and wine in the Holy Supper, and, as it is believed, when He was born of His mother [the most holy Virgin Mary].

101 Thirdly, the divine, heavenly mode, since He is one person with God, according to which, of course, all creatures must be far more penetrable and present to Him than they are according to the second mode. For if, according to that second mode, He can be in and with creatures in such a manner that they do not feel, touch, circumscribe, or comprehend Him, how much more wonderfully will He be in all creatures according to this sublime third mode, so that they do not circumscribe nor comprehend Him, but rather that He has them present before Himself, circumscribes and comprehends them! For you must place this being of Christ, who is one person with God [for you must place this mode of presence of Christ which He has by His personal union with God], very far, far outside of the creatures, as far as God is outside of them; and again as deep and near within all creatures as God is within them. For He is one inseparable person with God; where God is, there must He also be,

102 or our faith is false. But who will say or think how this occurs? We know indeed that it is so, that He is in God outside of all creatures, and one person with God, but how it occurs we do not know; it [this mystery] is above nature and reason, even above the reason of all the angels in heaven; it is understood and known only by God. Now, since it is unknown to us, and yet true, we should not deny His words before we know how to prove to a certainty that the body of Christ can by no means be where God is, and that this mode of being [presence] is false. This the fanatics must prove; but they will forego it.

103 Now, whether God has and knows still more modes in which Christ’s body is anywhere, I did not intend to deny herewith, but to indicate what awkward dolts our fanatics are, that they concede to the body of Christ no more than the first, comprehensible mode; although they cannot even prove that to be conflicting with our meaning. For in no way will I deny that the power of God may accomplish this much that a body might be in many places at the same time, even in a bodily, comprehensible way. For who will prove that this is impossible with God? Who has seen an end to His power? The fanatics indeed think thus: God cannot do it. But who will believe their thinking? With what do they make such thinking sure? Thus far Luther.

104 From these words of Dr. Luther this, too, is clear in what sense the word spiritual is employed in our churches with reference to this matter. For to the Sacramentarians this word spiritual means nothing else than the spiritual communion, when through faith true believers are in the Spirit incorporated into Christ, the Lord, and become true spiritual members of His body.

105 But when Dr. Luther or we employ this word spiritual in regard to this matter, we understand by it the spiritual, supernatural, heavenly mode, according to which Christ is present in the Holy Supper, working not only consolation and life in the believing, but also condemnation in the unbelieving; whereby we reject the Capernaitic thoughts of the gross [and] carnal presence which is ascribed to and forced upon our churches by the Sacramentarians against our manifold public protestations. In this sense we also say [wish the word spiritually to be understood when we say] that in the Holy Supper the body and blood of Christ are spiritually received, eaten, and drunk, although this participation occurs with the mouth, while the mode is spiritual.

106 Thus our faith in this article concerning the true presence of the body and blood of Christ in the Holy Supper is based upon the truth and omnipotence of the true, almighty God, our Lord and Savior Jesus Christ. These foundations are strong and firm enough to strengthen and establish our faith in all temptations concerning this article, and, on the contrary, to overthrow and refute all the counter-arguments and objections of the Sacramentarians, however agreeable and plausible they may be to our reason; and upon them a Christian heart also can securely and firmly rest and rely.

107 Accordingly, with heart and mouth we reject and condemn as false, erroneous, and misleading all errors which are not in accordance with, but contrary and opposed to, the doctrine above mentioned and founded upon God’s Word, such as,

108 1. The papistic transubstantiation, when it is taught that the consecrated or blessed bread and wine in the Holy Supper lose entirely their substance and essence, and are changed into the substance of the body and blood of Christ in such a way that only the mere form of bread and wine is left, or accidentia sine subiecto (the accidents without the object); under which form of the bread, which nevertheless is bread no longer, but according to their assertion has lost its natural essence, the body of Christ is present even apart from the administration of the Holy Supper, when the bread is enclosed in the pyx or is carried about for display and adoration. For nothing can be a sacrament without God’s command and the appointed use for which it is instituted in God’s Word, as was shown above.

109 2. We likewise reject and condemn all other papistic abuses of this Sacrament, as the abomination of the sacrifice of the mass for the living and dead.

110 3. Also, that contrary to the public command and institution of Christ only one form of the Sacrament is administered to the laity; as these papistic abuses have been thoroughly refuted by means of God’s Word and the testimonies of the ancient Church, in the common Confession and the Apology of our churches, the Smalcald Articles, and other writings of our theologians.

111 However, since we have undertaken in this document to present especially only our confession and explanation concerning the true presence of the body and blood of Christ against the Sacramentarians, some of whom shamelessly insinuate themselves into our churches under the name of the Augsburg Confession, we will also state and enumerate here especially the errors of the Sacramentarians, in order to warn our hearers to guard against and look out for them.

112 Accordingly, with heart and mouth we reject and condemn as false, erroneous, and misleading all Sacramentarian opiniones (opinions) and doctrines which are not in accordance with, but contrary and opposed to, the doctrine above presented and founded upon God’s Word:

113 1. As when they assert that the words of institution are not to be understood simply in their proper signification, as they read, of the true, essential presence of the body and blood of Christ in the Supper, but are to be wrested, by means of tropi (tropes) or figurative interpretations, to another new, strange sense. We hereby reject all such Sacramentarian opiniones (opinions) and self-contradictory notions [of which some even conflict with each other], however manifold and various they may be.

114 2. Also, that the oral participation of the body and blood of Christ in the Supper is denied [by the Sacramentarians], and it is taught, on the contrary, that the body of Christ in the Supper is partaken of only spiritually by faith, so that in the Supper our mouth receives only bread and wine.

115 3. Likewise, also, when it is taught that bread and wine in the Supper should be regarded as nothing more than tokens by which Christians are to recognize one another; or, 4. That they are only figures, similitudes, and representations (symbols, types] of the far-absent body of Christ, in such a manner that just as bread and wine are the outward food of our body, so also the absent body of Christ, with His merit, is the spiritual food of our souls.

116 5. Or that they are no more than tokens or memorials of the absent body of Christ, by which signs, as an external pledge, we should be assured that the faith which turns from the Supper and ascends beyond all heavens and there above becomes as truly participant of the body and blood of Christ as we truly receive with the mouth the external signs in the Supper; and that thus the assurance and confirmation of our faith occur in the Supper only through the external signs, and not through the true, present body and blood of Christ offered to us.

117 6. Or that in the Supper the power, efficacy, and merit of the far-absent body of Christ are distributed only to faith, and we thus become partakers of His absent body; and that, in this way just mentioned, unio sacramentalis, that is, the sacramental union, is to be understood de analogia signi et signati (with respect to the analogy of the sign and that which is signified), that is, as [far as] the bread and wine have a resemblance to the body and blood of Christ.

118 7. Or that the body and blood of Christ cannot be received and partaken of otherwise than only spiritually, by faith.

119 8. Likewise, when it is taught that because of His ascension into heaven Christ is so enclosed and circumscribed with His body in a definite place in heaven that with the same [His body] He cannot or will not be truly present with us in the Supper, which is celebrated according to the institution of Christ upon earth, but that He is as far and remote from it as heaven and earth are from one another, as some Sacramentarians have wilfully and wickedly falsified the text, Acts 3:21; oportet Christum coelum accipere, that is, Christ must occupy heaven, for the confirmation of their error, and instead thereof have rendered it: oportet Christum coelo capi, that is, Christ must be received or be circumscribed and enclosed by heaven or in heaven, in such a manner that in His human nature He can or will in no way be with us upon earth.

120 9. Likewise, that Christ has not promised the true, essential presence of His body and blood in His Supper, and that He neither can nor will afford it, because the nature and property of His assumed human nature could not suffer or admit of it.

121 10. Likewise, when it is taught that not only the Word and omnipotence of Christ, but faith, renders the body of Christ present in the Supper; on this account the words of institution in the administration of the Supper are omitted by some. For although the papistic consecration is justly rebuked and rejected, in which the power to produce a sacrament is ascribed to the speaking as the work of the priest, yet the words of institution can or should in no way be omitted in the administration of the Supper, as is shown in the preceding declaration.

122 11. Likewise, that believers are not to seek, by reason of the words of Christ’s institution, the body of Christ with the bread and wine of the Supper, but are directed with their faith away from the bread of the Supper to heaven, to the place where the Lord Christ is with His body, that they should become partakers of it there.

123 12. We reject also the teaching that unbelieving and impenitent, wicked Christians, who only bear the name of Christ, but do not have the right, true, living, and saving faith, receive in the Supper not the body and blood of Christ, but only bread and wine. And since there are only two kinds of guests found at this heavenly meal, the worthy and the unworthy, we reject also the distinction made among the unworthy [made by some who assert] that the godless Epicureans and scoffers at God’s Word, who are in the external fellowship of the Church, when using the Holy Supper, do not receive the body and blood of Christ for condemnation, but only bread and wine.

124 13. So, too, the teaching that worthiness consists not only in true faith, but in man’s own preparation.

125 14. Likewise, the teaching that even true believers, who have and keep a right, true, living faith, and yet lack the said sufficient preparation of their own, could, just as the unworthy guests, receive this Sacrament to condemnation.

126 15. Likewise, when it is taught that the elements or the visible species or forms of the consecrated bread and wine must be adored. However, no one, unless he be an Arian heretic, can and will deny that Christ Himself, true God and man, who is truly and essentially present in the Supper, should be adored in spirit and in truth in the true use of the same, as also in all other places, especially where His congregation is assembled.

127 16. We reject and condemn also all presumptuous, frivolous [sarcastically colored], blasphemous questions and expressions which are presented in a gross, carnal, Capernaitic way regarding the supernatural, heavenly mysteries of this Supper.

128 Other and additional antitheses, or rejected contrary doctrines, have been reproved and rejected in the preceding explanation, which, for the sake of brevity, we will not repeat here, and whatever other condemnable opiniones or erroneous opinions there may be still, over and above the foregoing, can be easily gathered and named from the preceding explanation; for we reject and condemn everything that is not in accordance with, but contrary and opposed to, the doctrine recorded above and thoroughly grounded in God’s Word.

90 Accordingly, it is a pernicious, shameless error that some from a cunning perversion of this familiar rule ascribe more to our faith, which [in their opinion] alone renders present and partakes of the body of Christ, than to the omnipotence of our Lord and Savior, Jesus Christ.

91 Now, as regards the various imaginary reasons and futile counter-arguments of the Sacramentarians concerning the essential and natural attributes of a human body, concerning the ascension of Christ, concerning His departure from this world, and such like, inasmuch as these have one and all been refuted thoroughly and in detail, from God’s Word, by Dr. Luther in his controversial writings: Against the Heavenly Prophets, That These Words, “This Is My Body,” Still Stand Firm; likewise in his Large and his Small Confession concerning the Holy Supper [published some years afterwards], and in other of his writings, and inasmuch as since his death nothing new has been advanced by the factious spirits, we would for the sake of brevity have the Christian reader directed to them and have referred to them.

92 For that we neither will, nor can, nor should allow ourselves to be led away by thoughts of human wisdom, whatever outward appearance or authority they may have, from the simple, distinct, and clear sense of the Word and testament of Christ to a strange opinion, other than the words read, but that, in accordance with what is above stated, we understand and believe them simply, our reasons upon which we have rested in this matter ever since the controversy concerning

93 this article arose, are those which Dr. Luther himself, in the very beginning, presented against the Sacramentarians in the following words (Dr. Luther in his Large Confession concerning the Holy Supper): My reasons upon which I rest in this matter are the following:

94 1. The first is this article of our faith: Jesus Christ is essential, natural, true, perfect God and man in one person, inseparable and undivided.

95 2. The second, that God’s right hand is everywhere.

96 3. The third, that God’s Word is not false, nor does it lie.

97 4. The fourth, that God has and knows of many modes of being in any place, and not only the single one concerning which the fanatics talk flippantly, and which philosophers call localem, or local.

98 Also: The one body of Christ [says Luther] has a threefold mode or all three modes of being anywhere.

99 First, the comprehensible, bodily mode, as He went about bodily upon earth, when, according to His size, He vacated and occupied space [was circumscribed by a fixed place]. This mode He can still use whenever He will, as He did after the resurrection, and will use at the last day, as Paul says, 1 Tim. 6:15: “Which in His times He shall show, who is the blessed God [and only Potentate, the King of kings and Lord of lords].” And to the Colossians, 3:4: “When Christ, who is our Life, shall appear.” In this manner He is not in God or with the Father, neither in heaven, as the mad spirits dream; for God is not a bodily space or place. And this is what the passages how Christ leaves the world and goes to the Father refer to which the false spirits cite.

100 Secondly, the incomprehensible, spiritual mode, according to which He neither occupies nor vacates space, but penetrates all creatures wherever He pleases [according to His most free will]; as, to make an imperfect comparison, my sight penetrates and is in air, light, or water, and does not occupy or vacate space; as a sound or tone penetrates and is in air or water or board and wall, and also does not occupy or vacate space; likewise, as light and heat penetrate and are in air, water, glass, crystal, and the like, and also do not vacate or occupy space; and much more of the like [many comparisons of this matter could be adduced]. This mode He used when He rose from the closed [and sealed] sepulcher, and passed through the closed door [to His disciples], and in the bread and wine in the Holy Supper, and, as it is believed, when He was born of His mother [the most holy Virgin Mary].

101 Thirdly, the divine, heavenly mode, since He is one person with God, according to which, of course, all creatures must be far more penetrable and present to Him than they are according to the second mode. For if, according to that second mode, He can be in and with creatures in such a manner that they do not feel, touch, circumscribe, or comprehend Him, how much more wonderfully will He be in all creatures according to this sublime third mode, so that they do not circumscribe nor comprehend Him, but rather that He has them present before Himself, circumscribes and comprehends them! For you must place this being of Christ, who is one person with God [for you must place this mode of presence of Christ which He has by His personal union with God], very far, far outside of the creatures, as far as God is outside of them; and again as deep and near within all creatures as God is within them. For He is one inseparable person with God; where God is, there must He also be,

102 or our faith is false. But who will say or think how this occurs? We know indeed that it is so, that He is in God outside of all creatures, and one person with God, but how it occurs we do not know; it [this mystery] is above nature and reason, even above the reason of all the angels in heaven; it is understood and known only by God. Now, since it is unknown to us, and yet true, we should not deny His words before we know how to prove to a certainty that the body of Christ can by no means be where God is, and that this mode of being [presence] is false. This the fanatics must prove; but they will forego it.

103 Now, whether God has and knows still more modes in which Christ’s body is anywhere, I did not intend to deny herewith, but to indicate what awkward dolts our fanatics are, that they concede to the body of Christ no more than the first, comprehensible mode; although they cannot even prove that to be conflicting with our meaning. For in no way will I deny that the power of God may accomplish this much that a body might be in many places at the same time, even in a bodily, comprehensible way. For who will prove that this is impossible with God? Who has seen an end to His power? The fanatics indeed think thus: God cannot do it. But who will believe their thinking? With what do they make such thinking sure? Thus far Luther.

104 From these words of Dr. Luther this, too, is clear in what sense the word spiritual is employed in our churches with reference to this matter. For to the Sacramentarians this word spiritual means nothing else than the spiritual communion, when through faith true believers are in the Spirit incorporated into Christ, the Lord, and become true spiritual members of His body.

105 But when Dr. Luther or we employ this word spiritual in regard to this matter, we understand by it the spiritual, supernatural, heavenly mode, according to which Christ is present in the Holy Supper, working not only consolation and life in the believing, but also condemnation in the unbelieving; whereby we reject the Capernaitic thoughts of the gross [and] carnal presence which is ascribed to and forced upon our churches by the Sacramentarians against our manifold public protestations. In this sense we also say [wish the word spiritually to be understood when we say] that in the Holy Supper the body and blood of Christ are spiritually received, eaten, and drunk, although this participation occurs with the mouth, while the mode is spiritual.

106 Thus our faith in this article concerning the true presence of the body and blood of Christ in the Holy Supper is based upon the truth and omnipotence of the true, almighty God, our Lord and Savior Jesus Christ. These foundations are strong and firm enough to strengthen and establish our faith in all temptations concerning this article, and, on the contrary, to overthrow and refute all the counter-arguments and objections of the Sacramentarians, however agreeable and plausible they may be to our reason; and upon them a Christian heart also can securely and firmly rest and rely.

107 Accordingly, with heart and mouth we reject and condemn as false, erroneous, and misleading all errors which are not in accordance with, but contrary and opposed to, the doctrine above mentioned and founded upon God’s Word, such as,

108 1. The papistic transubstantiation, when it is taught that the consecrated or blessed bread and wine in the Holy Supper lose entirely their substance and essence, and are changed into the substance of the body and blood of Christ in such a way that only the mere form of bread and wine is left, or accidentia sine subiecto (the accidents without the object); under which form of the bread, which nevertheless is bread no longer, but according to their assertion has lost its natural essence, the body of Christ is present even apart from the administration of the Holy Supper, when the bread is enclosed in the pyx or is carried about for display and adoration. For nothing can be a sacrament without God’s command and the appointed use for which it is instituted in God’s Word, as was shown above.

109 2. We likewise reject and condemn all other papistic abuses of this Sacrament, as the abomination of the sacrifice of the mass for the living and dead.

110 3. Also, that contrary to the public command and institution of Christ only one form of the Sacrament is administered to the laity; as these papistic abuses have been thoroughly refuted by means of God’s Word and the testimonies of the ancient Church, in the common Confession and the Apology of our churches, the Smalcald Articles, and other writings of our theologians.

111 However, since we have undertaken in this document to present especially only our confession and explanation concerning the true presence of the body and blood of Christ against the Sacramentarians, some of whom shamelessly insinuate themselves into our churches under the name of the Augsburg Confession, we will also state and enumerate here especially the errors of the Sacramentarians, in order to warn our hearers to guard against and look out for them.

112 Accordingly, with heart and mouth we reject and condemn as false, erroneous, and misleading all Sacramentarian opiniones (opinions) and doctrines which are not in accordance with, but contrary and opposed to, the doctrine above presented and founded upon God’s Word:

113 1. As when they assert that the words of institution are not to be understood simply in their proper signification, as they read, of the true, essential presence of the body and blood of Christ in the Supper, but are to be wrested, by means of tropi (tropes) or figurative interpretations, to another new, strange sense. We hereby reject all such Sacramentarian opiniones (opinions) and self-contradictory notions [of which some even conflict with each other], however manifold and various they may be.

114 2. Also, that the oral participation of the body and blood of Christ in the Supper is denied [by the Sacramentarians], and it is taught, on the contrary, that the body of Christ in the Supper is partaken of only spiritually by faith, so that in the Supper our mouth receives only bread and wine.

115 3. Likewise, also, when it is taught that bread and wine in the Supper should be regarded as nothing more than tokens by which Christians are to recognize one another; or,
4. That they are only figures, similitudes, and representations (symbols, types] of the far-absent body of Christ, in such a manner that just as bread and wine are the outward food of our body, so also the absent body of Christ, with His merit, is the spiritual food of our souls.

116 5. Or that they are no more than tokens or memorials of the absent body of Christ, by which signs, as an external pledge, we should be assured that the faith which turns from the Supper and ascends beyond all heavens and there above becomes as truly participant of the body and blood of Christ as we truly receive with the mouth the external signs in the Supper; and that thus the assurance and confirmation of our faith occur in the Supper only through the external signs, and not through the true, present body and blood of Christ offered to us.

117 6. Or that in the Supper the power, efficacy, and merit of the far-absent body of Christ are distributed only to faith, and we thus become partakers of His absent body; and that, in this way just mentioned, unio sacramentalis, that is, the sacramental union, is to be understood de analogia signi et signati (with respect to the analogy of the sign and that which is signified), that is, as [far as] the bread and wine have a resemblance to the body and blood of Christ.

118 7. Or that the body and blood of Christ cannot be received and partaken of otherwise than only spiritually, by faith.

119 8. Likewise, when it is taught that because of His ascension into heaven Christ is so enclosed and circumscribed with His body in a definite place in heaven that with the same [His body] He cannot or will not be truly present with us in the Supper, which is celebrated according to the institution of Christ upon earth, but that He is as far and remote from it as heaven and earth are from one another, as some Sacramentarians have wilfully and wickedly falsified the text, Acts 3:21; oportet Christum coelum accipere, that is, Christ must occupy heaven, for the confirmation of their error, and instead thereof have rendered it: oportet Christum coelo capi, that is, Christ must be received or be circumscribed and enclosed by heaven or in heaven, in such a manner that in His human nature He can or will in no way be with us upon earth.

120 9. Likewise, that Christ has not promised the true, essential presence of His body and blood in His Supper, and that He neither can nor will afford it, because the nature and property of His assumed human nature could not suffer or admit of it.

121 10. Likewise, when it is taught that not only the Word and omnipotence of Christ, but faith, renders the body of Christ present in the Supper; on this account the words of institution in the administration of the Supper are omitted by some. For although the papistic consecration is justly rebuked and rejected, in which the power to produce a sacrament is ascribed to the speaking as the work of the priest, yet the words of institution can or should in no way be omitted in the administration of the Supper, as is shown in the preceding declaration.

122 11. Likewise, that believers are not to seek, by reason of the words of Christ’s institution, the body of Christ with the bread and wine of the Supper, but are directed with their faith away from the bread of the Supper to heaven, to the place where the Lord Christ is with His body, that they should become partakers of it there.

123 12. We reject also the teaching that unbelieving and impenitent, wicked Christians, who only bear the name of Christ, but do not have the right, true, living, and saving faith, receive in the Supper not the body and blood of Christ, but only bread and wine. And since there are only two kinds of guests found at this heavenly meal, the worthy and the unworthy, we reject also the distinction made among the unworthy [made by some who assert] that the godless Epicureans and scoffers at God’s Word, who are in the external fellowship of the Church, when using the Holy Supper, do not receive the body and blood of Christ for condemnation, but only bread and wine.

124 13. So, too, the teaching that worthiness consists not only in true faith, but in man’s own preparation.

125 14. Likewise, the teaching that even true believers, who have and keep a right, true, living faith, and yet lack the said sufficient preparation of their own, could, just as the unworthy guests, receive this Sacrament to condemnation.

126 15. Likewise, when it is taught that the elements or the visible species or forms of the consecrated bread and wine must be adored. However, no one, unless he be an Arian heretic, can and will deny that Christ Himself, true God and man, who is truly and essentially present in the Supper, should be adored in spirit and in truth in the true use of the same, as also in all other places, especially where His congregation is assembled.

127 16. We reject and condemn also all presumptuous, frivolous [sarcastically colored], blasphemous questions and expressions which are presented in a gross, carnal, Capernaitic way regarding the supernatural, heavenly mysteries of this Supper.
128 Other and additional antitheses, or rejected contrary doctrines, have been reproved and rejected in the preceding explanation, which, for the sake of brevity, we will not repeat here, and whatever other condemnable opiniones or erroneous opinions there may be still, over and above the foregoing, can be easily gathered and named from the preceding explanation; for we reject and condemn everything that is not in accordance with, but contrary and opposed to, the doctrine recorded above and thoroughly grounded in God’s Word.

VIII. The Person of Christ

1 A controversy has also occurred among the theologians of the Augsburg Confession concerning the Person of Christ, which, however, did not first arise among them but sprang originally from the Sacramentarians [for which the Sacramentarians furnished the occasion].

2 For when Dr. Luther, in opposition to the Sacramentarians, had maintained the true, essential presence of the body and blood of Christ in the Supper with solid arguments from the words of institution, the objection was urged against him by the Zwinglians that, if the body of Christ were present at the same time in heaven and on earth in the Holy Supper, it could be no real, true human body; for such majesty was said to be peculiar to God alone, and the body of Christ not capable of it.

3 But while Dr. Luther contradicted and effectually refuted this, as his doctrinal and polemical writings concerning the Holy Supper show, which we hereby publicly confess [approve], as well as his doctrinal writings [and we wish this fact to be publicly attested],

4 some theologians of the Augsburg Confession after his death sought, though still unwilling to do so publicly and expressly, to confess themselves in agreement with the Sacramentarians concerning the Lord’s Supper; nevertheless they introduced and employed precisely the same false arguments concerning the person of Christ whereby the Sacramentarians dared to remove the true, essential presence of the body and blood of Christ from His Supper, namely, that nothing should be ascribed to the human nature in the person of Christ which is above or contrary to its natural, essential property; and on this account they have loaded the doctrine of Dr. Luther, and all those who follow it as in conformity with God’s Word, with the charge of almost all the ancient monstrous heresies.

5 To explain this controversy in a Christian way, in conformity with God’s Word, according to the guidance [analogy] of our simple Christian faith, and by God’s grace entirely to settle it, our unanimous doctrine, faith, and confession are as follows:

6 We believe, teach, and confess that the Son of God, although from eternity He has been a particular, distinct, entire divine person, and thus, with the Father and the Holy Ghost, true, essential, perfect God, nevertheless, in the fulness of time assumed also human nature into the unity of His person, not in such a way that there now are two persons or two Christs, but that Christ Jesus is now in one person at the same time true, eternal God, born of the Father from eternity, and a true man, born of the most blessed Virgin Mary, as it is written Rom. 9:5: Of whom, as concerning the flesh, Christ came, who is over all, God blessed forever.

7 We believe, teach, and confess that now, in this one undivided person of Christ, there are two distinct natures, the divine, which is from eternity, and the human, which in time was assumed into the unity of the person of the Son of God; which two natures in the person of Christ are never either separated from, or mingled with, one another, or changed the one into the other, but each abides in its nature and essence in the person of Christ to all eternity.

8 We believe, teach, and confess also that, as both natures mentioned remain unmingled and undestroyed in their nature and essence, each retains also its natural, essential properties, and does not lay them aside to all eternity, neither do the essential properties of the one nature ever become the essential properties of the other nature.

9 Accordingly, we believe, teach, and confess that to be almighty, eternal, infinite, to be of itself everywhere present at once naturally, that is, according to the property of its nature and its natural essence, and to know all things, are essential attributes of the divine nature, which never to eternity become essential properties of the human nature.

10 On the other hand, to be a corporeal creature, to be flesh and blood, to be finite and circumscribed, to suffer, to die, to ascend and descend, to move from one place to another, to suffer hunger, thirst, cold, heat, and the like, are properties of the human nature, which never become properties of the divine nature.

11 We believe, teach, and confess also that now, since the incarnation, each nature in Christ does not so subsist of itself that each is or constitutes a separate person, but that they are so united that they constitute one single person, in which the divine and the assumed human nature are and subsist at the same time, so that now, since the incarnation, there belongs to the entire person of Christ personally, not only His divine, but also His assumed human nature; and that, as without His divinity, so also without His humanity, the person of Christ or Filii Dei incarnati (of the incarnate Son of God), that is, of the Son of God who has assumed flesh and become man, is not entire. Hence Christ is not two distinct persons, but one single person, notwithstanding that two distinct natures are found in Him, unconfused in their natural essence and properties.

12 We believe, teach, and confess also that the assumed human nature in Christ not only has and retains its natural, essential properties, but that over and above these, through the personal union with the Deity, and afterwards through glorification, it has been exalted to the right hand of majesty, power, and might, over everything that can be named, not only in this world, but also in that which is to come [ Eph. 1:21 ].

13 Now as regards this majesty, to which Christ has been exalted according to His humanity, He did not first receive it when He arose from the dead and ascended into heaven, but when He was conceived in His mother’s womb and became man, and the divine and human natures were personally united with one another.

14 However, this personal union is not to be understood, as some incorrectly explain it, as though the two natures, the divine and the human, were united with one another, as two boards are glued together, so that they realiter, that is, in deed and truth, have no communion whatever with one another.

15 For this was the error and heresy of Nestorius and Samosatenus, who, as Suidas and Theodore, presbyter of Raithu, testify, taught and held: duvo fuvsei” ajkoinwnhvtou” prov” eJauta;” pantavpasin, hoc est, naturas omni modo incommunicables esse, that is, that the two natures have no communion whatever with one another. Thereby the natures are separated from one another, and thus two Christs are constituted, so that Christ is one, and God the Word, who dwells in Christ, another.

16 For thus Theodore the Presbyter writes: Paulus quidam iisdem, quibus Manes temporibus, Samosatenus quidem ortu, sed Antiochiae Syriae antistes, Dominum impie dixit nudum fuisse hominem, in quo Deus Verbum sicut et in singulis prophetis habitavit [habitaverit], ac proinde duas naturas separatas et citra omnem prorsus inter se communionem in Christo esse, quasi alius sit Christus, alius Deus Verbum in ipso habitans. That is: At the same time in which also the heretic Manes lived, one by the name of Paul, who, though born in Samosata, was a bishop at Antioch in Syria, wickedly taught that the Lord Christ was nothing else than a mere man in whom God the Word dwelt, just as in every prophet; therefore he also held that the divine and human natures are apart from one another and separate, and that in Christ they have no communion whatever with one another, just as though Christ were one, and God the Word, who dwells in Him, the other.

17 Against this condemned heresy the Christian Church always and at all times has simply believed and held that the divine and the human nature in the person of Christ are so united that they have a true communion with one another, whereby the natures [do not meet and] are not mingled in one essence, but, as Dr. Luther writes, in one person.

18 Accordingly, on account of this personal union and communion, the ancient teachers of the Church, before and after the Council of Chalcedon, frequently employed the word mixtio, mixture, in a good sense and with [true] discrimination. For proof of this, many testimonies of the Fathers, if necessary, could be adduced, which are to be found frequently also in the writings of our divines, and which explain the personal union and communion by the illustration animae et corporis and ferri candentis, that is, of the soul and body, and of glowing iron.

19 For the body and soul, as also fire and iron, have communion with each other, not per phrasin, or modum loquendi, or verbaliter (by a phrase or mode of speaking, or in mere words), that is, so that it is to be a mere form of speech and mere words, but vere and realiter (truly and really), that is, in deed and truth; and, nevertheless, no confusio or exaequatio naturarum, that is, a mixing or equalizing of the natures, is thereby introduced, as when hydromel is made from honey and water, which is no longer pure water or pure honey, but a mixed drink. Now, in the union of the divine and the human nature in the person of Christ it is far different. For it is a far different, more sublime, and [altogether] ineffable communion and union between the divine and the human nature in the person of Christ, on account of which union and communion God is man and man is God, yet neither the natures nor their properties are thereby intermingled, but each nature retains its essence and properties.

20 On account of this personal union, which cannot be thought of nor exist without such a true communion of the natures, not the mere human nature, whose property it is to suffer and die, has suffered for the sins of the world, but the Son of God Himself truly suffered, however, according to the assumed human nature, and (in accordance with our simple Christian faith) [as our Apostles’ Creed testifies] truly died, although the divine nature can neither suffer nor die.

21 This Dr. Luther has fully explained in his Large Confession concerning the Holy Supper in opposition to the blasphemous alloeosis of Zwingli, who taught that one nature should be taken and understood for the other, which Dr. Luther committed, as a devil’s mask, to the abyss of hell.

22 For this reason, then, the ancient teachers of the Church combined both words, koinwniva and e{nwsi”, communio et unio, that is, communion and union, in the explanation of this mystery, and have explained the one by the other. Irenaeus, lib. 4, chap. 37; Athanasius, in the Letter to Epictetus; Hilary, Concerning the Trinity, Book 9; Basil and Gregory of Nyssa, in Theodoret; Damascenus, Book 3, chap. 19.

23 On account of this personal union and communion of the divine and the human nature in Christ we believe, teach, and confess also, according to our simple Christian faith, what is said concerning the majesty of Christ according to His humanity, [by which He sits] at the right hand of the almighty power of God, and what is connected therewith [follows therefrom]; all of which would be naught and could not stand if this personal union and communion of the natures in the person of Christ did not exist realiter, that is, in deed and truth.

24 On account of this personal union and communion of the natures, Mary, the most blessed Virgin, bore not a mere man, but, as the angel [Gabriel] testifies, such a man as is truly the Son of the most high God, who showed His divine majesty even in His mother’s womb, inasmuch as He was born of a virgin, with her virginity inviolate. Therefore she is truly the mother of God, and nevertheless remained a virgin.

25 In virtue of this He also wrought all His miracles, and manifested this His divine majesty, according to His pleasure, when and as He willed, and therefore not first after His resurrection and ascension only, but also in His state of humiliation; for example, at the wedding at Cana of Galilee; also, when He was twelve years old, among the learned; also in the garden, when with a word He cast His enemies to the ground; likewise in death, when He died not simply as any other man, but in and with His death conquered sin, death, devil, hell, and eternal damnation; which the human nature alone would not have been able to do if it had not been thus personally united and had not had communion with the divine nature.

26 Hence also the human nature, after the resurrection from the dead, has its exaltation above all creatures in heaven and on earth; which is nothing else than that He entirely laid aside the form of a servant, and yet did not lay aside His human nature, but retains it to eternity, and is put in the full possession and use of the divine majesty according to His assumed human nature. However, this majesty He had immediately at His conception, even in His mother’s womb, but, as the apostle testifies [ Phil. 2:7 ], laid it aside; and, as Dr. Luther explains, He kept it concealed in the state of His humiliation, and did not employ it always, but only when He wished.

27 But now He does, since He has ascended, not merely as any other saint, to heaven, but, as the apostle testifies [ Eph. 4:10 ], above all heavens, and also truly fills all things, and being everywhere present, not only as God, but also as man [has dominion and] rules from sea to sea and to the ends of the earth; as the prophets predict, Ps. 8:1,6; 93:1f ; Zech. 9:10, and the apostles testify, Mark 16:20, that He everywhere wrought with them and confirmed their word with signs following.

28 Yet this occurred not in an earthly way, but, as Dr. Luther explains, according to the manner of the right hand of God, which is no fixed place in heaven, as the Sacramentarians assert without any ground in the Holy Scriptures, but nothing else than the almighty power of God, which fills heaven and earth, in [possession of] which Christ is installed according to His humanity, realiter, that is, in deed and truth, sine confusione et exaequatione naturarum, that is, without confusion and equalizing of the two natures in their essence and essential properties;

29 by this communicated [divine] power, according to the words of His testament, He can be and truly is present with His body and blood in the Holy Supper, to which He has directed us by His Word; this is possible to no other man, because no man is in such a way united with the divine nature, and installed in such divine almighty majesty and power through and in the personal union of the two natures in Christ, as Jesus, the Son of Mary.

30 For in Him the divine and the human nature are personally united with one another, so that in Christ dwelleth all the fulness of the Godhead bodily, Col. 2:9, and in this personal union have such a sublime, intimate, ineffable communion that even the angels are astonished at it, and, as St. Peter testifies, have their delight and joy in looking into it [ 1 Pet. 1:12 ]; all of which will shortly be explained in order and somewhat more fully.

31 From this basis of the personal union, as it has been stated and explained above, that is, from the manner in which the divine and the human nature in the person of Christ are united with one another, namely, that they have not only the names in common, but have also in deed and truth communion with one another, without any commingling or equalizing of the same in their essences, flows also the doctrine de communicatione idiomatum, that is, concerning the true communion of the properties of the natures, of which more is to be said hereafter.

32 For since this is verily so, quod propria non egrediantur sua subiecta (that properties do not leave their subjects), that is, that each nature retains its essential properties, and these are not separated from the nature and poured into the other nature, as water from one vessel into another, so also no communion of properties could be or subsist if the above-mentioned personal union or communion of the natures in the person of Christ were not true.

33 Next to the article of the Holy Trinity this is the greatest mystery in heaven and on earth, as Paul says: Without controversy, great is the mystery of godliness, that God was manifest in the flesh, 1 Tim. 3:16.

34 For since the Apostle Peter in clear words testifies [ 2 Pet. 1:4 ] that we also, in whom Christ dwells only by grace, on account of that sublime mystery, are in Christ, partakers of the divine nature, what kind of communion of the divine nature, then, must that be of which the apostle says that in Christ dwelt all the fulness of the Godhead bodily, so that God and man are one person?

35 But since it is highly important that this doctrine de communicatione idiomatum, that is, of the communion of the properties of both natures, be treated and explained with proper discrimination,-for the propositiones or praedicationes, that is, how to speak of the person of Christ, and of its natures and properties, are not all of one kind and mode, and when they are employed without proper discrimination, the doctrine becomes confused and the simple reader is easily led astray,-the following explanation should be carefully noted, which, for the purpose of making it plainer and simple, may well be comprised under three heads:

36 Namely, first, since in Christ two distinct natures exist and remain unchanged and unconfused in their natural essence and properties, and yet of both natures there is only one person, hence, that which is, indeed, an attribute of only one nature is ascribed not to that nature alone, as separate, but to the entire person, which is at the same time God and man (whether it is called God or man).

37 But in hoc genere, that is, in this mode of speaking, it does not follow that what is ascribed to the person is at the same time a property of both natures, but it is distinctively explained what nature it is according to which anything is ascribed to the person. Thus the Son of God was born of the seed of David according to the flesh, Rom. 1:3. Also: Christ was put to death according to the flesh, and hath suffered for us in, or according to, the flesh, 1 Pet. 3:18;4:1.

38 However, since beneath the words, when it is said that what is peculiar to one nature is ascribed to the entire person, secret and open Sacramentarians conceal their pernicious error, by naming indeed the entire person, but understanding thereby nevertheless only the one nature, and entirely excluding the other nature, as though the mere human nature had suffered for us, as Dr. Luther in his Large Confession concerning the Holy Supper has written concerning the alloeosis of Zwingli, we will here set down Luther’s own words, in order that the Church of God may be guarded in the best way against this error. His words are as follows:

39 Zwingli calls that an alloeosis when something is said of the divinity of Christ which really belongs to the humanity, or vice versa. As Luke 24:26: “Ought not Christ to have suffered these things, and to enter into His glory?” Here Zwingli juggles, asserting that [the word] Christ is understood of the human nature.

40 Beware, beware, I say, of the alloeosis! For it is a devil’s mask, for at last it manufactures such a Christ after whom I certainly would not be a Christian; namely, that henceforth Christ should be no more and do no more with His sufferings and life than any other mere saint. For if I believe this [permit myself to be persuaded] that only the human nature has suffered for me, then Christ is to me a poor Savior, then He Himself indeed needs a Savior. In a word, it is unspeakable what the devil seeks by the alloeosis.

41 And shortly afterwards: If the old weather-witch, Dame Reason, the grandmother of the alloeosis, would say, Yea, divinity cannot suffer nor die; you shall reply, That is true; yet, because in Christ divinity and humanity are one person, Scripture, on account of this personal union, ascribes also to divinity everything that happens to the humanity, and vice versa.

42 And it is so in reality; for you must certainly answer this, that the person (meaning Christ) suffers and dies. Now the person is true God; therefore it is rightly said: The Son of God suffers. For although the one part (to speak thus), namely, the divinity, does not suffer, yet the person, which is God, suffers in the other part, namely, in His humanity; for in truth God’s Son has been crucified for us, that is, the person which is God. For the person, the person, I say, was crucified according to the humanity.

43 And again, shortly afterwards: If the alloeosis is to stand as Zwingli teaches it, then Christ will have to be two persons, one divine and one human, because Zwingli applies the passages concerning suffering to the human nature alone, and diverts them entirely from the divinity. For if the works be parted and separated, the person must also be divided, since all the works or sufferings are ascribed not to the natures, but to the person. For it is the person that does and suffers everything, one thing according to one nature, and another according to the other nature, all of which the learned know well. Therefore we regard our Lord Christ as God and man in one person, non confundendo naturas nec dividendo personam, so that we neither confound the natures nor divide the person.

44 Dr. Luther says also in his book Of the Councils and the Church: We Christians must know that if God is not also in the balance, and gives the weight, we sink to the bottom with our scale. By this I mean: If it were not to be said [if these things were not true], God has died for us, but only a man, we would be lost. But if “God’s death” and “God died” lie in the scale of the balance, then He sinks down, and we rise up as a light, empty scale. But indeed He can also rise again or leap out of the scale; yet He could not sit in the scale unless He became a man like us, so that it could be said: “God died,” “God’s passion,” “God’s blood,” “God’s death.” For in His nature God cannot die; but now that God and man are united in one person, it is correctly called God’s death, when the man dies who is one thing or one person with God. Thus far Luther.

45 Hence it is manifest that it is incorrect to say or write that the above-mentioned expressions (God suffered, God died) are only praedicationes verbales (verbal assertions), that is, mere words, and that it is not so in fact. For our simple Christian faith proves that the Son of God, who became man, suffered for us, died for us and redeemed us with His blood.

46 Secondly, as to the execution of the office of Christ, the person does not act and work in, with, through, or according to only one nature, but in, according to, with, and through both natures, or, as the Council of Chalcedon expresses it, one nature operates in communion with the other what is a property of each.

47 Therefore Christ is our Mediator, Redeemer, King, High Priest, Head, Shepherd, etc., not according to one nature only, whether it be the divine or the human, but according to both natures, as this doctrine has been treated more fully in other places.

48 Thirdly, however, it is still a much different thing when the question, declaration, or discussion is, whether the natures in the personal union in Christ have nothing else or nothing more than only their natural, essential properties; for that they have and retain these has been mentioned above.

49 Now, as regards the divine nature in Christ, since in God there is no change, Jas. 1:17, His divine nature, in its essence and properties, suffered no subtraction nor addition by the incarnation; was not, in or by itself, either diminished or increased thereby.

50 But as regards the assumed human nature in the person of Christ, some have indeed wished to contend that even in the personal union with divinity it has nothing else and nothing more than only its natural, essential properties according to which it is in all things like its brethren; and that, on this account, nothing should or could be ascribed to the human nature in Christ which is beyond, or contrary to, its natural properties, even though the testimony of Scripture is to that effect.

51 But that this opinion is false and incorrect is so clear from God’s Word that even their own associates rebuke and reject this error. For the Holy Scriptures, and the ancient Fathers from the Scriptures [in which they were fully trained], testify forcefully that, for the reason and because of the fact that it has been personally united with the divine nature in Christ, the human nature in Christ, when it was glorified and exalted to the right hand of the majesty and power of God, after the form of a servant and humiliation had been laid aside, did receive, apart from, and over and above its natural, essential, permanent properties, also special, high, great, supernatural, inscrutable, ineffable, heavenly prerogativas (prerogatives) and excellences in majesty, glory, power, and might above everything that can be named, not only in this world, but also in that which is to come [Eph. 1:21]; and that, accordingly, in the operations of the office of Christ: the human nature in Christ, in its measure and mode, is equally employed [at the same time], and has also its efficaciam, that is, power and, efficacy, not only from, and according to, its natural, essential attributes, or only so far as their ability extends, but chiefly from, and according to, the majesty, glory, power, and might which it has received through the personal union, glorification, and exaltation.

52 And nowadays even the adversaries can or dare scarcely deny this, except that they dispute and contend that those are only created gifts or finitae qualitates (finite qualities), as in the saints, with which the human nature in Christ is endowed and adorned; and that, according to their [crafty] thoughts or from their own [silly] argumentationes (argumentations) or [fictitious] proofs, they wish to measure and calculate of what the human nature in Christ could or should be capable or incapable without becoming annihilated.

53 But the best, most certain, and surest way in this controversy is this, namely, that what Christ has received according to His assumed human nature through the personal union, glorification, or exaltation, and of what His assumed human nature is capable beyond the natural properties, without becoming annihilated, no one can know better or more thoroughly than the Lord Christ Himself; and He has revealed it in His Word, as much as is needful for us to know of it in this life. Now, everything for which we have in this instance clear, certain testimonies in the Scriptures, we must simply believe, and in no way argue against it, as though the human nature in Christ could not be capable of the same.

54 Now it is indeed correct and true what has been said concerning the created gifts which have been given and imparted to the human nature in Christ, that it possesses them in or of itself. But these do not reach unto the majesty which the Scriptures, and the ancient Fathers from Scripture, ascribe to the assumed human nature in Christ.

55 For to quicken, to have all judgment and all power in heaven and on earth, to have all things in His hands, to have all things in subjection beneath His feet, to cleanse from sin, etc., are not created gifts, but divine, infinite properties; and yet, according to the declaration of Scripture, these have been given and communicated to the man Christ, John 5:27; 6:39; Matt. 28:18; Dan. 7:14; John 3:35; 13:3; Matt. 11:27; Eph. 1:22; Heb. 2:8; 1 Cor. 15:27; John 1:3.

56 And that this communication is not to be understood per phrasin aut modum loquendi (as a phrase or mode of speaking), that is, only in words, with respect to the person according to the divine nature alone, but according to the assumed human nature, the three strong, irrefutable arguments and reasons, now following, show:

57 1. First, there is a unanimously received rule of the entire ancient orthodox Church that what Holy Scripture testifies that Christ received in time He received not according to the divine nature (according to which He has everything from eternity), but the person has received it in time ratione et respectu humanae naturae, that is, as referring, and with respect to, according to the assumed human nature.

58 2. Secondly, the Scriptures testify clearly, John 5:21f; 6:39f, that the power to quicken and to execute judgment has been given to Christ for the reason that He is the Son of Man, and in as far as He has flesh and blood.

59 3. Thirdly, the Scriptures speak not merely in general of the Son of Man, but also indicate expressly His assumed human nature, 1 John 1:7: The blood of Jesus Christ, His Son, cleanseth us from all sin, not only according to the merit [of the blood of Christ] which was once attained on the cross; but in this place John speaks of this, that in the work or act of justification not only the divine nature in Christ, but also His blood per modum efficaciae (by mode of efficacy), that is, actually, cleanses us from all sins. Thus in John 6:48-058 the flesh of Christ is a quickening food; as also the Council of Ephesus concluded from this [statement of the evangelist and apostle] that the flesh of Christ has power to quicken; and as many other glorious testimonies of the ancient orthodox Church concerning this article are cited elsewhere.

60 Now, that Christ, according to His human nature, has received this, and that it has been given and communicated to the assumed human nature in Christ, we shall and must believe according to the Scriptures. But, as above said, since the two natures in Christ are united in such a manner that they are not mingled with one another or changed one into the other, and each retains its natural, essential property, so that the properties of one nature never become properties of the other nature, this doctrine must also be rightly explained and diligently guarded against all heresies.

61 While we, then, invent nothing new of ourselves, but receive and repeat the explanations which the ancient orthodox Church has given hereof from the good foundation of Holy Scripture, namely, that this divine power, life, might, majesty, and glory was given to the assumed human nature in Christ, not in such a way as the Father from eternity has communicated to the Son, according to the divine nature, His essence and all divine attributes, whence He is of one essence with the Father and is equal to God (for Christ is equal to the Father only according to the divine nature, while according to the assumed human nature He is beneath God; from which it is manifest that we make no confusionem, exaequationem, abolitionem, that is, no confusion, equalization, or abolition of natures in Christ), so, too, the power to quicken is in the flesh of Christ not in that manner in which it is in His divine nature, namely, as an essential property.

62 Moreover, this communication or impartation has not occurred through an essential or natural infusion of the properties of the divine nature into the human, so that the humanity of Christ would have these by itself and apart from the divine essence, or as though the human nature in Christ had thereby [by this communication] entirely laid aside its natural, essential properties and were now either transformed into divinity, or had, with such communicated properties, in and by itself become equal to the same, or that there should now be for both natures identical or, at any rate, equal natural, essential properties and operations. For these and similar erroneous doctrines were justly rejected and condemned in the ancient approved councils on the basis of Holy Scripture. Nullo enim modo vel facienda vel admittenda est aut conversio aut confusio aut exaequatio sive naturarum in Christo sive essentialium proprietatum. That is: For in no way is conversion, confusion, or equalization of the natures in Christ or of their essential properties to be maintained [made] or admitted.

63 Accordingly, we have never understood the words realis communicatio or communicated realiter, that is, the impartation or communion which occurs in deed and truth, of any physica communicatio vel essentialis transfusio, physical communication or essential transfusion, that is, of an essential, natural communion or effusion, by which the natures would be commingled in their essence, and their essential properties, as some have craftily and wickedly, against their own conscience, perverted these words and phrases in order to make the pure doctrine suspected; but we have only opposed them to verbalis communicatio (verbal communication), that is, to this doctrine, when such persons assert that it is only phrasis and modus loquendi (a phrase and mode of speaking), that is, nothing more than mere words, titles, and names, upon which they have also laid so much stress that they would know of no other communion. Hence, for the true explanation of the majesty of Christ we have used such terms de reali communicatione (of real communion), and wished to indicate by them that this communion has occurred in deed and truth, however, without any confusion of natures and their essential properties.

64 We, therefore, hold and teach, in conformity with the ancient orthodox Church, as it has explained this doctrine from the Scriptures, that the human nature in Christ has received this majesty according to the manner of the personal union, namely, because the entire fulness of the divinity dwells in Christ, not as in other holy men or angels, but bodily, as in its own body, so that it shines forth with all its majesty, power, glory, and efficacy in the assumed human nature, voluntarily when and as He [Christ] wills, and in, with, and through the same manifests, exercises, and executes His divine power, glory, and efficacy, as the soul does in the body and fire in glowing iron (for by means of these illustrations, as was also mentioned above, the entire ancient Church has explained this doctrine).

65 This was concealed and withheld [for the greater part] at the time of the humiliation; but now, after the form of a servant [or exinanition] has been laid aside, it is fully, powerfully, and publicly exercised before all saints, in heaven and on earth; and in the life to come we shall also behold this His glory face to face, John 17:24.

66 Thus there is and remains in Christ only one divine omnipotence, power, majesty, and glory, which is peculiar to the divine nature alone; but it shines, manifests, and exercises itself fully, yet voluntarily, in, with, and through the assumed, exalted human nature in Christ. Just as in glowing iron there are not two kinds of power to shine and burn [as though the fire had a peculiar, and the iron also a peculiar and separate power of shining and burning], but the power to shine and to burn is a property of the fire; but since the fire is united with the iron, it manifests and exercises this its power to shine and to burn in, with, and through the glowing iron, so that thence and from this union also the glowing iron has the power to shine and to burn without conversion of the essence and of the natural properties of fire and iron.

67 For this reason we understand such testimonies of Scripture as speak of the majesty to which the human nature in Christ is exalted, not in such a way as if the divine majesty, which is peculiar to the divine nature of the Son of God, is in the person of the Son of Man to be ascribed [to Christ] simply and purely according to His divine nature, or that this majesty is to be in the human nature of Christ in such a manner only that from it His human nature should have but the mere title and name per phrasin et modum loquendi (by a phrase and mode of speaking), that is, only in words, but in deed and truth should have no communion whatever with it.

68 For in that way (since God is a spiritual, undivided essence, and therefore present everywhere and in all creatures, and wherever He is, dwelling, however, especially in believers and saints, there He has with Him such majesty of His) it might also be said with truth that in all creatures in whom God is, but especially in believers and saints, in whom He dwells, all the fulness of the Godhead dwells bodily, all treasures of wisdom and knowledge are hid, all power in heaven and earth is given, because the Holy Ghost, who has all power, is given them.

69 In this way, then, no distinction would be made between Christ according to His human nature and other holy men, and thus Christ would be deprived of His majesty, which He has received above all creatures, as a man or according to His human nature.

70 For no other creature, neither man nor angel, can or shall say: All power is given unto me in heaven and in earth, since, although God, with all the fulness of His Godhead, which He has everywhere with Himself, is in the saints, He does not dwell in them bodily, nor is personally united with them as in Christ. For from such personal union it follows that Christ says, even according to His human nature, Matt. 28:18: All power is given unto Me in heaven and in earth. Also John 13:3: Jesus knowing that the Father had given all things into His hands. Also Col. 2:9: In Him dwelleth all the fulness of the Godhead bodily. Also: Thou crownedst Him with glory and honor, and didst set Him over the works of Thy hands; Thou hast put all things in subjection under His feet. For in that He put all in subjection under Him, He left nothing that is not put under Him, Heb. 2:7f ; Ps. 8:6. He is excepted which did put all things under Him, 1 Cor. 15:27.

71 By no means, however, do we believe, teach, and confess such an infusion of the majesty of God and of all its properties into the human nature of Christ by which the divine nature is weakened [by which anything of the divine nature departs], or anything of its own is surrendered to another that it does not retain for itself, or that the human nature in its substance and essence should have received equal majesty, separate or distinct from the nature and essence of the Son of God, as when water, wine, or oil is poured from one vessel into another. For the human nature, as also no other creature, either in heaven or on earth, is capable of the omnipotence of God in such a manner that it would become in itself an almighty essence, or have in and by itself almighty properties; for thereby the human nature in Christ would be denied, and would be entirely converted into the divinity, which is contrary to our Christian faith, as also to the doctrine of all the prophets and apostles.

72 But we believe, teach, and confess that God the Father has so given His Spirit to Christ, His beloved Son, according to the assumed humanity (on account of which He is called also Messias, i.e., the Anointed), that He has not received His gifts by measure as other saints. For upon Christ the Lord, according to His assumed human nature (because, according to His divinity, He is of one essence with the Holy Ghost), rests the Spirit of wisdom and understanding, the Spirit of counsel and might, the Spirit of knowledge [and of the fear of the Lord, Col. 2:3; Is. 11:2; 61:1 ],

73 not in such a way that on this account, as a man, He knew and could do only some things, as other saints know and can do by the Spirit of God, who works in them only created gifts, but since Christ, according to His divinity, is the second person in the Holy Trinity, and from Him, as also from the Father, the Holy Ghost proceeds, and thus is and remains His and the Father’s own Spirit to all eternity, not separated from the Son of God, therefore (as the Fathers say) the entire fulness of the Spirit has been communicated by the personal union to Christ according to the flesh, which is personally united with the Son of God.

74 This voluntarily manifests and shows itself, with all its power therein, therewith and thereby [in, with, and through the human nature of Christ], so that He [Christ, according to His human nature] not only knows some things and is ignorant of others, can do some things and is unable to do others, but [according to the assumed human nature] knows and can do all things. For upon Him the Father poured without measure the Spirit of wisdom and power, so that, as man, He has received through this personal union all knowledge and all power in deed and truth. And thus all the treasures of wisdom are hidden in Him, thus all power is given to Him, and He is seated at the right hand of the majesty and power of God.

75 From history it can be learned that at the time of the Emperor Valens there was among the Arians a peculiar sect which was called the Agnoetae, because they imagined that the Son, the Word of the Father, knew indeed all things, but that His assumed human nature is ignorant of many things; against whom also Gregory the Great wrote.

76 On account of this personal union, and the communion resulting from it, which the divine and the human nature have with one another in the person of Christ in deed and truth, there is ascribed to Christ according to the flesh what His flesh, according to its nature and essence, cannot be of itself, and, apart from this union, cannot have, namely, that His flesh is a truly quickening food and His blood a truly quickening drink; as the two hundred Fathers of the Council of Ephesus have testified, carnem Christi esse vivificam seu vivificatricem, that is, that the flesh of Christ is a quickening flesh [or a quickener]. Hence, too, this man only, and no man besides, either in heaven or on earth, can say with truth, Matt. 18:20: Where two or three are gathered together in My name, there am I in the midst of them. Also Matt. 28:20: Lo, I am with you alway, even unto the end of the world.

77 And these testimonies we do not understand, as though only the divinity of Christ were present with us in the Christian Church and congregation, and such presence were to concern Christ according to His humanity in no way whatever; for in that manner Peter, Paul, and all the saints in heaven, since divinity which is everywhere present dwells in them, would also be with us on earth, which the Holy Scriptures, however, testify only of Christ, and of no other man besides.

78 But we hold that by these words [the above passages of Scripture] the majesty of the man Christ is declared, which Christ has received, according to His humanity, at the right hand of the majesty and power of God, namely, that also according to His assumed human nature and with the same, He can be, and also is, present where He will, and especially that in His Church and congregation on earth He is present as Mediator, Head, King, and High Priest, not in part, or one-half of Him only, but the entire person of Christ is present, to which both natures belong, the divine and the human; not only according to His divinity, but also according to, and with, His assumed human nature, according to which He is our

79 Brother, and we are flesh of His flesh and bone of His bone. Even as He has instituted His Holy Supper for the certain assurance and confirmation of this, that also according to that nature according to which He has flesh and blood He will be with us, and dwell, work, and be efficacious in us.

80 Upon this firm foundation Dr. Luther, of blessed memory, has also written [faithfully and clearly] concerning the majesty of Christ according to His human nature.

81 In the Large Confession concerning the Lord’s Supper he writes thus concerning the person of Christ: Now, since He [Christ] is such a man as is supernaturally one person with God, and apart from this man there is no God, it must follow that also according to the third, supernatural mode He is and can be in every place where God is, and all things are through and through full of Christ, also according to the humanity, not according to the first corporeal, comprehensible mode, but according to the supernatural, divine mode. Vol. 2, Wittenb. Germ., fol. 191.

82 For here you must stand [confess] and say: Wherever Christ according to the divinity is, there He is a natural, divine person, and He is there also naturally and personally, as His conception in His mother’s womb well shows. For if He were to be God’s son, He must, naturally and personally be in His mother’s womb and become man. Now, if He is naturally and personally wherever He is, He must also be man in the same place. For there are not [in Christ] two separate persons, but only one person: wherever it is, there it is the one undivided person; and wherever you can say, Here is God, there you must also say, Then Christ the man is also there. And if you would point out a place where God is, and not the man, the person would already be divided, because I could then say with truth: Here is God who is not man, and who never as yet has become man.

83 However, no such a God for me! For it would follow hence that space and place separated the two natures from one another, and divided the person, and yet even death and all devils could not divide or rend them from one another.

84 And there would remain to me a poor sort of Christ [a Christ of how much value, pray?], who would be a divine and human person at the same time in no more than in only one place, while in all other places He must be only a mere separate God and divine person without humanity. No, friend, wherever you place God, there you must also place with Him humanity; they do not allow themselves to be separated or divided from one another. There has been made [in Christ] one person, and it [the Son of God] does not separate from itself the [assumed] humanity.

85 In the little book concerning the Last Words of David, which Dr. Luther wrote shortly before his death, he says as follows: According to the other, the temporal, human birth, also the eternal power of God has been given Him; however, in time, and not from eternity. For the humanity of Christ has not been from eternity, like the divinity; but, as we reckon and write, Jesus, the Son of Mary, is 1543 years old this year. But from the instant when divinity and humanity were united in one person, the man, the Son of Mary, is and is called almighty, eternal God, who has eternal might, and has created and sustains all things per communicationem idiomatum for the reason that He is one person with the divinity, and is also true God. Of this He speaks Matt. 11:27: “All things are delivered unto Me of My Father”; and Matt. 28:18: “All power is given unto Me in heaven and in earth.” To which Me? To Me, Jesus of Nazareth, the Son of Mary, and born man. From eternity I have it of the Father, before I became man. But when I became man, I received it in time, according to humanity, and kept it concealed until My resurrection and ascension; when it was to be manifested and declared, as St. Paul says, Rom. 1:4: “He is declared and proved to be a Son of God with power.” John 17:10 calls it “glorified.” Vol. 5, Wittenb. Germ., fol. 545.

86 Similar testimonies are found in Dr. Luther’s writings, but especially in the book That These Words Still Stand Firm, and in the Large Confession concerning the Holy Supper; to which writings, as well-grounded explanations of the majesty of Christ at the right hand of God, and of His testament, we would be understood as having referred, for the sake of brevity, in this article, as well as in the Holy Supper, as has been heretofore mentioned.

87 Therefore we regard it as a pernicious error when such majesty is denied to Christ according to His humanity. For thereby the very great consolation is taken from Christians which they have in the aforecited promise concerning the presence and dwelling with them of their Head, King, and High Priest, who has promised them that not only His mere divinity would be with them, which to us poor sinners is as a consuming fire to dry stubble, but that He, He, the man who has spoken with them, who has tried all tribulations in His assumed human nature, and who can therefore have sympathy with us, as with men and His brethren,-He will be with us in all our troubles also according to the nature according to which He is our brother and we are flesh of His flesh.

88 Therefore we unanimously reject and condemn, with mouth and heart, all errors not in accordance with the doctrine presented, as contrary to the prophetic and apostolic Scriptures, the pure [received and approved] symbols, and our Christian Augsburg Confession

89 1. As, when it is believed or taught by any one that on account of the personal union the human nature is mingled with the divine or is changed into it.

90 2. Also, that the human nature in Christ is everywhere present in the same mode as the divinity, as an infinite essence, by essential power and property of its nature.

91 3. Also, that the human nature in Christ has become equal to and like the divine nature in its substance and essence or in its essential properties.

92 4. Also, that the humanity of Christ is locally extended in all places of heaven and earth; which is to be ascribed not even to the divinity. But that Christ, by His divine omnipotence can be present with His body, which He has placed at the right hand of the majesty and power of God, wherever He will, especially where He has, in His Word, promised this His presence, as in the Holy Supper, this His omnipotence and wisdom can well accomplish without change or abolition of His true human nature.

93 5. Also, that the mere human nature of Christ has suffered for us and redeemed us, with which the Son of God is said to have had no communion whatever in suffering.

94 6. Also, that Christ is present with us on earth in the Word preached and in the right use of the holy Sacraments only according to His divinity, and that this presence of Christ does not in any way pertain to His assumed human nature.

95 7. Also, that the assumed human nature in Christ has in deed and truth no communion whatever with the divine power, might, wisdom, majesty, and glory, but has in common only the mere title and name.
96 These errors, and all that are contrary and opposed to the [godly and pure] doctrine presented above, we reject and condemn as contrary to the pure Word of God, the Scriptures of the holy prophets and apostles, and our Christian faith and confession. And we admonish all Christians, since in the Holy Scriptures Christ is called a mystery upon which all heretics dash their heads, not to indulge in a presumptuous manner in subtile inquiries, concerning such mysteries, with their reason, but with the venerated apostles simply to believe, to close the eyes of their reason, and bring into captivity their understanding to the obedience of Christ, 2 Cor. 10:5, and to take comfort [seek most delightful and sure consolation], and hence to rejoice without ceasing in the fact that our flesh and blood is placed so high at the right hand of the majesty and almighty power of God. Thus we shall assuredly find constant consolation in every adversity, and remain well guarded from pernicious error.

IX. Christ’s Descent into Hell

1 And since even in the ancient Christian teachers of the Church, as well as in some among our teachers, dissimilar explanations of the article concerning the descent of Christ to hell are found, we abide in like manner by the simplicity of our Christian faith [comprised in the Creed], to which Dr. Luther in his sermon, which was delivered in the castle at Torgau in the year 1533, concerning the descent of Christ to hell, has pointed us, where we confess: I believe in the Lord Christ, God’s Son, our Lord, dead, buried, and descended into hell. For in this [Confession] the burial and descent of Christ to hell are distinguished as different articles;

2 and we simply believe that the entire person, God and man, after the burial descended into hell, conquered the devil, destroyed the power of hell, and took from the devil all his might.
3 We should not, however, trouble ourselves with high and acute thoughts as to how this occurred; for with our reason and our five senses this article can be comprehended as little as the preceding one, how Christ is placed at the right hand of the almighty power and majesty of God; but we are simply to believe it and adhere to the Word [in such mysteries of faith]. Thus we retain the substance [sound doctrine] and [true] consolation that neither hell nor the devil can take captive or injure us and all who believe in Christ.

X. Adiaphora

1 Concerning ceremonies and church rites which are neither commanded nor forbidden in God’s Word, but are introduced into the Church with a good intention, for the sake of good order and propriety, or otherwise to maintain Christian discipline, a dissension has likewise arisen among some theologians of the Augsburg Confession:

2 the one side holding that also in time of persecution and in case of confession [when confession of faith is to be made], even though the enemies of the Gospel do not come to an agreement with us in doctrine, yet some ceremonies, abrogated [long since], which in themselves are adiaphora, and neither commanded nor forbidden by God, may, without violence to conscience, be reestablished in compliance with the pressure and demand of the adversaries, and thus in such [things which are of themselves] adiaphora, or matters of indifference, we may indeed come to an agreement [have conformity] with them.

3 But the other side contended that in time of persecution, in case of confession, especially when it is the design of the adversaries, either through force and compulsion, or in an insidious manner, to suppress the pure doctrine, and gradually to introduce again into our churches their false doctrine, this, also in adiaphora, can in no way be done, as has been said, without violence to conscience and prejudice to the divine truth.

4 To explain this controversy, and by God’s grace finally to settle it, we present to the Christian reader this simple statement regarding the matter [in conformity with the Word of God]:

5 Namely, when under the title and pretext of external adiaphora such things are proposed as are in principle contrary to God’s Word (although painted another color), these are not to be regarded as adiaphora, in which one is free to act as he will, but must be avoided as things prohibited by God. In like manner, too, such ceremonies should not be reckoned among the genuine free adiaphora, or matters of indifference, as make a show or feign the appearance, as though our religion and that of the Papists were not far apart, thus to avoid persecution, or as though the latter were not at least highly offensive to us; or when such ceremonies are designed for the purpose, and required and received in this sense, as though by and through them both contrary religions were reconciled and became one body; or when a reentering into the Papacy and a departure from the pure doctrine of the Gospel and true religion should occur or gradually follow therefrom [when there is danger lest we seem to have reentered the Papacy, and to have departed, or to be on the point of departing gradually, from the pure doctrine of the Gospel].

6 For in this case what Paul writes, 2 Cor. 6:14-017, shall and must obtain: Be ye not unequally yoked together with unbelievers; for what communion hath light with darkness?Wherefore come out from among them and be ye separate, saith the Lord.

7 Likewise, when there are useless, foolish displays, that are profitable neither for good order nor Christian discipline, nor evangelical propriety in the Church, these also are not genuine adiaphora, or matters of indifference.

8 But as regards genuine adiaphora, or matters of indifference (as explained before), we believe, teach, and confess that such ceremonies, in and of themselves, are no worship of God, nor any part of it, but must be properly distinguished from such as are, as it is written: In vain they do worship Me, teaching for doctrines the commandments of men, Matt. 15:9.

9 Therefore we believe, teach, and confess that the congregation of God of every place and every time has, according to its circumstances, the good right, power, and authority [in matters truly adiaphora] to change, to diminish, and to increase them, without thoughtlessness and offense, in an orderly and becoming way, as at any time it may be regarded most profitable, most beneficial, and best for [preserving] good order, [maintaining] Christian discipline [and for eujtaxiva worthy of the profession of the Gospel], and the edification of the Church. Moreover, how we can yield and give way with a good conscience to the weak in faith in such external adiaphora, Paul teaches Rom. 14, and proves it by his example, Acts 16:3; 21:26; 1 Cor. 9:19.

10 We believe, teach, and confess also that at the time of confession [when a confession of the heavenly truth is required], when the enemies of God’s Word desire to suppress the pure doctrine of the holy Gospel, the entire congregation of God, yea, every Christian, but especially the ministers of the Word, as the leaders of the congregation of God [as those whom God has appointed to rule His Church], are bound by God’s Word to confess freely and openly the [godly] doctrine, and what belongs to the whole of [pure] religion, not only in words, but also in works and with deeds; and that then, in this case, even in such [things truly and of themselves] adiaphora, they must not yield to the adversaries, or permit these [adiaphora] to be forced upon them by their enemies, whether by violence or cunning, to the detriment of the true worship of God and the introduction and sanction of idolatry.

11 For it is written, Gal. 5:1: Stand fast, therefore, in the liberty wherewith Christ hath made us free, and be not again entangled in the yoke of bondage. Also Gal. 2:4f : And that because of false brethren unawares brought in, who came in privily to spy out our liberty which we have in Christ Jesus, that they might bring us into bondage; to whom we gave place by subjection, no, not for an hour, that the truth of the Gospel might continue with you.

12 [Now it is manifest that in that place Paul speaks concerning circumcision, which at that time had become an adiaphoron (1 Cor. 7:18f.), and which at other occasions was observed by Paul (however, with Christian and spiritual freedom, Acts 16:3). But when the false apostles urged circumcision for establishing their false doctrine, (that the works of the Law were necessary for righteousness and salvation,) and misused it for confirming their error in the minds of men, Paul says that he would not yield even for an hour, in order that the truth of the Gospel might continue unimpaired.]

13 Thus Paul yields and gives way to the weak as to food and [the observance of] times or days, Rom. 14:6. But to the false apostles, who wished to impose these upon the conscience as necessary things, he will yield not even in such things as in themselves are adiaphora, Col. 2:16: Let no man therefore judge you in meat, or in drink, or in respect of an holy day. And when Peter and Barnabas yielded somewhat [more than they ought] in such an emergency, Paul openly reproves them as those who in this matter were not walking aright, according to the truth of the Gospel, Gal. 2:11ff

14 For here it is no longer a question concerning external matters of indifference, which in their nature and essence are and remain of themselves free, and accordingly can admit of no command or prohibition that they be employed or omitted; but it is a question, in the first place, concerning the eminent article of our Christian faith, as the apostle testifies, that the truth of the Gospel might continue, which is obscured and perverted by such compulsion or command, because such adiaphora are then either publicly required for the sanction of false doctrine, superstition, and idolatry, and for the suppression of pure doctrine and Christian liberty, or at least are abused for this purpose by the adversaries, and are thus viewed [and are believed to be restored for this abuse and wicked end].

15 Likewise, the article concerning Christian liberty also is here at stake, which the Holy Ghost through the mouth of the holy apostle so earnestly charged His Church to preserve, as we have just heard. For as soon as this is weakened and the ordinances of men [human traditions] are forced upon the Church with coercion, as though it were wrong and a sin to omit them, the way is already prepared for idolatry, and by this means ordinances of men [human traditions] are afterwards multiplied and regarded as a divine worship, not only equal to the ordinances of God, but are even placed above them.

16 Moreover, by such [untimely] yielding and conformity in external things, where there has not been previously Christian union in doctrine, idolaters are confirmed in their idolatry; on the other hand, the true believers are grieved, offended, and weakened in their faith [their faith is grievously shaken, and made to totter as though by a battering-ram]; both of which every Christian for the sake of his soul’s welfare and salvation is bound to avoid, as it is written: Woe unto the world because of offenses! Also: Whoso shall offend one of these little ones which believe in Me, it were better for him that a millstone were hanged about his neck, and that he were drowned in the depth of the sea [Matt. 18:6, 7].

17 But it is to be especially remembered what Christ says: Whosoever therefore shalt confess Me before men, him will I confess also before My Father which is in heaven, Matt. 10:32.

18 However, that this has always and everywhere been the faith and confession, concerning such indifferent matters, of the chief teachers of the Augsburg Confession, into whose footsteps we have entered, and in whose Confession we intend by God’s grace to persevere, is shown [most clearly] by the following testimonies drawn from the Smalcald Articles, which were composed and subscribed in the year 1537:
From the Smalcald Articles, in the Year 1537, etc.

19 The Smalcald Articles (Of the Church) say concerning this as follows: We do not concede to them (the papal bishops) that they are the Church, and indeed they are not; nor will we listen to those things which, under the name of Church, they enjoin and forbid. For, thank God, [today] a child seven years old knows what the Church is, namely, the saints, believers, and lambs, who hear the voice of their Shepherd. And shortly before (Of Ordination and Vocation): If the bishops would be true bishops, and would devote themselves to the Church and the Gospel, it might be granted to them, for the sake of love and unity, but not from necessity, to ordain and confirm us and our preachers; omitting, however, all comedies and spectacular doings of an unchristian nature and display. But, because they neither are, nor wish to be, true bishops, but worldly lords and princes, who will neither preach, nor teach, nor baptize, nor administer the Lord’s Supper, nor perform any work or office of the Church, and, moreover, persecute and condemn those who, having been called to do so, discharge these functions, the Church ought not on their account to remain without ministers.

20 And in the article Of the Papacy, the Smalcald Articles say (475:14): Therefore, just as little as we can worship the devil himself as Lord and God, we can endure his apostle, the Pope, or Antichrist, in his rule as head or lord. For to lie and to kill and to destroy body and soul eternally, that is wherein his papal government really consists.

21 And in the treatise Concerning the Power and Primacy of the Pope, which is appended to the Smalcald Articles, and was also subscribed by the theologians then present with their own hands, are these words: No one is to burden the Church with his own traditions, but here the rule is to be that nobody’s power or authority is to avail more than the Word of God.

22 And shortly afterwards (517:41): This being the case, all Christians ought most diligently to beware of becoming partakers of the godless doctrine, blasphemies, and unjust cruelties of the Pope; but ought to desert and execrate the Pope with his members, or adherents, as the kingdom of Antichrist, just as Christ has commanded (Matt. 7:15): “Beware of false prophets.” And Paul commands us to avoid false teachers and execrate them as an abomination. And in 2 Cor. 6:14 he says: “Be ye not unequally yoked together with unbelievers; for what communion hath light with darkness?”

23 It is a grave matter wanting to separate one’s self from so many lands and nations, and to profess a separate doctrine; but here stands God’s command, that every one should beware and not agree with those who maintain false doctrine, or who think of supporting it by means of cruelty.

24 So Dr. Luther, too, has amply instructed the Church of God in a special treatise concerning what should be thought of ceremonies in general, and especially of adiaphora, Vol. 3, Jena, p. 523; as was also done in 1530, and can be seen in Tom. 3, Jena, German.

25 From this explanation every one can understand what every Christian congregation and every Christian man, especially in time of confession [when a confession of faith should be made], and, most of all, preachers, are to do or to leave undone, without injury to conscience, with respect to adiaphora, in order that God may not be angered [provoked to just indignation], love may not be injured, the enemies of God’s Word be not strengthened, nor the weak in faith offended.

26 1. Therefore we reject and condemn as wrong when the ordinances of men in themselves are regarded as a service or part of the service of God.

27 2. We reject and condemn also as wrong when these ordinances are by coercion forced upon the congregation of God as necessary.

28 3. We reject and condemn also as wrong the opinion of those who hold (what tends to the detriment of the truth) that at a time of persecution we may comply with the enemies of the holy Gospel in [restoring] such adiaphora, or come to an agreement with them.

29 4. We likewise regard it as a sin that deserves to be rebuked when in time of persecution anything is done either in indifferent matters or in doctrine, and in what otherwise pertains to religion, for the sake of the enemies of the Gospel, in word and act, contrary and opposed to the Christian confession.

30 5. We reject and condemn also [the madness] when these adiaphora are abrogated in such a manner as though it were not free to the congregation [church] of God at any time and place to employ one or more in Christian liberty, according to its circumstances, as may be most useful to the Church.
31 Thus [According to this doctrine] the churches will not condemn one another because of dissimilarity of ceremonies when, in Christian liberty, one has less or more of them, provided they are otherwise agreed with one another in the doctrine and all its articles, also in the right use of the holy Sacraments, according to the well-known saying: Dissonantia ieiunii non dissolvit consonantiam fidei; “Disagreement in fasting does not destroy agreement in the faith.”

XI. Election

1 Although among the theologians of the Augsburg Confession there has not occurred as yet any public dissension whatever concerning the eternal election of the children of God that has caused offense, and has become wide-spread, yet since this article has been brought into very painful controversy in other places, and even among our theologians there has been some agitation concerning it; moreover, since the same expressions were not always employed concerning it by the theologians; therefore, in order, by the aid of divine grace, to prevent disagreement and separation on its account in the future among our successors, we, as much as in us lies, have desired also to present an explanation of the same here, so that every one may know what is our unanimous doctrine, faith, and confession also concerning this article.

2 For the doctrine concerning this article, if taught from, and according to, the pattern of the divine Word [and analogy of God’s Word and of faith], neither can nor should be regarded as useless or unnecessary, much less as offensive or injurious, because the Holy Scriptures not only in but one place and incidentally, but in many places, thoroughly treat and urge [explain] the same.

3 Moreover, we should not neglect or reject the doctrine of the divine Word on account of abuse or misunderstanding, but precisely on that account, in order to avert all abuse and misunderstanding, the true meaning should and must be explained from the foundation of the Scriptures; and the plain sum and substance [of the heavenly doctrine] concerning this article, accordingly, consists in the following points:

4 First, the distinction between the eternal foreknowledge of God and the eternal election of His children to eternal salvation, is carefully to be observed. For praescientia vel praevisio (foreknowledge or prevision), that is, that God sees and knows everything before it happens, which is called God’s foreknowledge [prescience], extends over all creatures, good and bad; namely, that He foresees and foreknows everything that is or will be, that is occurring or will occur, whether it be good or bad, since before God all things, whether they be past or future, are manifest and present. Thus it is written, Matt. 10:29: Are not two sparrows sold for a farthing? And one of them shall not fall on the ground without your Father. And Ps. 139:16: Thine eyes did see my substance, yet being imperfect; and in Thy book all my members were written, which in continuance were fashioned, when as yet there were none of them. Also Is. 37:28: I know thy abode, and thy going out, and thy coming in, and thy rage against Me.

5 The eternal election of God, however, vel praedestinatio (or predestination), that is, God’s ordination to salvation, does not extend at once over the godly and the wicked, but only over the children of God, who were elected and ordained to eternal life before the foundation of the world was laid, as Paul says, Eph. 1:4. 5: He hath chosen us in Him, having predestinated us unto the adoption of children by Jesus Christ.

6 The foreknowledge of God (praescientia) foresees and foreknows also that which is evil; however, not in such a manner as though it were God’s gracious will that it should happen; but all that the perverse, wicked will of the devil and of men wills and desires to undertake and do, God sees and knows before; and His praescientia, that is, foreknowledge, observes its order also in wicked acts or works, inasmuch as a limit and measure is fixed by God to the evil which God does not will, how far it should go, and how long it should last, when and how He will hinder and punish it; for all of this God the Lord so overrules that it must redound to the glory of the divine name and to the salvation of His elect, and the godless, on that account, must be put to confusion.

7 However, the beginning and cause of evil is not God’s foreknowledge (for God does not create and effect [or work] evil, neither does He help or promote it); but the wicked, perverse will of the devil and of men [is the cause of evil], as it is written Hos. 13:9: O Israel, thou hast destroyed thyself; but in Me is thy help. Also: Thou art not a God that hath pleasure in wickedness. Ps. 5:4.

8 The eternal election of God, however, not only foresees and foreknows the salvation of the elect, but is also, from the gracious will and pleasure of God in Christ Jesus, a cause which procures, works, helps, and promotes our salvation and what pertains thereto; and upon this [divine predestination] our salvation is so founded that the gates of hell cannot prevail against it, Matt. 16:18, as is written John 10:28: Neither shall any man pluck My sheep out of My hand. And again, Acts 13:48: And as many as were ordained to eternal life, believed.

9 Nor is this eternal election or ordination of God to eternal life to be considered in God’s secret, inscrutable counsel in such a bare manner as though it comprised nothing further, or as though nothing more belonged to it, and nothing more were to be considered in it, than that God foresaw who and how many were to be saved, who and how many were to be damned, or that He only held a [sort of military] muster, thus: “This one shall be saved, that one shall be damned; this one shall remain steadfast [in faith to the end], that one shall not remain steadfast.”

10 For from this [notion] many derive and conceive strange, dangerous, and pernicious thoughts, which occasion and strengthen either security and impenitence or despondency and despair, so that they fall into troublesome thoughts and [for thus some think, with peril to themselves, nay, even sometimes] say: Since, before the foundation of the world was laid, Eph. 1:4, God has foreknown [predestinated] His elect to salvation, and God’s foreknowledge [election] cannot fail nor be hindered or changed by any one, Is. 14:27; Rom. 9:19, therefore, if I am foreknown [elected] to salvation, nothing can injure me with respect to it, even though I practise all sorts of sin and shame without repentance, have no regard for the Word and Sacraments, concern myself neither with repentance, faith, prayer, nor godliness; but I shall and must be saved nevertheless, because God’s foreknowledge [election] must come to pass. If, however, I am not foreknown [predestinated], it helps me nothing anyway, even though I would occupy myself with the Word, repent, believe, etc.; for I cannot hinder or change God’s foreknowledge [predestination].

11 And indeed also to godly hearts, even when, by God’s grace they have repentance, faith, and a good purpose [of living in a godly manner], such thoughts occur as these: If you are not foreknown [predestinated or elected] from eternity to salvation, everything [your every effort and entire labor] is of no avail. This occurs especially when they view their weakness and the examples of those who have not persevered [in faith to the end], but have fallen away again [from true godliness to ungodliness, and have become apostates].

12 To this false delusion and [dangerous] thought we should oppose the following clear argument, which is sure and cannot fail, namely: Since all Scripture, given by inspiration of God, is to serve, not for [cherishing] security and impenitence, but for reproof, for correction, for instruction in righteousness, 2 Tim. 3:16; also, since everything in God’s Word has been prescribed to us, not that we should thereby be driven to despair, but that we, through patience and comfort of the Scriptures, might have hope, Rom. 15:4, therefore it is without any doubt in no way the sound sense or right use of the doctrine concerning the eternal foreknowledge of God that either impenitence or despair should be occasioned or strengthened thereby. Accordingly, the Scriptures teach this doctrine in no other way than to direct us thereby to the [revealed] Word, Eph. 1:13; 1 Cor. 1:7; exhort to repentance, 2 Tim. 3:16; urge to godliness, Eph. 1:14; John 15:3; strengthen faith and assure us of our salvation, Eph. 1:13; John 10:27f ; 2 Thess. 2:13f.

13 Therefore, if we wish to think or speak correctly and profitably concerning eternal election, or the predestination and ordination of the children of God to eternal life, we should accustom ourselves not to speculate concerning the bare, secret, concealed, inscrutable foreknowledge of God, but how the counsel, purpose, and ordination of God in Christ Jesus, who is the true Book of Life, is revealed to us through the Word,

14 namely, that the entire doctrine concerning the purpose, counsel, will, and ordination of God pertaining to our redemption, call, justification, and salvation should be taken together; as Paul treats and has explained this article Rom. 8:29f ; Eph. 1:4f , as also Christ in the parable, Matt. 22:1ff , namely, that God in His purpose and counsel ordained [decreed]:

15 1. That the human race is truly redeemed and reconciled with God through Christ, who, by His faultless [innocency] obedience, suffering, and death, has merited for us the righteousness which avails before God, and eternal life.

16 2. That such merit and benefits of Christ shall be presented, offered, and distributed to us through His Word and Sacraments.

17 3. That by His Holy Ghost, through the Word, when it is preached, heard, and pondered, He will be efficacious and active in us, convert hearts to true repentance, and preserve them in the true faith.

18 4. That He will justify all those who in true repentance receive Christ by a true faith, and will receive them into grace, the adoption of sons, and the inheritance of eternal life.

19 5. That He will also sanctify in love those who are thus justified, as St. Paul says, Eph. 1:4.

20 6. That He also will protect them in their great weakness against the devil, the world, and the flesh, and rule and lead them in His ways, raise them again [place His hand beneath them], when they stumble, comfort them under the cross and in temptation, and preserve them [for life eternal].

21 7. That He will also strengthen, increase, and support to the end the good work which He has begun in them, if they adhere to God’s Word, pray diligently, abide in God’s goodness [grace], and faithfully use the gifts received.

22 8. That finally He will eternally save and glorify in life eternal those whom He has elected, called, and justified.

23 And [indeed] in this His counsel, purpose, and ordination God has prepared salvation not only in general, but has in grace considered and chosen to salvation each and every person of the elect who are to be saved through Christ, also ordained that in the way just mentioned He will, by His grace, gifts, and efficacy, bring them thereto [make them participants of eternal salvation], aid, promote, strengthen, and preserve them.

24 All this, according to the Scriptures, is comprised in the doctrine concerning the eternal election of God to adoption and eternal salvation, and is to be understood by it, and never excluded nor omitted, when we speak of God’s purpose, predestination, election, and ordination to salvation. And when our thoughts concerning this article are thus formed according to the Scriptures, we can by God’s grace simply [and correctly] adapt ourselves to it [and advantageously treat of it].

25 This also belongs to the further explanation and salutary use of the doctrine concerning God’s foreknowledge [predestination] to salvation: Since only the elect, whose names are written in the book of life, are saved, how, we can know, whence and whereby we can perceive who are the elect that can and should receive this doctrine for comfort.

26 And of this we should not judge according to our reason, nor according to the Law or from any external appearance. Neither should we attempt to investigate the secret, concealed abyss of divine predestination, but should give heed to the revealed will of God. For He has made known unto us the mystery of His will, and made it manifest through Christ that it might be preached, Eph. 1:9ff ; 2 Tim. 1:9f.

27 This, however, is revealed to us in the manner as Paul says, Rom. 8:29f : Whom God predestinated, elected, and foreordained, He also called. Now, God does not call without means, but through the Word, as He has commanded repentance and remission of sins to be preached in His name, Luke 24:47. St. Paul also testifies to like effect when he writes: We are ambassadors for Christ, as though God did beseech you by us; we pray you in Christ’s stead, Be ye reconciled to God. 2 Cor. 5:20. And the guests whom the King will have at the wedding of His Son He calls through His ministers sent forth, Matt. 22:2ff , some at the first and some at the second, third, sixth, ninth, and even at the eleventh hour, Matt. 20:3ff

28 Therefore, if we wish to consider our eternal election to salvation with profit, we must in every way hold sturdily and firmly to this, that, as the preaching of repentance, so also the promise of the Gospel is universalis (universal), that is, it pertains to all men, Luke 24:47. For this reason Christ has commanded that repentance and remission of sins should be preached in His name among all nations. For God loved the world and gave His Son, John 3:16. Christ bore the sins of the world, John 1:29, gave His flesh for the life of the world, John 6:51; His blood is the propitiation for the sins of the whole world, 1 John 1:7; 2:2. Christ says: Come unto Me, all ye that labor and are heavy laden, and I will give you rest, Matt. 11:28. God hath concluded them all in unbelief, that He might have mercy upon all, Rom. 11:32. The Lord is not willing that any should perish, but that all should come to repentance, 2 Pet. 3:9. The same Lord over all is rich unto all that call upon Him, Rom. 10:12. The righteousness of God, which is by faith of Jesus Christ, unto all and upon all them that believe, Rom. 3:22. This is the will of Him that sent Me, that every one that seeth the Son and believeth on Him may have everlasting life, John 6:40. Likewise it is Christ’s command that to all in common to whom repentance is preached this promise of the Gospel also should be offered Luke 24:47; Mark 16:15.

29 And this call of God, which is made through the preaching of the Word, we should not regard as jugglery, but know that thereby God reveals His will, that in those whom He thus calls He will work through the Word, that they may be enlightened, converted, and saved. For the Word, whereby we are called, is a ministration of the Spirit, that gives the Spirit, or whereby the Spirit is given, 2 Cor. 3:8, and a power of God unto salvation, Rom. 1:16. And since the Holy Ghost wishes to be efficacious through the Word, and to strengthen and give power and ability, it is God’s will that we should receive the Word, believe and obey it.

30 For this reason the elect are described thus, John 10:27f : My sheep hear My voice, and I know them, and they follow Me, and I give unto them eternal life. And Eph. 1:11. 13: Those who according to the purpose are predestinated to an inheritance hear the Gospel, believe in Christ, pray and give thanks, are sanctified in love, have hope, patience, and comfort under the cross, Rom. 8:25; and although all this is very weak in them, yet they hunger and thirst after righteousness, Matt. 5:6.

31 Thus the Spirit of God gives to the elect the testimony that they are children of God, and when they do not know for what they should pray as they ought, He intercedes for them with groanings that cannot be uttered, Rom. 8:16. 26.

32 Thus, also, Holy Scripture testifies that God, who has called us, is so faithful that, when He has begun the good work in us, He also will preserve it to the end and perfect it, if we ourselves do not turn from Him, but firmly retain to the end the work begun, for which He has promised His grace, 1 Cor. 1:9; Phil. 1:6 [ 1 Pet. 5:10 ]; 2 Pet. 3:9; Heb. 3:2.

33 With this revealed will of God we should concern ourselves, follow and be diligently engaged upon [eagerly con] it, because through the Word, whereby He calls us, the Holy Ghost bestows grace, power, and ability to this end, and should not [attempt to] sound the abyss of God’s hidden predestination, as it is written in Luke 13:24, where one asks: Lord, are there few that be saved? and Christ answers: Strive to enter in at the strait gate. Accordingly, Luther says [in the Preface to the Epistle to the Romans]: Follow the Epistle to the Romans in its order, concern yourself first with Christ and His Gospel, that you may recognize your sins and His grace; next, that you contend with sin, as Paul teaches from the first to the eighth chapter; then, when in the eighth chapter you will come into [will have been exercised by] temptation under the cross and afflictions, this will teach you in the ninth, tenth, and eleventh chapters how consolatory predestination is, etc.

34 However, that many are called and few chosen is not owing to the fact that the call of God, which is made through the Word, had the meaning as though God said: Outwardly, through the Word, I indeed call to My kingdom all of you to whom I give My Word; however, in My heart I do not mean this with respect to all, but only with respect to a few; for it is My will that the greatest part of those whom I call through the Word shall not be enlightened nor converted, but be and remain damned, although through the Word, in the call, I declare Myself to them otherwise. Hoc enim esset Deo contradictorias voluntates affingere, that is:

35 For this would be to assign contradictory wills to God. That is, in this way it would be taught that God, who surely is Eternal Truth, would be contrary to Himself [or say one thing, but revolve another in His heart], while, on the contrary, God [rebukes and] punishes also in men this wickedness [this wantonness, this dishonesty] when a person declares himself to one purpose, and thinks and means another in the heart, Ps. 5:9; 12:2f.

36 Thereby also the necessary consolatory foundation is rendered altogether uncertain and void, as we are daily reminded and admonished that only from God’s Word, through which He treats with us and calls us, we are to learn and conclude what His will towards us is, and that we should believe and not doubt what it affirms to us and promises.

37 For this reason also Christ causes the promise of the Gospel not only to be offered in general, but He seals it through the Sacraments which He attaches as seals of the promise, and thereby confirms it [the certainty of the promise of the Gospel] to every believer in particular.

38 On this account, as the Augsburg Confession in Art. 11 says, we also retain private absolution, and teach that it is God’s command that we believe such absolution, and should regard it as sure that, when we believe the word of absolution, we are as truly reconciled to God as though we had heard a voice from heaven, as the Apology explains this article. This consolation would be entirely taken from us if we were not to infer the will of God towards us from the call which is made through the Word and through the Sacraments.

39 There would also be overthrown and taken from us the foundation that the Holy Ghost wishes certainly to be present with the Word preached, heard, considered, and to be efficacious and operate through it. Therefore the meaning is not at all the one referred to above, namely, that the elect are to be such [among the elect are to be numbered such] as even despise the Word of God, thrust it from them, blaspheme and persecute it, Matt. 22:6; Acts 13:46; or, when they hear it, harden their hearts, Heb. 4:2. 7; resist the Holy Ghost, Acts 7:51; without repentance persevere in sins, Luke 14:18; do not truly believe in Christ, Mark 16:16; only make [godliness] an outward show, Matt. 7:22; 22:12; or seek other ways to righteousness and salvation outside of Christ, Rom. 9:31.

40 Moreover, even as God has ordained in His [eternal] counsel that the Holy Ghost should call, enlighten, and convert the elect through the Word, and that He will justify and save all those who by true faith receive Christ, so He also determined in His counsel that He will harden, reprobate, and condemn those who are called through the Word, if they reject the Word and resist the Holy Ghost, who wishes to be efficacious and to work in them through the Word and persevere therein. And in this manner many are called, but few are chosen.

41 For few receive the Word and follow it; the greatest number despise the Word, and will not come to the wedding, Matt. 22:3ff The cause for this contempt for the Word is not God’s foreknowledge [or predestination], but the perverse will of man, which rejects or perverts the means and instrument of the Holy Ghost, which God offers him through the call, and resists the Holy Ghost, who wishes to be efficacious, and works through the Word, as Christ says: How often would I have gathered you together, and ye would not! Matt. 23:37.

42 Thus many receive the Word with joy, but afterwards fall away again, Luke 8:13. But the cause is not as though God were unwilling to grant grace for perseverance to those in whom He has begun the good work, for that is contrary to St. Paul, Phil. 1:6; but the cause is that they wilfully turn away again from the holy commandment [of God], grieve and embitter the Holy Ghost, implicate themselves again in the filth of the world, and garnish again the habitation of the heart for the devil. With them the last state is worse than the first, 2 Pet. 2:10. 20; Eph. 4:30; Heb. 10:26; Luke 11:25.

43 Thus far is the mystery of predestination revealed to us in God’s Word, and if we abide thereby and cleave thereto, it is a very useful, salutary, consolatory doctrine; for it establishes very effectually the article that we are justified and saved without all works and merits of ours, purely out of grace alone, for Christ’s sake. For before the time of the world, before we existed, yea, before the foundation of the world was laid, when, of course, we could do nothing good, we were according to God’s purpose chosen by grace in Christ to salvation, Rom. 9:11; 2 Tim. 1:9.

44 Moreover, all opiniones (opinions) and erroneous doctrines concerning the powers of our natural will are thereby overthrown, because God in His counsel, before the time of the world, decided and ordained that He Himself, by the power of His Holy Ghost, would produce and work in us, through the Word, everything that pertains to our conversion.

45 Thus this doctrine affords also the excellent, glorious consolation that God was so greatly concerned about the conversion, righteousness, and salvation of every Christian, and so faithfully purposed it [provided therefor] that before the foundation of the world was laid, He deliberated concerning it, and in His [secret] purpose ordained how He would bring me thereto [call and lead me to salvation], and preserve me therein. Also, that He wished to secure my salvation so well and certainly that, since through the weakness and wickedness of our flesh it could easily be lost from our hands, or through craft and might of the devil and the world be snatched and taken from us, He ordained it in His eternal purpose, which cannot fail or be overthrown, and placed it for preservation in the almighty hand of our Savior Jesus Christ, from which no one can pluck us, John 10:28.

46 Hence Paul also says, Rom. 8:28. 39: Because we have been called according to the purpose of God, who will separate us from the love of God in Christ? [Paul builds the certainty of our blessedness upon the foundation of the divine purpose, when, from our being called according to the purpose of God, he infers that no one can separate us, etc.]

47 [Paul builds the certainty of our blessedness upon the foundation of the divine purpose, when, from our being called according to the purpose of God, he infers that no one can separate us, etc.]

48 Moreover, this doctrine affords glorious consolation under the cross and amid temptations, namely, that God in His counsel, before the time of the world, determined and decreed that He would assist us in all distresses [anxieties and perplexities], grant patience [under the cross], give consolation, excite [nourish and encourage] hope, and produce such an outcome as would contribute to our salvation.

49 Also, as Paul in a very consolatory way treats this, Rom. 8:28. 29. 35. 38. 39, that God in His purpose has ordained before the time of the world by what crosses and sufferings He would conform every one of His elect to the image of His Son, and that to every one His cross shall and must work together for good, because they are called according to the purpose, whence Paul has concluded that it is certain and indubitable that neither tribulation, nor distress, nor death, nor life, etc., shall be able to separate us from the love of God which is in Christ Jesus, our Lord.

50 This article also affords a glorious testimony that the Church of God will exist and abide in opposition to all the gates of hell, and likewise teaches which is the true Church of God, lest we be offended by the great authority [and majestic appearance] of the false Church, Rom. 9:24. 25.

51 From this article also powerful admonitions and warnings are derived, as Luke 7:30: They rejected the counsel of God against themselves. Luke 14:24: I say unto you that none of those men which were bidden shall taste of my supper. Also Matt. 20:16: Many be called, but few chosen. Also Luke 8:8. 18: He that hath ears to hear, let him hear, and: Take heed how ye hear. Thus the doctrine concerning this article can be employed profitably, comfortingly, and savingly [and can be transferred in many ways to our use].

52 But a distinction must be observed with especial care between that which is expressly revealed concerning it in God’s Word, and what is not revealed. For, in addition to what has been revealed in Christ concerning this, of which we have hitherto spoken, God has still kept secret and concealed much concerning this mystery, and reserved it for His wisdom and knowledge alone, which we should not investigate, nor should we indulge our thoughts in this matter, nor draw conclusions, nor inquire curiously, but should adhere [entirely] to the revealed Word [of God]. This admonition is most urgently needed.

53 For our curiosity has always much more pleasure in concerning itself with these matters [with investigating those things which are hidden and abstruse] than with what God has revealed to us concerning this in His Word, because we cannot harmonize it, which, moreover, we have not been commanded to do [since certain things occur in this mystery so intricate and involved that we are not able by the penetration of our natural ability to harmonize them; but this has not been demanded of us by God].

54 Thus there is no doubt that God most exactly and certainly foresaw before the time of the world, and still knows, which of those that are called will believe or will not believe; also which of the converted will persevere [in faith] and which will not persevere; which will return after a fall [into grievous sins], and which will fall into obduracy [will perish in their sins]. So, too, the number, how many there are of these on either side, is beyond all doubt perfectly known to God.

55 However, since God has reserved this mystery for His wisdom, and has revealed nothing to us concerning it in His Word, much less commanded us to investigate it with our thoughts, but has earnestly discouraged us therefrom, Rom. 11:33ff , we should not reason in our thoughts, draw conclusions, nor inquire curiously into these matters, but should adhere to His revealed Word, to which He points us.

56 Thus without any doubt God also knows and has determined for every one the time and hour of his call and conversion [and when He will raise again one who has lapsed]. But since this has not been revealed to us, we have the command always to keep urging the Word, but to entrust the time and hour [of conversion] to God, Acts 1:7.

57 Likewise, when we see that God gives His Word at one place [to one kingdom or realm], but not at another [to another nation]; removes it from one place [people], and allows it to remain at another; also, that one is hardened, blinded, given over to a reprobate mind, while another, who is indeed in the same guilt, is converted again, etc.,-in these and similar questions Paul [Rom. 11:22ff]

58 fixes a certain limit to us how far we should go, namely, that in the one part we should recognize God’s judgment [for He commands us to consider in those who perish the just judgment of God and the penalties of sins]. For they are well-deserved penalties of sins when God so punishes a land or nation for despising His Word that the punishment extends also to their posterity, as is to be seen in the Jews.

59 And thereby [by the punishments] God in some lands and persons exhibits His severity to those that are His [in order to indicate] what we all would have well deserved, and would be worthy and worth, since we act wickedly in opposition to God’s Word [are ungrateful for the revealed Word, and live unworthily of the Gospel] and often grieve the Holy Ghost sorely, in order that we may live in the fear of God, and acknowledge and praise God’s goodness, to the exclusion of, and contrary to, our merit in and with us, to whom He gives His Word, and with whom He leaves it, and whom He does not harden and reject.

60 For inasmuch as our nature has been corrupted by sin, and is worthy of, and subject to, God’s wrath and condemnation, God owes to us neither the Word, the Spirit, nor grace; and when He bestows these gifts out of grace, we often thrust them from us, and make ourselves unworthy of everlasting life, Acts 13:46. And this His righteous, well-deserved judgment He displays in some countries, nations, and persons, in order that, when we are placed alongside of them and compared with them [and found to be most similar to them], we may learn the more diligently to recognize and praise God’s pure [immense], unmerited grace in the vessels of mercy.

61 For no injustice is done those who are punished and receive the wages of their sins; but in the rest, to whom God gives and preserves His Word, by which men are enlightened, converted, and preserved, God commends His pure [immense] grace and mercy, without their merit.

62 When we proceed thus far in this article, we remain on the right [safe and royal] way, as it is written Hos. 13:9: O Israel, thou hast destroyed thyself; but in Me is thy help.

63 However, as regards these things in this disputation which would soar too high and beyond these limits, we should, with Paul, place the finger upon our lips, and remember and say, Rom. 9:20: O man, who art thou that repliest against God?

64 For that we neither can nor should investigate and fathom everything in this article, the great Apostle Paul declares [teaches by his own example], who, after having argued much concerning this article from the revealed Word of God, as soon as he comes to the point where he shows what God has reserved for His hidden wisdom concerning this mystery, suppresses and cuts it off with the following words, Rom. 11:33f : O the depth of the riches both of the wisdom and knowledge of God! How unsearchable are His judgments, and His ways past finding out! For who hath known the mind of the Lord? that is, outside of and beyond that which He has revealed to us in His Word.

65 Accordingly, this eternal election of God is to be considered in Christ, and not outside of or without Christ. For in Christ, the Apostle Paul testifies, Eph. 1:4f , He hath chosen us before the foundation of the world, as it is written: He hath made us accepted in the Beloved. This election, however, is revealed from heaven through the preaching of His Word, when the Father says, Matt. 17:6: This is My beloved Son, in whom I am well pleased; hear ye Him. And Christ says, Matt. 11:28: Come unto Me, all ye that labor and are heavy laden, and I will give you rest. And concerning the Holy Ghost Christ says, John 16:14: He shalt glorify Me; for He shall receive of Mine, and shall show it unto you.

66 Thus the entire Holy Trinity, God Father, Son, and Holy Ghost, directs all men to Christ, as to the Book of Life, in whom they should seek the eternal election of the Father. For this has been decided by the Father from eternity, that whom He would save He would save through Christ, as He [Christ] Himself says, John 14:6: No man cometh unto the Father but by Me. And again, John 10:9: I am the Door; by Me, if any man enter in, he shall be saved.

67 However, Christ, as the only-begotten Son of God, who is in the bosom of the Father, has announced to us the will of the Father, and thus also our eternal election to eternal life, namely, when He says, Mark 1:15: Repent ye, and believe the Gospel; the kingdom of God is at hand. Likewise He says, John 6:40: This is the will of Him that sent Me, that every one which seeth the Son and believeth on Him may have everlasting life. And again [John 3:16]: God so loved the world, etc. [that He gave His only-begotten Son, that whosoever believeth in Him should not perish, but have everlasting life].

68 This proclamation the Father wishes all men to hear and desires that they come to Christ; and these Christ does not drive from Him, as it is written John 6:37: Him that cometh to Me I will in no wise cast out.

69 And in order that we may come to Christ, the Holy Ghost works true faith through the hearing of the Word, as the apostle testifies when he says, Rom. 10:17: Faith cometh by hearing and hearing by the Word of God, [namely] when it is preached in its truth and purity.

70 Therefore, whoever would be saved should not trouble or harass himself with thoughts concerning the secret counsel of God, as to whether he also is elected and ordained to eternal life, with which miserable Satan usually attacks and annoys godly hearts. But they should hear Christ [and look upon Him as the Book of Life in which is written the eternal election], who is the Book of Life and of God’s eternal election of all of God’s children to eternal life: He testifies to all men without distinction that it is God’s will that all men should come to Him who labor and are heavy laden with sin, in order that He may give them rest and save them, Matt. 11:28.

71 According to this doctrine of His they should abstain from their sins, repent, believe His promise, and entirely trust in Him; and since we cannot do this by ourselves, of our own powers, the Holy Ghost desires to work these things, namely, repentance and faith, in us through the Word and Sacraments.

72 And in order that we may attain this, persevere in it, and remain steadfast, we should implore God for His grace, which He has promised us in Holy Baptism, and, no doubt, He will impart it to us according to His promise, as He has said, Luke 11:11ff : If a son shall ask bread of any of you that is a father, will he give him a stone? Or if he ask a fish, will he for a fish give him a serpent? Or if he shall ask an egg, will he offer him a scorpion? If ye, then, being evil, know how to give good gifts unto your children, how much more shall your heavenly Father give the Holy Spirit to them that ask Him!

73 And since the Holy Ghost dwells in the elect, who have become believers, as in His temple, and is not idle in them, but impels the children of God to obedience to God’s commands, believers, likewise, should not be idle, and much less resist the impulse of God’s Spirit, but should exercise themselves in all Christian virtues, in all godliness, modesty, temperance, patience, brotherly love, and give all diligence to make their calling and election sure, in order that they may doubt the less concerning it, the more they experience the power and strength of the Spirit within them.

74 For the Spirit bears witness to the elect that they are God’s children, Rom. 8:16. And although they sometimes fall into temptation so grievous that they imagine they perceive no more power of the indwelling Spirit of God, and say with David, Ps. 31:22: I said in my haste, I am cut off from before Thine eyes, yet they should, without regard to what they experience in themselves, again [be encouraged and] say with David, as is written ibidem, in the words immediately following: Nevertheless Thou heardest the voice of my supplications when I cried unto Thee.

75 And since our election to eternal life is founded not upon our godliness or virtue, but alone upon the merit of Christ and the gracious will of His Father, who cannot deny Himself, because He is unchangeable in will and essence, therefore, when His children depart from obedience and stumble, He has them called again to repentance through the Word, and the Holy Ghost wishes thereby to be efficacious in them for conversion; and when they turn to Him again in true repentance by a right faith, He will always manifest the old paternal heart to all those who tremble at His Word and from their heart turn again to Him, as it is written, Jer. 3:1: If a man put away his wife, and she go from him and become another man’s, shall he return unto her again? Shall not that land be greatly polluted? But thou hast played the harlot with many lovers; yet return again to Me, saith the Lord.

76 Moreover, the declaration, John 6:44, that no one can come to Christ except the Father draw him, is right and true. However, the Father will not do this without means, but has ordained for this purpose His Word and Sacraments as ordinary means and instruments; and it is the will neither of the Father nor of the Son that a man should not hear or should despise the preaching of His Word, and wait for the drawing of the Father without the Word and Sacraments. For the Father draws indeed by the power of His Holy Ghost, however, according to His usual order [the order decreed and instituted by Himself], by the hearing of His holy, divine Word, as with a net, by which the elect are plucked from the jaws of the devil.

77 Every poor sinner should therefore repair thereto [to holy preaching], hear it attentively, and not doubt the drawing of the Father. For the Holy Ghost will be with His Word in His power, and work by it; and that is the drawing of the Father.

78 But the reason why not all who hear it believe, and some are therefore condemned the more deeply [eternally to severer punishments], is not because God had begrudged them their salvation; but it is their own fault, as they have heard the Word in such a manner as not to learn, but only to despise, blaspheme, and disgrace it, and have resisted the Holy Ghost, who through the Word wished to work in them, as was the case at the time of Christ with the Pharisees and their adherents.

79 Hence the apostle distinguishes with especial care the work of God, who alone makes vessels of honor, and the work of the devil and of man, who by the instigation of the devil, and not of God, has made himself a vessel of dishonor. For thus it is written, Rom. 9:22f : God endured with much longsuffering the vessels of wrath fitted to destruction, that He might make known the riches of His glory on the vessels of mercy, which He had afore prepared unto glory.

80 Here, then, the apostle clearly says that God endured with much long-suffering the vessels of wrath, but does not say that He made them vessels of wrath; for if this had been His will, He would not have required any great long-suffering for it. The fault, however, that they are fitted for destruction belongs to the devil and to men themselves, and not to God.

81 For all preparation for condemnation is by the devil and man, through sin, and in no respect by God, who does not wish that any man be damned; how, then, should He Himself prepare any man for condemnation? For as God is not a cause of sins, so, too, He is no cause of punishment, of damnation; but the only cause of damnation is sin; for the wages of sin is death, Rom. 6:23. And as God does not will sin, and has no pleasure in sin, so He does not wish the death of the sinner either, Ezek. 33:11, nor has He pleasure in his condemnation. For He is not willing that any should perish, but that all should come to repentance, 2 Pet. 3:9. So, too, it is written in Ezek. 18:23; 33:11: As I live, saith the Lord God, I have no pleasure in the death of the wicked, but that the wicked turn from his way and live.

82 And St. Paul testifies in clear words that from vessels of dishonor vessels of honor may be made by God’s power and working, when he writes thus, 2 Tim. 2:21: If a man, therefore, purge himself from these, he shall be a vessel unto honor, sanctified and meet for the Master’s use, and prepared unto every good work. For he who is to purge himself must first have been unclean, and hence a vessel of dishonor. But concerning the vessels of mercy he says clearly that the Lord Himself has prepared them for glory, which he does not say concerning the damned, who themselves, and not God, have prepared themselves as vessels of damnation.

83 Moreover, it is to be diligently considered that when God punishes sin with sins, that is, when He afterwards punishes with obduracy and blindness those who had been converted, because of their subsequent security, impenitence, and wilful sins, this should not be interpreted to mean that it never had been God’s good pleasure that such persons should come to the knowledge of the truth and be saved. For both these facts are God’s revealed will:

First, that God will receive into grace all who repent and believe in Christ.

Secondly, that He also will punish those who wilfully turn away from the holy commandment, and again entangle themselves in the filth of the world, 2 Pet. 2:20, and garnish their hearts for Satan, Luke 11:25f , and do despite unto the Spirit of God, Heb. 10:29, and that they shall be hardened, blinded, and eternally condemned if they persist therein.

84 Accordingly, even Pharaoh (of whom it is written, Ex. 9:16; Rom. 9:17: In very deed for this cause have I raised thee up, for to show in thee My power, and that My name may be declared throughout all the earth) perished, not because God had begrudged him salvation, or because it had been His good pleasure that he should be damned and lost. For God is not willing that any should perish, 2 Pet. 3:9; He also has no pleasure in the death of the wicked, but that the wicked turn from his way and live, Ezek. 33:11.

85 But that God hardened Pharaoh’s heart, namely, that Pharaoh always sinned again and again, and became the more obdurate, the more he was admonished, that was a punishment of his antecedent sin and horrible tyranny, which in many and manifold ways he practised inhumanly and against the accusations of his heart towards the children of Israel. And since God caused His Word to be preached and His will to be proclaimed to him, and Pharaoh nevertheless wilfully reared up straightway against all admonitions and warnings, God withdrew His hand from him, and thus his heart became hardened and obdurate, and God executed His judgment upon him; for he was guilty of nothing else than hell-fire.

86 Accordingly, the holy apostle also introduces the example of Pharaoh for no other reason than to prove by it the justice of God which He exercises towards the impenitent and despisers of His Word; by no means, however, has he intended or understood it to mean that God begrudged salvation to him or any person, but had so ordained him to eternal damnation in His secret counsel that he should not be able, or that it should not be possible for him, to be saved.

87 By this doctrine and explanation of the eternal and saving choice [predestination] of the elect children of God His own glory is entirely and fully given to God, that in Christ He saves us out of pure [and free] mercy, without any merits or good works of ours, according to the purpose of His will, as it is written Eph. 1:5f : Having predestinated us unto the adoption of children by Jesus Christ to Himself, according to the good pleasure of His will, to the praise of the glory of His grace, wherein He hath made us accepted in the Beloved.

88 Therefore it is false and wrong [conflicts with the Word of God] when it is taught that not alone the mercy of God and the most holy merit of Christ, but that also in us there is a cause of God’s election, on account of which God has chosen us to eternal life. For not only before we had done anything good, but also before we were born, yea, even before the foundations of the world were laid, He elected us in Christ; and that the purpose of God according to election might stand, not of works, but of Him that calleth, it was said unto her, The elder shall serve the younger; as it is written concerning this matter, Jacob have I loved, but Esau have I hated, Rom. 9:11ff.; Gen. 25:23; Mal. 1:2f.

89 Moreover, this doctrine gives no one a cause either for despondency or for a shameless, dissolute life, namely, when men are taught that they must seek eternal election in Christ and His holy Gospel, as in the Book of Life, which excludes no penitent sinner, but beckons and calls all the poor, heavy-laden, and troubled sinners [who are disturbed by the sense of God’s wrath], to repentance and the knowledge of their sins and to faith in Christ, and promises the Holy Ghost for purification and renewal,

90 and thus gives the most enduring consolation to all troubled, afflicted men, that they know that their salvation is not placed in their own hands,-for otherwise they would lose it much more easily than was the case with Adam and Eve in paradise, yea, every hour and moment,-but in the gracious election of God, which He has revealed to us in Christ, out of whose hand no man shall pluck us, John 10:28; 2 Tim. 2:19.

91 Accordingly, if any one presents the doctrine concerning the gracious election of God in such a manner that troubled Christians cannot derive comfort from it, but are thereby incited to despair, or that the impenitent are confirmed in their wantonness, it is undoubtedly sure and true that such a doctrine is taught, not according to the Word and will of God, but according to [the blind judgment of human] reason and the instigation of the devil.

92 For, as the apostle testifies, Rom. 15:4: Whatsoever things were written aforetime were written for our learning, that we through patience and comfort of the Scriptures might have hope. But when this consolation and hope are weakened or entirely removed by Scripture, it is certain that it is understood and explained contrary to the will and meaning of the Holy Ghost.

93 By this simple, correct [clear], useful explanation which has a firm and good foundation in God’s revealed will, we abide; we flee from, and shun, all lofty, acute questions and disputations [useless for edifying]; and reject and condemn whatever is contrary to these simple, useful explanations.

94 So much concerning the controverted articles which have been discussed for many years already among the theologians of the Augsburg Confession, in which some have erred and severe controversiae (controversies), that is, religious disputes, have arisen.

95 From this our explanation, friends and enemies, and therefore every one, may clearly infer that we have no intention of yielding aught of the eternal, immutable truth of God for the sake of temporal peace, tranquillity, and unity (which, moreover, is not in our power to do). Nor would such peace and unity, since it is devised against the truth and for its suppression, have any permanency. Still less are we inclined to adorn and conceal a corruption of the pure doctrine and manifest, condemned errors.
96 But we entertain heartfelt pleasure and love for, and are on our part sincerely inclined and anxious to advance, that unity according to our utmost power, by which His glory remains to God uninjured, nothing of the divine truth of the Holy Gospel is surrendered, no room is given to the least error, poor sinners are brought to true, genuine repentance, raised up by faith, confirmed in new obedience, and thus justified and eternally saved alone through the sole merit of Christ.

XII. Other Factions, Heresies, and Sects

Which Never Embraced the Augsburg Confession.

1 However, as regards the sects and factions [sectarists and heretics] which never have embraced the Augsburg Confession, and of which express mention has not been made in this our explanation, such as are the Anabaptists, Schwenckfeldians, New Arians, and Anti-Trinitarians,

2 whose errors have been unanimously condemned by all churches of the Augsburg Confession, we have not wished to make particular and especial mention of them in this explanation, for the reason that at the present time this has been our only aim [that we might above all refute the charges of our adversaries, the Papists].

3 Since our opponents alleged with shameless mouths, and decried throughout all the world our churches and their teachers, claiming that not two preachers are found who agree in each and every article of the Augsburg Confession, but that they are rent asunder and separated from one another to such an extent that they themselves no longer know what is the Augsburg Confession and its proper [true, genuine, and germane] sense;

4 we have not made a joint confession only in brief words or names, but wished to make a pure, clear, distinct declaration concerning all the disputed articles which have been discussed and controverted only among the theologians of the Augsburg Confession,

5 in order that every one may see that we do not wish in a cunning manner to dissemble or cover up all this, or to come to an agreement only in appearance;

6 but to remedy the matter thoroughly, and have wished to set forth our opinion of these matters in such a manner that even our adversaries themselves must confess that in all this we abide by the true, simple, natural, and proper sense of the Augsburg Confession, in which we desire, moreover, by God’s grace, to persevere constantly until our end; and so far as it depends on our service, we will not connive at or be silent, lest anything contrary to the same [the genuine and sacred sense of the Augsburg Confession] is introduced into our churches and schools, in which the almighty God and Father of our Lord Jesus Christ has appointed us teachers and pastors.

7 However, lest there be silently ascribed to us the condemned errors of the above enumerated factions and sects [“of which evil the papistic tyranny, which persecutes the pure doctrine is the chief cause”],

8 -which as is the nature of such spirits, for the most part, secretly stole in at localities, and especially at a time when no place or room was given to the pure Word of the holy Gospel, but all its sincere teachers and confessors were persecuted, and the deep darkness of the Papacy still prevailed, and poor simple men who could not help but feel the manifest idolatry and false faith of the Papacy, in their simplicity, alas! embraced whatever was called the Gospel, and was not papistic,-we could not forbear testifying also against them publicly, before all Christendom, that we have neither part nor fellowship with their errors, be they many or few, but reject and condemn them, one and all, as wrong and heretical, and contrary to the Scriptures of the prophets and apostles, and to our Christian Augsburg Confession, well grounded in God’s Word.
Erroneous Articles of the Anabaptists.

9 Namely, for instance, the erroneous, heretical doctrines of the Anabaptists, which are to be tolerated and allowed neither in the Church, nor in the commonwealth, nor in domestic life, when they teach:

10 1. That our righteousness before God consists not only in the sole obedience and merit of Christ, but in our renewal and our own piety in which we walk before God; which they, for the most part, base upon their own peculiar ordinances and self-chosen spirituality, as upon a new sort of monkery.

11 2. That children who are not baptized are not sinners before God, but righteous and innocent, and thus are saved in their innocency without Baptism, which they do not need. Accordingly, they deny and reject the entire doctrine concerning original sin and what belongs to it.

12 3. That children are not to be baptized until they have attained the use of reason and can confess their faith themselves.

13 4. That the children of Christians, since they have been born of Christian and believing parents, are holy and the children of God even without and before Baptism; and for this reason they neither attach much importance to the baptism of children nor encourage it, contrary to the express words of the promise, which extends only to those who keep God’s covenant and do not despise it, Gen. 17:9.

14 5. That a congregation [church] in which sinners are still found is no true Christian assembly.

15 6. That no sermon should be heard or attended in those churches in which the papal masses have previously been said.

16 7. That no one should have anything to do with those ministers of the Church who preach the holy Gospel according to the Confession, and rebuke the errors of baptists; also, that no one should serve or in any way labor for them, but should flee from and shun them as perverters of God’s Word.

17 8. That under the New Testament the magistracy is not a godly estate.

18 9. That a Christian cannot with a good, inviolate conscience hold the office of magistrate.

19 10. That a Christian cannot without injury to conscience use the office of the magistracy in matters that may occur [when the matter so demands] against the wicked, neither can its subjects appeal to its power.

20 11. That a Christian cannot with a good conscience take an oath before a court, nor with an oath do homage to his prince or hereditary sovereign.

21 12. That magistrates cannot without injury to conscience inflict capital punishment upon evil-doers.

22 13. That a Christian cannot with a good conscience hold or possess any property, but is in duty bound to devote it to the common treasury.

23 14. That a Christian cannot with a good conscience be an inn-keeper, merchant, or cutler.

24 15. That married persons may be divorced on account of faith [diversity of religion], and that the one may abandon the other, and be married to another of his own faith.

25 16. That Christ did not assume His flesh and blood of the Virgin Mary, but brought them with Him from heaven.

26 17. That He is not true, essential God either, but only has more and higher gifts and glory than other men.

27 And still more articles of like kind; for they are divided among themselves into many bands [sects], and one has more and another fewer errors, and thus their entire sect is in reality nothing but a new kind of monkery.
Erroneous Articles of the Schwenckfeldians.

28 Likewise, when the Schwenckfeldians assert:

29 1. First, that all those have no knowledge of the reigning King of heaven, Christ, who regard Christ according to the flesh, or His assumed humanity, as a creature, and that the flesh of Christ has by exaltation so assumed all divine properties that in might, power, majesty, and glory He is in every respect, in degree and position of essence, equal to the Father and the eternal Word, so that there is the same essence, properties, will, and glory of both natures in Christ, and that the flesh of Christ belongs to the essence of the Holy Trinity.

30 2. That the ministry of the Church, the Word preached and heard, is not a means whereby God the Holy Ghost teaches men, and works in them saving knowledge of Christ, conversion, repentance, faith, and new obedience.

31 3. That the water of Baptism is not a means by which God the Lord seals adoption and works regeneration.

32 4. That bread and wine in the Holy Supper are not means by which Christ distributes His body and blood.

33 5. That a Christian man who is truly regenerated by God’s Spirit can in this life keep and fulfil the Law of God perfectly.

34 6. That a congregation in which no public excommunication or regular process of the ban is observed, is no true Christian congregation [church].

35 7. That the minister of the Church who is not on his part truly renewed, righteous, and godly cannot teach other men with profit or administer real, true sacraments.
Erroneous Articles of the New Arians.

36 Also, when the New Arians teach that Christ is not a true, essential, natural God, of one eternal divine essence with God the Father, but is only adorned with divine majesty inferior to, and beside, God the Father.
Erroneous Articles of the New Anti-Trinitarians.

37 1. Also, when some Anti-Trinitarians reject and condemn the ancient approved symbola, Nicaenum et Athanasianum (the Nicene and Athanasian creeds), as regards both their sense and words, and teach that there is not only one eternal divine essence of the Father, Son, and Holy Ghost, but as there are three distinct persons, God the Father, Son, and Holy Ghost, so each person has also its essence distinct and separate from the other persons; yet that all three are either as otherwise three men, distinct and separate in their essence, of the same power, wisdom, majesty, and glory [as some imagine], or in essence and properties unequal [as others think].

38 2. That the Father alone is true God.

39 These and like articles, one and all, with what pertains to them and follows from them, we reject and condemn as wrong, false, heretical, and contrary to the Word of God, the three Creeds, the Augsburg, Confession and Apology, the Smalcald Articles, and the Catechisms of Luther. Of these articles all godly Christians should and ought to beware, as much as the welfare and salvation of their souls is dear to them.
40 Since now, in the sight of God and of all Christendom [the entire Church of Christ], we wish to testify to those now living and those who shall come after us that this declaration herewith presented concerning all the controverted articles aforementioned and explained, and no other, is our faith, doctrine, and confession, in which we are also willing, by God’s grace, to appear with intrepid hearts before the judgment-seat of Jesus Christ, and give an account of it; and that we will neither privately nor publicly speak or write anything contrary to it, but, by the help of God’s grace, intend to abide thereby: therefore, after mature deliberation, we have, in God’s fear and with the invocation of His name, attached our signatures with our own hands.

