THE SMALCALD ARTICLES

By Martin Luther


     _The Smalcald Articles.
     Articles of Christian Doctrine
     which were to have been presented on our part
     to the Council, if any had been assembled at Mantua
     or elsewhere, indicating what we could accept
     or yield, and what we could not._

     by Dr. Martin Luther, 1537

     Translated by F. Bente and  W. H. T. Dau

     Published in:
     _Triglot Concordia: The Symbolical Books
     of the Ev. Lutheran Church_.
     (St. Louis: Concordia Publishing House, 1921),
     pp. 453-529.




Preface of Dr. Martin Luther.

Since Pope Paul III convoked a Council last year, to assemble
at Mantua about Whitsuntide, and afterwards transferred it
from Mantua, so that it is not yet known where he will or can
fix it, and we on our part either had to expect that we would
be summoned also to the Council or [to fear that we would] be
condemned unsummoned, I was directed to compile and collect
the articles of our doctrine [in order that it might be plain]
in case of deliberation as to what and how far we would be
both willing and able to yield to the Papists, and in what
points we intended to persevere and abide to the end.

I have accordingly compiled these articles and presented them
to our side. They have also been accepted and unanimously
confessed by our side, and it has been resolved that, in case
the Pope with his adherents should ever be so bold as
seriously and in good faith, without lying and cheating, to
hold a truly free [legitimate] Christian Council (as, indeed,
he would be in duty bound to do), they be publicly delivered
in order to set forth the Confession of our Faith.

But though the Romish court is so dreadfully afraid of a free
Christian Council, and shuns the light so shamefully, that it
has [entirely] removed, even from those who are on its side,
the hope that it will ever permit a free Council, much less
that it will itself hold one, whereat, as is just, they [many
Papists] are greatly offended and have no little trouble on
that account [are disgusted with this negligence of the Pope],
since they notice thereby that the Pope would rather see all
Christendom perish and all souls damned than suffer either
himself or his adherents to be reformed even a little, and his
[their] tyranny to be limited, nevertheless I have determined
meanwhile to publish these articles in plain print, so that,
should I die before there would be a Council (as I fully
expect and hope, because the knaves who flee the light and
shun the day take such wretched pains to delay and hinder the
Council), those who live and remain after me may have my
testimony and confession to produce, in addition to the
Confession which I have issued previously, whereby up to this
time I have abided, and, by God's grace, will abide.

For what shall I say? How shall I complain? I am still living,
writing, preaching, and lecturing daily; [and] yet there are
found such spiteful men, not only among the adversaries, but
also false brethren that profess to be on our side, as dare to
cite my writings and doctrine directly against myself, and let
me look on and listen, although they know well that I teach
otherwise, and as wish to adorn their venom with my labor, and
under my name to [deceive and] mislead the poor people. [Good
God!] Alas! what first will happen when I am dead?

Indeed, I ought to reply to everything while I am still
living. But, again, how can I alone stop all the mouths of the
devil? especially of those (as they all are poisoned) who will
not hear or notice what we write, but solely exercise
themselves with all diligence how they may most shamefully
pervert and corrupt our word in every letter. These I let the
devil answer, or at last Gods wrath, as they deserve. I often
think of the good Gerson who doubts whether anything good
should be [written and] published. If it is not done, many
souls are neglected who could be delivered: but if it is done,
the devil is there with malignant, villainous tongues without
number which envenom and pervert everything, so that
nevertheless the fruit [the usefulness of the writings] is
prevented. Yet what they gain thereby is manifest. For while
they have lied so shamefully against us and by means of lies
wished to retain the people, God has constantly advanced His
work, and been making their following ever smaller and ours
greater, and by their lies has caused and still causes them to
be brought to shame.

I must tell a story. There was a doctor sent here to
Wittenberg from France, who said publicly before us that his
king was sure and more than sure, that among us there is no
church, no magistrate, no married life, but all live
promiscuously as cattle, and each one does as he pleases.
Imagine now, how will those who by their writings have
instilled such gross lies into the king and other countries as
the pure truth, look at us on that day before the
judgment-seat of Christ? Christ, the Lord and Judge of us all,
knows well that they lie and have [always] lied, His sentence
they in turn, must hear; that I know certainly. God convert to
repentance those who can be converted! Regarding the rest it
will be said, Woe, and, alas! eternally.

But to return to the subject. I verily desire to see a truly
Christian Council [assembled some time], in order that many
matters and persons might be helped. Not that we need It, for
our churches are now, through God's grace, so enlightened and
equipped with the pure Word and right use of the Sacraments,
with knowledge of the various callings and of right works,
that we on our part ask for no Council, and on such points
have nothing better to hope or expect from a Council. But we
see in the bishoprics everywhere so many parishes vacant and
desolate that one's heart would break, and yet neither the
bishops nor canons care how the poor people live or die, for
whom nevertheless Christ has died, and who are not permitted
to hear Him speak with them as the true Shepherd with His
sheep. This causes me to shudder and fear that at some time He
may send a council of angels upon Germany utterly destroying
us, like Sodom and Gomorrah, because we so wantonly mock Him
with the Council.

Besides such necessary ecclesiastical affairs, there would be
also in the political estate innumerable matters of great
importance to improve. There is the disagreement between the
princes and the states; usury and avarice have burst in like a
flood, and have become lawful [are defended with a show of
right]; wantonness, lewdness, extravagance in dress, gluttony,
gambling, idle display, with all kinds of bad habits and
wickedness, insubordination of subjects, of domestics and
laborers of every trade, also the exactions [and most
exorbitant selling prices] of the peasants (and who can
enumerate all?) have so increased that they cannot be
rectified by ten Councils and twenty Diets. If such chief
matters of the spiritual and worldly estates as are contrary
to God would be considered in the Council, they would have all
hands so full that the child's play and absurdity of long
gowns [official insignia], large tonsures, broad cinctures [or
sashes], bishops' or cardinals' hats or maces, and like
jugglery would in the mean time be forgotten. If we first had
performed God's command and order in the spiritual and secular
estate we would find time enough to reform food, clothing,
tonsures, and surplices. But if we want to swallow such
camels, and, instead, strain at gnats, let the beams stand and
judge the motes, we also might indeed be satisfied with the
Council.

Therefore I have presented few articles; for we have without
this so many commands of God to observe in the Church, the
state and the family that we can never fulfil them. What,
then, is the use, or what does it profit that many decrees and
statutes thereon are made in the Council, especially when
these chief matters commanded of God are neither regarded nor
observed? Just as though He were bound to honor our jugglery
as a reward of our treading His solemn commandments under
foot. But our sins weigh upon us and cause God not to be
gracious to us; for we do not repent, and, besides, wish to
defend every abomination.

O Lord Jesus Christ, do Thou Thyself convoke a Council, and
deliver Thy servants by Thy glorious advent! The Pope and his
adherents are done for; they will have none of Thee. Do Thou,
then, help us, who are poor and needy, who sigh to Thee, and
beseech Thee earnestly, according to the grace which Thou hast
given us, through Thy Holy Ghost who liveth and reigneth with
Thee and the Father, blessed forever. Amen.




THE FIRST PART

Treats of the Sublime Articles Concerning the Divine Majesty,
as:

I.
That Father, Son, and Holy Ghost, three distinct persons in
one divine essence and nature, are one God, who has created
heaven and earth.

II.
That the Father is begotten of no one; the Son of the Father;
the Holy Ghost proceeds from Father and Son.

III.
That not the Father nor the Holy Ghost but the Son became man.

IV.
That the Son became man in this manner, that He was conceived,
without the cooperation of man, by the Holy Ghost, and was
born of the pure, holy [and always] Virgin Mary. Afterwards He
suffered, died, was buried, descended to hell, rose from the
dead, ascended to heaven, sits at the right hand of God, will
come to judge the quick and the dead, etc. as the Creed of the
Apostles, as well as that of St. Athanasius, and the Catechism
in common use for children, teach.

Concerning these articles there is no contention or dispute,
since we on both sides confess them. Therefore it is not
necessary now to treat further of them.




THE SECOND PART

Treats of the Articles which Refer to
the Office and Work of Jesus Christ,
or Our Redemption.

The first and chief article is this,

That Jesus Christ, our God and Lord, died for our sins, and
was raised again for our justification, Rom. 4, 25.

And He alone is the Lamb of God which taketh away the sins of
the world, John 1, 29; and God has laid upon Him the
iniquities of us all, Is. 53, 6.

Likewise: All have sinned and are justified without merit
[freely, and without their own works or merits] by His grace,
through the redemption that is in Christ Jesus, in His blood,
Rom. 3, 23 f.

Now, since it is necessary to believe this, and it cannot be
otherwise acquired or apprehended by any work, law, or merit,
it is clear and certain that this faith alone justifies us as
St. Paul says, Rom. 3, 28: For we conclude that a man is
justified by faith, without the deeds of the Law. Likewise v.
26: That He might be just, and the Justifier of him which
believeth in Christ.

Of this article nothing can be yielded or surrendered [nor can
anything be granted or permitted contrary to the same], even
though heaven and earth, and whatever will not abide, should
sink to ruin. For there is none other name under heaven, given
among men whereby we must be saved, says Peter, Acts 4, 12.
And with His stripes we are healed, Is. 53, 5. And upon this
article all things depend which we teach and practice in
opposition to the Pope, the devil, and the [whole] world.
Therefore, we must be sure concerning this doctrine, and not
doubt; for otherwise all is lost, and the Pope and devil and
all things gain the victory and suit over us.




Article II: Of the Mass.

That the Mass in the Papacy must be the greatest and most
horrible abomination, as it directly and powerfully conflicts
with this chief article, and yet above and before all other
popish idolatries it has been the chief and most specious. For
it has been held that this sacrifice or work of the Mass, even
though it be rendered by a wicked [and abandoned] scoundrel,
frees men from sins, both in this life and also in purgatory,
while only the Lamb of God shall and must do this, as has been
said above. Of this article nothing is to be surrendered or
conceded, because the first article does not allow it.

If, perchance, there were reasonable Papists we might speak
moderately and in a friendly way, thus: first, why they so
rigidly uphold the Mass. For it is but a pure invention of
men, and has not been commanded by God; and every invention of
man we may [safely] discard, as Christ declares, Matt. 15, 9:
In vain do they worship Me, teaching for doctrines the
commandments of men.

Secondly. It is an unnecessary thing, which can be omitted
without sin and danger.

Thirdly. The Sacrament can be received in a better and more
blessed way [more acceptable to God], (yea, the only blessed
way), according to the institution of Christ. Why, then, do
they drive the world to woe and [extreme] misery on account of
a fictitious, unnecessary matter, which can be well obtained
in another and more blessed way?

Let [care be taken that] it be publicly preached to the people
that the Mass as men's twaddle [commentitious affair or human
figment] can be omitted without sin, and that no one will be
condemned who does not observe it, but that he can be saved in
a better way without the Mass. I wager [Thus it will come to
pass] that the Mass will then collapse of itself, not only
among the insane [rude] common people, but also among all
pious, Christian, reasonable, God-fearing hearts; and that the
more, when they would hear that the Mass is a [very] dangerous
thing, fabricated and invented without the will and Word of
God.

Fourthly. Since such innumerable and unspeakable abuses have
arisen in the whole world from the buying and selling of
masses, the Mass should by right be relinquished, if for no
other purpose than to prevent abuses, even though in itself it
had something advantageous and good. How much more ought we to
relinquish it, so as to prevent [escape] forever these
horrible abuses, since it is altogether unnecessary, useless,
and dangerous, and we can obtain everything by a more
necessary, profitable, and certain way without the Mass.

Fifthly. But since the Mass is nothing else and can be nothing
else (as the Canon and all books declare), than a work of men
(even of wicked scoundrels), by which one attempts to
reconcile himself and others to God, and to obtain and merit
the remission of sins and grace (for thus the Mass is observed
when it is observed at the very best; otherwise what purpose
would it serve?), for this very reason it must and should
[certainly] be condemned and rejected. For this directly
conflicts with the chief article, which says that it is not a
wicked or a godly hireling of the Mass with his own work, but
the Lamb of God and the Son of God, that taketh away our sins.

But if any one should advance the pretext that as an act of
devotion he wishes to administer the Sacrament, or Communion,
to himself, he is not in earnest [he would commit a great
mistake, and would not be speaking seriously and sincerely].
For if he wishes to commune in sincerity, the surest and best
way for him is in the Sacrament administered according to
Christ's institution. But that one administer communion to
himself is a human notion, uncertain, unnecessary, yea, even
prohibited. And he does not know what he is doing, because
without the Word of God he obeys a false human opinion and
invention. So, too, it is not right (even though the matter
were otherwise correct) for one to use the common Sacrament of
[belonging to] the Church according to his own private
devotion, and without God's Word and apart from the communion
of the Church to trifle therewith.

This article concerning the Mass will be the whole business of
the Council. [The Council will perspire most over, and be
occupied with this article concerning the Mass.] For if it
were [although it would be] possible for them to concede to us
all the other articles, yet they could not concede this. As
Campegius said at Augsburg that he would be torn to pieces
before he would relinquish the Mass, so, by the help of God,
I, too, would suffer myself to be reduced to ashes before I
would allow a hireling of the Mass, be he good or bad, to be
made equal to Christ Jesus, my Lord and Savior, or to be
exalted above Him. Thus we are and remain eternally separated
and opposed to one another. They feel well enough that when
the Mass falls, the Papacy lies in ruins. Before they will
permit this to occur, they will put us all to death if they
can.

In addition to all this, this dragon's tail, [I mean] the
Mass, has begotten a numerous vermin-brood of manifold
idolatries.

First, purgatory. Here they carried their trade into purgatory
by masses for souls, and vigils, and weekly, monthly, and
yearly celebrations of obsequies, and finally by the Common
Week and All Souls Day, by soul-baths so that the Mass is used
almost alone for the dead, although Christ has instituted the
Sacrament alone for the living. Therefore purgatory, and every
solemnity, rite, and commerce connected with it, is to be
regarded as nothing but a specter of the devil. For it
conflicts with the chief article [which teaches] that only
Christ, and not the works of men, are to help [set free]
souls. Not to mention the fact that nothing has been
[divinely] commanded or enjoined upon us concerning the dead.
Therefore all this may be safely omitted, even if it were no
error and idolatry.

The Papists quote here Augustine and some of the Fathers who
are said to have written concerning purgatory, and they think
that we do not understand for what purpose and to what end
they spoke as they did. St. Augustine does not write that
there is a purgatory nor has he a testimony of Scripture to
constrain him thereto, but he leaves it in doubt whether there
is one, and says that his mother asked to be remembered at the
altar or Sacrament. Now, all this is indeed nothing but the
devotion of men, and that, too, of individuals, and does not
establish an article of faith, which is the prerogative of God
alone.

Our Papists, however, cite such statements [opinions] of men
in order that men should believe in their horrible,
blasphemous, and cursed traffic in masses for souls in
purgatory [or in sacrifices for the dead and oblations], etc.
But they will never prove these things from Augustine. Now,
when they have abolished the traffic in masses for purgatory,
of which Augustine never dreamt, we will then discuss with
them whether the expressions of Augustine without Scripture
[being without the warrant of the Word] are to be admitted,
and whether the dead should be remembered at the Eucharist.
For it will not do to frame articles of faith from the works
or words of the holy Fathers; otherwise their kind of fare, of
garments, of house, etc., would have to become an article of
faith, as was done with relies. [We have, however, another
rule, namely] The rule is: The Word of God shall establish
articles of faith, and no one else, not even an angel.

Secondly. From this it has followed that evil spirits have
perpetrated much knavery [exercised their malice] by appearing
as the souls of the departed, and with unspeakable [horrible]
lies and tricks demanded masses, vigils, pilgrimages, and
other alms. All of which we had to receive as articles of
faith, and to live accordingly; and the Pope confirmed these
things, as also the Mass and all other abominations. Here,
too, there is no [cannot and must not be any] yielding or
surrendering.

Thirdly. [Hence arose] the pilgrimages. Here, too, masses, the
remission of sins and the grace of God were sought, for the
Mass controlled everything. Now it is indeed certain that such
pilgrimages, without the Word of God, have not been commanded
us, neither are they necessary, since we can have these things
[the soul can be cared for] in a better way, and can omit
these pilgrimages without any sin and danger. Why therefore do
they leave at home [desert] their own parish [their called
ministers, their parishes], the Word of God, wives, children,
etc., who are ordained and [attention to whom is necessary and
has been] commanded, and run after these unnecessary,
uncertain, pernicious will-o'-the-wisps of the devil [and
errors]? Unless the devil was riding [made insane] the Pope,
causing him to praise and establish these practices, whereby
the people again and again revolted from Christ to their own
works, and became idolaters, which is worst of all; moreover,
it is neither necessary nor commanded, but is senseless and
doubtful, and besides harmful. Hence here, too, there can be
no yielding or surrendering [to yield or concede anything here
is not lawful], etc. And let this be preached, that such
pilgrimages are not necessary, but dangerous; and then see
what will become of them. [For thus they will perish of their
own accord.]

Fourthly. Fraternities [or societies], in which cloisters,
chapters, vicars have assigned and communicated (by a legal
contract and sale) all masses and good works, etc., both for
the living and the dead. This is not only altogether a human
bauble, without the Word of God, entirely unnecessary and not
commanded, but also contrary to the chief article, Of
Redemption. Therefore it is in no way to be tolerated.

Fifthly. The relics, in which there are found so many
falsehoods and tomfooleries concerning the bones of dogs and
horses, that even the devil has laughed at such rascalities,
ought long ago to have been condemned, even though there were
some good in them; and so much the more because they are
without the Word of God; being neither commanded nor
counseled, they are an entirely unnecessary and useless thing.
But the worst is that [they have imagined that] these relics
had to work indulgence and the forgiveness of sins [and have
revered them] as a good work and service of God, like the
Mass, etc.

Sixthly. Here belong the precious indulgences granted (but
only for money) both to the living and the dead, by which the
miserable [sacrilegious and accursed] Judas, or Pope, has sold
the merit of Christ, together with the superfluous merits of
all saints and of the entire Church, etc. All these things
[and every single one of them] are not to be borne, and are
not only without the Word of God, without necessity, not
commanded, but are against the chief article. For the merit of
Christ is [apprehended and] obtained not by our works or
pence, but from grace through faith, without money and merit;
and is offered [and presented] not through the power of the
Pope, but through the preaching of God's Word.


Of the Invocation of Saints.

The invocation of saints is also one of the abuses of
Antichrist conflicting with the chief article, and destroys
the knowledge of Christ. Neither is it commanded nor
counseled, nor has it any example [or testimony] in Scripture,
and even though it were a precious thing, as it is not [while,
on the contrary, it is a most harmful thing], in Christ we
have everything a thousandfold better [and surer, so that we
are not in need of calling upon the saints].

And although the angels in heaven pray for us (as Christ
Himself also does), as also do the saints on earth, and
perhaps also in heaven, yet it does not follow thence that we
should invoke and adore the angels and saints, and fast, hold
festivals, celebrate Mass in their honor, make offerings, and
establish churches, altars, divine worship, and in still other
ways serve them, and regard them as helpers in need [as
patrons and intercessors], and divide among them all kinds of
help, and ascribe to each one a particular form of assistance,
as the Papists teach and do. For this is idolatry, and such
honor belongs alone to God. For as a Christian and saint upon
earth you can pray for me, not only in one, but in many
necessities. But for this reason I am not obliged to adore and
invoke you, and celebrate festivals, fast, make oblations,
hold masses for your honor [and worship], and put my faith in
you for my salvation. I can in other ways indeed honor, love,
and thank you in Christ. If now such idolatrous honor were
withdrawn from angels and departed saints, the remaining honor
would be without harm and would quickly be forgotten. For when
advantage and assistance, both bodily and spiritual, are no
more to be expected, the saints will not be troubled [the
worship of the saints will soon vanish], neither in their
graves nor in heaven. For without a reward or out of pure love
no one will much remember, or esteem, or honor them [bestow on
them divine honor].

In short, the Mass itself and anything that proceeds from it,
and anything that is attached to it, we cannot tolerate, but
must condemn, in order that we may retain the holy Sacrament
pure and certain, according to the institution of Christ,
employed and received through faith.




Article III: Of Chapters and Cloisters.

That chapters and cloisters [colleges of canons and
communistic dwellings], which were formerly founded with the
good intention [of our forefathers] to educate learned men and
chaste [and modest] women, ought again to be turned to such
use, in order that pastors, preachers, and other ministers of
the churches may be had, and likewise other necessary persons
[fitted] for [the political administration of] the secular
government [or for the commonwealth] in cities and countries,
and well-educated, maidens for mothers and housekeepers, etc.

If they will not serve this purpose, it is better that they be
abandoned or razed, rather than [continued and], with their
blasphemous services invented by men, regarded as something
better than the ordinary Christian life and the offices and
callings ordained by God. For all this also is contrary to the
first chief article concerning the redemption made through
Jesus Christ. Add to this that (like all other human
inventions) these have neither been commanded; they are
needless and useless, and, besides, afford occasion for
dangerous and vain labor [dangerous annoyances and fruitless
worship], such services as the prophets call Aven, i.e., pain
and labor.




Article IV: Of the Papacy.

That the Pope is not, according to divine law or according to
the Word of God the head of all Christendom (for this [name]
belongs to One only, whose name is Jesus Christ), but is only
the bishop and pastor of the Church at Rome, and of those who
voluntarily or through a human creature (that is, a political
magistrate) have attached themselves to him, to be Christians,
not under him as a lord, but with him as brethren [colleagues]
and comrades, as the ancient councils and the age of St.
Cyprian show.

But to-day none of the bishops dare to address the Pope as
brother as was done at that time [in the age of Cyprian]; but
they must call him most gracious lord, even though they be
kings or emperors. This [Such arrogance] we will not, cannot,
must not take upon our conscience [with a good conscience
approve]. Let him, however, who will do it, do so without us
[at his own risk].

Hence it follows that all things which the Pope, from a power
so false, mischievous, blasphemous, and arrogant, has done and
undertaken, have been and still are purely diabolical affairs
and transactions (with the exception of such things as pertain
to the secular government, where God often permits much good
to be effected for a people, even through a tyrant and
[faithless] scoundrel) for the ruin of the entire holy
[catholic or] Christian Church (so far as it is in his power)
and for the destruction of the first and chief article
concerning the redemption made through Jesus Christ.

For all his bulls and books are extant, in which he roars like
a lion (as the angel in Rev. 12 depicts him), [crying out] that
no Christian can be saved unless he obeys him and is subject
to him in all things that he wishes, that he says, and that he
does. All of which amounts to nothing less than saying:
Although you believe in Christ, and have in Him [alone]
everything that is necessary to salvation, yet it is nothing
and all in vain unless you regard [have and worship] me as
your god, and be subject and obedient to me. And yet it is
manifest that the holy Church has been without the Pope for at
least more than five hundred years, and that even to the
present day the churches of the Greeks and of many other
languages neither have been nor are yet under the Pope.
Besides, as often remarked, it is a human figment which is not
commanded, and is unnecessary and useless; for the holy
Christian [or catholic] Church can exist very well without
such a head, and it would certainly have remained better
[purer, and its career would have been more prosperous] if
such a head had not been raised up by the devil. And the
Papacy is also of no use in the Church, because it exercises
no Christian office; and therefore it is necessary for the
Church to continue and to exist without the Pope.

And supposing that the Pope would yield this point, so as not
to be supreme by divine right or from Gods command, but that
we must have [there must be elected] a [certain] head, to whom
all the rest adhere [as their support] in order that the
[concord and] unity of Christians may be preserved against
sects and heretics, and that such a head were chosen by men,
and that it were placed within the choice and power of men to
change or remove this head, just as the Council of Constance
adopted nearly this course with reference to the Popes,
deposing three and electing a fourth; supposing, I say, that
the Pope and See at Rome would yield and accept this (which,
nevertheless, is impossible; for thus he would have to suffer
his entire realm and estate to be overthrown and destroyed,
with all his rights and books, a thing which, to speak in few
words, he cannot do), nevertheless, even in this way
Christianity would not be helped, but many more sects would
arise than before.

For since men would have to be subject to this head, not from
God's command, but from their personal good pleasure, it would
easily and in a short time be despised, and at last retain no
member; neither would it have to be forever confined to Rome
or any other place, but it might be wherever and in whatever
church God would grant a man fit for the [taking upon him such
a great] office. Oh, the complicated and confused state of
affairs [perplexity] that would result!

Therefore the Church can never be better governed and
preserved than if we all live under one head, Christ, and all
the bishops equal in office (although they be unequal in
gifts), be diligently joined in unity of doctrine, faith,
Sacraments, prayer, and works of love, etc., as St. Jerome
writes that the priests at Alexandria together and in common
governed the churches, as did also the apostles, and
afterwards all bishops throughout all Christendom, until the
Pope raised his head above all.

This teaching shows forcefully that the Pope is the very
Antichrist, who has exalted himself above, and opposed himself
against Christ because he will not permit Christians to be
saved without his power, which, nevertheless, is nothing, and
is neither ordained nor commanded by God. This is, properly
speaking to exalt himself above all that is called God as Paul
says, 2 Thess. 2, 4. Even the Turks or the Tartars, great
enemies of Christians as they are, do not do this, but they
allow whoever wishes to believe in Christ, and take bodily
tribute and obedience from Christians.

The Pope, however, prohibits this faith, saying that to be
saved a person must obey him. This we are unwilling to do,
even though on this account we must die in God s name. This
all proceeds from the fact that the Pope has wished to be
called the supreme head of the Christian Church by divine
right. Accordingly he had to make himself equal and superior
to Christ, and had to cause himself to be proclaimed the head
and then the lord of the Church, and finally of the whole
world, and simply God on earth, until he has dared to issue
commands even to the angels in heaven. And when we distinguish
the Pope s teaching from, or measure and hold it against, Holy
Scripture, it is found [it appears plainly] that the Pope s
teaching, where it is best, has been taken from the imperial
and heathen law and treats of political matters and decisions
or rights, as the Decretals show; furthermore, it teaches of
ceremonies concerning churches, garments, food, persons and
[similar] puerile, theatrical and comical things without
measure, but in all these things nothing at all of Christ,
faith, and the commandments of God. Lastly, it is nothing else
than the devil himself, because above and against God he urges
[and disseminates] his [papal] falsehoods concerning masses,
purgatory, the monastic life, one's own works and [fictitious]
divine worship (for this is the very Papacy [upon each of
which the Papacy is altogether founded and is standing]), and
condemns, murders and tortures all Christians who do not exalt
and honor these abominations [of the Pope] above all things.
Therefore, just as little as we can worship the devil himself
as Lord and God, we can endure his apostle, the Pope, or
Antichrist, in his rule as head or lord. For to lie and to
kill, and to destroy body and soul eternally, that is wherein
his papal government really consists, as I have very clearly
shown in many books.

In these four articles they will have enough to condemn in the
Council. For they cannot and will not concede us even the
least point in one of these articles. Of this we should be
certain, and animate ourselves with [be forewarned and made
firm in] the hope that Christ, our Lord, has attacked His
adversary, and he will press the attack home [pursue and
destroy him] both by His Spirit and coming. Amen.

For in the Council we will stand not before the Emperor or the
political magistrate, as at Augsburg (where the Emperor
published a most gracious edict, and caused matters to be
heard kindly [and dispassionately]), but [we will appear]
before the Pope and devil himself, who intends to listen to
nothing, but merely [when the case has been publicly
announced] to condemn, to murder and to force us to idolatry.
Therefore we ought not here to kiss his feet, or to say: Thou
art my gracious lord, but as the angel in Zechariah 3, 2 said
to Satan: The Lord rebuke thee, O Satan.




THE THIRD PART OF THE ARTICLES.

Concerning the following articles we may [will be able to]
treat with learned and reasonable men, or among ourselves. The
Pope and his [the Papal] government do not care much about
these. For with them conscience is nothing, but money, [glory]
honors, power are [to them] everything.

I. Of Sin.

Here we must confess, as Paul says in Rom. 5, 11, that sin
originated [and entered the world] from one man Adam, by whose
disobedience all men were made sinners, [and] subject to death
and the devil. This is called original or capital sin.

The fruits of this sin are afterwards the evil deeds which are
forbidden in the Ten Commandments, such as [distrust]
unbelief, false faith, idolatry, to be without the fear of
God, presumption [recklessness], despair, blindness [or
complete loss of sight], and, in short not to know or regard
God; furthermore to lie, to swear by [to abuse] God's name [to
swear falsely], not to pray, not to call upon God, not to
regard [to despise or neglect] God's Word, to be disobedient
to parents, to murder, to be unchaste, to steal, to deceive,
etc.

This hereditary sin is so deep and [horrible] a corruption of
nature that no reason can understand it, but it must be
[learned and] believed from the revelation of Scriptures, Ps.
51, 5; Rom. 6, 12 ff.; Ex. 33, 3; Gen. 3, 7 ff. Hence, it is
nothing but error and blindness in regard to this article what
the scholastic doctors have taught, namely:

That since the fall of Adam the natural powers of man have
remained entire and incorrupt, and that man by nature has a
right reason and a good will; which things the philosophers
teach.

Again that man has a free will to do good and omit evil, and,
conversely, to omit good and do evil.

Again, that man by his natural powers can observe and keep
[do] all the commands of God.

Again, that, by his natural powers, man can love God above all
things and his neighbor as himself.

Again, if a man does as much as is in him, God certainly
grants him His grace.

Again, if he wishes to go to the Sacrament, there is no need
of a good intention to do good, but it is sufficient if he has
not a wicked purpose to commit sin; so entirely good is his
nature and so efficacious the Sacrament.

[Again,] that it is not founded upon Scripture that for a good
work the Holy Ghost with His grace is necessary.

Such and many similar things have arisen from want of
understanding and ignorance as regards both this sin and
Christ, our Savior and they are truly heathen dogmas, which we
cannot endure. For if this teaching were right [approved],
then Christ has died in vain, since there is in man no defect
nor sin for which he should have died; or He would have died
only for the body, not for the soul, inasmuch as the soul is
[entirely] sound, and the body only is subject to death.


II. Of the Law

Here we hold that the Law was given by God, first, to restrain
sin by threats and the dread of punishment, and by the promise
and offer of grace and benefit. But all this miscarried on
account of the wickedness which sin has wrought in man. For
thereby a part [some] were rendered worse, those, namely, who
are hostile to [hate] the Law, because it forbids what they
like to do, and enjoins what they do not like to do.
Therefore, wherever they can escape [if they were not
restrained by] punishment, they [would] do more against the
Law than before. These, then, are the rude and wicked
[unbridled and secure] men, who do evil wherever they [notice
that they] have the opportunity.

The rest become blind and arrogant [are smitten with arrogance
and blindness], and [insolently] conceive the opinion that
they observe and can observe the Law by their own powers, as
has been said above concerning the scholastic theologians;
thence come the hypocrites and [self-righteous or] false
saints.

But the chief office or force of the Law is that it reveal
original sin with all its fruits, and show man how very low
his nature has fallen, and has become [fundamentally and]
utterly corrupted; as the Law must tell man that he has no God
nor regards [cares for] God, and worships other gods, a matter
which before and without the Law he would not have believed.
In this way he becomes terrified, is humbled, desponds,
despairs, and anxiously desires aid, but sees no escape; he
begins to be an enemy of [enraged at] God, and to murmur, etc.
This is what Paul says, Rom. 4, 15: The Law worketh wrath. And
Rom. 5, 20: Sin is increased by the Law. [The Law entered that
the offense might abound.]


III. Of Repentance.

This office [of the Law] the New Testament retains and urges,
as St. Paul, Rom. 1, 18 does, saying: The wrath of God is
revealed from heaven against all ungodliness and
unrighteousness of men. Again, 3, 19: All the world is guilty
before God. No man is righteous before Him. And Christ says,
John 16, 8: The Holy Ghost will reprove the world of sin.

This, then, is the thunderbolt of God by which He strikes in a
heap [hurls to the ground] both manifest sinners and false
saints [hypocrites], and suffers no one to be in the right
[declares no one righteous], but drives them all together to
terror and despair. This is the hammer, as Jeremiah says, 23,
29: Is not My Word like a hammer that breaketh the rock in
pieces? This is not activa contritio or manufactured
repentance, but passiva contritio [torture of conscience],
true sorrow of heart, suffering and sensation of death.

This, then, is what it means to begin true repentance; and
here man must hear such a sentence as this: You are all of no
account, whether you be manifest sinners or saints [in your
own opinion]; you all must become different and do otherwise
than you now are and are doing [no matter what sort of people
you are], whether you are as great, wise, powerful, and holy
as you may. Here no one is [righteous, holy], godly, etc.

But to this office the New Testament immediately adds the
consolatory promise of grace through the Gospel, which must be
believed, as Christ declares, Mark 1,15: Repent and believe
the Gospel, i.e., become different and do otherwise, and
believe My promise. And John, preceding Him, is called a
preacher of repentance, however, for the remission of sins,
i.e., John was to accuse all, and convict them of being
sinners, that they might know what they were before God, and
might acknowledge that they were lost men, and might thus be
prepared for the Lord, to receive grace, and to expect and
accept from Him the remission of sins. Thus also Christ
Himself says, Luke 24, 47: Repentance and remission of sins
must be preached in My name among all nations.

But whenever the Law alone, without the Gospel being added
exercises this its office there is [nothing else than] death
and hell, and man must despair, like Saul and Judas; as St.
Paul, Rom. 7, 10, says: Through sin the Law killeth. On the
other hand, the Gospel brings consolation and remission not
only in one way, but through the word and Sacraments, and the
like, as we shall hear afterward in order that [thus] there is
with the Lord plenteous redemption, as Ps. 130, 7 says against
the dreadful captivity of sin.

However, we must now contrast the false repentance of the
sophists with true repentance, in order that both may be the
better understood.

Of the False Repentance of the Papists.

It was impossible that they should teach correctly concerning
repentance, since they did not [rightly] know the real sins
[the real sin]. For, as has been shown above, they do not
believe aright concerning original sin, but say that the
natural powers of man have remained [entirely] unimpaired and
incorrupt; that reason can teach aright, and the will can in
accordance therewith do aright [perform those things which are
taught], that God certainly bestows His grace when a man does
as much as is in him, according to his free will.

It had to follow thence [from this dogma] that they did [must
do] penance only for actual sins such as wicked thoughts to
which a person yields (for wicked emotion [concupiscence,
vicious feelings, and inclinations], lust and improper
dispositions [according to them] are not sins ), and for
wicked words and wicked deeds, which free will could readily
have omitted.

And of such repentance they fix three parts contrition,
confession, and satisfaction, with this [magnificent]
consolation and promise added: If man truly repent, [feel
remorse,] confess, render satisfaction, he thereby would have
merited forgiveness, and paid for his sins before God [atoned
for his sins and obtained a plenary redemption]. Thus in
repentance they instructed men to repose confidence in their
own works. Hence the expression originated, which was employed
in the pulpit when public absolution was announced to the
people: Prolong O God, my life, until I shall make
satisfaction for my sins and amend my life.

There was here [profound silence and] no mention of Christ nor
faith; but men hoped by their own works to overcome and blot
out sins before God. And with this intention we became priests
and monks, that we might array ourselves against sin.

As to contrition, this is the way it was done: Since no one
could remember all his sins (especially as committed through
an entire year), they inserted this provision, namely, that if
an unknown sin should be remembered later [if the remembrance
of a concealed sin should perhaps return], this also must be
repented of and confessed etc. Meanwhile they were [the person
was] commended to the grace of God.

Moreover, since no one could know how great the contrition
ought to be in order to be sufficient before God, they gave
this consolation: He who could not have contrition, at least
ought to have attrition, which I may call half a contrition or
the beginning of contrition, for they have themselves
understood neither of these terms nor do they understand them
now, as little as I. Such attrition was reckoned as contrition
when a person went to confession.

And when it happened that any one said that he could not have
contrition nor lament his sins (as might have occurred in
illicit love or the desire for revenge, etc.), they asked
whether he did not wish or desire to have contrition [lament].
When one would reply Yes (for who, save the devil himself,
would here say No?), they accepted this as contrition, and
forgave him his sins on account of this good work of his
[which they adorned with the name of contrition]. Here they
cited the example of St. Bernard, etc.

Here we see how blind reason, in matters pertaining to God,
gropes about, and, according to its own imagination, seeks for
consolation in its own works, and cannot think of [entirely
forgets] Christ and faith. But if it be [clearly] viewed in
the light, this contrition is a manufactured and fictitious
thought [or imagination], derived from man's own powers,
without faith and without the knowledge of Christ. And in it
the poor sinner, when he reflected upon his own lust and
desire for revenge, would sometimes [perhaps] have laughed
rather than wept [either laughed or wept, rather than to think
of something else], except such as either had been truly
struck by [the lightning of] the Law, or had been vainly vexed
by the devil with a sorrowful spirit. Otherwise [with the
exception of these persons] such contrition was certainly mere
hypocrisy, and did not mortify the lust for sins [flames of
sin]; for they had to grieve, while they would rather have
continued to sin, if it had been free to them.

As regards confession, the procedure was this: Every one had
[was enjoined] to enumerate all his sins (which is an
impossible thing). This was a great torment. From such as he
had forgotten [But if any one had forgotten some sins] he
would be absolved on the condition that, if they would occur
to him, he must still confess them. In this way he could never
know whether he had made a sufficiently pure confession
[perfectly and correctly], or when confessing would ever have
an end. Yet he was pointed to his own works, and comforted
thus: The more fully [sincerely and frankly] one confesses,
and the more he humiliates himself and debases himself before
the priest, the sooner and better he renders satisfaction for
his sins; for such humility certainly would earn grace before
God.

Here, too, there was no faith nor Christ, and the virtue of
the absolution was not declared to him, but upon his
enumeration of sins and his self-abasement depended his
consolation. What torture, rascality, and idolatry such
confession has produced is more than can be related.

As to satisfaction, this is by far the most involved
[perplexing] part of all. For no man could know how much to
render for a single sin, not to say how much for all. Here
they have resorted to the device of imposing a small
satisfaction, which could indeed be rendered, as five
Paternosters, a day's fast, etc.; for the rest [that was
lacking] of the [in their] repentance they were directed to
purgatory.

Here, too, there was nothing but anguish and [extreme] misery.
[For] some thought that they would never get out of purgatory,
because, according to the old canons seven years' repentance
is required for a single mortal sin. Nevertheless, confidence
was placed upon our work of satisfaction, and if the
satisfaction could have been perfect, confidence would have
been placed in it entirely, and neither faith nor Christ would
have been of use. But this confidence was impossible. For
although any one had done penance in that way for a hundred
years, he would still not have known whether he had finished
his penance. That meant forever to do penance and never to
come to repentance.

Here now the Holy See at Rome, coming to the aid of the poor
Church, invented indulgences, whereby it forgave and remitted
[expiation or] satisfaction, first, for a single instance, for
seven years, for a hundred years and distributed them among
the cardinals and bishops, so that one could grant indulgence
for a hundred years and another for a hundred days. But he
reserved to himself alone the power to remit the entire
satisfaction.

Now, since this began to yield money, and the traffic in bulls
became profitable he devised the golden jubilee year [a truly
goldbearing year], and fixed it at Rome. He called this the
remission of all punishment and guilt. Then the people came
running, because every one would fain have been freed from
this grievous, unbearable burden. This meant to find [dig up]
and raise the treasures of the earth. Immediately the Pope
pressed still further, and multiplied the golden years one
upon another. But the more he devoured money, the wider grew
his maw.

Later, therefore, he issued them [those golden years of his]
by his legates [everywhere] to the countries, until all
churches and houses were full of the Golden Year. At last he
also made an inroad into purgatory among the dead, first, by
founding masses and vigils, afterwards, by indulgences and the
Golden Year, and finally souls became so cheap that he
released one for a farthing.

But all this, too, was of no avail. For although the Pope
taught men to depend upon, and trust in, these indulgences
[for salvation], yet he rendered the [whole] matter again
uncertain. For in his bulls he declares: Whoever would share
in the indulgences or a Golden Year must be contrite, and have
confessed, and pay money. Now, we have heard above that this
contrition and confession are with them uncertain and
hypocrisy. Likewise, also no one knew what soul was in
purgatory, and if some were therein, no one knew which had
properly repented and confessed. Thus he took the precious
money [the Pope snatched up the holy pence], and comforted
them meanwhile with [led them to confidence in] his power and
indulgence, and [then again led them away from that and]
directed them again to their uncertain work.

If, now [although], there were some who did not believe
[acknowledge] themselves guilty of such actual sins in
[committed by] thoughts, words, and works,--as I, and such
as I, in monasteries and chapters [fraternities or colleges of
priests], wished to be monks and priests, and by fasting,
watching, praying, saying Mass, coarse garments, and hard
beds, etc., fought against [strove to resist] evil thoughts,
and in full earnest and with force wanted to be holy, and yet
the hereditary, inborn evil sometimes did in sleep what it is
wont to do (as also St. Augustine and Jerome among others
confess),--still each one held the other in esteem, so that
some, according to our teaching, were regarded as holy,
without sin and full of good works, so much so that with this
mind we would communicate and sell our good works to others,
as being superfluous to us for heaven. This is indeed true,
and seals, letters, and instances [that this happened] are at
hand.

[When there were such, I say] These did not need repentance.
For of what would they repent, since they had not indulged
wicked thoughts? What would they confess [concerning words not
uttered], since they had avoided words? For what should they
render satisfaction, since they were so guiltless of any deed
that they could even sell their superfluous righteousness to
other poor sinners? Such saints were also the Pharisees and
scribes in the time of Christ.

Here comes the fiery angel, St. John [Rev. 10], the true
preacher of [true] repentance, and with one [thunderclap and]
bolt hurls both [those selling and those buying works] on one
heap, and says: Repent! Matt. 3, 2. Now, the former [the poor
wretches] imagine: Why, we have repented! The latter [the
rest] say: We need no repentance. John says: Repent ye, both
of you, for ye are false penitents; so are these [the rest]
false saints [or hypocrites], and all of you on either side
need the forgiveness of sins, because neither of you know what
true sin is not to say anything about your duty to repent of
it and shun it. For no one of you is good; you are full of
unbelief, stupidity, and ignorance of God and God's will. For
here He is present of whose fulness have all we received, and
grace for grace, John 1, 16, and without Him no man can be
just before God. Therefore, if you wish to repent, repent
aright- your penance will not accomplish anything [is
nothing]. And you hypocrites, who do not need repentance, you
serpents' brood, who has assured you that you will escape the
wrath to come? etc. Matt. 3, 7; Luke 3, 7.

In the same way Paul also preaches, Rom. 3, 10-12: There is
none righteous, there is none that understandeth, there is
none that seeketh after God, there is none that doeth good, no
not one; they are all gone out of the way; they are together
become unprofitable. And Acts 17, 30: God now commandeth all
men everywhere to repent. "All men," he says; no one excepted
who is a man. This repentance teaches us to discern sin,
namely, that we are altogether lost, and that there is nothing
good in us from head to foot [both within and without], and
that we must absolutely become new and other men.

This repentance is not piecemeal [partial] and beggarly
[fragmentary], like that which does penance for actual sins,
nor is it uncertain like that. For it does not debate what is
or is not sin, but hurls everything on a heap, and says: All
in us is nothing but sin [affirms that, with respect to us,
all is simply sin (and there is nothing in us that is not sin
and guilt)]. What is the use of [For why do we wish]
investigating, dividing, or distinguishing a long time? For
this reason, too, this contrition is not [doubtful or]
uncertain. For there is nothing left with which we can think
of any good thing to pay for sin, but there is only a sure
despairing concerning all that we are, think, speak, or do
[all hope must be cast aside in respect of everything], etc.

In like manner confession, too, cannot be false, uncertain, or
piecemeal [mutilated or fragmentary]. For he who confesses
that all in him is nothing but sin comprehends all sins
excludes none, forgets none. Neither can the satisfaction be
uncertain, because it is not our uncertain, sinful work, but
it is the suffering and blood of the [spotless and] innocent
Lamb of God who taketh away the sin of the world.

Of this repentance John preaches, and afterwards Christ in the
Gospel, and we also. By this [preaching of] repentance we dash
to the ground the Pope and everything that is built upon our
good works. For all is built upon a rotten and vain
foundation, which is called a good work or law, even though no
good work is there, but only wicked works, and no one does the
Law (as Christ, John 7, 19, says), but all transgress it.
Therefore the building [that is raised upon it] is nothing but
falsehood and hypocrisy, even [in the part] where it is most
holy and beautiful.

And in Christians this repentance continues until death,
because, through the entire life it contends with sin
remaining in the flesh, as Paul, Rom. 7, 14-25, [shows]
testifies that he wars with the law in his members, etc.; and
that, not by his own powers, but by the gift of the Holy Ghost
that follows the remission of sins. This gift daily cleanses
and sweeps out the remaining sins, and works so as to render
man truly pure and holy.

The Pope, the theologians, the jurists, and every other man
know nothing of this [from their own reason], but it is a
doctrine from heaven, revealed through the Gospel, and must
suffer to be called heresy by the godless saints [or
hypocrites].

On the other hand, if certain sectarists would arise, some of
whom are perhaps already extant, and in the time of the
insurrection [of the peasants] came to my own view, holding
that all those who had once received the Spirit or the
forgiveness of sins, or had become believers, even though they
should afterwards sin, would still remain in the faith, and
such sin would not harm them, and [hence] crying thus: "Do
whatever you please; if you believe, it all amounts to
nothing; faith blots out all sins," etc.--they say, besides,
that if any one sins after he has received faith and the
Spirit, he never truly had the Spirit and faith: I have had
before me [seen and heard] many such insane men, and I fear
that in some such a devil is still remaining [hiding and
dwelling].

It is, accordingly, necessary to know and to teach that when
holy men, still having and feeling original sin, also daily
repenting of and striving with it, happen to fall into
manifest sins, as David into adultery, murder, and blasphemy,
that then faith and the Holy Ghost has departed from them
[they cast out faith and the Holy Ghost]. For the Holy Ghost
does not permit sin to have dominion, to gain the upper hand
so as to be accomplished, but represses and restrains it so
that it must not do what it wishes. But if it does what it
wishes, the Holy Ghost and faith are [certainly] not present.
For St. John says, 1 Ep. 3, 9: Whosoever is born of God doth
not commit sin,... and he cannot sin. And yet it is also the
truth when the same St. John says, 1 Ep. 1, 8: If we say that
we have no sin, we deceive ourselves and the truth is not in
us.


IV. Of the Gospel.

We will now return to the Gospel, which not merely in one way
gives us counsel and aid against sin; for God is
superabundantly rich [and liberal] in His grace [and
goodness]. First, through the spoken Word by which the
forgiveness of sins is preached [He commands to be preached]
in the whole world; which is the peculiar office of the
Gospel. Secondly, through Baptism. Thirdly, through the holy
Sacrament of the Altar. Fourthly, through the power of the
keys, and also through the mutual conversation and consolation
of brethren, Matt. 18, 20: Where two or three are gathered
together, etc.


V. Of Baptism.

Baptism is nothing else than the Word of God in the water,
commanded by His institution, or, as Paul says, a washing in
the Word; as also Augustine says: Let the Word come to the
element, and it becomes a Sacrament. And for this reason we do
not hold with Thomas and the monastic preachers [or
Dominicans] who forget the Word (God's institution) and say
that God has imparted to the water a spiritual power, which
through the water washes away sin. Nor [do we agree] with
Scotus and the Barefooted monks [Minorites or Franciscan
monks], who teach that, by the assistance of the divine will,
Baptism washes away sins, and that this ablution occurs only
through the will of God, and by no means through the Word or
water. Of the baptism of children we hold that children ought
to be baptized. For they belong to the promised redemption
made through Christ, and the Church should administer it
[Baptism and the announcement of that promise] to them.


VI. Of the Sacrament of the Altar.

Of the Sacrament of the Altar we hold that bread and wine in
the Supper are the true body and blood of Christ, and are
given and received not only by the godly, but also by wicked
Christians.

And that not only one form is to be given. [For] we do not
need that high art [specious wisdom] which is to teach us that
under the one form there is as much as under both, as the
sophists and the Council of Constance teach. For even if it
were true that there is as much under one as under both, yet
the one form only is not the entire ordinance and institution
[made] ordained and commanded by Christ. And we especially
condemn and in God's name execrate those who not only omit
both forms but also quite autocratically [tyrannically]
prohibit, condemn, and blaspheme them as heresy, and so exalt
themselves against and above Christ, our Lord and God
[opposing and placing themselves ahead of Christ], etc.

As regards transubstantiation, we care nothing about the
sophistical subtlety by which they teach that bread and wine
leave or lose their own natural substance, and that there
remain only the appearance and color of bread, and not true
bread. For it is in perfect agreement with Holy Scriptures
that there is, and remains, bread, as Paul himself calls it,
1 Cor. 10, 16: The bread which we break. And 1 Cor. 11, 28:
Let him so eat of that bread.


VII. Of the Keys.

The keys are an office and power given by Christ to the Church
for binding and loosing sin, not only the gross and well-known
sins, but also the subtle, hidden, which are known only to
God, as it is written in Ps. 19, 13: Who can understand his
errors? And in Rom. 7, 25 St. Paul himself complains that with
the flesh he serves the law of sin. For it is not in our
power, but belongs to God alone, to judge which, how great,
and how many the sins are, as it is written in Ps. 143, 2:
Enter not into judgment with Thy servant; for in Thy sight
shall no man living be justified. And Paul, 1 Cor. 4, 4, says:
For I know nothing by myself; yet am I not hereby justified.

VIII. Of Confession.

Since Absolution or the Power of the Keys is also an aid and
consolation against sin and a bad conscience, ordained by
Christ [Himself] in the Gospel, Confession or Absolution ought
by no means to be abolished in the Church, especially on
account of [tender and] timid consciences and on account of
the untrained [and capricious] young people, in order that
they may be examined, and instructed in the Christian
doctrine.

But the enumeration of sins ought to be free to every one, as
to what he wishes to enumerate or not to enumerate. For as
long as we are in the flesh, we shall not lie when we say: "I
am a poor man [I acknowledge that I am a miserable sinner],
full of sin." Rom. 7, 23: I see another law in my members,
etc. For since private absolution originates in the Office of
the Keys, it should not be despised [neglected], but greatly
and highly esteemed [of the greatest worth], as [also] all
other offices of the Christian Church.

And in those things which concern the spoken, outward Word, we
must firmly hold that God grants His Spirit or grace to no
one, except through or with the preceding outward Word, in
order that we may [thus] be protected against the enthusiasts,
i.e., spirits who boast that they have the Spirit without and
before the Word, and accordingly judge Scripture or the spoken
Word, and explain and stretch it at their pleasure, as Muenzer
did, and many still do at the present day, who wish to be
acute judges between the Spirit and the letter, and yet know
not what they say or declare. For [indeed] the Papacy also is
nothing but sheer enthusiasm, by which the Pope boasts that
all rights exist in the shrine of his heart, and whatever he
decides and commands with [in] his church is spirit and right,
even though it is above and contrary to Scripture and the
spoken Word.

All this is the old devil and old serpent, who also converted
Adam and Eve into enthusiasts, and led them from the outward
Word of God to spiritualizing and self-conceit, and
nevertheless he accomplished this through other outward words.
Just as also our enthusiasts [at the present day] condemn the
outward Word, and nevertheless they themselves are not silent,
but they fill the world with their pratings and writings, as
though, indeed, the Spirit could not come through the writings
and spoken word of the apostles, but [first] through their
writings and words he must come. Why [then] do not they also
omit their own sermons and writings, until the Spirit Himself
come to men, without their writings and before them, as they
boast that Me has come into them without the preaching of the
Scriptures? But of these matters there is not time now to
dispute at greater length; we have elsewhere sufficiently
urged this subject.

For even those who believe before Baptism, or become believing
in Baptism, believe through the preceding outward Word, as the
adults, who have come to reason, must first have heard: He
that believeth and is baptized shall be saved, even though
they are at first unbelieving, and receive the Spirit and
Baptism ten years afterwards. Cornelius, Acts 10, 1 ff., had
heard long before among the Jews of the coming Messiah,
through whom he was righteous before God, and in such faith
his prayers and alms were acceptable to God (as Luke calls him
devout and God-fearing), and without such preceding Word and
hearing could not have believed or been righteous. But St.
Peter had to reveal to him that the Messiah (in whom, as one
that was to come, he had hitherto believed) now had come, lest
his faith concerning the coming Messiah hold him captive among
the hardened and unbelieving Jews, but know that he was now to
be saved by the present Messiah, and must not, with the
[rabble of the] Jews deny nor persecute Him.

In a word, enthusiasm inheres in Adam and his children from
the beginning [from the first fall] to the end of the world,
[its poison] having been implanted and infused into them by
the old dragon, and is the origin, power [life], and strength
of all heresy, especially of that of the Papacy and Mahomet.
Therefore we ought and must constantly maintain this point,
that God does not wish to deal with us otherwise than through
the spoken Word and the Sacraments. It is the devil himself
whatsoever is extolled as Spirit without the Word and
Sacraments. For God wished to appear even to Moses through the
burning bush and spoken Word; and no prophet neither Elijah
nor Elisha, received the Spirit without the Ten Commandments
[or spoken Word]. Neither was John the Baptist conceived
without the preceding word of Gabriel, nor did he leap in his
mother's womb without the voice of Mary. And Peter says,
2. Ep. 1, 21: The prophecy came not by the will of man; but
holy men of God spake as they were moved by the Holy Ghost.
Without the outward Word, however, they were not holy, much
less would the Holy Ghost have moved them to speak when they
still were unholy [or profane]; for they were holy, says he,
since the Holy Ghost spake through them.

IX. Of Excommunication.

The greater excommunication, as the Pope calls it, we regard
only as a civil penalty, and it does not concern us ministers
of the Church. But the lesser, that is, the true Christian
excommunication, consists in this, that manifest and obstinate
sinners are not admitted to the Sacrament and other communion
of the Church until they amend their lives and avoid sin. And
ministers ought not to mingle secular punishments with this
ecclesiastical punishment, or excommunication.

X. Of Ordination and the Call.

If the bishops would be true bishops [would rightly discharge
their office], and would devote themselves to the Church and
the Gospel, it might be granted to them for the sake of love
and unity, but not from necessity, to ordain and confirm us
and our preachers; omitting, however, all comedies and
spectacular display [deceptions, absurdities, and appearances]
of unchristian [heathenish] parade and pomp. But because they
neither are, nor wish to be, true bishops, but worldly lords
and princes, who will neither preach, nor teach, nor baptize,
nor administer the Lord's Supper, nor perform any work or
office of the Church, and, moreover, persecute and condemn
those who discharge these functions, having been called to do
so, the Church ought not on their account to remain without
ministers [to be forsaken by or deprived of ministers].

Therefore, as the ancient examples of the Church and the
Fathers teach us, we ourselves will and ought to ordain
suitable persons to this office; and, even according to their
own laws, they have not the right to forbid or prevent us. For
their laws say that those ordained even by heretics should be
declared [truly] ordained and stay ordained [and that such
ordination must not be changed], as St. Jerome writes of the
Church at Alexandria, that at first it was governed in common
by priests and preachers, without bishops.

XI. Of the Marriage of Priests.

To prohibit marriage, and to burden the divine order of
priests with perpetual celibacy, they have had neither
authority nor right [they have done out of malice, without any
honest reason], but have acted like antichristian, tyrannical,
desperate scoundrels [have performed the work of antichrist,
of tyrants and the worst knaves], and have thereby caused all
kinds of horrible, abominable, innumerable sins of unchastity
[depraved lusts], in which they still wallow. Now, as little
as we or they have been given the power to make a woman out of
a man or a man out of a woman, or to nullify either sex, so
little have they had the power to [sunder and] separate such
creatures of God, or to forbid them from living [and
cohabiting] honestly in marriage with one another. Therefore
we are unwilling to assent to their abominable celibacy, nor
will we [even] tolerate it, but we wish to have marriage free
as God has instituted [and ordained] it, and we wish neither
to rescind nor hinder His work; for Paul says, 1 Tim. 4, 1
ff., that this [prohibition of marriage] is a doctrine of
devils.

XII. Of the Church.

We do not concede to them that they are the Church, and [in
truth] they are not [the Church]; nor will we listen to those
things which, under the name of Church, they enjoin or forbid.
For, thank God, [to-day] a child seven years old knows what
the Church is, namely, the holy believers and lambs who hear
the voice of their Shepherd. For the children pray thus: I
believe in one holy [catholic or] Christian Church. This
holiness does not consist in albs, tonsures, long gowns, and
other of their ceremonies devised by them beyond Holy
Scripture, but in the Word of God and true faith.

XIII. How One is Justified before God, and of Good Works.

What I have hitherto and constantly taught concerning this I
know not how to change in the least, namely, that by faith, as
St. Peter says, we acquire a new and clean heart, and God will
and does account us entirely righteous and holy for the sake
of Christ, our Mediator. And although sin in the flesh has not
yet been altogether removed or become dead, yet He will not
punish or remember it.

And such faith, renewal, and forgiveness of sins is followed
by good works. And what there is still sinful or imperfect
also in them shall not be accounted as sin or defect, even
[and that, too] for Christ's sake; but the entire man, both as
to his person and his works, is to be called and to be
righteous and holy from pure grace and mercy, shed upon us
[unfolded] and spread over us in Christ. Therefore we cannot
boast of many merits and works, if they are viewed apart from
grace and mercy, but as it is written, 1 Cor. 1, 31: He that
glorieth, let him glory in the Lord, namely, that he has a
gracious God. For thus all is well. We say, besides, that if
good works do not follow, faith is false and not true.

XIV. Of Monastic Vows.

As monastic vows directly conflict with the first chief
article, they must be absolutely abolished. For it is of them
that Christ says, Matt. 24, 5. 23 ff.: I am Christ, etc. For
he who makes a vow to live as a monk believes that he will
enter upon a mode of life holier than ordinary Christians
lead, and wishes to earn heaven by his own works not only for
himself, but also for others; this is to deny Christ. And they
boast from their St. Thomas that a monastic vow is equal to
Baptism. This is blasphemy [against God].

XV. Of Human Traditions.

The declaration of the Papists that human traditions serve for
the remission of sins, or merit salvation, is [altogether]
unchristian and condemned, as Christ says Matt. 15, 9: In vain
they do worship Me, teaching for doctrines the commandments of
men. Again, Titus 1, 14: That turn from the truth. Again, when
they declare that it is a mortal sin if one breaks these
ordinances [does not keep these statutes], this, too, is not
right.

These are the articles on which I must stand, and, God
willing, shall stand even to my death; and I do not know how
to change or to yield anything in them. If any one wishes to
yield anything, let him do it at the peril of his conscience.

Lastly, there still remains the Pope's bag of impostures
concerning foolish and childish articles, as, the dedication
of churches, the baptism of bells, the baptism of the
altarstone, and the inviting of sponsors to these rites, who
would make donations towards them. Such baptizing is a
reproach and mockery of Holy Baptism, hence should not be
tolerated. Furthermore, concerning the consecration of
wax-tapers, palm-branches, cakes, oats, [herbs,] spices, etc.,
which indeed, cannot be called consecrations, but are sheer
mockery and fraud. And such deceptions there are without
number, which we commend for adoration to their god and to
themselves, until they weary of it. We will [ought to] have
nothing to do with them.


     Dr. Martin Luther subscribed.

     Dr. Justus Jonas, Rector, subscribed with his own hand.

     Dr. John Bugenhagen, Pomeranus, subscribed.

     Dr. Caspar Creutziger subscribed.

     Nicholas Amsdorf of Magdeburg subscribed.

     George Spalatin of Altenburg subscribed.

     I, Philip Melanchthon, also regard [approve] the above
     articles as right and Christian. But regarding the Pope I hold
     that, if he would allow the Gospel, his superiority over the
     bishops which he has otherwise, is conceded to him by human
     right also by us, for the sake of the peace and general unity
     of those Christians who are also under him, and may be under
     him hereafter.

     John Agricola of Eisleben subscribed.
     Gabriel Didymus subscribed.

     I, Dr. Urban Rhegius, Superintendent of the churches in the
     Duchy of Lueneburg, subscribe in my own name and in the name
     of my brethren, and of the Church of Hanover.

     I, Stephen Agricola, Minister at Hof, subscribe.

     Also I, John Draconites, Professor and Minister at Marburg,
     subscribe.

     I, Conrad Figenbotz, for the glory of God subscribe that I
     have thus believed, and am still preaching and firmly
     believing as above.

     I, Andrew Osiander of Nuernberg, subscribe.
     I, Magister Veit Dieterich, Minister at Nuernberg, subscribe.
     I, Erhard Schnepf, Preacher at Stuttgart, subscribe.
     Conrad Oettinger, Preacher of Duke Ulrich at Pforzheim.
     Simon Schneeweiss, Pastor of the Church at Crailsheim.

     I, John Schlagenhaufen, Pastor of the Church at Koethen,
     subscribe.

     The Reverend Magister George Helt of Forchheim.
     The Reverend Magister Adam of Fulda, Preacher in Hesse.
     The Reverend Magister Anthony Corvinus, Preacher in Hesse.

     I, Doctor John Bugenhagen, Pomeranus, again subscribe in the
     name of Magister John Brentz, as on departing from Smalcald he
     directed me orally and by a letter, which I have shown to
     these brethren who have subscribed.

     I, Dionysius Melander, subscribe to the Confession, the
     Apology, and the Concordia on the subject of the Eucharist.

     Paul Rhodius, Superintendent of Stettin.
     Gerard Oemcken, Superintendent of the Church at Minden.

     I, Brixius Northanus, Minister of the Church of Christ which
     is at Soest, subscribe to the Articles of the Reverend Father
     Martin Luther, and confess that hitherto I have thus believed
     and taught, and by the Spirit of Christ I shall continue thus
     to believe and teach.

     Michael Caelius, Preacher at Mansfeld, subscribed.
     The Reverend Magister Peter Geltner Preacher at Frankfort,
     subscribed.
     Wendal Faber, Pastor of Seeburg in Mansfeld.

     I, John Aepinus, subscribe.
     Likewise, I, John Amsterdam of Bremen.

     I, Frederick Myconius, Pastor of the Church at Gotha in
     Thuringia, subscribe in my own name and in that of Justus
     Menius of Eisenach.

     I, Doctor John Lang, Preacher of the Church at Erfurt,
     subscribe with my own hand in my own name, and in that of my
     other coworkers in the Gospel, namely:
     The Reverend Licentiate Ludwig Platz of Melsungen.
     The Reverend Magister Sigismund Kirchner,
     The Reverend Wolfgang Kiswetter,
     The Reverend Melchior Weitmann
     The Reverend John Thall.
     The Reverend John Kilian.
     The Reverend Nicholas Faber.
     The Reverend Andrew Menser.

     And I, Egidius Mechler, have subscribed with my own hand.
